%%%%%%%%%%%%%%%%%%%%%%%%%%%%%%%%%%%%%%%%%%%%%%%%%%%%%%%%%%%%%%%%%%%%%%%%%%%%%%%%
%%%%%%%%%%%%%%%%%%%%%%%%%%%%%%%%%%%%%%%%%%%%%%%%%%%%%%%%%%%%%%%%%%%%%%%%%%%%%%%%
%%%%%%%%%%%%%%%%%%%%%%%%%%%%%%%%%%%%%%%%%%%%%%%%%%%%%%%%%%%%%%%%%%%%%%%%%%%%%%%%
\section{Contexte}
%%%%%%%%%%%%%%%%%%%%%%%%%%%%%%%%%%%%%%%%%%%%%%%%%%%%%%%%%%%%%%%%%%%%%%%%%%%%%%%%
%%%%%%%%%%%%%%%%%%%%%%%%%%%%%%%%%%%%%%%%%%%%%%%%%%%%%%%%%%%%%%%%%%%%%%%%%%%%%%%%
%%%%%%%%%%%%%%%%%%%%%%%%%%%%%%%%%%%%%%%%%%%%%%%%%%%%%%%%%%%%%%%%%%%%%%%%%%%%%%%%

Dans le cadre d'un asservissement à retour unitaire, la fonction de transfert 
complexe en boucle fermée $H_{BF}(\jw)$ peut être établie à partir de la 
fonction de transfert complexe en boucle ouverte $H_{BO}(\jw)$.
\[
    H_{BF}(\jw)=\dfrac{H_{BO}(\jw)}{1+H_{BO}(\jw)}
\]
On formalise cette relation par l'application de $\mathbb{C}$ dans  $\mathbb{C}$:
%-------------------------------------------------------------------------------
\begin{equation*}
    z\to\dfrac{z}{1+z}
\end{equation*}
%-------------------------------------------------------------------------------
C'est cette application que nous allons étudier dans cette annexe, nous la 
représenterons à la fois sur les diagramme de Nyquist et celui de Black.
L'idée est de tracer sur le même diagramme les contours d'isogain et d'isophase 
(respectivement les contours de module et d'arguments constant 
en boucle fermée).
%%%%%%%%%%%%%%%%%%%%%%%%%%%%%%%%%%%%%%%%%%%%%%%%%%%%%%%%%%%%%%%%%%%%%%%%%%%%%%%%
%%%%%%%%%%%%%%%%%%%%%%%%%%%%%%%%%%%%%%%%%%%%%%%%%%%%%%%%%%%%%%%%%%%%%%%%%%%%%%%%
%%%%%%%%%%%%%%%%%%%%%%%%%%%%%%%%%%%%%%%%%%%%%%%%%%%%%%%%%%%%%%%%%%%%%%%%%%%%%%%%
\section{Dans le plan complexe (\emph{Hall circles})}
%%%%%%%%%%%%%%%%%%%%%%%%%%%%%%%%%%%%%%%%%%%%%%%%%%%%%%%%%%%%%%%%%%%%%%%%%%%%%%%%
%%%%%%%%%%%%%%%%%%%%%%%%%%%%%%%%%%%%%%%%%%%%%%%%%%%%%%%%%%%%%%%%%%%%%%%%%%%%%%%%
%%%%%%%%%%%%%%%%%%%%%%%%%%%%%%%%%%%%%%%%%%%%%%%%%%%%%%%%%%%%%%%%%%%%%%%%%%%%%%%%
Dans le diagramme de Nyquist (représentant les parties réelles et imaginaires de 
la fonction de transfert en boucle ouverte), les contours d'isogain et 
d'isophase de la boucle fermée sont représentés par des cercles. 
On parle de respectivement de M-cercles et de N-cercles. 
%%%%%%%%%%%%%%%%%%%%%%%%%%%%%%%%%%%%%%%%%%%%%%%%%%%%%%%%%%%%%%%%%%%%%%%%%%%%%%%%
%%%%%%%%%%%%%%%%%%%%%%%%%%%%%%%%%%%%%%%%%%%%%%%%%%%%%%%%%%%%%%%%%%%%%%%%%%%%%%%%
\subsection{M-cercles}
%%%%%%%%%%%%%%%%%%%%%%%%%%%%%%%%%%%%%%%%%%%%%%%%%%%%%%%%%%%%%%%%%%%%%%%%%%%%%%%%
%%%%%%%%%%%%%%%%%%%%%%%%%%%%%%%%%%%%%%%%%%%%%%%%%%%%%%%%%%%%%%%%%%%%%%%%%%%%%%%%
On se donne une valeur $M$ représentant le gain en boucle fermée (en \si{\dB}). 
Les contours de module constant $M$ sur le diagramme de Nyquist 
sont données par :
\[
    M=|H_{BF}(\jw)|=\dfrac{|H_{B0}(\jw)|}{|1+H_{B0}(\jw)|}
\]
%-------------------------------------------------------------------------------
\begin{figure}[!h]
\begin{center}
    \includegraphics[width=0.8\textwidth]{notebook/fig/black_nichols_3_0.eps}
\end{center}
\end{figure}
%-------------------------------------------------------------------------------
%%%%%%%%%%%%%%%%%%%%%%%%%%%%%%%%%%%%%%%%%%%%%%%%%%%%%%%%%%%%%%%%%%%%%%%%%%%%%%%%
%%%%%%%%%%%%%%%%%%%%%%%%%%%%%%%%%%%%%%%%%%%%%%%%%%%%%%%%%%%%%%%%%%%%%%%%%%%%%%%%
\subsection{N-cercles}
%%%%%%%%%%%%%%%%%%%%%%%%%%%%%%%%%%%%%%%%%%%%%%%%%%%%%%%%%%%%%%%%%%%%%%%%%%%%%%%%
%%%%%%%%%%%%%%%%%%%%%%%%%%%%%%%%%%%%%%%%%%%%%%%%%%%%%%%%%%%%%%%%%%%%%%%%%%%%%%%%
On se donne une valeur $N$ représentant la phase en boucle fermée 
(en \si{\degree}). Les contours de phase constante $N$ sur le diagramme 
de Nyquist sont données par :
\[
    N=\arg{H_{BF}(\jw)}=\arg{\dfrac{H_{B0}(\jw)}{1+H_{B0}(\jw)}}
\]
%-------------------------------------------------------------------------------
\begin{figure}[!h]
    \begin{center}
        \includegraphics[width=0.8\textwidth]{notebook/fig/black_nichols_5_0.eps}
    \end{center}
\end{figure}
%-------------------------------------------------------------------------------
\clearpage
%%%%%%%%%%%%%%%%%%%%%%%%%%%%%%%%%%%%%%%%%%%%%%%%%%%%%%%%%%%%%%%%%%%%%%%%%%%%%%%%
%%%%%%%%%%%%%%%%%%%%%%%%%%%%%%%%%%%%%%%%%%%%%%%%%%%%%%%%%%%%%%%%%%%%%%%%%%%%%%%%
%%%%%%%%%%%%%%%%%%%%%%%%%%%%%%%%%%%%%%%%%%%%%%%%%%%%%%%%%%%%%%%%%%%%%%%%%%%%%%%%
\section{Abaque de Black-Nichols}
%%%%%%%%%%%%%%%%%%%%%%%%%%%%%%%%%%%%%%%%%%%%%%%%%%%%%%%%%%%%%%%%%%%%%%%%%%%%%%%%
%%%%%%%%%%%%%%%%%%%%%%%%%%%%%%%%%%%%%%%%%%%%%%%%%%%%%%%%%%%%%%%%%%%%%%%%%%%%%%%%
%%%%%%%%%%%%%%%%%%%%%%%%%%%%%%%%%%%%%%%%%%%%%%%%%%%%%%%%%%%%%%%%%%%%%%%%%%%%%%%%
L'abaque de Black-Nichols permet de repérer par un système de doubles 
coordonnées les valeurs des modules et de la phase de la FTBF correspondant à 
un lieu de Black particulier de la FTBO. Les contours d'isogain sont 
concentriques autour du point critique $(-180,\SI{0}{dB})$. Les contours 
d'isophase sont orthogonaux aux contours d'isogain.
On constate que le gain en boucle fermée augmente rapidement et tend vers 
$\infty$ quand on s'approche du point critique.
%-------------------------------------------------------------------------------
\begin{figure}[!h]
\begin{center}
    \includegraphics[width=\textwidth]{notebook/fig/black_nichols_8_0.eps}
\end{center}
    \caption{Abaque de Black-Nichols.}
\end{figure}
%-------------------------------------------------------------------------------
\clearpage
%%%%%%%%%%%%%%%%%%%%%%%%%%%%%%%%%%%%%%%%%%%%%%%%%%%%%%%%%%%%%%%%%%%%%%%%%%%%%%%%
%%%%%%%%%%%%%%%%%%%%%%%%%%%%%%%%%%%%%%%%%%%%%%%%%%%%%%%%%%%%%%%%%%%%%%%%%%%%%%%%
%%%%%%%%%%%%%%%%%%%%%%%%%%%%%%%%%%%%%%%%%%%%%%%%%%%%%%%%%%%%%%%%%%%%%%%%%%%%%%%%
\section{Exemples d'application}\label{exemples-dapplication}
%%%%%%%%%%%%%%%%%%%%%%%%%%%%%%%%%%%%%%%%%%%%%%%%%%%%%%%%%%%%%%%%%%%%%%%%%%%%%%%%
%%%%%%%%%%%%%%%%%%%%%%%%%%%%%%%%%%%%%%%%%%%%%%%%%%%%%%%%%%%%%%%%%%%%%%%%%%%%%%%%
%%%%%%%%%%%%%%%%%%%%%%%%%%%%%%%%%%%%%%%%%%%%%%%%%%%%%%%%%%%%%%%%%%%%%%%%%%%%%%%%

%%%%%%%%%%%%%%%%%%%%%%%%%%%%%%%%%%%%%%%%%%%%%%%%%%%%%%%%%%%%%%%%%%%%%%%%%%%%%%%%
%%%%%%%%%%%%%%%%%%%%%%%%%%%%%%%%%%%%%%%%%%%%%%%%%%%%%%%%%%%%%%%%%%%%%%%%%%%%%%%%
\subsection{Lecture du gain et du déphasage en boucle fermée}
%%%%%%%%%%%%%%%%%%%%%%%%%%%%%%%%%%%%%%%%%%%%%%%%%%%%%%%%%%%%%%%%%%%%%%%%%%%%%%%%
%%%%%%%%%%%%%%%%%%%%%%%%%%%%%%%%%%%%%%%%%%%%%%%%%%%%%%%%%%%%%%%%%%%%%%%%%%%%%%%%
%-------------------------------------------------------------------------------
\begin{figure}[!h]
\begin{center}
    \includegraphics[width=0.45\textwidth]{notebook/fig/black_nichols_11_1.eps}
    \includegraphics[width=0.45\textwidth]{notebook/fig/black_nichols_12_1.eps}
\end{center}
    \caption{Diagrammes de Bode (à gauche) en boucle ouverte et (à droite) 
    boucle fermée.}
\end{figure}
%-------------------------------------------------------------------------------


%-------------------------------------------------------------------------------
\begin{figure}[!h]
\begin{center}
    \includegraphics[width=0.8\textwidth]{notebook/fig/black_nichols_13_1.eps}
\end{center}
\end{figure}
%-------------------------------------------------------------------------------
    
%-------------------------------------------------------------------------------
\begin{figure}[!h]
\begin{center}
    \includegraphics[width=0.45\textwidth]{notebook/fig/black_nichols_14_1.eps}
    \includegraphics[width=0.45\textwidth]{notebook/fig/black_nichols_15_1.eps}
\end{center}
\end{figure}
%-------------------------------------------------------------------------------

%-------------------------------------------------------------------------------
\begin{figure}[!h]
\begin{center}
    \includegraphics[width=0.8\textwidth]{notebook/fig/black_nichols_16_2.eps}
\end{center}
\end{figure}
%-------------------------------------------------------------------------------

%-------------------------------------------------------------------------------
\begin{center}
\begin{tabular}{ccc}
\hline
$\omega$ [\si{\radian\per\second}] & Gain [\si{\decibel}] & Phase [\si{\degree}]\\
\hline
1.00 & 6.02 & -90.00\\
1.04 & 5.56 & -99.21\\
1.08 & 4.88 & -107.99\\
1.13 & 4.04 & -116.00\\
1.18 & 3.08 & -123.09\\
1.22 & 2.04 & -129.23\\
1.28 & 0.97 & -134.50\\
1.33 & -0.11 & -139.01\\
1.38 & -1.18 & -142.86\\
1.44 & -2.24 & -146.17\\
1.50 & -3.27 & -149.04\\
\hline
\end{tabular}
\end{center}
%-------------------------------------------------------------------------------

%-------------------------------------------------------------------------------
\begin{figure}[!h]
\begin{center}
    \includegraphics[width=0.8\textwidth]{notebook/fig/black_nichols_18_2.eps}
\end{center}
\end{figure}
%-------------------------------------------------------------------------------

%-------------------------------------------------------------------------------
\begin{center}
\begin{tabular}{ccc}
\hline
$\omega$ \si{\radian\per\second} & Gain\si{\decibel} & Phase\si{\degree}\\
\hline
0.10 & -8.01 & -10.29\\
0.16 & -8.09 & -16.27\\
0.25 & -8.29 & -25.60\\
0.40 & -8.76 & -39.91\\
0.63 & -9.84 & -60.99\\
1.00 & -12.04 & -90.00\\
1.58 & -16.10 & -126.32\\
2.51 & -22.73 & -165.72\\
3.98 & -31.94 & -200.17\\
6.31 & -42.76 & -225.05\\
10.00 & -54.29 & -241.46\\
\hline
\end{tabular}
\end{center}
%-------------------------------------------------------------------------------

%-------------------------------------------------------------------------------
\begin{figure}[!h]
\begin{center}
    \includegraphics[width=0.8\textwidth]{notebook/fig/black_nichols_20_2.eps}
\end{center}
\end{figure}
%-------------------------------------------------------------------------------
    

