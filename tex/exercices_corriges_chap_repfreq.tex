%%%%%%%%%%%%%%%%%%%%%%%%%%%%%%%%%%%%%%%%%%%%%%%%%%%%%%%%%%%%%%%%%%%%%%%%%%%%%%%%
%%%%%%%%%%%%%%%%%%%%%%%%%%%%%%%%%%%%%%%%%%%%%%%%%%%%%%%%%%%%%%%%%%%%%%%%%%%%%%%%
\exercice{Représentation graphique d'un signal harmonique~\facile}
%%%%%%%%%%%%%%%%%%%%%%%%%%%%%%%%%%%%%%%%%%%%%%%%%%%%%%%%%%%%%%%%%%%%%%%%%%%%%%%%
%%%%%%%%%%%%%%%%%%%%%%%%%%%%%%%%%%%%%%%%%%%%%%%%%%%%%%%%%%%%%%%%%%%%%%%%%%%%%%%%
%%%%%%%%%%%%%%%%%%%%%%%%%%%%%%%%%%%%%%%%%%%%%%%%%%%%%%%%%%%%%%%%%%%%%%%%%%%%%%%%
\question{\textbf{Déterminer le gain naturel $G(\omega)=|H(\jw)|$ et la 
phase $\phi(\omega)$ à partir de la fonction de transfert.}}
%%%%%%%%%%%%%%%%%%%%%%%%%%%%%%%%%%%%%%%%%%%%%%%%%%%%%%%%%%%%%%%%%%%%%%%%%%%%%%%%
Le gain naturel et la phase de la réponse harmonique sont déterminés 
repectivement par le module et l'argument du nombre complexe $H(\jw)$, ici 
tel que :
\[
H(\jw)=\dfrac{1}{1+2\jw}
\]
On obtient alors pour le gain :
\[
G(\omega)=|H(\jw)|=\dfrac{1}{\sqrt{1+4\omega^2}}
\]
et pour la phase :
\[
\phi(\omega)=\arctan{2\omega}
\]
%%%%%%%%%%%%%%%%%%%%%%%%%%%%%%%%%%%%%%%%%%%%%%%%%%%%%%%%%%%%%%%%%%%%%%%%%%%%%%%%
\question{\textbf{Tracer graphiquement la sollicitation $e(t)$ pour $E_0=1$,
$\omega=\SI{1}{\radian\per\second}$ et  $\omega=\SI{2}{\radian\per\second}$}.}
%%%%%%%%%%%%%%%%%%%%%%%%%%%%%%%%%%%%%%%%%%%%%%%%%%%%%%%%%%%%%%%%%%%%%%%%%%%%%%%%
%-------------------------------------------------------------------------------
\begin{center}
    \tikzsetnextfilename
    {sollicitation_harmonique-exercices_corriges-chap_repfreq-ext}
    \newcommand{\rps}{\radian\per\second}
\begin{tikzpicture}
    \begin{axis}[
        axis line style = thick,
        height=8cm,
        width=12cm,
        axis x line=center,
        axis y line=center,
        xmin=0,
        xmax=12,
        ymin=-1.25,
        ymax=1.25,
        xlabel={$t$},
        ylabel={$e(t)$},
        xlabel style={below right},
        ylabel style={left},
        grid=both,
        minor tick num=4,
        clip=false,
        ]
    \addplot[signalb,domain=0:12]{sin(deg(x))};
    \addplot[signalr,domain=0:12]{sin(2*deg(x))};
    \node[left,color=col1] at (axis cs:16,1.0)  {\large $\omega=\SI{1}{\rps}$};
    \node[left,color=col4] at (axis cs:16,0.75) {\large $\omega=\SI{2}{\rps}$};
    \end{axis}
\end{tikzpicture}

\end{center}
%-------------------------------------------------------------------------------
%%%%%%%%%%%%%%%%%%%%%%%%%%%%%%%%%%%%%%%%%%%%%%%%%%%%%%%%%%%%%%%%%%%%%%%%%%%%%%%%
\question{\textbf{Déterminer numériquement le gain et le déphasage 
pour  $\omega=\SI{0.1}{\radian\per\second}$, $\omega=\SI{1}{\radian\per\second}$
et $\omega=\SI{2}{\radian\per\second}$.}}
%%%%%%%%%%%%%%%%%%%%%%%%%%%%%%%%%%%%%%%%%%%%%%%%%%%%%%%%%%%%%%%%%%%%%%%%%%%%%%%%
%-------------------------------------------------------------------------------
\begin{center}
\begin{tabular}{P{3cm}P{3cm}P{3cm}}
    \hline
    $\omega$ (\si{\radian\per\second}) & 
    $G(\omega)$                        & 
    $\phi(\omega)$ (\degreeSI)\tabularnewline
    \hline
    0.1 & 0.98 & -11.3 \tabularnewline
    1.0 & 0.45 & -63   \tabularnewline
    2.0 & 0.24 & -75   \tabularnewline
    \hline
\end{tabular}
\end{center}
%-------------------------------------------------------------------------------
%%%%%%%%%%%%%%%%%%%%%%%%%%%%%%%%%%%%%%%%%%%%%%%%%%%%%%%%%%%%%%%%%%%%%%%%%%%%%%%%
\question{\textbf{Tracer graphiquement la réponse harmonique avec la 
sollicitation pour $\omega=\SI{0.1}{\radian\per\second}$, 
$\omega=\SI{1}{\radian\per\second}$ et $\omega=\SI{2}{\radian\per\second}$. 
On choisira judicieusement l'échelle temporelle pour représenter une periode 
complète.}}
%%%%%%%%%%%%%%%%%%%%%%%%%%%%%%%%%%%%%%%%%%%%%%%%%%%%%%%%%%%%%%%%%%%%%%%%%%%%%%%%
%-------------------------------------------------------------------------------
\begin{center}
    \tikzsetnextfilename
    {reponse1_harmonique-exercices_corriges-chap_repfreq-ext}
    \begin{tikzpicture}
    \begin{axis}[
        axis line style = thick,
        height=8cm,
        width=12cm,
        axis x line=center,
        axis y line=center,
        xmin=0,
        xmax=120,
        ymin=-1.25,
        ymax=1.25,
        xlabel={$t$},
        ylabel={$e(t)$},
        xlabel style={below right},
        ylabel style={left},
        grid=both,
        minor tick num=4,
        ]
        \addplot[signalb,domain=0:120]{sin(0.1*deg(x))};
        \addplot[signalr,domain=0:120]{0.98*sin(0.1*deg(x)-11.3)};
    \end{axis}
\end{tikzpicture}

\end{center}
%-------------------------------------------------------------------------------
%-------------------------------------------------------------------------------
\begin{center}
    \tikzsetnextfilename
    {reponse2_harmonique-exercices_corriges-chap_repfreq-ext}
    \begin{tikzpicture}
    \begin{axis}[
        axis line style = thick,
        height=8cm,
        width=12cm,
        axis x line=center,
        axis y line=center,
        xmin=0,
        xmax=12,
        ymin=-1.25,
        ymax=1.25,
        xlabel={$t$},
        ylabel={$e(t)$},
        xlabel style={below right},
        ylabel style={left},
        grid=both,
        minor tick num=4,
        ]
    \addplot[signalb,domain=0:12]{sin(deg(x))};
    \addplot[signalr,domain=0:12]{0.45*sin(deg(x)-63)};
    \end{axis}
\end{tikzpicture}

\end{center}
%-------------------------------------------------------------------------------
%-------------------------------------------------------------------------------
\begin{center}
    \tikzsetnextfilename
    {reponse3_harmonique-exercices_corriges-chap_repfreq-ext}
    \begin{tikzpicture}
    \begin{axis}[
        axis line style = thick,
        height=8cm,
        width=12cm,
        axis x line=center,
        axis y line=center,
        xmin=0,
        xmax=6,
        ymin=-1.25,
        ymax=1.25,
        xlabel={$t$},
        ylabel={$e(t)$},
        xlabel style={below right},
        ylabel style={left},
        grid=both,
        minor tick num=4,
        ]
        \addplot[signalb,domain=0:6]{sin(2*deg(x))};
        \addplot[signalr,domain=0:6]{0.24*sin(2*deg(x)-75)};
    \end{axis}
\end{tikzpicture}

\end{center}
%-------------------------------------------------------------------------------
\clearpage
%%%%%%%%%%%%%%%%%%%%%%%%%%%%%%%%%%%%%%%%%%%%%%%%%%%%%%%%%%%%%%%%%%%%%%%%%%%%%%%%
%%%%%%%%%%%%%%%%%%%%%%%%%%%%%%%%%%%%%%%%%%%%%%%%%%%%%%%%%%%%%%%%%%%%%%%%%%%%%%%%
\exercice{Diagramme de Bode~\moyen}
%%%%%%%%%%%%%%%%%%%%%%%%%%%%%%%%%%%%%%%%%%%%%%%%%%%%%%%%%%%%%%%%%%%%%%%%%%%%%%%%
%%%%%%%%%%%%%%%%%%%%%%%%%%%%%%%%%%%%%%%%%%%%%%%%%%%%%%%%%%%%%%%%%%%%%%%%%%%%%%%%
\textbf{Tracer les diagrammes asymptotiques et réels de Bode des systèmes 
définis par les fonctions de transfert suivantes :}
%% QUESTION 1 %%%%%%%%%%%%%%%%%%%%%%%%%%%%%%%%%%%%%%%%%%%%%%%%%%%%%%%%%%%%%%%%%%
%%%%%%%%%%%%%%%%%%%%%%%%%%%%%%%%%%%%%%%%%%%%%%%%%%%%%%%%%%%%%%%%%%%%%%%%%%%%%%%%
\question{}
%%%%%%%%%%%%%%%%%%%%%%%%%%%%%%%%%%%%%%%%%%%%%%%%%%%%%%%%%%%%%%%%%%%%%%%%%%%%%%%%
\[
\boldsymbol{H(p) = \dfrac{1}{10p+1}}
\]
%%%%%%%%%%%%%%%%%%%%%%%%%%%%%%%%%%%%%%%%%%%%%%%%%%%%%%%%%%%%%%%%%%%%%%%%%%%%%%%%
\paragraph{Recherche des asymptotes}
%%%%%%%%%%%%%%%%%%%%%%%%%%%%%%%%%%%%%%%%%%%%%%%%%%%%%%%%%%%%%%%%%%%%%%%%%%%%%%%%
%-------------------------------------------------------------------------------
\begin{itemize}
    \item $\textrm{gain statique :} K=1      \rightarrow 
           \textrm{toute fréquence}             \rightarrow 
           \textrm{0 dB à basse fréquence}      \rightarrow 
           \textrm{pas d'effet sur la phase}$
    \item $\textrm{1er ordre : }\tau_1=10        \rightarrow 
           \omega_1=0.1                         \rightarrow 
           \textrm{pente du gain : }
           \SI[per-mode=symbol]{-20}{\dB\per\dec} \rightarrow 
        \phi\mathrel{+}=-90\si{\degree}$ 
\end{itemize}
%-------------------------------------------------------------------------------
%-------------------------------------------------------------------------------
\begin{center}
    \tikzsetnextfilename{gainQ1-exercices_corriges-chap_repfreq-ext}
    \begin{tikzpicture}[trim axis left]
    \begin{axis}[
    ticklabel style = {font=\footnotesize},
    width=0.9\textwidth,
    height=0.24\textheight,
    ylabel={Gain (\si{\decibel})},
    xtick={1e-4,1e-3,1e-2,1e-1,1,1e1,1e2},
    ytick={-60,-50,-40,-30,-20,-10,0,10,20},
    xticklabels={$10^{-4}$,
                 $10^{-3}$,
                 $10^{-2}$,
                 $10^{-1}$,
                 $10^{0}$,
                 $10^{1}$,
                 $10^{2}$},
    yticklabels={-60,,-40,,-20,,0,,20},
    xmode=log,ymode=normal,
    xmin=1e-4, xmax=1e2,
    ymin=-60, ymax=10,
    grid=both,
    major grid style={black!40}
    ]
    \addplot[signalb,domain=1e-4:1e5] {-10*log10(1+100*x*x)};
    \addplot[signalr,line width=2pt,dashed,domain=1e-4:1e-1] {0};
    \addplot[signalr,line width=2pt,dashed,domain=1e-1:1e5] {-20-20*log10(x)};
    \draw[draw=none,fill=col1] (axis cs:0.01,-0.0432) circle[radius=2pt];
    \draw[draw=none,fill=col1] (axis cs:0.1,-3.0103)  circle[radius=2pt];
    \draw[draw=none,fill=col1] (axis cs:1,-20.0432)   circle[radius=2pt];
    \end{axis}
\end{tikzpicture}


    \tikzsetnextfilename{phaseQ1-exercices_corriges-chap_repfreq-ext}
    \begin{tikzpicture}[trim axis left]
\begin{axis}[
    ticklabel style = {font=\footnotesize},
    width=0.9\textwidth,
    height=0.24\textheight,
    xlabel={Pulsation (\si{\radian\per\second})},
    ylabel={Phase (\si{degree})},
    xtick={1e-4,1e-3,1e-2,1e-1,1,1e1,1e2},
    ytick={-90,-75,-60,-45,-30,-15,0},
    yticklabels={-90,-75,-60,-45,-30,-15,0},
    xticklabels={$10^{-4}$,$10^{-3}$,$10^{-2}$,$10^{-1}$,
                 $10^{0}$,$10^{1}$,$10^{2}$},
    xmode=log,ymode=normal,
    xmin=1e-4, xmax=1e2,
    ymin=-90, ymax=0,
    grid=both,
    major grid style={black!40}
]
    \addplot[signalb,domain=1e-4:1e5] {-atan2(10*x,1)};
    \addplot[signalr,line width=2pt,dashed,domain=1e-4:1e-1] {0};
    \addplot[signalr,line width=2pt,dashed,domain=1e-1:1e5] {-90};
    \draw[line width=2pt,red,dashed] (axis cs:1e-1,0)  -- (axis cs:1e-1,-90);
    \draw[draw=none,fill=col1] (axis cs:0.01,-5.71) circle[radius=2pt];
    \draw[draw=none,fill=col1] (axis cs:0.1,-45)    circle[radius=2pt];
    \draw[draw=none,fill=col1] (axis cs:1,-84.29)   circle[radius=2pt];
\end{axis}
\end{tikzpicture}

\end{center}
%-------------------------------------------------------------------------------
%%%%%%%%%%%%%%%%%%%%%%%%%%%%%%%%%%%%%%%%%%%%%%%%%%%%%%%%%%%%%%%%%%%%%%%%%%%%%%%%
\paragraph{Fonctions réelles du gain et du déphasage}
%%%%%%%%%%%%%%%%%%%%%%%%%%%%%%%%%%%%%%%%%%%%%%%%%%%%%%%%%%%%%%%%%%%%%%%%%%%%%%%%
\[
G(\omega)=|H(\jw)|=\dfrac{1}{\sqrt{1+\tau_1^2\omega^2}}
\]
\[
G_{dB}(\omega)=-10\log{(1+\tau_1^2\omega^2)}
\]
\[
\phi(\omega)=\arg{H(\jw)}=-\arctan{\tau_1\omega}
\]
%%%%%%%%%%%%%%%%%%%%%%%%%%%%%%%%%%%%%%%%%%%%%%%%%%%%%%%%%%%%%%%%%%%%%%%%%%%%%%%%
\paragraph{Quelques valeurs particulières (calculées) :}
%%%%%%%%%%%%%%%%%%%%%%%%%%%%%%%%%%%%%%%%%%%%%%%%%%%%%%%%%%%%%%%%%%%%%%%%%%%%%%%%
%-------------------------------------------------------------------------------
\begin{center}
\begin{tabular}{P{3.5cm}P{2cm}P{2cm}P{2cm}}
\hline
\hline
Pulsation (\si{\radian\per\second}) & $10^{-2}$ & $\omega_1=0.1$ & 1 \\[1em]
Gain (\si{\decibel})                &     0     &     -3     &   -20 \\[1em]
Déphasage (\si{\degree})            &    -6     &     -45    &   -84 \\[1em]
\hline
\hline
\end{tabular}
\end{center}
%-------------------------------------------------------------------------------
\newpage
%%% QUESTION 2 %%%%%%%%%%%%%%%%%%%%%%%%%%%%%%%%%%%%%%%%%%%%%%%%%%%%%%%%%%%%%%%%%
%%%%%%%%%%%%%%%%%%%%%%%%%%%%%%%%%%%%%%%%%%%%%%%%%%%%%%%%%%%%%%%%%%%%%%%%%%%%%%%%
\question{}
%%%%%%%%%%%%%%%%%%%%%%%%%%%%%%%%%%%%%%%%%%%%%%%%%%%%%%%%%%%%%%%%%%%%%%%%%%%%%%%%
\[
\boldsymbol{H(p) = \dfrac{100}{(10p+1)(100p+1)}}
\]
%%%%%%%%%%%%%%%%%%%%%%%%%%%%%%%%%%%%%%%%%%%%%%%%%%%%%%%%%%%%%%%%%%%%%%%%%%%%%%%%
\paragraph{Recherche des asymptotes}
%%%%%%%%%%%%%%%%%%%%%%%%%%%%%%%%%%%%%%%%%%%%%%%%%%%%%%%%%%%%%%%%%%%%%%%%%%%%%%%%
%-------------------------------------------------------------------------------
\begin{itemize}
\item $\textrm{gain statique : } K=100      \rightarrow 
     \textrm{toute fréquence}           \rightarrow 
     \textrm{40 dB à basse fréquence}   \rightarrow 
     \textrm{pas d'effet sur la phase}$
\item $\textrm{1er ordre : }\tau_1=100     \rightarrow 
     \omega_1=10^{-2}                   \rightarrow 
     \textrm{pente du gain : } 
     \SI[per-mode=symbol]{-20}{\dB\per\dec}  \rightarrow 
     \phi\mathrel{+}=-90\si{\degree}$ 
\item $ \textrm{1er ordre : }\tau_2=10     \rightarrow 
     \omega_2=10^{-1}                   \rightarrow 
     \textrm{pente du gain : } 
     \SI[per-mode=symbol]{-20}{\dB\per\dec} \rightarrow 
     \phi\mathrel{+}=-90\si{\degree}$ 
\end{itemize}
%-------------------------------------------------------------------------------
%-------------------------------------------------------------------------------
\begin{center}
    \tikzsetnextfilename{gainQ2-exercices_corriges-chap_repfreq-ext}
    \begin{tikzpicture}[trim axis left]
\begin{axis}[
    ticklabel style = {font=\footnotesize},
    width=0.9\textwidth,
    height=0.24\textheight,
    ylabel={Gain (\si{\decibel})},
    xtick={1e-5,1e-4,1e-3,1e-2,1e-1,1e0,1e1,1e2,1e3,1e4},
    ytick={-80,-60,-40,-20,0,20,40,60},
    xticklabels={$10^{-5}$,
                 $10^{-4}$,
                 $10^{-3}$,
                 $10^{-2}$,
                 $10^{-1}$,
                 $10^{0}$,
                 $10^{1}$,
                 $10^{2}$,
                 $10^{3}$,
                 $10^{4}$},
    yticklabels={-80,-60,-40,-20,0,20,40,60},
    xmode=log,ymode=normal,
    xmin=1e-4, xmax=1e1,
    ymin=-60, ymax=60,
    grid=both,
    major grid style={black!40}
]
   \addplot[signalb,domain=1e-5:1e4] 
   {40-10*log10(1+100*x*x)-10*log10(1+10000*x*x)};
   \addplot[signalr,line width=2pt,dashed,domain=1e-5:1e-2] {40};
   \addplot[signalr,line width=2pt,dashed,domain=1e-2:1e-1] {-20*log10(x)};
   \addplot[signalr,line width=2pt,dashed,domain=1e-1:1e4 ] {-20-40*log10(x)};
   \draw[draw=none,fill=col1] (axis cs:0.001,39.95) circle[radius=2pt];
    \draw[draw=none,fill=col1] (axis cs:0.01,36.95) circle[radius=2pt];
    \draw[draw=none,fill=col1] (axis cs:0.1,16.95) circle[radius=2pt];
    \draw[draw=none,fill=col1] (axis cs:1,-20.04) circle[radius=2pt];
\end{axis}
\end{tikzpicture}


    \tikzsetnextfilename{phaseQ2-exercices_corriges-chap_repfreq-ext}
    \begin{tikzpicture}[trim axis left]
\begin{axis}[
    ticklabel style = {font=\footnotesize},
    width=0.9\textwidth,
    height=0.24\textheight,
    xlabel={Pulsation (\si{\radian\per\second})},
    ylabel={Phase (\si{degree})},
    xtick={1e-5,1e-4,1e-3,1e-2,1e-1,1e0,1e1,1e2,1e3,1e4},
    ytick={-180,-150,-120,-90,-60,-30,0},
    yticklabels={-180,-150,-120,-90,-60,-30,0},
    xticklabels={$10^{-5}$,
                 $10^{-4}$,
                 $10^{-3}$,
                 $10^{-2}$,
                 $10^{-1}$,
                 $10^{0}$,
                 $10^{1}$,
                 $10^{2}$,
                 $10^{3}$,
                 $10^{4}$},
    xmode=log,ymode=normal,
    xmin=1e-4, xmax=1e1,
    ymin=-180, ymax=0,
    grid=both,
    major grid style={black!40}
]
    \addplot[signalb,domain=1e-5:1e4] {-atan2(10*x,1)-atan2(100*x,1)};
    \addplot[signalr,line width=2pt,dashed,domain=1e-5:1e-2] {0};
    \addplot[signalr,line width=2pt,dashed,domain=1e-2:1e-1] {-90};
    \addplot[signalr,line width=2pt,dashed,domain=1e-1:1e5 ] {-180};
    \draw[signalr,line width=2pt,dashed](axis cs:0.01,0)  -- (axis cs:0.01,-90);
    \draw[signalr,line width=2pt,dashed](axis cs:0.1,-90) -- (axis cs:0.1,-180);
    \draw[draw=none,fill=col1] (axis cs:0.001,-6.28) circle[radius=2pt];
    \draw[draw=none,fill=col1] (axis cs:0.01,-50.71) circle[radius=2pt];
    \draw[draw=none,fill=col1] (axis cs:0.1,-129.29) circle[radius=2pt];
    \draw[draw=none,fill=col1] (axis cs:1,-173.72) circle[radius=2pt];
\end{axis}
\end{tikzpicture}

\end{center}
%-------------------------------------------------------------------------------
%%%%%%%%%%%%%%%%%%%%%%%%%%%%%%%%%%%%%%%%%%%%%%%%%%%%%%%%%%%%%%%%%%%%%%%%%%%%%%%%
\paragraph{Fonctions réelles du gain et du déphasage}
%%%%%%%%%%%%%%%%%%%%%%%%%%%%%%%%%%%%%%%%%%%%%%%%%%%%%%%%%%%%%%%%%%%%%%%%%%%%%%%%
\[
G(\omega)=|H(\jw)|=
\dfrac{100}{\sqrt{1+\tau_2^2\omega^2}\sqrt{1+\tau_1^2\omega^2}}
\]
\[
G_{dB}(\omega)=40-10\log{(1+\tau_2^2\omega^2)}-10\log{(1+\tau_1^2\omega^2)}
\]
\[
\phi(\omega)=\arg{H(\jw)}=-\arctan{\tau_1\omega}-\arctan{\tau_2\omega}
\]
%%%%%%%%%%%%%%%%%%%%%%%%%%%%%%%%%%%%%%%%%%%%%%%%%%%%%%%%%%%%%%%%%%%%%%%%%%%%%%%%
\paragraph{Quelques valeurs particulières (calculées) :}
%%%%%%%%%%%%%%%%%%%%%%%%%%%%%%%%%%%%%%%%%%%%%%%%%%%%%%%%%%%%%%%%%%%%%%%%%%%%%%%%
%-------------------------------------------------------------------------------
\begin{center}
\begin{tabular}{P{3.5cm}P{2cm}P{2cm}P{2cm}P{2cm}}
\hline
\hline
Pulsation (\si{\radian\per\second}) 
& $10^{-3}$ & $\omega_1=10^{-2}$ & $\omega_2=10^{-1}$ &     1  \\[1em]
Gain (\si{\decibel})       
&   40      &    37              &    17              &   -20  \\[1em]
Déphasage (\si{\degree})   
&  -6       &   -51              &  -129              &  -174  \\[1em]
\hline
\hline
\end{tabular}
\end{center}
%-------------------------------------------------------------------------------
\newpage
%%% QUESTION 3 %%%%%%%%%%%%%%%%%%%%%%%%%%%%%%%%%%%%%%%%%%%%%%%%%%%%%%%%%%%%%%%%%
%%%%%%%%%%%%%%%%%%%%%%%%%%%%%%%%%%%%%%%%%%%%%%%%%%%%%%%%%%%%%%%%%%%%%%%%%%%%%%%%
\question{}
%%%%%%%%%%%%%%%%%%%%%%%%%%%%%%%%%%%%%%%%%%%%%%%%%%%%%%%%%%%%%%%%%%%%%%%%%%%%%%%%
\[
\boldsymbol{H(p) = \dfrac{100(p+1)^2}{(100p+1)(10p+1)(0.01p+1)}}
\]
%%%%%%%%%%%%%%%%%%%%%%%%%%%%%%%%%%%%%%%%%%%%%%%%%%%%%%%%%%%%%%%%%%%%%%%%%%%%%%%%
\paragraph{Recherche des asymptotes}
%%%%%%%%%%%%%%%%%%%%%%%%%%%%%%%%%%%%%%%%%%%%%%%%%%%%%%%%%%%%%%%%%%%%%%%%%%%%%%%%
%-------------------------------------------------------------------------------
\begin{itemize}
\item $\textrm{gain statique : } K=100      \rightarrow 
      \textrm{toute fréquence}           \rightarrow 
      \textrm{40 dB à basse fréquence}   \rightarrow 
      \textrm{pas d'effet sur la phase}$
\item $\textrm{1er ordre : }\tau_1=100     \rightarrow 
      \omega_1=10^{-2}                   \rightarrow 
      \textrm{pente du gain : } 
      \SI[per-mode=symbol]{-20}{\dB\per\dec} \rightarrow 
      \phi\mathrel{+}=-90\si{\degree}$ 
\item $ \textrm{1er ordre : }\tau_2=10     \rightarrow 
      \omega_2=10^{-1}                   \rightarrow 
      \textrm{pente du gain : } 
      \SI[per-mode=symbol]{-20}{\dB\per\dec} \rightarrow 
      \phi\mathrel{+}=-90\si{\degree}$ 
\item $ \textrm{2nd ordre : }\tau_3=1   \rightarrow 
      \omega_3=1                         \rightarrow 
      \textrm{pente du gain : } 
      $+$\SI[per-mode=symbol]{40}{\dB\per\dec}  \rightarrow 
      \phi\mathrel{+}=+180\si{\degree}$ 
\item $ \textrm{1er ordre : }\tau_4=0.01     \rightarrow 
      \omega_4=10^{2}                   \rightarrow 
      \textrm{pente du gain : } 
      \SI[per-mode=symbol]{-20}{\dB\per\dec} \rightarrow 
      \phi\mathrel{+}=-90\si{\degree}$ 
\end{itemize}
%-------------------------------------------------------------------------------
%-------------------------------------------------------------------------------
\begin{center}
    \tikzsetnextfilename{gainQ3-exercices_corriges-chap_repfreq-ext}
    \tikzstyle{dr}=[signalr,dashed,line width=2pt]
\begin{tikzpicture}[trim axis left]
\begin{axis}[
    ticklabel style = {font=\footnotesize},
    width=0.9\textwidth,
    height=0.24\textheight,
    ylabel={Gain (\si{\decibel})},
    xtick={1e-4,1e-3,1e-2,1e-1,1,1e1,1e2,1e3,1e4,1e5},
    ytick={-80,-60,-40,-20,0,20,40,60},
    xticklabels={$10^{-4}$,$10^{-3}$,$10^{-2}$,$10^{-1}$,$10^{0}$,$10^{1}$,$10^{2}$,$10^{3}$,$10^{4}$,$10^{5}$},
    yticklabels={-80,-60,-40,-20,0,20,40,60},
    xmode=log,ymode=normal,
    xmin=1e-4, xmax=1e5,
    ymin=-80, ymax=60,
    grid=both,
    major grid style={black!40}
]
    \addplot[signalb,domain=1e-4:1e5] {40+20*log10(1+x*x)
                                         -10*log10(1+10000*x*x)
                                         -10*log10(1+100*x*x)
                                         -10*log10(1+0.0001*x*x)};
    \addplot[dr,domain=1e-4:1e-2] {40};
    \addplot[dr,domain=1e-2:1e-1] {-20*log10(x)};
    \addplot[dr,domain=1e-1:1e0]  {-40*log10(x)-20*log10(10)};
    \addplot[dr,domain=1e0:1e2] {-20};
    \addplot[dr,domain=1e2:1e5] {-20*log10(x)+20*log10(10)};
    \draw[draw=none,fill=col1] (axis cs:0.001,39.9564) circle[radius=2pt];
    \draw[draw=none,fill=col1] (axis cs:0.01,36.9474) circle[radius=2pt];
    \draw[draw=none,fill=col1] (axis cs:0.1,17.0329) circle[radius=2pt];
    \draw[draw=none,fill=col1] (axis cs:1,-14.0235) circle[radius=2pt];
    \draw[draw=none,fill=col1] (axis cs:10,-19.9572) circle[radius=2pt];
    \draw[draw=none,fill=col1] (axis cs:100,-23.0094) circle[radius=2pt];
    \draw[draw=none,fill=col1] (axis cs:1000,-40.0432) circle[radius=2pt];
\end{axis}
\end{tikzpicture}


    \tikzsetnextfilename{phaseQ3-exercices_corriges-chap_repfreq-ext}
    \tikzstyle{dr}=[signalr,dashed,line width=2pt]
\begin{tikzpicture}[trim axis left]
\begin{axis}[
    ticklabel style = {font=\footnotesize},
    width=0.9\textwidth,
    height=0.24\textheight,
    xlabel={Pulsation (\si{\radian\per\second})},
    ylabel={Phase (\si{degree})},
    xtick={1e-4,1e-3,1e-2,1e-1,1,1e1,1e2,1e3,1e4,1e5},
    ytick={-180,-150,-120,-90,-60,-30,0},
    yticklabels={-180,-150,-120,-90,-60,-30,0},
    xticklabels={$10^{-4}$,
                 $10^{-3}$,
                 $10^{-2}$,
                 $10^{-1}$,
                 $10^{0}$,
                 $10^{1}$,
                 $10^{2}$,
                 $10^{3}$,
                 $10^{4}$,
                 $10^{5}$},
    xmode=log,ymode=normal,
    xmin=1e-4, xmax=1e5,
    ymin=-180, ymax=0,
    grid=both,
    major grid style={black!40}
]
    \addplot[signalb,domain=1e-4:1e5] {2*atan2(x,1)
                                        -atan2(100*x,1)
                                        -atan2(10*x,1)
                                        -atan2(0.01*x,1)};
    \addplot[dr,domain=1e-4:1e-2] {0};
    \addplot[dr,domain=1e-2:1e-1] {-90};
    \addplot[dr,domain=1e-1:1e0 ] {-180};
    \addplot[dr,domain=1e0:1e2] {0};
    \addplot[dr,domain=1e2:1e5] {-90};
    \draw[dr] (axis cs:0.01,0)  -- (axis cs:0.01,-90);
    \draw[dr] (axis cs:0.1,-90) -- (axis cs:0.1,-180);
    \draw[dr] (axis cs:1,-180)  -- (axis cs:1,0);
    \draw[dr] (axis cs:100,0)   -- (axis cs:100,-90);
    \draw[draw=none,fill=col1] (axis cs:0.001,-6.1695) circle[radius=2pt];
    \draw[draw=none,fill=col1] (axis cs:0.01,-49.5704) circle[radius=2pt];
    \draw[draw=none,fill=col1] (axis cs:0.1,-117.9255) circle[radius=2pt];
    \draw[draw=none,fill=col1] (axis cs:1,-84.2894) circle[radius=2pt];
    \draw[draw=none,fill=col1] (axis cs:10,-16.5015) circle[radius=2pt];
    \draw[draw=none,fill=col1] (axis cs:100,-46.0829) circle[radius=2pt];
    \draw[draw=none,fill=col1] (axis cs:1000,-84.3977) circle[radius=2pt];
\end{axis}
\end{tikzpicture}

\end{center}
%-------------------------------------------------------------------------------
%%%%%%%%%%%%%%%%%%%%%%%%%%%%%%%%%%%%%%%%%%%%%%%%%%%%%%%%%%%%%%%%%%%%%%%%%%%%%%%%
\paragraph{Fonctions réelles du gain et du déphasage}$\,$\newline
%%%%%%%%%%%%%%%%%%%%%%%%%%%%%%%%%%%%%%%%%%%%%%%%%%%%%%%%%%%%%%%%%%%%%%%%%%%%%%%%
\[
G(\omega)=|H(\jw)|=
\dfrac{100(1+\tau_3^2\omega^2)}
{\sqrt{1+\tau_1^2\omega^2}\sqrt{1+\tau_2^2\omega^2}\sqrt{1+\tau_4^2\omega^2}}
\]
\[
G_{dB}(\omega)=40+
               20\log{(1+\tau_3^2\omega^2)}
              -10\log{(1+\tau_1^2\omega^2)}
              -10\log{(1+\tau_2^2\omega^2)}
              -10\log{(1+\tau_4^2\omega^2)}
\]
\[
\phi(\omega)=\arg{H(\jw)}=
             2\arctan{\tau_3\omega}
             -\arctan{\tau_1\omega}
             -\arctan{\tau_2\omega}
             -\arctan{\tau_4\omega}
\]
%%%%%%%%%%%%%%%%%%%%%%%%%%%%%%%%%%%%%%%%%%%%%%%%%%%%%%%%%%%%%%%%%%%%%%%%%%%%%%%%
\paragraph{Quelques valeurs particulières (calculées) :}
%%%%%%%%%%%%%%%%%%%%%%%%%%%%%%%%%%%%%%%%%%%%%%%%%%%%%%%%%%%%%%%%%%%%%%%%%%%%%%%%
%-------------------------------------------------------------------------------
\begin{center}
\resizebox{\textwidth}{!}{%
\begin{tabular}{P{3cm}P{2cm}P{2cm}P{2cm}P{2cm}P{2cm}P{2cm}P{2cm}}
\hline
\hline
Pulsation (\si{\radian\per\second}) & 
$10^{-3}$ & $\omega_1=10^{-2}$ & $\omega_2=10^{-1}$ & 
$\omega_3=1$  & 10 & $\omega_4=10^{2}$ & $10^{3}$ \\[1em]
Gain (\si{\decibel})         &  40 &  37 &   17 & -14 & -20 & -23 & -40 \\[1em]
Déphasage (\si{\degree}) &  -6 & -50 & -118 & -84 & -16 & -46 & -84 \\[1em]
\hline
\hline
\end{tabular}%
}
\end{center}
%-------------------------------------------------------------------------------
\newpage
% QUESTION 4
%%%%%%%%%%%%%%%%%%%%%%%%%%%%%%%%%%%%%%%%%%%%%%%%%%%%%%%%%%%%%%%%%%%%%%%%%%%%%%%%
\question{}
%%%%%%%%%%%%%%%%%%%%%%%%%%%%%%%%%%%%%%%%%%%%%%%%%%%%%%%%%%%%%%%%%%%%%%%%%%%%%%%%
\[
\boldsymbol{H(p) = \dfrac{(10p+1)(0.1p+1)}{p(0.001p+1)^2}}
\]
%%%%%%%%%%%%%%%%%%%%%%%%%%%%%%%%%%%%%%%%%%%%%%%%%%%%%%%%%%%%%%%%%%%%%%%%%%%%%%%%
\paragraph{Recherche des asymptotes}
%%%%%%%%%%%%%%%%%%%%%%%%%%%%%%%%%%%%%%%%%%%%%%%%%%%%%%%%%%%%%%%%%%%%%%%%%%%%%%%%
%-------------------------------------------------------------------------------
\begin{itemize}
\item $\textrm{intégrateur } \frac{K}{p}     \rightarrow 
      \textrm{toute fréquence}               \rightarrow 
      \textrm{pente -20dB/dec}               \rightarrow 
      \phi\mathrel{+}=-90\si{\degree}$
\item $\textrm{1er ordre : }\tau_1=10          \rightarrow 
      \omega_1=10^{-1}                       \rightarrow 
      \textrm{pente du gain : } 
      $+$\SI[per-mode=symbol]{20}{\dB\per\dec}    \rightarrow 
      \phi\mathrel{+}=+90\si{\degree}$ 
\item $ \textrm{1er ordre : }\tau_2=0.1        \rightarrow 
      \omega_2=10                            \rightarrow 
      \textrm{pente du gain : } 
      $+$\SI[per-mode=symbol]{20}{\dB\per\dec}     \rightarrow 
      \phi\mathrel{+}=+90\si{\degree}$ 
\item $ \textrm{2nd ordre : }\tau_3=10^{-3} \rightarrow 
      \omega_3=10^3                          \rightarrow 
      \textrm{pente du gain : } 
      \SI[per-mode=symbol]{-40}{\dB\per\dec}     \rightarrow 
      \phi\mathrel{+}=-180\si{\degree}$ 
\end{itemize}
%-------------------------------------------------------------------------------
%-------------------------------------------------------------------------------
\begin{center}
    \tikzsetnextfilename{gainQ4-exercices_corriges-chap_repfreq-ext}
    \tikzstyle{dr}=[signalr,dashed,line width=2pt]
\begin{tikzpicture}[trim axis left]
\begin{axis}[
    ticklabel style = {font=\footnotesize},
    width=0.9\textwidth,
    height=0.24\textheight,
    ylabel={Gain (\si{\decibel})},
    xtick={1e-4,1e-3,1e-2,1e-1,1,1e1,1e2,1e3,1e4,1e5},
    ytick={20,30,40,50,60,70,80,90},
    xticklabels={$10^{-4}$,
                 $10^{-3}$,
                 $10^{-2}$,
                 $10^{-1}$,
                 $10^{0}$,
                 $10^{1}$,
                 $10^{2}$,
                 $10^{3}$,
                 $10^{4}$,
                 $10^{5}$},
    yticklabels={20,30,40,50,60,70,80,90},
    xmode=log,ymode=normal,
    xmin=1e-4, xmax=1e5,
    ymin=20, ymax=90,
    grid=both,
    major grid style={black!40}
]
    \addplot[signalb,domain=1e-4:1e5] 
    {10*log10(1+100*x*x)+10*log10(1+0.01*x*x)-20*log10(x)-20*log10(1+1e-6*x*x)};
    \addplot[dr,domain=1e-4:1e-1] {-20*log10(x)};
    \addplot[dr,domain=1e-1:1e1 ] {20};
    \addplot[dr,domain=1e1:1e3  ] {20*log10(x)};
    \addplot[dr,domain=1e3:1e5  ] {120-20*log10(x)};
    \draw[draw=none,fill=col1] (axis cs:0.01,40.0432) circle[radius=2pt];
    \draw[draw=none,fill=col1] (axis cs:0.1,23.0107) circle[radius=2pt];
    \draw[draw=none,fill=col1] (axis cs:1,20.0864) circle[radius=2pt];
    \draw[draw=none,fill=col1] (axis cs:10,23.0099) circle[radius=2pt];
    \draw[draw=none,fill=col1] (axis cs:100,39.9568) circle[radius=2pt];
    \draw[draw=none,fill=col1] (axis cs:1000,53.9798) circle[radius=2pt];
    \draw[draw=none,fill=col1] (axis cs:10000,39.9136) circle[radius=2pt];
\end{axis}
\end{tikzpicture}


    \tikzsetnextfilename{phaseQ4-exercices_corriges-chap_repfreq-ext}
    \tikzstyle{dr}=[signalr,dashed,line width=2pt]
\begin{tikzpicture}[trim axis left]
\begin{axis}[
    ticklabel style = {font=\footnotesize},
    width=0.9\textwidth,
    height=0.24\textheight,
    xlabel={Pulsation (\si{\radian\per\second})},
    ylabel={Phase (\si{degree})},
    xtick={1e-4,1e-3,1e-2,1e-1,1,1e1,1e2,1e3,1e4,1e5},
    ytick={-90,-60,-30,0,30,60,90}, 
    yticklabels={-90,-60,-30,0,30,60,90}, 
    xticklabels={$10^{-4}$,$10^{-3}$,$10^{-2}$,$10^{-1}$,$10^{0}$,$10^{1}$,$10^{2}$,$10^{3}$,$10^{4}$,$10^{5}$},
    xmode=log,ymode=normal,
    xmin=1e-4, xmax=1e5,
    ymin=-90, ymax=90,
    grid=both,
    major grid style={black!40}
]
    \addplot[signalb,domain=1e-4:1e5] {-90+atan2(10*x,1)
                                          +atan2(0.1*x,1)
                                        -2*atan2(0.001*x,1)};
    \addplot[dr,domain=1e-4:1e-1] {-90};
    \addplot[dr,domain=1e-1:1e1] {0};
    \addplot[dr,domain=1e1:1e3] {90};
    \addplot[dr,domain=1e3:1e5] {-90};
    \draw[dr] (axis cs:0.1,-90)  -- (axis cs:0.1,0);
    \draw[dr] (axis cs:10,0) -- (axis cs:10,90);
    \draw[dr] (axis cs:1000,90)  -- (axis cs:1000,-90);
    \draw[draw=none,fill=col1] (axis cs:0.01,-84.2333) circle[radius=2pt];
    \draw[draw=none,fill=col1] (axis cs:0.1,-44.4385) circle[radius=2pt];
    \draw[draw=none,fill=col1] (axis cs:1,-0.1146) circle[radius=2pt];
    \draw[draw=none,fill=col1] (axis cs:10,43.2812) circle[radius=2pt];
    \draw[draw=none,fill=col1] (axis cs:100,72.8109) circle[radius=2pt];
    \draw[draw=none,fill=col1] (axis cs:1000,-0.5787) circle[radius=2pt];
    \draw[draw=none,fill=col1] (axis cs:10000,-78.6367) circle[radius=2pt];
\end{axis}
\end{tikzpicture}

\end{center}
%-------------------------------------------------------------------------------
%%%%%%%%%%%%%%%%%%%%%%%%%%%%%%%%%%%%%%%%%%%%%%%%%%%%%%%%%%%%%%%%%%%%%%%%%%%%%%%%
\paragraph{Fonctions réelles du gain et du déphasage}$\,$\newline
%%%%%%%%%%%%%%%%%%%%%%%%%%%%%%%%%%%%%%%%%%%%%%%%%%%%%%%%%%%%%%%%%%%%%%%%%%%%%%%%
\[
G(\omega)=|H(\jw)|=
\dfrac{\sqrt{1+\tau_1^2\omega^2}\sqrt{1+\tau_2^2\omega^2}}
      {\omega(1+\tau_3^2\omega^2)}
\]
\[
G_{dB}(\omega)=10\log{(1+\tau_1^2\omega^2)}
              +10\log{(1+\tau_2^2\omega^2)}
              -20\log{\omega}
              -20\log{(1+\tau_3^2\omega^2)}
\]
\[
\phi(\omega)=\arg{H(\jw)}=-\frac{\pi}{2}+\arctan{\tau_1\omega}
                                        +\arctan{\tau_2\omega}
                                       -2\arctan{\tau_3\omega}
\]
%%%%%%%%%%%%%%%%%%%%%%%%%%%%%%%%%%%%%%%%%%%%%%%%%%%%%%%%%%%%%%%%%%%%%%%%%%%%%%%%
\paragraph{Quelques valeurs particulières (calculées) :}
%%%%%%%%%%%%%%%%%%%%%%%%%%%%%%%%%%%%%%%%%%%%%%%%%%%%%%%%%%%%%%%%%%%%%%%%%%%%%%%%
%-------------------------------------------------------------------------------
\begin{center}
\resizebox{\textwidth}{!}{%
\begin{tabular}{P{3.5cm}P{2cm}P{2cm}P{2cm}P{2cm}P{2cm}P{2cm}P{2cm}}
\hline
\hline
Pulsation (\si{\radian\per\second}) & $10^{-2}$ 
                                    & $\omega_1=10^{-1}$ 
                                    & 1 & $\omega_1=10$ 
                                    & $10^{2}$ 
                                    & $\omega_3=10^{3}$ 
                                    & $10^{4}$  \\[1em]
Gain (\si{\decibel})                &    40 
                                    &    23 
                                    &    20
                                    &    23 
                                    &    40 
                                    &    54 
                                    &    40 \\[1em]
Déphasage (\si{\degree})            &   -84 
                                    &   -44 
                                    &    0 
                                    &    43 
                                    &    73 
                                    &    1  
                                    &   -79 \\[1em]
\hline
\hline
\end{tabular}%
}
\end{center}
%-------------------------------------------------------------------------------
\newpage
%% QUESTION 5 %%%%%%%%%%%%%%%%%%%%%%%%%%%%%%%%%%%%%%%%%%%%%%%%%%%%%%%%%%%%%%%%%%
%%%%%%%%%%%%%%%%%%%%%%%%%%%%%%%%%%%%%%%%%%%%%%%%%%%%%%%%%%%%%%%%%%%%%%%%%%%%%%%%
\question{}
%%%%%%%%%%%%%%%%%%%%%%%%%%%%%%%%%%%%%%%%%%%%%%%%%%%%%%%%%%%%%%%%%%%%%%%%%%%%%%%%
\[
\boldsymbol{H(p) = \dfrac{10(10p+1)}{(p+1)(100p+1)}}
\]
%%%%%%%%%%%%%%%%%%%%%%%%%%%%%%%%%%%%%%%%%%%%%%%%%%%%%%%%%%%%%%%%%%%%%%%%%%%%%%%%
\paragraph{Recherche des asymptotes}
%%%%%%%%%%%%%%%%%%%%%%%%%%%%%%%%%%%%%%%%%%%%%%%%%%%%%%%%%%%%%%%%%%%%%%%%%%%%%%%%
%-------------------------------------------------------------------------------
\begin{itemize}
\item $K=10                                    \rightarrow 
      \textrm{toute fréquence}                 \rightarrow 
      \textrm{20 dB à basse fréquence}         \rightarrow 
      \textrm{pas d'effet sur la phase}$
\item $\textrm{1er ordre : }\tau_1=100         \rightarrow 
      \omega_1=10^{-2}                         \rightarrow 
      \textrm{pente du gain : } 
      \SI[per-mode=symbol]{-20}{\dB\per\dec}   \rightarrow 
      \phi\mathrel{+}=-90\si{\degree}$ 
\item $\textrm{1er ordre : }\tau_2=10          \rightarrow 
      \omega_2=10^{-1}                         \rightarrow 
      \textrm{pente du gain : } 
      $+$\SI[per-mode=symbol]{20}{\dB\per\dec} \rightarrow 
      \phi\mathrel{+}=+90\si{\degree}$ 
\item $\textrm{1er ordre : }\tau_3=1           \rightarrow 
      \omega_3=1                               \rightarrow 
      \textrm{pente du gain : } 
      \SI[per-mode=symbol]{-20}{\dB\per\dec}   \rightarrow 
      \phi\mathrel{+}=-90\si{\degree}$ 
\end{itemize}
%-------------------------------------------------------------------------------
%-------------------------------------------------------------------------------
\begin{center}
    \tikzsetnextfilename{gainQ5-exercices_corriges-chap_repfreq-ext}
    \tikzstyle{dr}=[signalr,dashed,line width=2pt]
\begin{tikzpicture}[trim axis left]
\begin{axis}[
    ticklabel style = {font=\footnotesize},
    width=0.9\textwidth,
    height=0.24\textheight,
    ylabel={Gain (\si{\decibel})},
    xtick={1e-4,1e-3,1e-2,1e-1,1,1e1,1e2},
    ytick={-60,-40,-20,0,20,40},
    xticklabels={$10^{-4}$,
                 $10^{-3}$,
                 $10^{-2}$,
                 $10^{-1}$,
                 $10^{0}$,
                 $10^{1}$,
                 $10^{2}$},
    yticklabels={-60,-40,-20,0,20,40},
    xmode=log,ymode=normal,
    xmin=1e-4, xmax=1e2,
    ymin=-60, ymax=40,
    grid=both,
    major grid style={black!40}
]
    \addplot[signalb,domain=1e-4:1e5] 
    {20+10*log10(1+100*x*x)-10*log10(1+x*x)-10*log10(1+1e4*x*x)};
    \addplot[dr,domain=1e-4:1e-2] {20};
    \addplot[dr,domain=1e-2:1e-1] {-20*log10(x)-20};
    \addplot[dr,domain=1e-1:1e0 ] {0};
    \addplot[dr,domain=1e0:1e2  ] {-20*log10(x)};
    \draw[draw=none,fill=col1] (axis cs:0.001,19.9572) circle[radius=2pt];
    \draw[draw=none,fill=col1] (axis cs:0.01,17.0325) circle[radius=2pt];
    \draw[draw=none,fill=col1] (axis cs:0.1,2.9239) circle[radius=2pt];
    \draw[draw=none,fill=col1] (axis cs:1,-2.9675) circle[radius=2pt];
    \draw[draw=none,fill=col1] (axis cs:10,-20.0428) circle[radius=2pt];
\end{axis}
\end{tikzpicture}

    
    \tikzsetnextfilename{phaseQ5-exercices_corriges-chap_repfreq-ext}
    \tikzstyle{dr}=[signalr,dashed,line width=2pt]
\begin{tikzpicture}[trim axis left]
\begin{axis}[
    ticklabel style = {font=\footnotesize},
    width=0.9\textwidth,
    height=0.24\textheight,
    xlabel={Pulsation (\si{\radian\per\second})},
    ylabel={Phase (\si{degree})},
    xtick={1e-4,1e-3,1e-2,1e-1,1,1e1,1e2},
    ytick={-90,-75,-60,-45,-30,-15,0},
    yticklabels={-90,-75,-60,-45,-30,-15,0},
    xticklabels={$10^{-4}$,$10^{-3}$,$10^{-2}$,$10^{-1}$,
                 $10^{0}$,$10^{1}$,$10^{2}$},
    xmode=log,ymode=normal,
    xmin=1e-4, xmax=1e2,
    ymin=-90, ymax=0,
    grid=both,
    major grid style={black!40}
    ]
    \addplot[signalb,domain=1e-4:1e5] {atan2(10*x,1)-atan2(x,1)-atan2(100*x,1)};
    \addplot[dr,domain=1e-4:1e-2] {0};
    \addplot[dr,domain=1e-2:1e-1] {-90};
    \addplot[dr,domain=1e-1:1e0 ] {0};
    \addplot[dr,domain=1e0:1e4  ] {-90};
    \draw[dr] (axis cs:0.01,0)  -- (axis cs:0.01,-90);
    \draw[dr] (axis cs:0.1,-90) -- (axis cs:0.1,0);
    \draw[dr] (axis cs:1,0)     -- (axis cs:1,-90);
    \draw[draw=none,fill=col1] (axis cs:0.001,-5.1950) circle[radius=2pt];
    \draw[draw=none,fill=col1] (axis cs:0.01,-39.8623) circle[radius=2pt];
    \draw[draw=none,fill=col1] (axis cs:0.1,-45.0000) circle[radius=2pt];
    \draw[draw=none,fill=col1] (axis cs:1,-50.1377) circle[radius=2pt];
    \draw[draw=none,fill=col1] (axis cs:10,-84.8050) circle[radius=2pt];
\end{axis}
\end{tikzpicture}

\end{center}
%-------------------------------------------------------------------------------
%%%%%%%%%%%%%%%%%%%%%%%%%%%%%%%%%%%%%%%%%%%%%%%%%%%%%%%%%%%%%%%%%%%%%%%%%%%%%%%%
\paragraph{Fonctions réelles du gain et du déphasage}
%%%%%%%%%%%%%%%%%%%%%%%%%%%%%%%%%%%%%%%%%%%%%%%%%%%%%%%%%%%%%%%%%%%%%%%%%%%%%%%%
\[
G(\omega)
=|H(\jw)|=\dfrac{10\sqrt{1+\tau_2^2\omega^2}}
                {\sqrt{1+\tau_3^2\omega^2}\sqrt{1+\tau_1^2\omega^2}}
\]
\[
G_{dB}(\omega)=20+10\log{(1+\tau_2^2\omega^2)}
                 -10\log{(1+\tau_3^2\omega^2)}
                 -10\log{(1+\tau_1^2\omega^2)}
\]
\[
\phi(\omega)=\arg{H(\jw)}=\arctan{\tau_2\omega}
                         -\arctan{\tau_3\omega}
                         -\arctan{\tau_1\omega}
\]
%%%%%%%%%%%%%%%%%%%%%%%%%%%%%%%%%%%%%%%%%%%%%%%%%%%%%%%%%%%%%%%%%%%%%%%%%%%%%%%%
\paragraph{Quelques valeurs particulières (calculées) :}
%%%%%%%%%%%%%%%%%%%%%%%%%%%%%%%%%%%%%%%%%%%%%%%%%%%%%%%%%%%%%%%%%%%%%%%%%%%%%%%%
%-------------------------------------------------------------------------------
\begin{center}
\resizebox{\textwidth}{!}{%
\begin{tabular}{P{3cm}P{2cm}P{2cm}P{2cm}P{2cm}P{2cm}}
\hline
\hline
Pulsation (\si{\radian\per\second}) 
& $10^{-3}$ & 
$\omega_1=10^{-2}$ & 
$\omega_2=10^{-1}$ & 
$\omega_3=1$ & 10 \\[1em]
Gain (\si{\decibel})       & 20 &  17 &   3 &  -3 & -20 \\[1em]
Déphasage (\si{\degree})   & -5 & -40 & -45 & -50 & -85 \\[1em]
\hline
\hline
\end{tabular}%
}
\end{center}
%-------------------------------------------------------------------------------
\clearpage
%%%%%%%%%%%%%%%%%%%%%%%%%%%%%%%%%%%%%%%%%%%%%%%%%%%%%%%%%%%%%%%%%%%%%%%%%%%%%%%%
%%%%%%%%%%%%%%%%%%%%%%%%%%%%%%%%%%%%%%%%%%%%%%%%%%%%%%%%%%%%%%%%%%%%%%%%%%%%%%%%
\exercice{Diagrammes de Nyquist~\difficile}
%%%%%%%%%%%%%%%%%%%%%%%%%%%%%%%%%%%%%%%%%%%%%%%%%%%%%%%%%%%%%%%%%%%%%%%%%%%%%%%%
%%%%%%%%%%%%%%%%%%%%%%%%%%%%%%%%%%%%%%%%%%%%%%%%%%%%%%%%%%%%%%%%%%%%%%%%%%%%%%%%
%%%%%%%%%%%%%%%%%%%%%%%%%%%%%%%%%%%%%%%%%%%%%%%%%%%%%%%%%%%%%%%%%%%%%%%%%%%%%%%%
\question{\textbf{Tracer le diagramme de Nyquist de la 
fonction de transfert suivante:}}
%%%%%%%%%%%%%%%%%%%%%%%%%%%%%%%%%%%%%%%%%%%%%%%%%%%%%%%%%%%%%%%%%%%%%%%%%%%%%%%%
La fonction de transfert complexe de $H(p)$ s'écrit :
\[
    H(\jw)=\dfrac{1}{\jw(\jw+1)}
\]
s'écrit, en identifiant explicitement la partie réelle et imaginaire, sous la 
forme :
\[
    H(\jw)=-\dfrac{1}{1+\omega^2}-j\dfrac{\frac{1}{\omega}}{1+\omega^2}
\]
Le diagramme de Nyquist consiste simplement a tracé la partie réelle et 
imaginaire pour différentes valeurs de la pulsation de la sollicitation.
%-------------------------------------------------------------------------------
\begin{figure}[!h]
    \centering
    \tikzsetnextfilename{ex2_1-exercices_corriges-chap_repfreq-ext}
    \input{tikz/ex2_1-exercices_corriges-chap_repfreq.tex}
\end{figure}
%-------------------------------------------------------------------------------
On remarque pour $\omega\rightarrow0^\pm$ la partie réelle tend vers -1 et la 
partie imaginaire vers $\mp\infty$ (selon le signe d'approche de 0).
Pour compléter la description, déterminons le gain $G(\omega)$ et le déphasage 
$\phi(\omega)$ de ce système:
\[
    G(\omega)=\dfrac{1}{\omega\sqrt{1+\omega^2}}
\]
\[
    \phi(\omega)=-\arctan{\left(-\dfrac{1}{\omega}\right)}-\pi
\]
Le déphasage permet de déterminer l'allure du diagramme de Nyquist 
lorsque $\omega$ approche zéro ou l'infini. 
