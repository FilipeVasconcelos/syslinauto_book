%%%%%%%%%%%%%%%%%%%%%%%%%%%%%%%%%%%%%%%%%%%%%%%%%%%%%%%%%%%%%%%%%%%%%%%%%%%%%%%%
%%%%%%%%%%%%%%%%%%%%%%%%%%%%%%%%%%%%%%%%%%%%%%%%%%%%%%%%%%%%%%%%%%%%%%%%%%%%%%%%
%%%%%%%%%%%%%%%%%%%%%%%%%%%%%%%%%%%%%%%%%%%%%%%%%%%%%%%%%%%%%%%%%%%%%%%%%%%%%%%%
%%%%%%%%%%%%%%%%%%%%%%%%%%%%%%%%%%%%%%%%%%%%%%%%%%%%%%%%%%%%%%%%%%%%%%%%%%%%%%%%
\chapter{Équations différentielles à coefficients constants
        \label{annexe-eqndiff}}
%%%%%%%%%%%%%%%%%%%%%%%%%%%%%%%%%%%%%%%%%%%%%%%%%%%%%%%%%%%%%%%%%%%%%%%%%%%%%%%%
%%%%%%%%%%%%%%%%%%%%%%%%%%%%%%%%%%%%%%%%%%%%%%%%%%%%%%%%%%%%%%%%%%%%%%%%%%%%%%%%
%%%%%%%%%%%%%%%%%%%%%%%%%%%%%%%%%%%%%%%%%%%%%%%%%%%%%%%%%%%%%%%%%%%%%%%%%%%%%%%%
%%%%%%%%%%%%%%%%%%%%%%%%%%%%%%%%%%%%%%%%%%%%%%%%%%%%%%%%%%%%%%%%%%%%%%%%%%%%%%%%
\chaptermark{Équations différentielles}
La forme générale d'une équation différentielle à coefficients constants est 
donnée par :
\begin{align}
\sum_{i=0}^{n}a_i\devi{s(t)}{i}=\sum_{i=0}^{m}b_i\devi{e(t)}{i}
\end{align}

avec $a_i$ et $b_i$ des coefficients réels et $m$ et $n$ des entiers naturels 
tels que $m\le n$. L'ordre de l'équation est donnée par $n$.

Nous allons dans cette annexe, présenter la résolution direct 
des formes canoniques du premier et du second ordre. La méthodologie consiste
à déterminer la solution générale de l'équation homogène (c.a.d sans second 
membre) puis de chercher une solution particulière pour l'équation avec second 
membre. Nous nous limiterons à l'étude des solutions pour des seconds membres 
correspondant aux signaux usuels présentés dans ce document 
au~\Cref{chap-slci} (c.a.d Impulsion de Dirac, échelon-unité et rampe).

%%%%%%%%%%%%%%%%%%%%%%%%%%%%%%%%%%%%%%%%%%%%%%%%%%%%%%%%%%%%%%%%%%%%%%%%%%%%%%%%
%%%%%%%%%%%%%%%%%%%%%%%%%%%%%%%%%%%%%%%%%%%%%%%%%%%%%%%%%%%%%%%%%%%%%%%%%%%%%%%%
%%%%%%%%%%%%%%%%%%%%%%%%%%%%%%%%%%%%%%%%%%%%%%%%%%%%%%%%%%%%%%%%%%%%%%%%%%%%%%%%
\section{Résolution équation différentielle du premier ordre}
%%%%%%%%%%%%%%%%%%%%%%%%%%%%%%%%%%%%%%%%%%%%%%%%%%%%%%%%%%%%%%%%%%%%%%%%%%%%%%%%
%%%%%%%%%%%%%%%%%%%%%%%%%%%%%%%%%%%%%%%%%%%%%%%%%%%%%%%%%%%%%%%%%%%%%%%%%%%%%%%%
%%%%%%%%%%%%%%%%%%%%%%%%%%%%%%%%%%%%%%%%%%%%%%%%%%%%%%%%%%%%%%%%%%%%%%%%%%%%%%%%

La forme générale d'une équation différentielle du premier ordre est
donnée par : 
$$
a_1\devi{s(t)}{}+a_0s(t)=b_0e(t)
$$
Il est toujours possible de simplifier une telle équation sous la forme :
$$
\devi{s(t)}{}+\dfrac{a_0}{a_1}s(t)=\dfrac{b_0}{a_1}e(t)
$$
La forme canonique d'une équation différentielle du premier ordre est alors :
\begin{bequation}[ams align]
\devi{s(t)}{}+as(t)=be(t)
\end{bequation}


%%%%%%%%%%%%%%%%%%%%%%%%%%%%%%%%%%%%%%%%%%%%%%%%%%%%%%%%%%%%%%%%%%%%%%%%%%%%%%%%
%%%%%%%%%%%%%%%%%%%%%%%%%%%%%%%%%%%%%%%%%%%%%%%%%%%%%%%%%%%%%%%%%%%%%%%%%%%%%%%%
\subsection{Sans second membre}
%%%%%%%%%%%%%%%%%%%%%%%%%%%%%%%%%%%%%%%%%%%%%%%%%%%%%%%%%%%%%%%%%%%%%%%%%%%%%%%%
%%%%%%%%%%%%%%%%%%%%%%%%%%%%%%%%%%%%%%%%%%%%%%%%%%%%%%%%%%%%%%%%%%%%%%%%%%%%%%%%

La solution générale de l'équation sans second membre est de la forme :
$$
s(t) = C e^{-\frac{a_0}{a_1}t}\,\,\,\,\text{avec}\,\,\,\,C\in\mathbb{R}
$$

On considère l'équation différentielle suivante régissant l'entrée
et la sortie d'un SLCI\footnote{Système Linéaire Continu et Invariant} :
$$
\devi{s(t)}{}+as(t)=be(t)
$$
avec pour condition initiale $s(0)=s_0$.
On cherche à déterminer la réponse indicielle de ce système.
La réponse libre du système $s_1(t)$ satisfait l'équation différentielle
$$
\devi{s_1(t)}{}=-as_1(t)
$$
Cette réponse ne dépend que des conditions initiales, ici $s(0)=s_0$.
Cette solution est donc de la forme :
$$
s_1(t)=Ce^{-at}
$$
avec $C$ une constante réel. La réponse doit satisfaire la condition 
initiale $s(0)=s_0$. Ce qui impose :
$$
s_1(t)=s_0e^{-at}
$$
La réponse indicielle est donnée pour $e(t)=u(t)$. Pour $t>0$, l'équation 
différentielle avec second membre est alors :
$$
\devi{s(t)}{}+as(t)=b
$$
Déterminons une solution particulière $s_2(t)$ de cette équation 
différentielle de la forme :
$$
s_2(t)=\lambda(t)e^{-at}.
$$
En introduisant, celle-ci dans l'équation différentielle on a alors :
$$
\lambda'(t)e^{-at}-a\lambda(t)e^{-at}+a\lambda(t)e^{-at}=b
$$
On cherche une primitive de la dérivée de $\lambda$ à partir de :
$$
\lambda'(t)=be^{at}
$$
soit alors :
$$
\lambda(t)=\dfrac{b}{a}e^{at}+C
$$
avec $C$ une constante d'intégration.
$$
s_2(t)=\dfrac{b}{a}+Ce^{-at}.
$$
La solution générale est donnée par la somme des deux réponses précédentes :
\begin{align*}
    s(t)&=s_1(t)+s_2(t)\\
    s(t)&=s_0e^{-at}+\dfrac{b}{a}+Ce^{-at}
\end{align*}
La constante $C$ se détermine à partir de la condition initiale:
$$
s(t)=s_0e^{-at}+\dfrac{b}{a}\left(1-e^{-at}\right)
$$
\begin{center}
\tikzsetnextfilename{solutions_equadiff-annexeF-ext}
\tikzstyle{thinfunc}=[thin,col1,samples=201]
\begin{tikzpicture}
    \pgfmathsetmacro{\t}{8}             
    \begin{axis}
    [   height=7.5cm,
        width=10cm,
        grid=both,
        minor tick num=4,
        xmin=0,
        xmax=\t,
        ymin=0,
        ymax=1.0,
        xlabel={$t$},
        ylabel={$s(t)$},
    ]
    \addplot [very thick,col1,domain=-1:0, samples=101]{0};
    \pgfmathsetmacro{\a}{0.5}             
    \pgfmathsetmacro{\b}{0.25}           
                    
    \foreach\s in {-2,-1.8,...,3.0}
    {
    \addplot[thinfunc,domain=0:\t] {\s*exp(-\a*x)+(\b/\a)*(1-exp(-\a*x))};
    }
    \end{axis}
    \node[] at (9,3 ) {$\dfrac{b}{a}$};
    \end{tikzpicture}
\end{center}
Sur les trois termes de la solution générale, un seul est non
nul pour $t\to\infty$. Il correspond au régime permanent de
la réponse. Les deux autres correspondent au régime transitoire.
\begin{align*}
    s(t)=\textcolor{orange}{s_0e^{-at}-\dfrac{b}{a}e^{-at}}&+
    \textcolor{magenta}{\dfrac{b}{a}}\\
    \textcolor{orange}{transitoire} &\quad\textcolor{magenta}{permanent}
\end{align*}
%%%%%%%%%%%%%%%%%%%%%%%%%%%%%%%%%%%%%%%%%%%%%%%%%%%%%%%%%%%%%%%%%%%%%%%%%%%%%%%%
%%%%%%%%%%%%%%%%%%%%%%%%%%%%%%%%%%%%%%%%%%%%%%%%%%%%%%%%%%%%%%%%%%%%%%%%%%%%%%%%
%%%%%%%%%%%%%%%%%%%%%%%%%%%%%%%%%%%%%%%%%%%%%%%%%%%%%%%%%%%%%%%%%%%%%%%%%%%%%%%%
%%%%%%%%%%%%%%%%%%%%%%%%%%%%%%%%%%%%%%%%%%%%%%%%%%%%%%%%%%%%%%%%%%%%%%%%%%%%%%%%
%annexe_equa_diff.tex
