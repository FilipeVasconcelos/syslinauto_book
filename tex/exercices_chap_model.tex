%%%%%%%%%%%%%%%%%%%%%%%%%%%%%%%%%%%%%%%%%%%%%%%%%%%%%%%%%%%%%%%%%%%%%%%%%%%%%%%%
%%%%%%%%%%%%%%%%%%%%%%%%%%%%%%%%%%%%%%%%%%%%%%%%%%%%%%%%%%%%%%%%%%%%%%%%%%%%%%%%
\exercice{Analyse de la réponse temporelle d'un SLCI~\moyen}
%%%%%%%%%%%%%%%%%%%%%%%%%%%%%%%%%%%%%%%%%%%%%%%%%%%%%%%%%%%%%%%%%%%%%%%%%%%%%%%%
%%%%%%%%%%%%%%%%%%%%%%%%%%%%%%%%%%%%%%%%%%%%%%%%%%%%%%%%%%%%%%%%%%%%%%%%%%%%%%%%
Nous avons relevé la réponse à un échelon unitaire, 
déclenchée à $t=0$, d'un système linéaire.
Cette réponse est donnée par la figure ci-dessous.
L'objectif de cet exercice est de caractériser le système 
à partir de cette réponse temporelle.
%-------------------------------------------------------------------------------
\begin{center}
    \tikzsetnextfilename{reponse_2nd_ordre-exercices-chap_model-ext}
    \input{tikz/reponse_2nd_ordre-exercices-chap_model.tex}
\end{center}
%-------------------------------------------------------------------------------
%%%%%%%%%%%%%%%%%%%%%%%%%%%%%%%%%%%%%%%%%%%%%%%%%%%%%%%%%%%%%%%%%%%%%%%%%%%%%%%%
\question{Déterminer :}
%%%%%%%%%%%%%%%%%%%%%%%%%%%%%%%%%%%%%%%%%%%%%%%%%%%%%%%%%%%%%%%%%%%%%%%%%%%%%%%%
%-------------------------------------------------------------------------------
\begin{itemize}
    \item l'ordre du système,
    \item le gain statique $K$,
    \item le temps de réponse à 5\%,
    \item et le dépassement relatif en \%.
\end{itemize}
%-------------------------------------------------------------------------------
%%%%%%%%%%%%%%%%%%%%%%%%%%%%%%%%%%%%%%%%%%%%%%%%%%%%%%%%%%%%%%%%%%%%%%%%%%%%%%%%
\question{En déduire l'amortissement $\xi$, la pulsation propre $\omega_0$ et 
écrire la fonction de transfert.}
%%%%%%%%%%%%%%%%%%%%%%%%%%%%%%%%%%%%%%%%%%%%%%%%%%%%%%%%%%%%%%%%%%%%%%%%%%%%%%%%
Nous rappelons que le temps du premier maximum est 
donné par $t_1=\dfrac{\pi}{\omega_d}=\dfrac{\pi}{\omega_0\sqrt{1-\xi^2}}$
et le dépassement par $D=e^{\dfrac{-\xi\pi}{\sqrt{1-\xi^2}}}$
On pourra s'aider des abaques 1 et 2 présentés en annexe.
%%%%%%%%%%%%%%%%%%%%%%%%%%%%%%%%%%%%%%%%%%%%%%%%%%%%%%%%%%%%%%%%%%%%%%%%%%%%%%%%
\question{Calculer les pôles du système et les placer les sur une cartes des
pôles (plan complexe).}
%%%%%%%%%%%%%%%%%%%%%%%%%%%%%%%%%%%%%%%%%%%%%%%%%%%%%%%%%%%%%%%%%%%%%%%%%%%%%%%%
\clearpage
%%%%%%%%%%%%%%%%%%%%%%%%%%%%%%%%%%%%%%%%%%%%%%%%%%%%%%%%%%%%%%%%%%%%%%%%%%%%%%%%
%%%%%%%%%%%%%%%%%%%%%%%%%%%%%%%%%%%%%%%%%%%%%%%%%%%%%%%%%%%%%%%%%%%%%%%%%%%%%%%%
%%%%%%%%%%%%%%%%%%%%%%%%%%%%%%%%%%%%%%%%%%%%%%%%%%%%%%%%%%%%%%%%%%%%%%%%%%%%%%%%
\exercice{Réponse temporelle d'un système du second ordre~\moyen}
%%%%%%%%%%%%%%%%%%%%%%%%%%%%%%%%%%%%%%%%%%%%%%%%%%%%%%%%%%%%%%%%%%%%%%%%%%%%%%%%
%%%%%%%%%%%%%%%%%%%%%%%%%%%%%%%%%%%%%%%%%%%%%%%%%%%%%%%%%%%%%%%%%%%%%%%%%%%%%%%%
On considère un système régi par l'équation différentielle :
\[
\devi{s(t)}{2}+3\devi{s(t)}{}+2s(t)=e(t)
\]
%%%%%%%%%%%%%%%%%%%%%%%%%%%%%%%%%%%%%%%%%%%%%%%%%%%%%%%%%%%%%%%%%%%%%%%%%%%%%%%%
\question{Déterminer les paramètres du second ordre ($K,\omega_0, \xi$) 
de cette équation différentielle.}
%%%%%%%%%%%%%%%%%%%%%%%%%%%%%%%%%%%%%%%%%%%%%%%%%%%%%%%%%%%%%%%%%%%%%%%%%%%%%%%%
%%%%%%%%%%%%%%%%%%%%%%%%%%%%%%%%%%%%%%%%%%%%%%%%%%%%%%%%%%%%%%%%%%%%%%%%%%%%%%%%
\question{Déterminer la fonction de transfert de ce système.}
%%%%%%%%%%%%%%%%%%%%%%%%%%%%%%%%%%%%%%%%%%%%%%%%%%%%%%%%%%%%%%%%%%%%%%%%%%%%%%%%
%%%%%%%%%%%%%%%%%%%%%%%%%%%%%%%%%%%%%%%%%%%%%%%%%%%%%%%%%%%%%%%%%%%%%%%%%%%%%%%%
\question{Déterminer la réponse indicielle de ce système pour une 
entrée $e(t)=1$ et tracer son allure.}
%%%%%%%%%%%%%%%%%%%%%%%%%%%%%%%%%%%%%%%%%%%%%%%%%%%%%%%%%%%%%%%%%%%%%%%%%%%%%%%%
%%%%%%%%%%%%%%%%%%%%%%%%%%%%%%%%%%%%%%%%%%%%%%%%%%%%%%%%%%%%%%%%%%%%%%%%%%%%%%%%
\exercice{Modélisation de la chute libre~\moyen}
%%%%%%%%%%%%%%%%%%%%%%%%%%%%%%%%%%%%%%%%%%%%%%%%%%%%%%%%%%%%%%%%%%%%%%%%%%%%%%%%
On cherche à déterminer l'expression de la vitesse d'un corps en chute libre 
dans l'air. 
Lorsqu'on lâche, sans vitesse initiale, ce corps $M$ (hypothèse du corps 
ponctuel) de masse $m$
dans cet environnement, il est soumis à deux forces :
%-------------------------------------------------------------------------------
\begin{itemize}
    \item son poids $m\vect{g}$
    \item $-k\vect{v}$ qui s'oppose au mouvement et qui 
        est proportionnelle à la vitesse, $k$ dépend généralement 
        de la forme du corps et de la composition de l'atmosphère
\end{itemize}
%-------------------------------------------------------------------------------
%%%%%%%%%%%%%%%%%%%%%%%%%%%%%%%%%%%%%%%%%%%%%%%%%%%%%%%%%%%%%%%%%%%%%%%%%%%%%%%%
\question{Tracer un schéma représentant le bilan des forces mis en jeu. 
On orientera l'axe de déplacement du haut vers le bas.}
%%%%%%%%%%%%%%%%%%%%%%%%%%%%%%%%%%%%%%%%%%%%%%%%%%%%%%%%%%%%%%%%%%%%%%%%%%%%%%%%
%%%%%%%%%%%%%%%%%%%%%%%%%%%%%%%%%%%%%%%%%%%%%%%%%%%%%%%%%%%%%%%%%%%%%%%%%%%%%%%%
\question{Déterminer quel système SLCI pourrait être envisagé. 
Quel serait l'entrée et la sortie ?}
%%%%%%%%%%%%%%%%%%%%%%%%%%%%%%%%%%%%%%%%%%%%%%%%%%%%%%%%%%%%%%%%%%%%%%%%%%%%%%%%
%%%%%%%%%%%%%%%%%%%%%%%%%%%%%%%%%%%%%%%%%%%%%%%%%%%%%%%%%%%%%%%%%%%%%%%%%%%%%%%%
\question{Déterminer l'équation différentielle régissant le système. 
Quels sont son ordre et sa classe ? Quels sont les paramètres de sa forme
canonique?}
%%%%%%%%%%%%%%%%%%%%%%%%%%%%%%%%%%%%%%%%%%%%%%%%%%%%%%%%%%%%%%%%%%%%%%%%%%%%%%%%
%%%%%%%%%%%%%%%%%%%%%%%%%%%%%%%%%%%%%%%%%%%%%%%%%%%%%%%%%%%%%%%%%%%%%%%%%%%%%%%%
\question{Déterminer la forme de la sortie à partir de la réponse temporelle 
de la forme canonique de ce système. Tracer sa solution.}
%%%%%%%%%%%%%%%%%%%%%%%%%%%%%%%%%%%%%%%%%%%%%%%%%%%%%%%%%%%%%%%%%%%%%%%%%%%%%%%%
\clearpage
%%%%%%%%%%%%%%%%%%%%%%%%%%%%%%%%%%%%%%%%%%%%%%%%%%%%%%%%%%%%%%%%%%%%%%%%%%%%%%%%
%%%%%%%%%%%%%%%%%%%%%%%%%%%%%%%%%%%%%%%%%%%%%%%%%%%%%%%%%%%%%%%%%%%%%%%%%%%%%%%%
\exercice{\'Etude des systèmes du 1er ordre~\facile}
%%%%%%%%%%%%%%%%%%%%%%%%%%%%%%%%%%%%%%%%%%%%%%%%%%%%%%%%%%%%%%%%%%%%%%%%%%%%%%%%
%%%%%%%%%%%%%%%%%%%%%%%%%%%%%%%%%%%%%%%%%%%%%%%%%%%%%%%%%%%%%%%%%%%%%%%%%%%%%%%%
On considère quatre systèmes du 1er ordre ayant les caractéristiques suivantes:
%-------------------------------------------------------------------------------
\begin{itemize}
    \item Système 1 ($S_1$): de gain $K$ et de constante de temps $T$
    \item Système 2 ($S_2$): de gain $K$ et de constante de temps $2T$
    \item Système 3 ($S_3$): de gain $2K$ et de constante de temps $T$
    \item Système 4 ($S_4$): de gain $K$ et de constante de temps $T/2$
\end{itemize}
%-------------------------------------------------------------------------------
%%%%%%%%%%%%%%%%%%%%%%%%%%%%%%%%%%%%%%%%%%%%%%%%%%%%%%%%%%%%%%%%%%%%%%%%%%%%%%%%
\question{Pour chacun des systèmes:}
%%%%%%%%%%%%%%%%%%%%%%%%%%%%%%%%%%%%%%%%%%%%%%%%%%%%%%%%%%%%%%%%%%%%%%%%%%%%%%%%
%-------------------------------------------------------------------------------
\begin{itemize}
    \item[(a)] placer son pôle sur la carte des pôles et zéros.
    \item[(b)] donner l'allure de la réponse indicielle et estimer le temps
               de réponse à 5\%.
\end{itemize}
%-------------------------------------------------------------------------------
%%%%%%%%%%%%%%%%%%%%%%%%%%%%%%%%%%%%%%%%%%%%%%%%%%%%%%%%%%%%%%%%%%%%%%%%%%%%%%%%
\question{Comparer les quatre systèmes en termes de rapidité}
%%%%%%%%%%%%%%%%%%%%%%%%%%%%%%%%%%%%%%%%%%%%%%%%%%%%%%%%%%%%%%%%%%%%%%%%%%%%%%%%
%%%%%%%%%%%%%%%%%%%%%%%%%%%%%%%%%%%%%%%%%%%%%%%%%%%%%%%%%%%%%%%%%%%%%%%%%%%%%%%%
%%%%%%%%%%%%%%%%%%%%%%%%%%%%%%%%%%%%%%%%%%%%%%%%%%%%%%%%%%%%%%%%%%%%%%%%%%%%%%%%
\exercice{Caractéristiques de la réponse temporelle d'un système du second 
ordre~\moyen}
%%%%%%%%%%%%%%%%%%%%%%%%%%%%%%%%%%%%%%%%%%%%%%%%%%%%%%%%%%%%%%%%%%%%%%%%%%%%%%%%
%%%%%%%%%%%%%%%%%%%%%%%%%%%%%%%%%%%%%%%%%%%%%%%%%%%%%%%%%%%%%%%%%%%%%%%%%%%%%%%%
%%%%%%%%%%%%%%%%%%%%%%%%%%%%%%%%%%%%%%%%%%%%%%%%%%%%%%%%%%%%%%%%%%%%%%%%%%%%%%%%
\question{Déterminer la fonction de transfert du second ordre donnant lieu 
à une réponse indicielle aux caractéristiques suivantes :}
%%%%%%%%%%%%%%%%%%%%%%%%%%%%%%%%%%%%%%%%%%%%%%%%%%%%%%%%%%%%%%%%%%%%%%%%%%%%%%%%
%-------------------------------------------------------------------------------
\noindent Dépassement de 10\%                   \\
Temps de réponse à 5\% de \SI{8}{\milli\second} \\
Une sortie égale à l'entrée en régime permanent \\
%-------------------------------------------------------------------------------
%%%%%%%%%%%%%%%%%%%%%%%%%%%%%%%%%%%%%%%%%%%%%%%%%%%%%%%%%%%%%%%%%%%%%%%%%%%%%%%%
%%%%%%%%%%%%%%%%%%%%%%%%%%%%%%%%%%%%%%%%%%%%%%%%%%%%%%%%%%%%%%%%%%%%%%%%%%%%%%%%
\exercice{Réponse temporelle d'un système du premier ordre~\moyen}
%%%%%%%%%%%%%%%%%%%%%%%%%%%%%%%%%%%%%%%%%%%%%%%%%%%%%%%%%%%%%%%%%%%%%%%%%%%%%%%%
%%%%%%%%%%%%%%%%%%%%%%%%%%%%%%%%%%%%%%%%%%%%%%%%%%%%%%%%%%%%%%%%%%%%%%%%%%%%%%%%
Un système linéaire est caractérisé par l'équation :
\[
    0.5\devi{s(t)}{}+s(t)=15e(t)
\]
%%%%%%%%%%%%%%%%%%%%%%%%%%%%%%%%%%%%%%%%%%%%%%%%%%%%%%%%%%%%%%%%%%%%%%%%%%%%%%%%
\question{Donner l'expression de la fonction de transfert du système, la classe,
l'ordre, le gain et la constante de temps de ce système.} 
%%%%%%%%%%%%%%%%%%%%%%%%%%%%%%%%%%%%%%%%%%%%%%%%%%%%%%%%%%%%%%%%%%%%%%%%%%%%%%%%
On applique une entrée quelconque $x$ pour une durée de \SI{0.1}{\second}, 
suivie par une consigne de 5 pour une durée indéterminée, le graphe de ce
signal est donnée par :
%-------------------------------------------------------------------------------
\begin{center}
    \tikzsetnextfilename{sollicitation_1er_ordre-exercices-chap_model-ext}
    \begin{tikzpicture}[baseline=0]
   \begin{axis}[
        height=4cm,
        width=6cm,
        axis x line=center,
        axis y line=center,
        xmin=-1,
        xmax=3,
        ymin=-0.5,
        ymax=1.7,
        xlabel={$t$},
        ylabel={$e(t)$},
        xlabel style={below right},
        ylabel style={left},
        yticklabels={5,$x$},
        ytick={0.8,1.2},
        y tick label style={left},
        xticklabels={\SI{0.1}{\second}},
        xtick={1},
        x tick label style={below},
        ]
        \addplot [very thick,col1,domain=-1:0, samples=50]{0.01};
        \addplot [very thick,col1,domain=0:1, samples=50]{1.2};
        \addplot [very thick,col1,domain=1:2.9, samples=50]{0.8};
        \end{axis}
\end{tikzpicture}

\end{center}
%-------------------------------------------------------------------------------
%%%%%%%%%%%%%%%%%%%%%%%%%%%%%%%%%%%%%%%%%%%%%%%%%%%%%%%%%%%%%%%%%%%%%%%%%%%%%%%%
\question{Donner l'expression de $s(t)$ en fonction de $x$}
%%%%%%%%%%%%%%%%%%%%%%%%%%%%%%%%%%%%%%%%%%%%%%%%%%%%%%%%%%%%%%%%%%%%%%%%%%%%%%%%
%%%%%%%%%%%%%%%%%%%%%%%%%%%%%%%%%%%%%%%%%%%%%%%%%%%%%%%%%%%%%%%%%%%%%%%%%%%%%%%%
\question{Trouver $x$ pour qu'à $t=\SI{0.1}{\second}$ la réponse atteigne 
sa valeur finale et ne la quitte pas}
%%%%%%%%%%%%%%%%%%%%%%%%%%%%%%%%%%%%%%%%%%%%%%%%%%%%%%%%%%%%%%%%%%%%%%%%%%%%%%%%
