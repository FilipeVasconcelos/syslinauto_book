\chapter{Unités du Système International\label{annexe-usi}}
\chaptermark{Unités du Système International}

La Conférence générale des poids et mesures (CGPM) 
qui se réunit tous les quatre ans depuis 1889, est un organe décisionnel chargé de prendre 
les décisions en matière de métrologie et en particulier en ce 
qui concerne le Système international d'unités (SI)\footnote{La conférence se réunit 
dans les locaux du Bureau international des poids et mesures (BIPM), au Pavillon de Breteuil, 
à Sèvres (France).}.

Ces conférences ont établi les différentes unités fondamentales 
à partir desquelles un grand nombre de grandeurs peuvent être dérivées. 
Dans cette annexe, nous présentons ces unités fondamentales ainsi que les grandeurs dérivées utiles à la mécanique.

\section*{Les unités fondamentales du Système International}
Le Système International compte sept unités de base quantifiant des grandeurs physiques indépendantes.

\begin{table}[!h]
    \centering
    \begin{tabular*}{15cm}{M{6cm}M{2.5cm}M{2.5cm}M{2.5cm}N}
        \hhline{====}
        Grandeur & Nom & Symbole & Dimension \\
        \hhline{----}
        Longeur & mètre & \si{\meter} & [L] \\
        Masse & kilogramme & \si{\kilogram} & [M] \\
        Temps & seconde & \si{\second} & [T] \\
        Intensité du courant électrique & ampère & \si{\ampere} & [I] \\
        Température thermodynamique & kelvin & \si{\kelvin} & [$\Theta$] \\
        Quantité de matière & mole & \si{\mole} & [N] \\
        Intensité lumineuse & candela & \si{\candela} & [J] \\
        \hhline{====}
    \end{tabular*}
    \caption{Les unités de base du Système International}
\end{table}

\newpage
\section*{Grandeurs dérivées du système international}

\begin{table}[!h]
    \centering
    \begin{tabular*}{15cm}{M{5.5cm}M{4cm}M{1.5cm}M{2cm}N}
        \hhline{====}
        Grandeur & Nom & Symbole & Dimension \\
        \hhline{----}
        Vitesse               & mètre par seconde      & \si{\meter\per\second}         & [LT$^{-1}$] \\
        Accélération          & mètre par second carré & \si{\meter\per\second\squared} & [LT$^{-2}$] \\ 
        Force                 & newton                 & \si{\newton}                   & [MLT$^{-2}$] \\
        Moment de force       & newton-mètre           & \si{\newton\meter}            & [ML$^2$T$^{-2}$] \\
        Travail \'Energie     & joule                  & \si{\joule}                    & [ML$^2$T$^{-2}$] \\
        Puissance             & watt                   & \si{\watt}                     & [ML$^2$T$^{-3}$] \\
        Pression              & pascal                 & \si{\pascal}                   & [ML$^{-1}$T$^{-2}$] \\
        Moment d'inertie      & kilogramme-mètre carré & \si{\kilogram\meter\squared}  & [ML$^2$] \\
        Quantité de mouvement & newton-seconde         & \si{\newton\second}           & [MLT$^{-1}$] \\
        \hhline{====}

    \end{tabular*}
    \caption{Grandeurs dérivées du système international.}
\end{table}



