\chapter{Décomposition en éléments simples\label{annexe-DES}}

\section{Avant-propos}
En automatique, la détermination d'une réponse temporelle $s(t)$
correspond à déterminer la transformée 
de Laplace inverse d'une fraction rationnelle $S(p)$ 
définie dans le domaine de Laplace. Autrement dit, 
$$
s(t)=\laplacei{S(p)}
$$
Cette inversion passe généralement par l'utiliation des tables de transformées de Laplace 
(c.f \Cref{annexe-lap}).
Ces tables peuvent ne pas être complètes. La~\gls{des} de $S(p)$, nous permet alors de 
réécrire cette fraction rationnelle sous une forme comportant 
des fractions rationnelles usuellement présente dans les tables.

Dans cette annexe, nous présenterons les techniques de~\gls{des} les plus couramment
recontrées dans l'étude des~\gls{slci}. 
%Ces systèmes ne s'intéressent qu'à 
%une famille de fraction rationnelle.
Cette présentation n'est pas exhaustive, et ne remplacera donc 
pas la lecture du chapitre du cours de mathématiques qui lui est consacré.

\section{Fractions rationnelles rencontrées en automatique}
Dans le cas qui nous intéresse la fraction rationnelle est la réponse $S(p)$ 
défini dans le domaine de Laplace.
Cette grandeur est de la forme,
$$
S(p)=\dfrac{N(p)}{D(p)}
$$
où $N(p)$ et $D(p)$ sont deux polynômes de degrés $m$ et $n$ respectivement.
En générale, nous aurons à faire à des systèmes pour lesquels $m<n$. 
Une des conséquences est que \textbf{la décomposition en élements simples ne comportera
pas de partie entière.}\footnote{}

%%%%%%%%%%%%%%%%%%%%%%%%%%%%%%%%%%%%%%%%%%%%%%%%%%%%%%%%%%%%%%%%%%%%%%%%%%%%%%%%%%%%%%%%%%%%%%%%%%
\section{Décomposition en éléments simples}
%%%%%%%%%%%%%%%%%%%%%%%%%%%%%%%%%%%%%%%%%%%%%%%%%%%%%%%%%%%%%%%%%%%%%%%%%%%%%%%%%%%%%%%%%%%%%%%%%%

Soit $S(p)=\dfrac{N(p)}{D(p)}$ une fraction rationnelle. On considère la décomposition 
de $D(p)$ en produit de polynômes irréductibles\footnote{Nous rappelons que 
dans $\mathbb{R}[p]$, les polynômes irréductibles sont 
les polynômes de degré 1 et les polynômes de degré 2 de discriminant négatif}unitaire\footnote{Un polynôme 
unitaire est un polynôme dont le coefficient de degré le plus grand est 1}
de la forme:
$$
D(p)=a\prod_{k=1}^r(p-\alpha_k)^{m_k}\prod_{k=1}^s(p^2+\beta_lp+\gamma_l)^{n_k}
$$
où $a$ est une constante qui est le coefficient du terme de plus haut degré de $D(p)$, les $\alpha_k$
sont les pôles de multiplicités $m_k$,  les polynômes de degré 2 sont sans pôles réels (i.e $\beta_i-4\gamma_i<0$).

Alors il existe une famille unique de rééls $A_{k,i}$, $B_{l,j}$ et $C_{l,j}$ telles que :
\begin{align}
S(p)=\sum_{k=1}^r\sum_{i=1}^{m_k} \dfrac{A_{k,i}}{\left(p-\alpha_k\right)^i}+\sum_{l=1}^s\sum_{j=1}^{n_l} \dfrac{B_{l,j}p+C_{l,j}}{\left(p^2+\beta_lp+\gamma_l\right)^j}
\end{align}

On appelle cette écriture la \textbf{décomposition en éléments simples} de $S(p)$ sur $\mathbb{R}$.

\paragraph{Exemple 1}
Soit $S(p)$ tel que:
$$
S(p)=\dfrac{1}{(p^2-1)(p^2+1)^2}
$$
où $D(p)$ se factorise sous la forme :
$$
D(p)=(p^2-1)(p^2+1)^2=(p-1)(p+1)(p^2+1)^2
$$
On obtient une décomposition en éléments simples de $S(p)$ de la forme :
$$
S(p)=\dfrac{A}{p-1}+\dfrac{B}{p+1}+\dfrac{Cp+D}{p^2+1}+\dfrac{Ep+F}{p^2+1}
$$

\paragraph{Exemple 2}
Soit $S(p)$ tel que:
$$
S(p)=\dfrac{4p^3}{(p^2-1)^2}
$$
où $D(p)$ se factorise sous la forme :
$$
D(p)=(p^2-1)^2=\left((p-1)(p+1)\right)^2
$$
On obtient une décomposition en éléments simples de $S(p)$ de la forme :
$$
S(p)=\dfrac{A}{p-1}+\dfrac{B}{(p-1)^2}+\dfrac{C}{p+1}+\dfrac{D}{(p+1)^2}
$$
%%%%%%%%%%%%%%%%%%%%%%%%%%%%%%%%%%%%%%%%%%%%%%%%%%%%%%%%%%%%%%%%%%%%%%%%%%%%%%%%%%%%%%%%%%%%%%%%%%
\section[Détermination des coefficients de la DES]{Détermination des coefficients de la décomposition en éléments simples}
%%%%%%%%%%%%%%%%%%%%%%%%%%%%%%%%%%%%%%%%%%%%%%%%%%%%%%%%%%%%%%%%%%%%%%%%%%%%%%%%%%%%%%%%%%%%%%%%%%
\acpl
%%%%%%%%%%%%%%%%%%%%%%%%%%%%%%%%%%%%%%%%%%%%%%%%%%%%%%%%%%%%%%%%%%%%%%%%%%%%%%%%%%%%%%%%%%%%%%%%%%
\subsection{Par identification}
%%%%%%%%%%%%%%%%%%%%%%%%%%%%%%%%%%%%%%%%%%%%%%%%%%%%%%%%%%%%%%%%%%%%%%%%%%%%%%%%%%%%%%%%%%%%%%%%%%
\acpl
%%%%%%%%%%%%%%%%%%%%%%%%%%%%%%%%%%%%%%%%%%%%%%%%%%%%%%%%%%%%%%%%%%%%%%%%%%%%%%%%%%%%%%%%%%%%%%%%%%
\subsection{Multiplication/Substitution}
%%%%%%%%%%%%%%%%%%%%%%%%%%%%%%%%%%%%%%%%%%%%%%%%%%%%%%%%%%%%%%%%%%%%%%%%%%%%%%%%%%%%%%%%%%%%%%%%%%
\acpl
%%%%%%%%%%%%%%%%%%%%%%%%%%%%%%%%%%%%%%%%%%%%%%%%%%%%%%%%%%%%%%%%%%%%%%%%%%%%%%%%%%%%%%%%%%%%%%%%%%
\subsection{\'Evaluation}
%%%%%%%%%%%%%%%%%%%%%%%%%%%%%%%%%%%%%%%%%%%%%%%%%%%%%%%%%%%%%%%%%%%%%%%%%%%%%%%%%%%%%%%%%%%%%%%%%%
\acpl
%%%%%%%%%%%%%%%%%%%%%%%%%%%%%%%%%%%%%%%%%%%%%%%%%%%%%%%%%%%%%%%%%%%%%%%%%%%%%%%%%%%%%%%%%%%%%%%%%%
\subsection{Parité}
%%%%%%%%%%%%%%%%%%%%%%%%%%%%%%%%%%%%%%%%%%%%%%%%%%%%%%%%%%%%%%%%%%%%%%%%%%%%%%%%%%%%%%%%%%%%%%%%%%
\acpl
%%%%%%%%%%%%%%%%%%%%%%%%%%%%%%%%%%%%%%%%%%%%%%%%%%%%%%%%%%%%%%%%%%%%%%%%%%%%%%%%%%%%%%%%%%%%%%%%%%
\subsection{Passage à la limite}
%%%%%%%%%%%%%%%%%%%%%%%%%%%%%%%%%%%%%%%%%%%%%%%%%%%%%%%%%%%%%%%%%%%%%%%%%%%%%%%%%%%%%%%%%%%%%%%%%%
\acpl

