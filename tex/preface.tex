\chapter*{Préface}
\thispagestyle{plain}
\addstarredchapter{Préface}
J'ai commencé l'enseignement des systèmes linéaires et automatiques
à l'ESME de Lille en septembre 2017 au sein du module \og Génie des Systèmes\fg.
Une grande partie du premier semestre des étudiants de première année
était consacrée à l'étude des systèmes linéaires.

Très rapidement, je me suis lancé dans la tâche
de composer des documents \LaTeX~pour ces étudiants.
Au cours de l'année scolaire 2017-2018, de nombreuses fiches
d'exercices (TD) ainsi que des rappels de cours ont été rédigés au
fil des séances en face à face (comme on le disait encore à l'époque). 
À l'été 2018, le polycopié que j'imprimais
chapitre par chapitre l'année scolaire précédente faisait déjà 170 pages.
\begin{itemize}
    \item 2018-2019 -- 170 pages
    \item 2019-2020 -- 238 pages
    \item 2020-2021 -- 360 pages
    \item Actuellement -- 397 pages
\end{itemize}
Le premier \og git commit\fg lié à la production de ce document date
cependant du 26 octobre 2018. Je décidais d'utiliser de façon systématique
les outils de \og versioning\fg pour la production de documents
\LaTeX~comme j'avais pu déjà le faire auparavant pour la rédaction
d'articles scientifiques. Depuis lors, cette approche s'est généralisée pour tout un
tas de modules de l'école avec la contribution d'un grand nombre de collègues.

En prenant en charge l'enseignement des \og Systèmes Mécaniques et Automatiques\fg (SMA)
des étudiants de deuxième année, les documents ont commencé à être distribués
à l'échelle nationale pour tous les campus de l'école.
Depuis lors, le document et le contenu du dépôt n'ont cessé de s'enrichir avec l'ajout
de nouveaux chapitres, de nouvelles figures tikz, d'exercices corrigés, d'annexes
traitant de sujets hors programme ou encore de notebooks d'exploration
numérique de notions liées aux systèmes et à l'automatique linéaires.
L'enseignement sous la forme du module SMA s'est arrêté lors de l'année
scolaire 2021-2022, les \og Systèmes Techniques\fg (l'intitulé du
nouveau module) ne couvrant désormais que la partie mécanique.
La création des classes anglophones a amené automatiquement un besoin
d'outils de composition de documents en deux langues (que ce soit pour les examens
ou les fiches de \og Travaux Dirigés\fg). Tout cela explique pourquoi la production
de contenu au sein du dépôt s'est mécaniquement fortement réduite,
malgré l'envie de terminer la tâche que je m'étais fixée quatre ans auparavant.

La fin d'année scolaire 2024-2025 marque la fin de ma
participation en tant que responsable du module \og Systèmes Techniques\fg. C'est pourquoi
j'ai voulu marquer le coup en partageant de façon publique sur la plateforme GitHub
l'ensemble des ressources liées à la production de ce document.
Je tiens par la même occasion à remercier tous les collègues enseignants des modules
de \og Génie des Systèmes\fg, \og Systèmes Mécaniques et Automatiques\fg
et \og Systèmes Techniques\fg
de ces huit dernières années, qui m'ont aidé et soutenu dans la tâche de production
de contenu ou de coordination. Plus particulièrement
Abdelmoumène Bellagha, Andréa Barregi, Stéphane Feret, Daniel Abécassis,
Amine Noussri, Omar Kraiem, Denis Wagner, Sadek Nourine, Mohamed Saad,
Rachid Britaa, Olubukola Akangbe, Ahlem Sassi, Michel Rossigneux,
Nabih Jaafar, Julien Duboc, Valentin Blonz et Romain Lhomer
et ceux/celles que j'aurais oubliés.

Sans oublier le soutien de tous les collègues de Lille :
Guillaume Roux, Vincent Froger, Mathilde Capelle, Lucie Briolet,
Brahim Jawad, Marie Warembourg, Shabab Samimi, Marianne Weidel,
Riheb Cherif, Lorraine Bayard, Issyan Tekaya, Karina Lefièvre, Amélie Bisson,
Juliette Ducrocq, Kamel Guerchouche et Alexandre Ba et ceux/celles que j'aurais oubliés.

\hspace{1cm}
\hfill Lille, le 05 juin 2025.

\hfill Filipe Vasconcelos
