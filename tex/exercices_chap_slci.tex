%%%%%%%%%%%%%%%%%%%%%%%%%%%%%%%%%%%%%%%%%%%%%%%%%%%%%%%%%%%%%%%%%%%%%%%%%%%%%%%%
%%%%%%%%%%%%%%%%%%%%%%%%%%%%%%%%%%%%%%%%%%%%%%%%%%%%%%%%%%%%%%%%%%%%%%%%%%%%%%%%
\subsection*{Signaux}
%%%%%%%%%%%%%%%%%%%%%%%%%%%%%%%%%%%%%%%%%%%%%%%%%%%%%%%%%%%%%%%%%%%%%%%%%%%%%%%%
%%%%%%%%%%%%%%%%%%%%%%%%%%%%%%%%%%%%%%%%%%%%%%%%%%%%%%%%%%%%%%%%%%%%%%%%%%%%%%%%
%%%%%%%%%%%%%%%%%%%%%%%%%%%%%%%%%%%%%%%%%%%%%%%%%%%%%%%%%%%%%%%%%%%%%%%%%%%%%%%%
%%%%%%%%%%%%%%%%%%%%%%%%%%%%%%%%%%%%%%%%%%%%%%%%%%%%%%%%%%%%%%%%%%%%%%%%%%%%%%%%
\exercice{Décomposition d'un signal~\facile}
%%%%%%%%%%%%%%%%%%%%%%%%%%%%%%%%%%%%%%%%%%%%%%%%%%%%%%%%%%%%%%%%%%%%%%%%%%%%%%%%
%%%%%%%%%%%%%%%%%%%%%%%%%%%%%%%%%%%%%%%%%%%%%%%%%%%%%%%%%%%%%%%%%%%%%%%%%%%%%%%%
Soit le signal d'entrée $e_1(t)$ défini par le graphe ci-contre.
%%%%%%%%%%%%%%%%%%%%%%%%%%%%%%%%%%%%%%%%%%%%%%%%%%%%%%%%%%%%%%%%%%%%%%%%%%%%%%%%
\question{Décomposer $e_1(t)$ à l'aide de fonctions échelons.}
%%%%%%%%%%%%%%%%%%%%%%%%%%%%%%%%%%%%%%%%%%%%%%%%%%%%%%%%%%%%%%%%%%%%%%%%%%%%%%%%
%-------------------------------------------------------------------------------
\begin{marginfigure}
    \centering
    \tikzsetnextfilename{fonction_e1-exercices-chap_slci-ext}
    \input{tikz/fonction_e1-exercices-chap_slci.tex}
\end{marginfigure}
%-------------------------------------------------------------------------------
%%%%%%%%%%%%%%%%%%%%%%%%%%%%%%%%%%%%%%%%%%%%%%%%%%%%%%%%%%%%%%%%%%%%%%%%%%%%%%%%
\question{Donner la transformée de Laplace $E_1(p)$ de ce signal.}
%%%%%%%%%%%%%%%%%%%%%%%%%%%%%%%%%%%%%%%%%%%%%%%%%%%%%%%%%%%%%%%%%%%%%%%%%%%%%%%%
%%%%%%%%%%%%%%%%%%%%%%%%%%%%%%%%%%%%%%%%%%%%%%%%%%%%%%%%%%%%%%%%%%%%%%%%%%%%%%%%
%%%%%%%%%%%%%%%%%%%%%%%%%%%%%%%%%%%%%%%%%%%%%%%%%%%%%%%%%%%%%%%%%%%%%%%%%%%%%%%%
\exercice{Décomposition d'un signal parabolique~\moyen}
%%%%%%%%%%%%%%%%%%%%%%%%%%%%%%%%%%%%%%%%%%%%%%%%%%%%%%%%%%%%%%%%%%%%%%%%%%%%%%%%
%%%%%%%%%%%%%%%%%%%%%%%%%%%%%%%%%%%%%%%%%%%%%%%%%%%%%%%%%%%%%%%%%%%%%%%%%%%%%%%%
Soit le signal d'entrée $e_2(t)$ défini par le graphe ci-contre.
%%%%%%%%%%%%%%%%%%%%%%%%%%%%%%%%%%%%%%%%%%%%%%%%%%%%%%%%%%%%%%%%%%%%%%%%%%%%%%%%
\question{Décomposer $e_2(t)$ à l'aide de fonctions échelons.}
%%%%%%%%%%%%%%%%%%%%%%%%%%%%%%%%%%%%%%%%%%%%%%%%%%%%%%%%%%%%%%%%%%%%%%%%%%%%%%%%
%-------------------------------------------------------------------------------
\begin{marginfigure}
    \centering
    \tikzsetnextfilename{fonction_e2-exercices-chap_slci-ext}
    \input{tikz/fonction_e2-exercices-chap_slci.tex}
\end{marginfigure}
%-------------------------------------------------------------------------------
%%%%%%%%%%%%%%%%%%%%%%%%%%%%%%%%%%%%%%%%%%%%%%%%%%%%%%%%%%%%%%%%%%%%%%%%%%%%%%%%
\question{Donner la transformée de Laplace $E_2(p)$ de ce signal.}
%%%%%%%%%%%%%%%%%%%%%%%%%%%%%%%%%%%%%%%%%%%%%%%%%%%%%%%%%%%%%%%%%%%%%%%%%%%%%%%%
%%%%%%%%%%%%%%%%%%%%%%%%%%%%%%%%%%%%%%%%%%%%%%%%%%%%%%%%%%%%%%%%%%%%%%%%%%%%%%%%
%%%%%%%%%%%%%%%%%%%%%%%%%%%%%%%%%%%%%%%%%%%%%%%%%%%%%%%%%%%%%%%%%%%%%%%%%%%%%%%%
\exercice{Décompositon d'un signal et transformée de Laplace~\difficile}
%%%%%%%%%%%%%%%%%%%%%%%%%%%%%%%%%%%%%%%%%%%%%%%%%%%%%%%%%%%%%%%%%%%%%%%%%%%%%%%%
%%%%%%%%%%%%%%%%%%%%%%%%%%%%%%%%%%%%%%%%%%%%%%%%%%%%%%%%%%%%%%%%%%%%%%%%%%%%%%%%
On sollicite un système linéaire continu et invariant par deux signaux $e_3(t)$ 
et $e_4(t)$ (d'amplitude maximale et de durée unité) définits 
par les graphes ci-contre.
%%%%%%%%%%%%%%%%%%%%%%%%%%%%%%%%%%%%%%%%%%%%%%%%%%%%%%%%%%%%%%%%%%%%%%%%%%%%%%%%
\question{Décomposer $e_3(t)$ et $e_4(t)$ à l'aide de la fonction échelon 
unitaire (c.-à-d. $u(t)$).}
%%%%%%%%%%%%%%%%%%%%%%%%%%%%%%%%%%%%%%%%%%%%%%%%%%%%%%%%%%%%%%%%%%%%%%%%%%%%%%%%
%-------------------------------------------------------------------------------
\begin{marginfigure}
    \centering
    \tikzsetnextfilename{fonction_e3-exercices-chap_slci-ext}
    \begin{tikzpicture}
    \begin{axis}
    [
        height=4cm,
        width=6cm,
        axis x line=center,
        axis y line=center,
        xmin=-0.5,
        xmax=2.5,
        ymin=-0.5,
        ymax=1.5,
        xlabel={$t$},
        ylabel={$e_1(t)$},
        xlabel style={below right},
        ylabel style={above left},
        yticklabels={0,1},
        ytick={0,1},
        y tick label style={anchor=east},
        xticklabels={1},
        xtick={1},
        x tick label style={below},
    ]
        \addplot [signalb,domain=-1:0]{0};
        \addplot [signalb,domain=0:1]{-x+1};
        \addplot [signalb,domain=1:2.45]{0};
        \draw[signalb] (axis cs:0,0) -- (axis cs:0,1);
    \end{axis}
\end{tikzpicture}

\end{marginfigure}
%-------------------------------------------------------------------------------
%%%%%%%%%%%%%%%%%%%%%%%%%%%%%%%%%%%%%%%%%%%%%%%%%%%%%%%%%%%%%%%%%%%%%%%%%%%%%%%%
\question{Déterminer les fonctions $E_3(p)$ et $E_4(p)$ les transformées de 
Laplace respectives de $e_3(t)$ et $e_4(t)$.}
%%%%%%%%%%%%%%%%%%%%%%%%%%%%%%%%%%%%%%%%%%%%%%%%%%%%%%%%%%%%%%%%%%%%%%%%%%%%%%%%
%-------------------------------------------------------------------------------
\begin{marginfigure}
    \centering
    \tikzsetnextfilename{fonction_e3-exercices-chap_slci-ext}
    \begin{tikzpicture}
    \begin{axis}
    [
        height=4cm,
        width=6cm,
        axis x line=center,
        axis y line=center,
        xmin=-0.5,
        xmax=2.5,
        ymin=-0.5,
        ymax=1.5,
        xlabel={$t$},
        ylabel={$e_1(t)$},
        xlabel style={below right},
        ylabel style={above left},
        yticklabels={0,1},
        ytick={0,1},
        y tick label style={anchor=east},
        xticklabels={1},
        xtick={1},
        x tick label style={below},
    ]
        \addplot [signalb,domain=-1:0]{0};
        \addplot [signalb,domain=0:1]{-x+1};
        \addplot [signalb,domain=1:2.45]{0};
        \draw[signalb] (axis cs:0,0) -- (axis cs:0,1);
    \end{axis}
\end{tikzpicture}

\end{marginfigure}
%-------------------------------------------------------------------------------
On sollicite maintenant le système par la fonction $e(t)$ constituée 
de la somme des signaux $e_3(t)$ et $e_4(t-1)$. 
%%%%%%%%%%%%%%%%%%%%%%%%%%%%%%%%%%%%%%%%%%%%%%%%%%%%%%%%%%%%%%%%%%%%%%%%%%%%%%%%
\question{Déterminer à partir des résultats précédents la transformée
de Laplace $E(p)$ correspondant à la somme des signaux $e_3(t)$ et $e_4(t-1)$.}
%%%%%%%%%%%%%%%%%%%%%%%%%%%%%%%%%%%%%%%%%%%%%%%%%%%%%%%%%%%%%%%%%%%%%%%%%%%%%%%%
\clearpage
%%%%%%%%%%%%%%%%%%%%%%%%%%%%%%%%%%%%%%%%%%%%%%%%%%%%%%%%%%%%%%%%%%%%%%%%%%%%%%%%
%%%%%%%%%%%%%%%%%%%%%%%%%%%%%%%%%%%%%%%%%%%%%%%%%%%%%%%%%%%%%%%%%%%%%%%%%%%%%%%%
\exercice{L'impulsion de Dirac approchée~\difficile}
%%%%%%%%%%%%%%%%%%%%%%%%%%%%%%%%%%%%%%%%%%%%%%%%%%%%%%%%%%%%%%%%%%%%%%%%%%%%%%%%
%%%%%%%%%%%%%%%%%%%%%%%%%%%%%%%%%%%%%%%%%%%%%%%%%%%%%%%%%%%%%%%%%%%%%%%%%%%%%%%%
L'impulsion de Dirac peut être approchée par la fonction porte 
$\delta_a(t)$ définie par le graphe ci-contre.
%%%%%%%%%%%%%%%%%%%%%%%%%%%%%%%%%%%%%%%%%%%%%%%%%%%%%%%%%%%%%%%%%%%%%%%%%%%%%%%%
\question{Décomposer $\delta_a(t)$ à l'aide de fonctions échelons.}
%%%%%%%%%%%%%%%%%%%%%%%%%%%%%%%%%%%%%%%%%%%%%%%%%%%%%%%%%%%%%%%%%%%%%%%%%%%%%%%%
%-------------------------------------------------------------------------------
\begin{marginfigure}
    \centering
    \tikzsetnextfilename{fonction_porte-exercices-chap_slci-ext}
    \input{tikz/fonction_porte-exercices-chap_slci.tex}
\end{marginfigure}
%-------------------------------------------------------------------------------
%%%%%%%%%%%%%%%%%%%%%%%%%%%%%%%%%%%%%%%%%%%%%%%%%%%%%%%%%%%%%%%%%%%%%%%%%%%%%%%%
\question{Donner sa transformée de Laplace en fonction du paramètre $a$.}
%%%%%%%%%%%%%%%%%%%%%%%%%%%%%%%%%%%%%%%%%%%%%%%%%%%%%%%%%%%%%%%%%%%%%%%%%%%%%%%%
%%%%%%%%%%%%%%%%%%%%%%%%%%%%%%%%%%%%%%%%%%%%%%%%%%%%%%%%%%%%%%%%%%%%%%%%%%%%%%%%
\question{En considérant la limite $a\rightarrow0$, montrer que la transformée
          de Laplace de $\delta_a(t)$ tend bien vers 1.}
%%%%%%%%%%%%%%%%%%%%%%%%%%%%%%%%%%%%%%%%%%%%%%%%%%%%%%%%%%%%%%%%%%%%%%%%%%%%%%%%
%%%%%%%%%%%%%%%%%%%%%%%%%%%%%%%%%%%%%%%%%%%%%%%%%%%%%%%%%%%%%%%%%%%%%%%%%%%%%%%%
%%%%%%%%%%%%%%%%%%%%%%%%%%%%%%%%%%%%%%%%%%%%%%%%%%%%%%%%%%%%%%%%%%%%%%%%%%%%%%%%
\subsection*{\'Equations différentielles}
%%%%%%%%%%%%%%%%%%%%%%%%%%%%%%%%%%%%%%%%%%%%%%%%%%%%%%%%%%%%%%%%%%%%%%%%%%%%%%%%
%%%%%%%%%%%%%%%%%%%%%%%%%%%%%%%%%%%%%%%%%%%%%%%%%%%%%%%%%%%%%%%%%%%%%%%%%%%%%%%%
%%%%%%%%%%%%%%%%%%%%%%%%%%%%%%%%%%%%%%%%%%%%%%%%%%%%%%%%%%%%%%%%%%%%%%%%%%%%%%%%
%%%%%%%%%%%%%%%%%%%%%%%%%%%%%%%%%%%%%%%%%%%%%%%%%%%%%%%%%%%%%%%%%%%%%%%%%%%%%%%%
\exercice{Décharge d'un condensateur~\facile}
%%%%%%%%%%%%%%%%%%%%%%%%%%%%%%%%%%%%%%%%%%%%%%%%%%%%%%%%%%%%%%%%%%%%%%%%%%%%%%%%
%%%%%%%%%%%%%%%%%%%%%%%%%%%%%%%%%%%%%%%%%%%%%%%%%%%%%%%%%%%%%%%%%%%%%%%%%%%%%%%%
On reprend la mise en équation de la décharge d'un condensateur 
dans un circuit RC comme présentée dans ce chapitre 
(c.f \cref{para-decharge}). L'équation différentielle
de la charge $q(t)$ au borne du condensateur est donnée par :
\[
    RC\devi{q(t)}{}+q(t)=0
\]
Le condensateur est initialement chargé $q(0)=q_0$. On remarquera que le 
produit $RC$ est de dimension d'un temps. On notera cette constante $\tau=RC$.
%-------------------------------------------------------------------------------
\begin{marginfigure}
    \centering
    \tikzsetnextfilename{decharge_condensateur-exercices-chap_slci-ext}
    \input{tikz/decharge_condensateur-exercices-chap_slci.tex}
\end{marginfigure}
%-------------------------------------------------------------------------------

%%%%%%%%%%%%%%%%%%%%%%%%%%%%%%%%%%%%%%%%%%%%%%%%%%%%%%%%%%%%%%%%%%%%%%%%%%%%%%%%
\question{Déterminer la transformée de Laplace $Q(p)$ de la charge au borne du 
condensateur $q(t)$.}
%%%%%%%%%%%%%%%%%%%%%%%%%%%%%%%%%%%%%%%%%%%%%%%%%%%%%%%%%%%%%%%%%%%%%%%%%%%%%%%%
%%%%%%%%%%%%%%%%%%%%%%%%%%%%%%%%%%%%%%%%%%%%%%%%%%%%%%%%%%%%%%%%%%%%%%%%%%%%%%%%
\question{Déterminer alors la fonction $q(t)$ solution de l'équation 
différentielle de la décharge d'un condensateur.}
%%%%%%%%%%%%%%%%%%%%%%%%%%%%%%%%%%%%%%%%%%%%%%%%%%%%%%%%%%%%%%%%%%%%%%%%%%%%%%%%
\clearpage
%%%%%%%%%%%%%%%%%%%%%%%%%%%%%%%%%%%%%%%%%%%%%%%%%%%%%%%%%%%%%%%%%%%%%%%%%%%%%%%%
%%%%%%%%%%%%%%%%%%%%%%%%%%%%%%%%%%%%%%%%%%%%%%%%%%%%%%%%%%%%%%%%%%%%%%%%%%%%%%%%
\exercice{Système masse-ressort~\difficile}
%%%%%%%%%%%%%%%%%%%%%%%%%%%%%%%%%%%%%%%%%%%%%%%%%%%%%%%%%%%%%%%%%%%%%%%%%%%%%%%%
%%%%%%%%%%%%%%%%%%%%%%%%%%%%%%%%%%%%%%%%%%%%%%%%%%%%%%%%%%%%%%%%%%%%%%%%%%%%%%%%
On reprend ici, la mise en équation du système mécanique masse-ressort 
présentée dans ce chapitre (c.f \cref{para-masse_ressort}). L'équation 
différentielle reliant la position de la masse $z(t)$ et 
la force appliquée $f(t)$ est donnée par 
\[
m\devi{z(t)}{2}+b\devi{z(t)}{}+kz(t)=f(t)
\]
où $m$ est la masse, $b$ le coefficient d'amortissement visqueux et $k$ la
constante de raideur du ressort. Ce système peut être considéré comme
un système dont l'entrée est la force $f(t)$ et la sortie est la position
$x(t)$ de la masse.
%-------------------------------------------------------------------------------
\begin{marginfigure}
    \centering
    \tikzsetnextfilename{masse_ressort-exercices-chap_slci-ext}
    \input{tikz/masse_ressort-exercices-chap_slci.tex}
\end{marginfigure}
%-------------------------------------------------------------------------------
On se place dans le cas de conditions initiales nulles. 
%%%%%%%%%%%%%%%%%%%%%%%%%%%%%%%%%%%%%%%%%%%%%%%%%%%%%%%%%%%%%%%%%%%%%%%%%%%%%%%%
\question{Déterminer la transformée de Laplace $Z(p)$ solution algébrique 
de la transformée de Laplace de l'équation différentielle de ce système.}
%%%%%%%%%%%%%%%%%%%%%%%%%%%%%%%%%%%%%%%%%%%%%%%%%%%%%%%%%%%%%%%%%%%%%%%%%%%%%%%%
%%%%%%%%%%%%%%%%%%%%%%%%%%%%%%%%%%%%%%%%%%%%%%%%%%%%%%%%%%%%%%%%%%%%%%%%%%%%%%%%
\question{Indentifier la fonction de transfert $H(p)$ du système et 
donner sa forme canonique}
%%%%%%%%%%%%%%%%%%%%%%%%%%%%%%%%%%%%%%%%%%%%%%%%%%%%%%%%%%%%%%%%%%%%%%%%%%%%%%%%
%%%%%%%%%%%%%%%%%%%%%%%%%%%%%%%%%%%%%%%%%%%%%%%%%%%%%%%%%%%%%%%%%%%%%%%%%%%%%%%%
\question{Indentifier la solution algébrique $Z(p)$ pour une cette 
solicitation.}
%%%%%%%%%%%%%%%%%%%%%%%%%%%%%%%%%%%%%%%%%%%%%%%%%%%%%%%%%%%%%%%%%%%%%%%%%%%%%%%%
\question{En comparant, la solution algébrique précédente à la forme 
\[
    \dfrac{K}{\dfrac{1}{\omega^2}p^2+\dfrac{2\xi}{\omega} p+1},
\] 
identifier les paramètres $K$, $\xi$ et $\omega$ en fonction des 
paramètres du problème.}
%%%%%%%%%%%%%%%%%%%%%%%%%%%%%%%%%%%%%%%%%%%%%%%%%%%%%%%%%%%%%%%%%%%%%%%%%%%%%%%%
%\newpage
%\restoregeometry
%\captionsetup{width=0.9\linewidth}
%%%%%%%%%%%%%%%%%%%%%%%%%%%%%%%%%%%%%%%%%%%%%%%%%%%%%%%%%%%%%%%%%%%%%%%%%%%%%%%%
\question{En utilisant, la transformée de Laplace inverse suivante : 
\[
    \laplacei{\dfrac{1}{\dfrac{1}{\omega^2}p^2+\dfrac{2\xi}{\omega} p+1}}=
    \dfrac{\omega}{\sqrt{1-\xi^2}}
    e^{-\xi\omega t}\sin{\omega\sqrt{1-\xi^2} t}
\quad\quad \text{pour}\quad\xi<1, 
\]
déterminer $z(t)$ la solution temporelle de l'équation différentielle 
pour une telle sollicitation.}
%%%%%%%%%%%%%%%%%%%%%%%%%%%%%%%%%%%%%%%%%%%%%%%%%%%%%%%%%%%%%%%%%%%%%%%%%%%%%%%%
$\,$\newline
On réalise une experience avec une masse $m=\SI{0.5}{\kilogram}$, 
un ressort de constante de raideur $k=\SI{10}{\newton\per\meter}$ et 
un amortissement visqueux $b=\SI{0.894}{\kilogram\per\second}$. 
%%%%%%%%%%%%%%%%%%%%%%%%%%%%%%%%%%%%%%%%%%%%%%%%%%%%%%%%%%%%%%%%%%%%%%%%%%%%%%%%
\question{Déterminer les valeurs numériques des paramètres $\xi$ et $\omega$}
%%%%%%%%%%%%%%%%%%%%%%%%%%%%%%%%%%%%%%%%%%%%%%%%%%%%%%%%%%%%%%%%%%%%%%%%%%%%%%%%
%%%%%%%%%%%%%%%%%%%%%%%%%%%%%%%%%%%%%%%%%%%%%%%%%%%%%%%%%%%%%%%%%%%%%%%%%%%%%%%%
\question{Tracer la réponse $z(t)$ et déterminer le temps pour que la masse 
retourne à son position d'équilibre (à 5\% de son déplacement maximal)}
%%%%%%%%%%%%%%%%%%%%%%%%%%%%%%%%%%%%%%%%%%%%%%%%%%%%%%%%%%%%%%%%%%%%%%%%%%%%%%%%
%%%%%%%%%%%%%%%%%%%%%%%%%%%%%%%%%%%%%%%%%%%%%%%%%%%%%%%%%%%%%%%%%%%%%%%%%%%%%%%%
%%%%%%%%%%%%%%%%%%%%%%%%%%%%%%%%%%%%%%%%%%%%%%%%%%%%%%%%%%%%%%%%%%%%%%%%%%%%%%%%
\subsection*{Fonction de transfert}
%%%%%%%%%%%%%%%%%%%%%%%%%%%%%%%%%%%%%%%%%%%%%%%%%%%%%%%%%%%%%%%%%%%%%%%%%%%%%%%%
%%%%%%%%%%%%%%%%%%%%%%%%%%%%%%%%%%%%%%%%%%%%%%%%%%%%%%%%%%%%%%%%%%%%%%%%%%%%%%%%
%%%%%%%%%%%%%%%%%%%%%%%%%%%%%%%%%%%%%%%%%%%%%%%%%%%%%%%%%%%%%%%%%%%%%%%%%%%%%%%%
%%%%%%%%%%%%%%%%%%%%%%%%%%%%%%%%%%%%%%%%%%%%%%%%%%%%%%%%%%%%%%%%%%%%%%%%%%%%%%%%
\exercice{Forme canonique d'une fonction de transfert du premier ordre~\facile}
%%%%%%%%%%%%%%%%%%%%%%%%%%%%%%%%%%%%%%%%%%%%%%%%%%%%%%%%%%%%%%%%%%%%%%%%%%%%%%%%
%%%%%%%%%%%%%%%%%%%%%%%%%%%%%%%%%%%%%%%%%%%%%%%%%%%%%%%%%%%%%%%%%%%%%%%%%%%%%%%%
Soit un système régi par l'équation différentielle suivante 
\[
    a_1\devi{s(t)}{} + a_0 s(t) = b_0 e(t)
\]
%%%%%%%%%%%%%%%%%%%%%%%%%%%%%%%%%%%%%%%%%%%%%%%%%%%%%%%%%%%%%%%%%%%%%%%%%%%%%%%%
\question{Déterminer la forme canonique de fonction de transfert 
$H(p)$ associée à cette équation différentielle}
%%%%%%%%%%%%%%%%%%%%%%%%%%%%%%%%%%%%%%%%%%%%%%%%%%%%%%%%%%%%%%%%%%%%%%%%%%%%%%%%
%%%%%%%%%%%%%%%%%%%%%%%%%%%%%%%%%%%%%%%%%%%%%%%%%%%%%%%%%%%%%%%%%%%%%%%%%%%%%%%%
\question{Déterminer les paramètres $K$ et $\tau$ en fonction des coefficients
$a_i$ et $b_i$ de tels sorte que $H(p)$ s'écrivent sous la forme}
%%%%%%%%%%%%%%%%%%%%%%%%%%%%%%%%%%%%%%%%%%%%%%%%%%%%%%%%%%%%%%%%%%%%%%%%%%%%%%%%
%%%%%%%%%%%%%%%%%%%%%%%%%%%%%%%%%%%%%%%%%%%%%%%%%%%%%%%%%%%%%%%%%%%%%%%%%%%%%%%%
\question{Déterminer les dimensions de ces paramètres quelques soient celles 
des signaux d'entrée $e(t)$ et de sortie $s(t)$}
%%%%%%%%%%%%%%%%%%%%%%%%%%%%%%%%%%%%%%%%%%%%%%%%%%%%%%%%%%%%%%%%%%%%%%%%%%%%%%%%
%%%%%%%%%%%%%%%%%%%%%%%%%%%%%%%%%%%%%%%%%%%%%%%%%%%%%%%%%%%%%%%%%%%%%%%%%%%%%%%%
%%%%%%%%%%%%%%%%%%%%%%%%%%%%%%%%%%%%%%%%%%%%%%%%%%%%%%%%%%%%%%%%%%%%%%%%%%%%%%%%
\exercice{Carte des pôles d'une fonction de transfert~\moyen}
%%%%%%%%%%%%%%%%%%%%%%%%%%%%%%%%%%%%%%%%%%%%%%%%%%%%%%%%%%%%%%%%%%%%%%%%%%%%%%%%
%%%%%%%%%%%%%%%%%%%%%%%%%%%%%%%%%%%%%%%%%%%%%%%%%%%%%%%%%%%%%%%%%%%%%%%%%%%%%%%%
Soit la carte des pôles et zéros ci-contre 
%%%%%%%%%%%%%%%%%%%%%%%%%%%%%%%%%%%%%%%%%%%%%%%%%%%%%%%%%%%%%%%%%%%%%%%%%%%%%%%%
\question{Donner les zéros et les pôles de la fonction de transfert.}
%%%%%%%%%%%%%%%%%%%%%%%%%%%%%%%%%%%%%%%%%%%%%%%%%%%%%%%%%%%%%%%%%%%%%%%%%%%%%%%%
%%%%%%%%%%%%%%%%%%%%%%%%%%%%%%%%%%%%%%%%%%%%%%%%%%%%%%%%%%%%%%%%%%%%%%%%%%%%%%%%
\question{\'Ecrire la fonction de transfert $H(p)$ sachant que le gain de 
sa forme factorisée vaut 3.}
%%%%%%%%%%%%%%%%%%%%%%%%%%%%%%%%%%%%%%%%%%%%%%%%%%%%%%%%%%%%%%%%%%%%%%%%%%%%%%%%
%-------------------------------------------------------------------------------
\begin{marginfigure}
    \centering
    \tikzsetnextfilename{carte_exo-exercices-chap_slci-ext}
    \input{tikz/carte_exo-exercices-chap_slci.tex}
\end{marginfigure}
%-------------------------------------------------------------------------------
%%%%%%%%%%%%%%%%%%%%%%%%%%%%%%%%%%%%%%%%%%%%%%%%%%%%%%%%%%%%%%%%%%%%%%%%%%%%%%%%
\question{Déterminer la forme canonique de $H(p)$ et déduire le gain statique, 
la classe et l'ordre de ce système.}
%%%%%%%%%%%%%%%%%%%%%%%%%%%%%%%%%%%%%%%%%%%%%%%%%%%%%%%%%%%%%%%%%%%%%%%%%%%%%%%%
