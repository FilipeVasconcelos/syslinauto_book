%%%%%%%%%%%%%%%%%%%%%%%%%%%%%%%%%%%%%%%%%%%%%%%%%%%%%%%%%%%%%%%%%%%%%%%%%%%%%%%%
%%%%%%%%%%%%%%%%%%%%%%%%%%%%%%%%%%%%%%%%%%%%%%%%%%%%%%%%%%%%%%%%%%%%%%%%%%%%%%%%
\subsection*{Signaux}
%%%%%%%%%%%%%%%%%%%%%%%%%%%%%%%%%%%%%%%%%%%%%%%%%%%%%%%%%%%%%%%%%%%%%%%%%%%%%%%%
%%%%%%%%%%%%%%%%%%%%%%%%%%%%%%%%%%%%%%%%%%%%%%%%%%%%%%%%%%%%%%%%%%%%%%%%%%%%%%%%
%%%%%%%%%%%%%%%%%%%%%%%%%%%%%%%%%%%%%%%%%%%%%%%%%%%%%%%%%%%%%%%%%%%%%%%%%%%%%%%%
%%%%%%%%%%%%%%%%%%%%%%%%%%%%%%%%%%%%%%%%%%%%%%%%%%%%%%%%%%%%%%%%%%%%%%%%%%%%%%%%
\exercice{Décomposition d'un signal~\facile}
%%%%%%%%%%%%%%%%%%%%%%%%%%%%%%%%%%%%%%%%%%%%%%%%%%%%%%%%%%%%%%%%%%%%%%%%%%%%%%%%
%%%%%%%%%%%%%%%%%%%%%%%%%%%%%%%%%%%%%%%%%%%%%%%%%%%%%%%%%%%%%%%%%%%%%%%%%%%%%%%%
Soit le signal d'entrée $e(t)$ définie par le graphe ci-contre.

%%%%%%%%%%%%%%%%%%%%%%%%%%%%%%%%%%%%%%%%%%%%%%%%%%%%%%%%%%%%%%%%%%%%%%%%%%%%%%%%
\question{Décomposer $e(t)$ à l'aide de fonctions échelons.}
%%%%%%%%%%%%%%%%%%%%%%%%%%%%%%%%%%%%%%%%%%%%%%%%%%%%%%%%%%%%%%%%%%%%%%%%%%%%%%%%
%-------------------------------------------------------------------------------
\begin{marginfigure}
    \centering
    \tikzsetnextfilename{fonction_triangle2-chap-slci-ext}
    \begin{tikzpicture}[baseline=0]
   \begin{axis}[
        height=4cm,
        width=6cm,
        axis x line=center,
        axis y line=center,
        xmin=-1,
        xmax=5,
        ymin=-1.5,
        ymax=1.5,
        xlabel={$t$},
        ylabel={$e(t)$},
        xlabel style={below right},
        ylabel style={left},
        yticklabels={-1,1},
        ytick={-1,1},
        y tick label style={left},
        xticklabels={1,2,3,4},
        xtick={1,2,3,4},
        x tick label style={below},
        ]
        \addplot [very thick,col1,domain=-1:0, samples=50]{0.01};
        \addplot [very thick,col1,domain=0:1, samples=50]{x};
        \addplot [very thick,col1,domain=1:3, samples=50]{2-x};
        \addplot [very thick,col1,domain=3:4, samples=50]{x-4};
        \addplot [very thick,col1,domain=4:4.9, samples=50]{0.01};
        \end{axis}
\end{tikzpicture}

\end{marginfigure}
%-------------------------------------------------------------------------------

%%%%%%%%%%%%%%%%%%%%%%%%%%%%%%%%%%%%%%%%%%%%%%%%%%%%%%%%%%%%%%%%%%%%%%%%%%%%%%%%
\question{Donner la transformée de Laplace $E(p)$ de ce signal.}
%%%%%%%%%%%%%%%%%%%%%%%%%%%%%%%%%%%%%%%%%%%%%%%%%%%%%%%%%%%%%%%%%%%%%%%%%%%%%%%%

%%%%%%%%%%%%%%%%%%%%%%%%%%%%%%%%%%%%%%%%%%%%%%%%%%%%%%%%%%%%%%%%%%%%%%%%%%%%%%%%
%%%%%%%%%%%%%%%%%%%%%%%%%%%%%%%%%%%%%%%%%%%%%%%%%%%%%%%%%%%%%%%%%%%%%%%%%%%%%%%%
\exercice{Décomposition d'un signal parabolique~\moyen}
%%%%%%%%%%%%%%%%%%%%%%%%%%%%%%%%%%%%%%%%%%%%%%%%%%%%%%%%%%%%%%%%%%%%%%%%%%%%%%%%
%%%%%%%%%%%%%%%%%%%%%%%%%%%%%%%%%%%%%%%%%%%%%%%%%%%%%%%%%%%%%%%%%%%%%%%%%%%%%%%%

%%%%%%%%%%%%%%%%%%%%%%%%%%%%%%%%%%%%%%%%%%%%%%%%%%%%%%%%%%%%%%%%%%%%%%%%%%%%%%%%
\question{Décomposer $e(t)$ à l'aide de fonctions échelons.}
%%%%%%%%%%%%%%%%%%%%%%%%%%%%%%%%%%%%%%%%%%%%%%%%%%%%%%%%%%%%%%%%%%%%%%%%%%%%%%%%
%-------------------------------------------------------------------------------
\begin{marginfigure}
    \centering
    \tikzsetnextfilename{fonction_parabole-chap-slci-ext}
    \input{tikz/fonction_parabole-chap-slci.tex}
\end{marginfigure}
%-------------------------------------------------------------------------------

%%%%%%%%%%%%%%%%%%%%%%%%%%%%%%%%%%%%%%%%%%%%%%%%%%%%%%%%%%%%%%%%%%%%%%%%%%%%%%%%
\question{Donner la transformée de Laplace $E(p)$ de ce signal.}
%%%%%%%%%%%%%%%%%%%%%%%%%%%%%%%%%%%%%%%%%%%%%%%%%%%%%%%%%%%%%%%%%%%%%%%%%%%%%%%%

%%%%%%%%%%%%%%%%%%%%%%%%%%%%%%%%%%%%%%%%%%%%%%%%%%%%%%%%%%%%%%%%%%%%%%%%%%%%%%%%
%%%%%%%%%%%%%%%%%%%%%%%%%%%%%%%%%%%%%%%%%%%%%%%%%%%%%%%%%%%%%%%%%%%%%%%%%%%%%%%%
\exercice{Décompositon d'un signal et transformée de Laplace~\difficile}
%%%%%%%%%%%%%%%%%%%%%%%%%%%%%%%%%%%%%%%%%%%%%%%%%%%%%%%%%%%%%%%%%%%%%%%%%%%%%%%%
%%%%%%%%%%%%%%%%%%%%%%%%%%%%%%%%%%%%%%%%%%%%%%%%%%%%%%%%%%%%%%%%%%%%%%%%%%%%%%%%

On sollicite un système linéaire continu et invariant par deux signaux $e_1(t)$ 
et $e_2(t)$ (d'amplitude maximale et de durée unité) définits 
par les graphes ci-contre.

%%%%%%%%%%%%%%%%%%%%%%%%%%%%%%%%%%%%%%%%%%%%%%%%%%%%%%%%%%%%%%%%%%%%%%%%%%%%%%%%
\question{Décomposer $e_1(t)$ et $e_2(t)$ à l'aide de la fonction échelon 
unitaire (i.e $u(t)$).}
%%%%%%%%%%%%%%%%%%%%%%%%%%%%%%%%%%%%%%%%%%%%%%%%%%%%%%%%%%%%%%%%%%%%%%%%%%%%%%%%
%-------------------------------------------------------------------------------
\begin{marginfigure}
    \centering
    \tikzsetnextfilename{fonction_e1-chap-slci-ext}
    \input{tikz/fonction_e1-chap-slci.tex}
\end{marginfigure}
%-------------------------------------------------------------------------------


%%%%%%%%%%%%%%%%%%%%%%%%%%%%%%%%%%%%%%%%%%%%%%%%%%%%%%%%%%%%%%%%%%%%%%%%%%%%%%%%
\question{Déterminer les fonctions $E_1(p)$ et $E_2(p)$ les transformées de 
Laplace respectives de $e_1(t)$ et $e_2(t)$.}
%%%%%%%%%%%%%%%%%%%%%%%%%%%%%%%%%%%%%%%%%%%%%%%%%%%%%%%%%%%%%%%%%%%%%%%%%%%%%%%%
%-------------------------------------------------------------------------------
\begin{marginfigure}
    \centering
    \tikzsetnextfilename{fonction_e2-chap-slci-ext}
    \begin{tikzpicture}
    \begin{axis}
    [
        height=4cm,
        width=6cm,
        axis x line=center,
        axis y line=center,
        xmin=-1,
        xmax=1.5,
        ymin=-0.5,
        ymax=1.5,
        xlabel={$t$},
        ylabel={$e_2(t)$},
        xlabel style={below right},
        ylabel style={above left},
        yticklabels={0,1},
        ytick={0,1},
        y tick label style={anchor=east},
        xticklabels={1},
        xtick={1},
        x tick label style={below},
    ]
        \addplot [very thick,col1,domain=-1:0, samples=51]{0};
        \addplot [very thick,col1,domain=0:1, samples=51]{x};
        \addplot [very thick,col1,domain=1:1.45, samples=51]{0};
        \draw[very thick,col1] (axis cs:1,1) -- (axis cs:1,0);
    \end{axis}
\end{tikzpicture}

\end{marginfigure}
%-------------------------------------------------------------------------------

%%%%%%%%%%%%%%%%%%%%%%%%%%%%%%%%%%%%%%%%%%%%%%%%%%%%%%%%%%%%%%%%%%%%%%%%%%%%%%%%
%\question{Généraliser ces entrées pour une amplitude maximale $E_0$ 
%et une durée du signal $\tau$ quelconques.} 
%%%%%%%%%%%%%%%%%%%%%%%%%%%%%%%%%%%%%%%%%%%%%%%%%%%%%%%%%%%%%%%%%%%%%%%%%%%%%%%%

On sollicite maintenant le système par la fonction $e(t)$ constituée 
des signaux $e_1(t)$ et $e_2(t-1)$. 

%-------------------------------------------------------------------------------
%\begin{marginfigure}
%    \centering
%    \tikzsetnextfilename{fonction_e3-chap-slci-ext}
%    \input{tikz/fonction_e3-chap-slci.tex}
%\end{marginfigure}
%-------------------------------------------------------------------------------

%%%%%%%%%%%%%%%%%%%%%%%%%%%%%%%%%%%%%%%%%%%%%%%%%%%%%%%%%%%%%%%%%%%%%%%%%%%%%%%%
\question{Déterminer à partir des résultats précédents la transformée 
de Laplace $E(p)$.}
%%%%%%%%%%%%%%%%%%%%%%%%%%%%%%%%%%%%%%%%%%%%%%%%%%%%%%%%%%%%%%%%%%%%%%%%%%%%%%%%

%%%%%%%%%%%%%%%%%%%%%%%%%%%%%%%%%%%%%%%%%%%%%%%%%%%%%%%%%%%%%%%%%%%%%%%%%%%%%%%%
%%%%%%%%%%%%%%%%%%%%%%%%%%%%%%%%%%%%%%%%%%%%%%%%%%%%%%%%%%%%%%%%%%%%%%%%%%%%%%%%
\exercice{L'impulsion de Dirac approchée~\difficile}
%%%%%%%%%%%%%%%%%%%%%%%%%%%%%%%%%%%%%%%%%%%%%%%%%%%%%%%%%%%%%%%%%%%%%%%%%%%%%%%%
%%%%%%%%%%%%%%%%%%%%%%%%%%%%%%%%%%%%%%%%%%%%%%%%%%%%%%%%%%%%%%%%%%%%%%%%%%%%%%%%
L'impulsion de Dirac peut être approchée par la fonction porte 
$\delta_a(t)$ définie par le graphe ci-contre.

%%%%%%%%%%%%%%%%%%%%%%%%%%%%%%%%%%%%%%%%%%%%%%%%%%%%%%%%%%%%%%%%%%%%%%%%%%%%%%%%
\question{Décomposer $\delta_a(t)$ à l'aide de fonctions échelons.}
%%%%%%%%%%%%%%%%%%%%%%%%%%%%%%%%%%%%%%%%%%%%%%%%%%%%%%%%%%%%%%%%%%%%%%%%%%%%%%%%
%-------------------------------------------------------------------------------
\begin{marginfigure}
    \centering
    \tikzsetnextfilename{fonction_porte-chap-slci-ext}
    \begin{tikzpicture}[baseline=0]
   \begin{axis}[
        height=4cm,
        width=5cm,
        axis x line=center,
        axis y line=center,
        xmin=-1,
        xmax=5,
        ymin=-0.5,
        ymax=2.0,
        xlabel={$t$},
        ylabel={$\delta_a(t)$},
        xlabel style={below right},
        ylabel style={left},
        yticklabels={$\dfrac{1}{a}$},
        ytick={1},
        y tick label style={anchor=east},
        xticklabels={$a$},
        xtick={2},
        x tick label style={anchor=north},
        ]
        \addplot [very thick,col1,const plot] coordinates 
        {(-1,0.01) (0,0.01) (0,1) (2,1)  (2,0.01) (5,0.01) };
        \end{axis}
\end{tikzpicture}

\end{marginfigure}
%-------------------------------------------------------------------------------

%%%%%%%%%%%%%%%%%%%%%%%%%%%%%%%%%%%%%%%%%%%%%%%%%%%%%%%%%%%%%%%%%%%%%%%%%%%%%%%%
\question{Donner sa transformée de Laplace en fonction du paramètre $a$.}
%%%%%%%%%%%%%%%%%%%%%%%%%%%%%%%%%%%%%%%%%%%%%%%%%%%%%%%%%%%%%%%%%%%%%%%%%%%%%%%%

%%%%%%%%%%%%%%%%%%%%%%%%%%%%%%%%%%%%%%%%%%%%%%%%%%%%%%%%%%%%%%%%%%%%%%%%%%%%%%%%
\question{En considérant la limite $a\rightarrow0$, montrer que la transformée
          de Laplace de $\delta_a(t)$ tend bien vers 1.}
%%%%%%%%%%%%%%%%%%%%%%%%%%%%%%%%%%%%%%%%%%%%%%%%%%%%%%%%%%%%%%%%%%%%%%%%%%%%%%%%

%%%%%%%%%%%%%%%%%%%%%%%%%%%%%%%%%%%%%%%%%%%%%%%%%%%%%%%%%%%%%%%%%%%%%%%%%%%%%%%%
%%%%%%%%%%%%%%%%%%%%%%%%%%%%%%%%%%%%%%%%%%%%%%%%%%%%%%%%%%%%%%%%%%%%%%%%%%%%%%%%
\subsection*{\'Equations différentielles}
%%%%%%%%%%%%%%%%%%%%%%%%%%%%%%%%%%%%%%%%%%%%%%%%%%%%%%%%%%%%%%%%%%%%%%%%%%%%%%%%
%%%%%%%%%%%%%%%%%%%%%%%%%%%%%%%%%%%%%%%%%%%%%%%%%%%%%%%%%%%%%%%%%%%%%%%%%%%%%%%%

%%%%%%%%%%%%%%%%%%%%%%%%%%%%%%%%%%%%%%%%%%%%%%%%%%%%%%%%%%%%%%%%%%%%%%%%%%%%%%%%
%%%%%%%%%%%%%%%%%%%%%%%%%%%%%%%%%%%%%%%%%%%%%%%%%%%%%%%%%%%%%%%%%%%%%%%%%%%%%%%%
\exercice{Décharge d'un condensateur~\facile}
%%%%%%%%%%%%%%%%%%%%%%%%%%%%%%%%%%%%%%%%%%%%%%%%%%%%%%%%%%%%%%%%%%%%%%%%%%%%%%%%
%%%%%%%%%%%%%%%%%%%%%%%%%%%%%%%%%%%%%%%%%%%%%%%%%%%%%%%%%%%%%%%%%%%%%%%%%%%%%%%%
On reprend la mise en équation de la décharge d'un condensateur 
dans un circuit RC comme présentée dans ce chapitre 
(c.f \cref{para-decharge}). L'équation différentielle
de la charge $q(t)$ au borne du condensateur est donnée par :
\[
    RC\devi{q(t)}{}+q(t)=0
\]
Le condensateur est initialement chargé $q(0)=q_0$. On remarquera que le 
produit $RC$ est de dimension d'un temps. On notera cette constante $\tau=RC$.
%-------------------------------------------------------------------------------
\begin{marginfigure}
    \centering
    \tikzsetnextfilename{decharge_condensateur-chap_slci-ext}
    %               K           
%     A ------B    C---------D
%     |                      |
%     |                      |
% C1--E--C2                --J-R2 
%                          |   |
% C3--F--C4               R1-I--
%     |                      |
%     |                      |
%     G----------------------H
%
%
\begin{tikzpicture}

    \pgfmathsetmacro{\pC}{0.75}   % distance de C(1,2,3,4) en x relativement E/F
    \pgfmathsetmacro{\eC}{0.5}    % ecartement du condensateur E-F
    \pgfmathsetmacro{\lR}{1.25}   % longueur de la resistance
    \pgfmathsetmacro{\pR}{0.25}   % position de R1 / I
    \pgfmathsetmacro{\eR}{2*\pR}
    \pgfmathsetmacro{\ab}{1.5}
    \pgfmathsetmacro{\cb}{1.0}
    \pgfmathsetmacro{\gh}{2*\ab+\cb}
    \pgfmathsetmacro{\fg}{1}
    \pgfmathsetmacro{\tmpu}{0.5*\eC}
    \pgfmathsetmacro{\tmpd}{-0.5*\lR}
    \pgfmathsetmacro{\dj}{\fg+\tmpu+\tmpd}

    \coordinate (A)  at (0,0);
    \coordinate (B)  at ($(A)+(\ab,0)$);
    \coordinate (C)  at ($(B)+(\cb,0)$);
    \coordinate (D)  at ($(C)+(\ab,0)$);
    \coordinate (E)  at ($(A)+(0,-\fg)$);
    \coordinate (F)  at ($(E)+(0,-\eC)$);
    \coordinate (C1) at ($(E)+(-\pC,0)$);
    \coordinate (C2) at ($(E)+(\pC,0)$);
    \coordinate (C3) at ($(F)+(-\pC,0)$);
    \coordinate (C4) at ($(F)+(\pC,0)$);
    \coordinate (G)  at ($(F)+(0,-\fg)$);
    \coordinate (H)  at ($(G)+(\gh,0)$);
    \coordinate (I)  at ($(H)+(0,\dj)$);
    \coordinate (J)  at ($(I)+(0,\lR)$);
    \coordinate (K)  at ($(C)+(-0.75,0.5)$);
    \coordinate (R1) at ($(I)+(-\pR,0)$);
    \coordinate (R2) at ($(R1)+(\eR,\lR)$);

    %\node[circle,fill=white,draw=black,ultra thick] at (B) {};
    \draw[thick] (A)  -- (B);
    \draw[thick] (C)  -- (D);
    \draw[thick] (A)  -- (E);
    \draw[very thick,col1] (C1) -- (C2) 
    node[col1,xshift=1em,yshift=-0.5em] {$C$};
    \draw[very thick,col1] (C3) -- (C4);
    \draw[thick] (F)  -- (G);
    \draw[thick] (G)  -- (H);
    \draw[thick] (I)  -- (H);
    \draw[thick] (J)  -- (D);
    \draw[thick] (K)  -- (C);
    \draw[very thick,col4] (R1) rectangle (R2) 
    node[yshift=-1.5em,xshift=0.75em] {$R$};
    \draw[draw=black,fill=white] (B) circle (2pt);
    \draw[draw=black,fill=white] (C) circle (2pt);
\end{tikzpicture}

\end{marginfigure}
%-------------------------------------------------------------------------------

%%%%%%%%%%%%%%%%%%%%%%%%%%%%%%%%%%%%%%%%%%%%%%%%%%%%%%%%%%%%%%%%%%%%%%%%%%%%%%%%
\question{Déterminer la transformée de Laplace $Q(p)$ de la charge au borne du 
condensateur $q(t)$.}
%%%%%%%%%%%%%%%%%%%%%%%%%%%%%%%%%%%%%%%%%%%%%%%%%%%%%%%%%%%%%%%%%%%%%%%%%%%%%%%%

%%%%%%%%%%%%%%%%%%%%%%%%%%%%%%%%%%%%%%%%%%%%%%%%%%%%%%%%%%%%%%%%%%%%%%%%%%%%%%%%
\question{Déterminer alors la fonction $q(t)$ solution de l'équation 
différentielle de la décharge d'un condensateur.}
%%%%%%%%%%%%%%%%%%%%%%%%%%%%%%%%%%%%%%%%%%%%%%%%%%%%%%%%%%%%%%%%%%%%%%%%%%%%%%%%


%%%%%%%%%%%%%%%%%%%%%%%%%%%%%%%%%%%%%%%%%%%%%%%%%%%%%%%%%%%%%%%%%%%%%%%%%%%%%%%%
%%%%%%%%%%%%%%%%%%%%%%%%%%%%%%%%%%%%%%%%%%%%%%%%%%%%%%%%%%%%%%%%%%%%%%%%%%%%%%%%
\exercice{Système masse-ressort~\difficile}
%%%%%%%%%%%%%%%%%%%%%%%%%%%%%%%%%%%%%%%%%%%%%%%%%%%%%%%%%%%%%%%%%%%%%%%%%%%%%%%%
%%%%%%%%%%%%%%%%%%%%%%%%%%%%%%%%%%%%%%%%%%%%%%%%%%%%%%%%%%%%%%%%%%%%%%%%%%%%%%%%

On reprend ici, la mise en équation du système mécanique masse-ressort 
présentée dans ce chapitre (c.f \cref{para-masse_ressort}). L'équation 
différentielle de la position de la masse est donnée par 
\[
m\devi{x(t)}{2}+b\devi{x(t)}{}+kx(t)=f(t)
\]
où $m$ est la masse, $b$ le coefficient d'amortissement visqueux et $k$ la
constante de raideur du ressort. Ce système peut être considéré comme
un système dont l'entrée est la force $f(t)$ et la sortie est la position
$x(t)$ de la masse.
%-------------------------------------------------------------------------------
\begin{marginfigure}
    \centering
    \tikzsetnextfilename{masse_ressort-chap_slci-ext}
    \input{tikz/masse_ressort-chap_slci.tex}
\end{marginfigure}
%-------------------------------------------------------------------------------
On se place dans le cas de conditions initiales nulles. 

%%%%%%%%%%%%%%%%%%%%%%%%%%%%%%%%%%%%%%%%%%%%%%%%%%%%%%%%%%%%%%%%%%%%%%%%%%%%%%%%
\question{Déterminer la transformée de Laplace $X(p)$ solution algébrique 
de la transformée de Laplace de l'équation différentielle de ce système.}
%%%%%%%%%%%%%%%%%%%%%%%%%%%%%%%%%%%%%%%%%%%%%%%%%%%%%%%%%%%%%%%%%%%%%%%%%%%%%%%%
La transformée de Laplace de l'équation différentielle nous donnes : 
\[
    X(p)\left(mp^2+bp+k\right)=F(p)
\]
ainsi on peut écrire $X(p)$ sous la forme :
\[
    X(p)=\dfrac{1}{mp^2+bp+k}F(p)
\]

%%%%%%%%%%%%%%%%%%%%%%%%%%%%%%%%%%%%%%%%%%%%%%%%%%%%%%%%%%%%%%%%%%%%%%%%%%%%%%%%
\question{Indentifier la fonction de transfert $H(p)$ du système et 
donner sa forme canonique}
%%%%%%%%%%%%%%%%%%%%%%%%%%%%%%%%%%%%%%%%%%%%%%%%%%%%%%%%%%%%%%%%%%%%%%%%%%%%%%%%
On identifie simplement la fonction de transfert du système comme la fraction 
rationnelle définie par le rapport $\dfrac{X(p)}{F(p)}$
\[
    H(p)=\dfrac{1}{mp^2+bp+k}
\]

Sous forme canonique, cette fonction de transfert s'écrit :
\[
    H(p)=\dfrac{1}{k\left(\dfrac{m}{k}p^2+\dfrac{b}{k}p+1\right)}
\]


On sollicite le système par une impulsion de Dirac. Dans les faits, on peut 
voir cette sollicitation comme l'application d'une force $f(t)$ constante 
d'intensité $\dfrac{1}{a}$ intense pendant un temps $a$ très court 
(i.e $a\to0$).    
%%%%%%%%%%%%%%%%%%%%%%%%%%%%%%%%%%%%%%%%%%%%%%%%%%%%%%%%%%%%%%%%%%%%%%%%%%%%%%%%
\question{Indentifier la solution $x(t)$ pour une cette solicitation.}
%%%%%%%%%%%%%%%%%%%%%%%%%%%%%%%%%%%%%%%%%%%%%%%%%%%%%%%%%%%%%%%%%%%%%%%%%%%%%%%%
La transformée de Laplace de cette sollicitation $F(p)$ est égale à 1 à 
la limite $a\to0$. On obtient alors pour la transformée de Laplace de la position
\[
    X(p)=\dfrac{1}{mp^2+bp+k}
\]
ou encore sous la forme
\[
    X(p)=\dfrac{1}{k}\dfrac{\dfrac{k}{m}}{\left(p^2+\dfrac{b}{m}p+\dfrac{k}{m}\right)}
\]
Pour obtenir $x(t)$, la position dans le domaine temporelle, il faut 
appliquer la transformée de Laplace inverse. On identifie pour $X(p)$, 
la forme de la ligne 31 de la table des transformées de Laplace 
(c.f \cref{annexe-lap}).

\[
\laplacei{\dfrac{\omega^2}{p^2+2\xi\omega p+\omega^2}}=
    \dfrac{\omega}{\sqrt{1-\xi^2}}
    e^{-\xi\omega t}\sin{\omega\sqrt{1-\xi^2} t}
\]

Identifions les paramètres, $\xi$ et $\omega$ 
\[
    \omega^2=\dfrac{k}{m}
\]
\[
    2\xi\omega=\dfrac{b}{m}
\]
\[
    \xi=\dfrac{b}{2\sqrt{mk}}
\]
\[
    \xi^2=\dfrac{b^2}{4mk}
\]
\[
    1-\xi^2=1-\dfrac{b^2}{4mk}
\]
\[
    \sqrt{1-\xi^2}=\sqrt{1-\dfrac{b^2}{4mk}}
\]
\[
    \xi\omega=\dfrac{b}{2m}
\]
\[
    \sin{\omega\sqrt{1-\xi^2}t}=\sin{\sqrt{\left(\dfrac{k}{m}-\dfrac{b^2}{4m^2}\right)t}}
\]
\[
    \dfrac{\omega}{\sqrt{1-\xi^2}}=\dfrac{\sqrt{k}}{\sqrt{m-\dfrac{b^2}{4k}}}
\]
\[
    \dfrac{1}{k}\dfrac{\omega}{\sqrt{1-\xi^2}}=\dfrac{1}{\sqrt{mk-\dfrac{b^2}{4}}}
\]
Dans les conditions pour lequelle $\xi<1$ 
La solution est donc de la forme :
\[
    x(t)=\dfrac{1}{k}\dfrac{\omega}{\sqrt{1-\xi^2}}                                                                                                     e^{-\xi\omega t}\sin{\omega\sqrt{1-\xi^2} t} 
\]
ou encore 
\[
    x(t)=\dfrac{1}{\sqrt{mk-\dfrac{b^2}{4}}}e^{-\dfrac{b}{2m}t}\sin{\sqrt{\left(\dfrac{k}{m}-\dfrac{b^2}{4m^2}\right)t}}
\]

