%%%%%%%%%%%%%%%%%%%%%%%%%%%%%%%%%%%%%%%%%%%%%%%%%%%%%%%%%%%%%%%%%%%%%%%%%%%%%%%%
%%%%%%%%%%%%%%%%%%%%%%%%%%%%%%%%%%%%%%%%%%%%%%%%%%%%%%%%%%%%%%%%%%%%%%%%%%%%%%%%
\exercice{Décomposition d'un signal}
%%%%%%%%%%%%%%%%%%%%%%%%%%%%%%%%%%%%%%%%%%%%%%%%%%%%%%%%%%%%%%%%%%%%%%%%%%%%%%%%
%%%%%%%%%%%%%%%%%%%%%%%%%%%%%%%%%%%%%%%%%%%%%%%%%%%%%%%%%%%%%%%%%%%%%%%%%%%%%%%%
%%%%%%%%%%%%%%%%%%%%%%%%%%%%%%%%%%%%%%%%%%%%%%%%%%%%%%%%%%%%%%%%%%%%%%%%%%%%%%%%
\question{\textbf{Décomposer $e_1(t)$ à l'aide de fonctions échelons.}}
%%%%%%%%%%%%%%%%%%%%%%%%%%%%%%%%%%%%%%%%%%%%%%%%%%%%%%%%%%%%%%%%%%%%%%%%%%%%%%%%
Le signal $e(t)$ est défini par :
\[
e_1(t)=\begin{cases}
    0&\quad t<0 \\
    t&\quad t\in[0,1]\\
    2-t&\quad t\in[1,3]\\
    t-4&\quad t\in[3,4]\\
    0&\quad t>4
     \end{cases}
\]
On peut alors décomposer $e_1(t)$ en somme d'échelon :
\[
    e_1(t)=t    \left(u(t)-u(t-1)  \right)+(2-t)\left(u(t-1)-u(t-3)\right)+
         (t-4)\left(u(t-3)-u(t-4)\right) 
\]
En regroupant les termes, on obtient alors :
\[
    e_1(t)=tu(t)+2(1-t)u(t-1)+2(t-3)u(t-3)-(t-4)u(t-4)
\]
%%%%%%%%%%%%%%%%%%%%%%%%%%%%%%%%%%%%%%%%%%%%%%%%%%%%%%%%%%%%%%%%%%%%%%%%%%%%%%%%
\question{\textbf{Donner la transformée de Laplace $E_1(p)$ de ce signal.}}
%%%%%%%%%%%%%%%%%%%%%%%%%%%%%%%%%%%%%%%%%%%%%%%%%%%%%%%%%%%%%%%%%%%%%%%%%%%%%%%%
La transformée de Laplace $E_1(p)$ est simplement donnée par la transformée de
chacun des termes.
\[
    E_1(p)=\dfrac{1}{p^2}+\dfrac{2e^{-p}}{p^2}\left(p-1\right)
                       +\dfrac{2e^{-3p}}{p^2}\left(1-3p\right)
                       -\dfrac{e^{-4p}}{p^2}\left(1-4p\right)
\]
ou encore
\[
    E_1(p)=\dfrac{1}{p^2}
    \left(1+2e^{-p}(p-1)+2e^{-3p}(1-3p)-e^{-4p}(1-4p)\right)
\]
%%%%%%%%%%%%%%%%%%%%%%%%%%%%%%%%%%%%%%%%%%%%%%%%%%%%%%%%%%%%%%%%%%%%%%%%%%%%%%%%
%%%%%%%%%%%%%%%%%%%%%%%%%%%%%%%%%%%%%%%%%%%%%%%%%%%%%%%%%%%%%%%%%%%%%%%%%%%%%%%%
\exercice{Décomposition d'un signal parabolique}
%%%%%%%%%%%%%%%%%%%%%%%%%%%%%%%%%%%%%%%%%%%%%%%%%%%%%%%%%%%%%%%%%%%%%%%%%%%%%%%%
%%%%%%%%%%%%%%%%%%%%%%%%%%%%%%%%%%%%%%%%%%%%%%%%%%%%%%%%%%%%%%%%%%%%%%%%%%%%%%%%
%%%%%%%%%%%%%%%%%%%%%%%%%%%%%%%%%%%%%%%%%%%%%%%%%%%%%%%%%%%%%%%%%%%%%%%%%%%%%%%%
\question{\textbf{Décomposer $e_2(t)$ à l'aide de fonctions échelons.}}
%%%%%%%%%%%%%%%%%%%%%%%%%%%%%%%%%%%%%%%%%%%%%%%%%%%%%%%%%%%%%%%%%%%%%%%%%%%%%%%%
Le signal est composé de deux paraboles. 
%-------------------------------------------------------------------------------
\begin{figure}[!h]
    \centering
    \tikzsetnextfilename{fonction_e2-exercices_corriges-chap_slci-ext}
    \input{tikz/fonction_e2-exercices_corriges-chap_slci.tex}
\end{figure}
%-------------------------------------------------------------------------------
\[
e_2(t)=
\begin{cases}
    0&\quad t<0\\
    t^2&\quad t\in[0,1]\\
    (t-2)^2&\quad t\in[1,2]\\
    0&\quad t>2
\end{cases}
\]
\`A l'aide des fonctions échelons le signal s'écrit alors 
\[
    e_2(t)=t^2\left(u(t)-u(t-1)\right)+(t-2)^2\left(u(t-1)-u(t-2)\right).
\]
En regroupant les termes on peut réecrire cette expression sous la forme :
\[
    e_2(t)=t^2u(t)-4(t-1)u(t-1)-t^2u(t-2)+4(t-1)u(t-2)
\]
%%%%%%%%%%%%%%%%%%%%%%%%%%%%%%%%%%%%%%%%%%%%%%%%%%%%%%%%%%%%%%%%%%%%%%%%%%%%%%%%
\question{\textbf{Donner la transformée de Laplace $E_2(p)$ de ce signal.}}
%%%%%%%%%%%%%%%%%%%%%%%%%%%%%%%%%%%%%%%%%%%%%%%%%%%%%%%%%%%%%%%%%%%%%%%%%%%%%%%%
On rapelle que la transformé de Laplace de la fonction $f(t)=t^2$ 
est $F(p)=\dfrac{2}{p^3}$. On obtient alors pour la transformée 
$E_2(p)$ de $e_2(t)$ :
\[
    E_2(p)=\dfrac{2}{p^3}-4e^{-p}\left(\dfrac{1}{p^2}-\dfrac{1}{p}\right)
        -e^{-2p}\dfrac{2}{p^3}+4e^{-2p}\left(\dfrac{1}{p^2}-\dfrac{1}{p}\right)
\]
\[
    E_2(p)=\dfrac{2}{p^3}\left(1-e^{-2p}\right)
        -4\left(\dfrac{1}{p^2}-\dfrac{1}{p}\right)\left(e^{-p}-e^{-2p}\right)
\]
%%%%%%%%%%%%%%%%%%%%%%%%%%%%%%%%%%%%%%%%%%%%%%%%%%%%%%%%%%%%%%%%%%%%%%%%%%%%%%%%
%%%%%%%%%%%%%%%%%%%%%%%%%%%%%%%%%%%%%%%%%%%%%%%%%%%%%%%%%%%%%%%%%%%%%%%%%%%%%%%%
\exercice{Décompositon d'un signal et transformée de Laplace}
%%%%%%%%%%%%%%%%%%%%%%%%%%%%%%%%%%%%%%%%%%%%%%%%%%%%%%%%%%%%%%%%%%%%%%%%%%%%%%%%
%%%%%%%%%%%%%%%%%%%%%%%%%%%%%%%%%%%%%%%%%%%%%%%%%%%%%%%%%%%%%%%%%%%%%%%%%%%%%%%%
%%%%%%%%%%%%%%%%%%%%%%%%%%%%%%%%%%%%%%%%%%%%%%%%%%%%%%%%%%%%%%%%%%%%%%%%%%%%%%%%
\question{\textbf{Décomposer $e_3(t)$ et $e_4(t)$ à l'aide de la fonction 
échelon unitaire (c.-à-d. $u(t)$).}}
%%%%%%%%%%%%%%%%%%%%%%%%%%%%%%%%%%%%%%%%%%%%%%%%%%%%%%%%%%%%%%%%%%%%%%%%%%%%%%%%
Décomposons $e_3(t)$ et $e_4(t)$ par morceaux. On a alors :
\[
e_3(t)=
\begin{cases}
    0&\quad t<0\\
    1-t&\quad t\in[0,1]\\
    0&\quad t>1
\end{cases}
\]
et
\[
e_4(t)=
\begin{cases}
    0&\quad t<0\\
    t&\quad t\in[0,1]\\
    0&\quad t>1
\end{cases}
\]
On peut alors écrire ces signaux comme la somme des signaux usuels comme:
\[
    e_3(t)=(1-t)\left(u(t)-u(t-1)\right)=u(t)-r(t)-u(t-1)+r(t-1)
\]
et
\[
    e_4(t)=t\left(u(t)-u(t-1)\right)=r(t)-r(t-1)
\]
où $u(t)$ est la fonction échelon et $r(t)$ la fonction rampe.
%%%%%%%%%%%%%%%%%%%%%%%%%%%%%%%%%%%%%%%%%%%%%%%%%%%%%%%%%%%%%%%%%%%%%%%%%%%%%%%%
\question{\textbf{Déterminer les fonctions $E_3(p)$ et $E_4(p)$ les 
transformées de Laplace respectives de $e_3(t)$ et $e_4(t)$.}}
%%%%%%%%%%%%%%%%%%%%%%%%%%%%%%%%%%%%%%%%%%%%%%%%%%%%%%%%%%%%%%%%%%%%%%%%%%%%%%%%
Les transformées sont simplement données par la propriété des fonctions 
retardés et des signaux rampes. On a notamment,
\[
    E_3(p)=\dfrac{(p-1)}{p^2}\left(1-e^{-p}\right)
\]
et
\[
    E_3(p)=\dfrac{1}{p^2}\left(1-e^{-p}\right)
\]
%%%%%%%%%%%%%%%%%%%%%%%%%%%%%%%%%%%%%%%%%%%%%%%%%%%%%%%%%%%%%%%%%%%%%%%%%%%%%%%%
%%%%%%%%%%%%%%%%%%%%%%%%%%%%%%%%%%%%%%%%%%%%%%%%%%%%%%%%%%%%%%%%%%%%%%%%%%%%%%%%
%Pour une amplitude maximale de $E_0$ et une durée de signal $\tau$, les
%signaux temporels deviennent :
%\[
%    e_3(t)=\dfrac{E_0}{\tau}(1-t)\left(u(t)-u(t-\tau)\right)
%\]
%\[
%    e_4(t)=\dfrac{E_0}{\tau}t\left(u(t)-u(t-\tau)\right)
%\]
%%%%%%%%%%%%%%%%%%%%%%%%%%%%%%%%%%%%%%%%%%%%%%%%%%%%%%%%%%%%%%%%%%%%%%%%%%%%%%%%
\question{\textbf{Déterminer à partir des résultats précédents la transformée 
de Laplace $E(p)$ correspondant à la somme des signaux $e_3(t)$ et $e_4(t-1)$.}}
%%%%%%%%%%%%%%%%%%%%%%%%%%%%%%%%%%%%%%%%%%%%%%%%%%%%%%%%%%%%%%%%%%%%%%%%%%%%%%%%
On sollicite le système par le signal $e(t)$ défini par :
\[
    e(t)=e_3(t)+e_4(t-1)
\]
La transformée de Laplace $E(p)$ est simplement donnée par la somme,
\[
    E(p)=E_3(p)+e^{-p} E_4(p)
        =\dfrac{(p-1)}{p^2}\left(1-e^{-p}\right)+
         \dfrac{e^{-p}}{p^2}\left(1-e^{-p}\right)
        =\dfrac{1-e^{-p}}{p^2}\left(e^{-p}+p-1\right)
\]
%%%%%%%%%%%%%%%%%%%%%%%%%%%%%%%%%%%%%%%%%%%%%%%%%%%%%%%%%%%%%%%%%%%%%%%%%%%%%%%%
%%%%%%%%%%%%%%%%%%%%%%%%%%%%%%%%%%%%%%%%%%%%%%%%%%%%%%%%%%%%%%%%%%%%%%%%%%%%%%%%
\exercice{L'impulsion de Dirac approchée}
%%%%%%%%%%%%%%%%%%%%%%%%%%%%%%%%%%%%%%%%%%%%%%%%%%%%%%%%%%%%%%%%%%%%%%%%%%%%%%%%
%%%%%%%%%%%%%%%%%%%%%%%%%%%%%%%%%%%%%%%%%%%%%%%%%%%%%%%%%%%%%%%%%%%%%%%%%%%%%%%%
%%%%%%%%%%%%%%%%%%%%%%%%%%%%%%%%%%%%%%%%%%%%%%%%%%%%%%%%%%%%%%%%%%%%%%%%%%%%%%%%
\question{\textbf{Décomposer $\delta_a(t)$ à l'aide de fonctions échelons.}}
%%%%%%%%%%%%%%%%%%%%%%%%%%%%%%%%%%%%%%%%%%%%%%%%%%%%%%%%%%%%%%%%%%%%%%%%%%%%%%%%
%-------------------------------------------------------------------------------
\begin{figure}[!h]
    \centering
    \tikzsetnextfilename
    {fonction_porte_decomposition-exercices_corriges-chap_slci-ext}
    \input{tikz/fonction_porte_decomposition-exercices_corriges-chap_slci.tex}
    \tikzsetnextfilename
    {fonction_porte_decomposition2-exercices_corriges-chap_slci-ext}
    \begin{tikzpicture}
   \begin{axis}[
   height=4cm,
   width=5cm,
      axis x line=center,
        axis y line=center,
        xmin=-1,
        xmax=5,
        ymin=-1.5,
        ymax=2.0,
        xlabel={$t$},
        ylabel={$\delta_a(t)$},
        xlabel style={below right},
        ylabel style={left},
        yticklabels={\small$\dfrac{1}{a}$},
        ytick={1},
        y tick label style={anchor=east},
        xticklabels={$a$},
        xtick={2},
        x tick label style={below left},
        ]
        \addplot [very thick,color=col4,const plot]   
        coordinates { (-1, 0.01) (0, 0.01) (0 ,1)  (5,1) };
        \addplot [very thick,color=col3,const plot] 
        coordinates { (-1,-0.01) (0,-0.01) (2,-1) (5,-1) };
        \end{axis}
\end{tikzpicture}

\end{figure}
%-------------------------------------------------------------------------------
La fonction $\delta_a(t)$ est simplement donnée par 
\[
    \color{col1}p_a(t)
    =\color{col4}\dfrac{1}{a}u(t)\color{col3}-\dfrac{1}{a}u(t-a)
    =\normalcolor\dfrac{1}{a}\left(u(t)-u(t-a)\right)
\] 
où $u(t)$ est la fonction échelon unité.
%%%%%%%%%%%%%%%%%%%%%%%%%%%%%%%%%%%%%%%%%%%%%%%%%%%%%%%%%%%%%%%%%%%%%%%%%%%%%%%%
\question{\textbf{Donner sa transformée de Laplace en fonction du 
paramètre $a$.}}
%%%%%%%%%%%%%%%%%%%%%%%%%%%%%%%%%%%%%%%%%%%%%%%%%%%%%%%%%%%%%%%%%%%%%%%%%%%%%%%%
On rappel la transformée de Laplace d'une fonction retardée :
\[
    \laplace{f(t-\tau)} = e^{-\tau p}F(p)
\]
La transformée de Laplace de $\delta_a(t)$ est donc donnée par : 
\[
    \Delta_a(p) = \laplace{\delta_a(t)}
                = \dfrac{1}{a}\left(\laplace{u(t)}-\laplace{u(t-a)}\right)
                = \dfrac{1}{ap}(1-e^{-ap})
\]
%%%%%%%%%%%%%%%%%%%%%%%%%%%%%%%%%%%%%%%%%%%%%%%%%%%%%%%%%%%%%%%%%%%%%%%%%%%%%%%%
\question{\textbf{En considérant la limite $a\rightarrow0$, montrer que la 
transformée de Laplace de $\delta_a(t)$ tend bien vers 1.}}
%%%%%%%%%%%%%%%%%%%%%%%%%%%%%%%%%%%%%%%%%%%%%%%%%%%%%%%%%%%%%%%%%%%%%%%%%%%%%%%%
Le développement limité d'ordre 1 de $e^{x}=1+x+o\left(x\right)$ permet de
déterminer la limite pour $a\rightarrow0$ de la transformée de Laplace de 
$\delta_a(t)$. En effet, 
\[
    \lim_{a\to0} \Delta_a(p) = \dfrac{1}{ap}(1-1+ap)=1.
\]
Puisque la transformée de Laplace d'une impulsion de Dirac est égale à 1, on a
bien vérifié que $\lim_{a\to0} \delta_a(p)=\delta(t)$ où $\delta(t)$ est 
l'impulsion de Dirac.
\clearpage
%%%%%%%%%%%%%%%%%%%%%%%%%%%%%%%%%%%%%%%%%%%%%%%%%%%%%%%%%%%%%%%%%%%%%%%%%%%%%%%%
%%%%%%%%%%%%%%%%%%%%%%%%%%%%%%%%%%%%%%%%%%%%%%%%%%%%%%%%%%%%%%%%%%%%%%%%%%%%%%%%
\exercice{Décharge d'un condensateur}
%%%%%%%%%%%%%%%%%%%%%%%%%%%%%%%%%%%%%%%%%%%%%%%%%%%%%%%%%%%%%%%%%%%%%%%%%%%%%%%%
%%%%%%%%%%%%%%%%%%%%%%%%%%%%%%%%%%%%%%%%%%%%%%%%%%%%%%%%%%%%%%%%%%%%%%%%%%%%%%%%
Dans cet exercice nous allons appliquer la méthode \fg classique\og pour la
résolution d'une équation différentielle à coefficients constants.

%%%%%%%%%%%%%%%%%%%%%%%%%%%%%%%%%%%%%%%%%%%%%%%%%%%%%%%%%%%%%%%%%%%%%%%%%%%%%%%%
\question{\textbf{Déterminer la transformée de Laplace $Q(p)$ de la charge au 
borne du condensateur $q(t)$.}}
%%%%%%%%%%%%%%%%%%%%%%%%%%%%%%%%%%%%%%%%%%%%%%%%%%%%%%%%%%%%%%%%%%%%%%%%%%%%%%%%
La transformée de Laplace de l'équation différentielle de la décharge d'un
condensateur est donnée par :
\[
    \tau(pQ(p)-q_0)+Q(p)=0
\]
où $q_0$ est la condition initialle de $q(t)$.
La résolution algébrique de cette équation nous donne 
\[
    Q(p)(\tau p+1)=q_0\tau
\]
ou encore :
\[
    Q(p)=q_0\dfrac{\tau}{\tau p+1}
\]
Remarquons que cette solution ne dépend que des conditions initiales, il n'est 
pas possible de définir une fonction de transfert dans ce cas particulier.

%%%%%%%%%%%%%%%%%%%%%%%%%%%%%%%%%%%%%%%%%%%%%%%%%%%%%%%%%%%%%%%%%%%%%%%%%%%%%%%%
\question{\textbf{Déterminer alors la fonction $q(t)$ solution de l'équation 
différentielle de la décharge d'un condensateur.}}
%%%%%%%%%%%%%%%%%%%%%%%%%%%%%%%%%%%%%%%%%%%%%%%%%%%%%%%%%%%%%%%%%%%%%%%%%%%%%%%%
Pour déterminer la solution $q(t)$ de l'équation différentielle, il suffit de 
déterminer la transformée inverse de $Q(p)$.

La transformée inverse de la fonction $F(p)=\dfrac{1}{p+a}$ est $f(t)=e^{-a}$.
Donc sous la forme,
\[
    Q(p)=q_0\dfrac{1}{p+\dfrac{1}{\tau}},
\]
la transformée inverse de $Q(p)$ est simplement donnée par 
\[
    q(t)=q_0e^{-\dfrac{t}{\tau}}
\]
%-------------------------------------------------------------------------------
\begin{center}
    \tikzsetnextfilename{laplace_schema_decharge_condensateur-chap_slci-ext}
    \input{tikz/laplace_schema_decharge_condensateur.tex}
\end{center}
%-------------------------------------------------------------------------------
La~\cref{fig-decharge} suivante présente la représentation graphique de 
cette solution pour $\tau=\SI{1}{\second}$.
%-------------------------------------------------------------------------------
\begin{figure}[!h]
    \centering
    \tikzsetnextfilename{decharge_condensateur-exercices_corriges-chap_slci-ext}
    \input{tikz/decharge_condensateur-exercices_corriges-chap_slci.tex}
    \caption{Charge aux bornes du condensateur en fonction du temps.
    \label{fig-decharge}}
\end{figure}
%-------------------------------------------------------------------------------
\clearpage
%%%%%%%%%%%%%%%%%%%%%%%%%%%%%%%%%%%%%%%%%%%%%%%%%%%%%%%%%%%%%%%%%%%%%%%%%%%%%%%%
%%%%%%%%%%%%%%%%%%%%%%%%%%%%%%%%%%%%%%%%%%%%%%%%%%%%%%%%%%%%%%%%%%%%%%%%%%%%%%%%
\exercice{Système masse-ressort}
%%%%%%%%%%%%%%%%%%%%%%%%%%%%%%%%%%%%%%%%%%%%%%%%%%%%%%%%%%%%%%%%%%%%%%%%%%%%%%%%
%%%%%%%%%%%%%%%%%%%%%%%%%%%%%%%%%%%%%%%%%%%%%%%%%%%%%%%%%%%%%%%%%%%%%%%%%%%%%%%%
%%%%%%%%%%%%%%%%%%%%%%%%%%%%%%%%%%%%%%%%%%%%%%%%%%%%%%%%%%%%%%%%%%%%%%%%%%%%%%%%
\question{\textbf{Déterminer la transformée de Laplace $Z(p)$ solution 
algébrique de la transformée de Laplace de l'équation différentielle de 
ce système.}}
%%%%%%%%%%%%%%%%%%%%%%%%%%%%%%%%%%%%%%%%%%%%%%%%%%%%%%%%%%%%%%%%%%%%%%%%%%%%%%%%
La transformée de Laplace de l'équation différentielle nous donne : 
\[
    Z(p)\left(mp^2+bp+k\right)=F(p)
\]
ainsi on peut écrire $X(p)$ sous la forme :
\[
    Z(p)=\dfrac{1}{mp^2+bp+k}F(p)
\]
%%%%%%%%%%%%%%%%%%%%%%%%%%%%%%%%%%%%%%%%%%%%%%%%%%%%%%%%%%%%%%%%%%%%%%%%%%%%%%%%
\question{\textbf{Indentifier la fonction de transfert $H(p)$ du système et 
donner sa forme canonique}}
%%%%%%%%%%%%%%%%%%%%%%%%%%%%%%%%%%%%%%%%%%%%%%%%%%%%%%%%%%%%%%%%%%%%%%%%%%%%%%%%
On identifie simplement la fonction de transfert du système comme la fraction 
rationnelle définie par le rapport $\dfrac{Z(p)}{F(p)}$
\[
    H(p)=\dfrac{1}{mp^2+bp+k}
\]

Sous forme canonique, cette fonction de transfert s'écrit :
\[
    H(p)=\dfrac{1}{k\left(\dfrac{m}{k}p^2+\dfrac{b}{k}p+1\right)}
\]
On sollicite le système par une impulsion de Dirac. Dans les faits, on peut 
voir cette sollicitation comme l'application d'une force $f(t)$ constante 
d'intensité $\dfrac{1}{a}$ intense pendant un temps $a$ très court 
(c.-à-d. $a\to0$).    

%%%%%%%%%%%%%%%%%%%%%%%%%%%%%%%%%%%%%%%%%%%%%%%%%%%%%%%%%%%%%%%%%%%%%%%%%%%%%%%%
\question{\textbf{Indentifier la solution algébrique $Z(p)$ pour une cette 
solicitation.}}
%%%%%%%%%%%%%%%%%%%%%%%%%%%%%%%%%%%%%%%%%%%%%%%%%%%%%%%%%%%%%%%%%%%%%%%%%%%%%%%%
La transformée de Laplace de cette sollicitation $F(p)$ est égale à 1 à 
la limite $a\to0$. On obtient alors pour la transformée de Laplace de la 
position :
\[
    Z(p)=\dfrac{1}{mp^2+bp+k}
\]
ou encore en utilisant la forme canonique de la fonction de transfert :
\[
    Z(p)=\dfrac{1}{k\left(\dfrac{m}{k}p^2+\dfrac{b}{k}p+1\right)}
\]
%%%%%%%%%%%%%%%%%%%%%%%%%%%%%%%%%%%%%%%%%%%%%%%%%%%%%%%%%%%%%%%%%%%%%%%%%%%%%%%%
\question{\textbf{Identifier les paramètres $K$, $\xi$ et $\omega$ en 
fonction des paramètres du problème.}}
%%%%%%%%%%%%%%%%%%%%%%%%%%%%%%%%%%%%%%%%%%%%%%%%%%%%%%%%%%%%%%%%%%%%%%%%%%%%%%%%
\[
    \begin{cases}
        K=\dfrac{1}{k}\\[1.5em]
        \omega=\left(\dfrac{k}{m}\right)^{\frac{1}{2}}\\[1.5em]
        \xi=\dfrac{b}{2k}\omega
    \end{cases}
\]
%%%%%%%%%%%%%%%%%%%%%%%%%%%%%%%%%%%%%%%%%%%%%%%%%%%%%%%%%%%%%%%%%%%%%%%%%%%%%%%%
\question{\textbf{Déterminer $z(t)$ la solution temporelle de l'équation 
différentielle pour une telle sollicitation.}}
%%%%%%%%%%%%%%%%%%%%%%%%%%%%%%%%%%%%%%%%%%%%%%%%%%%%%%%%%%%%%%%%%%%%%%%%%%%%%%%%
Dans les conditions pour lesquels les paramètres $m, b ,k$ donne $\xi<1$ 
La solution est donc de la forme :
\[
    z(t)=K\dfrac{\omega}{\sqrt{1-\xi^2}}
         e^{-\xi\omega t}\sin{\omega\sqrt{1-\xi^2} t} 
\]
ou encore 
\[
    z(t)=\dfrac{1}{m\left(\dfrac{k}{m}-\dfrac{b^2}{4m^2}\right)^{\frac{1}{2}}}
         e^{\frac{-b}{2m}t}\sin{\left(\dfrac{k}{m}-\dfrac{b^2}{4m^2}\right)^
         {\frac{1}{2}}t}
\]

On réalise une experience avec une masse $m=\SI{0.5}{\kilogram}$, un ressort 
de constante de raideur $k=\SI{10}{\newton\per\meter}$ et un amortissement 
visqueux $b=\SI{0.894}{\kilogram\per\second}$
%%%%%%%%%%%%%%%%%%%%%%%%%%%%%%%%%%%%%%%%%%%%%%%%%%%%%%%%%%%%%%%%%%%%%%%%%%%%%%%%
\question{\textbf{Déterminer les valeurs numériques des paramètres 
$\xi$ et $\omega$}}
%%%%%%%%%%%%%%%%%%%%%%%%%%%%%%%%%%%%%%%%%%%%%%%%%%%%%%%%%%%%%%%%%%%%%%%%%%%%%%%%
\[
    \omega=\SI{4.47}{\radian\per\second}
\]
\[
    \xi=0.2
\]
%-------------------------------------------------------------------------------
\begin{figure}[!h]
    \centering
    \tikzsetnextfilename{masse_ressort_solution-chap_slci-ext}
    \pgfmathdeclarefunction{func}{2}{%
    \pgfmathparse{%
    ((#2/sqrt(1-#1*#1))*exp(-#1*#2*x)*
    sin(#2*deg(x)*sqrt(1-#1*#1))))
    }%
}
\begin{tikzpicture}
    \begin{axis}
    [
        height=6cm,
        width=9cm,
        axis x line=center,
        axis y line=center,
        xmin=-1,
        xmax=6,
        ymin=-0.25,
        ymax=0.5,
        xlabel={$t$},
        ylabel={$z(t)$ [\si{\centi\meter}]},
        xlabel style={below right},
        ylabel style={above left},
        yticklabels={-20,0,20,40},
        ytick={-0.2,0,0.2,0.4},
        y tick label style={anchor=east},
        xticklabels={0,2,4,6,8},
        xtick={0,2,4,6,8},
        minor xtick={-1,1,3,5,7},
        minor ytick={-0.3,-0.1,0.1,0.3},
        x tick label style={below},
        grid=both,
    ]
        \addplot[signalb,domain=-5:0]{0};
        \addplot[signalb,domain=0:6,samples=200]{0.1*func(0.2,4.47)};
    \end{axis}
\end{tikzpicture}

    \caption{Réponse temporelle du système masse ressort
    \label{fig-masse_ressort_reptemp}}
\end{figure}
%-------------------------------------------------------------------------------
%%%%%%%%%%%%%%%%%%%%%%%%%%%%%%%%%%%%%%%%%%%%%%%%%%%%%%%%%%%%%%%%%%%%%%%%%%%%%%%%
\question{\textbf{Tracer la réponse $z(t)$ et déterminer le temps pour que la 
masse retourne à son position d'équilibre (à 5\% de son déplacement maximal)}}
%%%%%%%%%%%%%%%%%%%%%%%%%%%%%%%%%%%%%%%%%%%%%%%%%%%%%%%%%%%%%%%%%%%%%%%%%%%%%%%%
Le déplacement maximal est d'environ \SI{35}{\centi\meter}. On 
constate~\cref{fig-masse_ressort_reptemp} que la masse retourne 
à sa position d'équilibre (à \SI{1.75}{\centi\meter} près) après 
approximativement \SI{4}{\second}.
\clearpage
%%%%%%%%%%%%%%%%%%%%%%%%%%%%%%%%%%%%%%%%%%%%%%%%%%%%%%%%%%%%%%%%%%%%%%%%%%%%%%%%
%%%%%%%%%%%%%%%%%%%%%%%%%%%%%%%%%%%%%%%%%%%%%%%%%%%%%%%%%%%%%%%%%%%%%%%%%%%%%%%%
\exercice{Forme canonique d'une fonction de transfert du premier ordre}
%%%%%%%%%%%%%%%%%%%%%%%%%%%%%%%%%%%%%%%%%%%%%%%%%%%%%%%%%%%%%%%%%%%%%%%%%%%%%%%%
%%%%%%%%%%%%%%%%%%%%%%%%%%%%%%%%%%%%%%%%%%%%%%%%%%%%%%%%%%%%%%%%%%%%%%%%%%%%%%%%
%%%%%%%%%%%%%%%%%%%%%%%%%%%%%%%%%%%%%%%%%%%%%%%%%%%%%%%%%%%%%%%%%%%%%%%%%%%%%%%%
\question{\textbf{Déterminer la forme canonique de fonction de transfert 
$H(p)$ associée à cette équation différentielle}}
%%%%%%%%%%%%%%%%%%%%%%%%%%%%%%%%%%%%%%%%%%%%%%%%%%%%%%%%%%%%%%%%%%%%%%%%%%%%%%%%
La transformée de Laplace de l'équation différentielle nous amène à définir
la fonction de transfert :
\[
    H(p)= \dfrac{S(p)}{E(p)}=\dfrac{b_0}{a_1p+a_0}
\]
sous forme canonique elle s'écrit :
\[
    H(p)=\dfrac{b_0}{a_0\left(\dfrac{a_1}{a_0}p+1\right)}
\]
%%%%%%%%%%%%%%%%%%%%%%%%%%%%%%%%%%%%%%%%%%%%%%%%%%%%%%%%%%%%%%%%%%%%%%%%%%%%%%%%
\question{\textbf{Déterminer les paramètres $K$ et $\tau$ en fonction des 
coefficients $a_i$ et $b_i$ de tels sorte que $H(p)$ s'écrivent sous la forme}}
%%%%%%%%%%%%%%%%%%%%%%%%%%%%%%%%%%%%%%%%%%%%%%%%%%%%%%%%%%%%%%%%%%%%%%%%%%%%%%%%
Par identification, les paramètres $K$ et $\tau$ sont données par 
\[
    \begin{cases}
        K&=\dfrac{b_0}{a_0}\\[1.5em]
     \tau&=\dfrac{a_1}{a_0}
    \end{cases}
\]
%%%%%%%%%%%%%%%%%%%%%%%%%%%%%%%%%%%%%%%%%%%%%%%%%%%%%%%%%%%%%%%%%%%%%%%%%%%%%%%%
\question{\textbf{Déterminer les dimensions de ces paramètres quelques soient 
celles des signaux d'entrée $e(t)$ et de sortie $s(t)$}}
%%%%%%%%%%%%%%%%%%%%%%%%%%%%%%%%%%%%%%%%%%%%%%%%%%%%%%%%%%%%%%%%%%%%%%%%%%%%%%%%
L'étude dimensionnelle de l'équation différentielle nous permet d'identifier
les relations suivantes :
\[
    [a_1][s(t)] T^{-1} = [a_0][s(t)] = [b_0][e(t)].
\]
Ainsi $\tau$ a toujours la dimension d'un temps quelque soit la dimension des 
signaux d'entrée  $e(t)$ et de sortie $s(t)$.
\[
    [\tau] = \dfrac{[a_1]}{[a_0]}= T.
\]
La dimension du paramètre $K$ dépend elle du rapport des dimensions de ces 
signaux. En effet,
\[
    [K]=\dfrac{[b_0]}{[a_0]}=\dfrac{[s(t)]}{[e(t)]}
\]
On peut dire que la dimension de la fonction de transfert est donnée par 
le paramètre $K$.

%%%%%%%%%%%%%%%%%%%%%%%%%%%%%%%%%%%%%%%%%%%%%%%%%%%%%%%%%%%%%%%%%%%%%%%%%%%%%%%%
%%%%%%%%%%%%%%%%%%%%%%%%%%%%%%%%%%%%%%%%%%%%%%%%%%%%%%%%%%%%%%%%%%%%%%%%%%%%%%%%
\exercice{Carte des pôles d'une fonction de transfert}
%%%%%%%%%%%%%%%%%%%%%%%%%%%%%%%%%%%%%%%%%%%%%%%%%%%%%%%%%%%%%%%%%%%%%%%%%%%%%%%%
%%%%%%%%%%%%%%%%%%%%%%%%%%%%%%%%%%%%%%%%%%%%%%%%%%%%%%%%%%%%%%%%%%%%%%%%%%%%%%%%
%%%%%%%%%%%%%%%%%%%%%%%%%%%%%%%%%%%%%%%%%%%%%%%%%%%%%%%%%%%%%%%%%%%%%%%%%%%%%%%%
\question{\textbf{Donner les zéros et les pôles de la fonction de transfert.}}
%%%%%%%%%%%%%%%%%%%%%%%%%%%%%%%%%%%%%%%%%%%%%%%%%%%%%%%%%%%%%%%%%%%%%%%%%%%%%%%%
D'après la carte des pôles et zéros, le système est composé de trois zéros
$p_1=-3$, $p_2=-1+2j$ et $p_3=-1-2j$ et de deux zéros $z_1=-5+j$ et $z_2=-5-j$.

%%%%%%%%%%%%%%%%%%%%%%%%%%%%%%%%%%%%%%%%%%%%%%%%%%%%%%%%%%%%%%%%%%%%%%%%%%%%%%%%
\question{\textbf{\'Ecrire la fonction de transfert $H(p)$ sachant que le gain 
de sa forme factorisée vaut 3.}}
%%%%%%%%%%%%%%%%%%%%%%%%%%%%%%%%%%%%%%%%%%%%%%%%%%%%%%%%%%%%%%%%%%%%%%%%%%%%%%%%
La forme factorisée de la fonction de transfert est alors donnée par 
\[
    H(p)=3\dfrac{(p-z_1)(p-z_2)}{(p-p_1)(p-p_2)(p-p_3)}
\]
ou encore en développant les différents termes,
\[
    H(p)=3\dfrac{(p+5+j)(p+5-j)}{(p+3)(p+1+2j)(p+1-2j)}
        %=3\dfrac{p^2+10p+26}{(p+3)(p^2+2p+5)}
        =3\dfrac{p^2+10p+26}{p^3+5p^2+11p+15}
\]
%%%%%%%%%%%%%%%%%%%%%%%%%%%%%%%%%%%%%%%%%%%%%%%%%%%%%%%%%%%%%%%%%%%%%%%%%%%%%%%%
\question{\textbf{Déterminer la forme canonique de $H(p)$ et déduire le gain 
statique, la classe et l'ordre de ce système.}}
%%%%%%%%%%%%%%%%%%%%%%%%%%%%%%%%%%%%%%%%%%%%%%%%%%%%%%%%%%%%%%%%%%%%%%%%%%%%%%%%
La forme canonique de $H(p)$ est donnée par 
\[
    H(p)=\dfrac{26}{5}\dfrac{\frac{1}{26}p^2+\frac{10}{26}p+1}
                            {\frac{1}{15}p^3+\frac{1}{3}p^2+\frac{11}{15}p+1}
\]
On alors à faire à un système d'ordre 3 de classe 0 et de gain statique 
de valeur 5.2. 
Notons que dans le domaine de l'automatique les fractions sont communément
données sous forme décimales. On écrira donc plus facilement cette fonction
de transfert sous la forme :
\[
    H(p)=5.2\dfrac{0.0385p^2+0.385p+1}{0.066p^3+0.33p^2+0.733p+1}
\]
Aussi sous cette forme, la question de la précision des valeurs décimales 
peut se poser. 
