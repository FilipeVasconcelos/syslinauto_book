\chapter*{Avant-propos}
%\addcontentsline{toc}{chapter}{Introduction}
\addstarredchapter{Avant-propos}

Ce cours est une introduction à l'\textbf{automatique}
pour des étudiants de deuxième année de classe préparatoire scientifique.
Il repose sur une structure sous-jacente formée de 
trois piliers : la \textbf{modélisation}, l'\textbf{analyse} et 
le \textbf{contrôle}. Les chapitres suivent un découpage classique
autour de ces trois problématiques de l'automatique (c.f~\Cref{fig-diagramme_cours}).

L'objectif principal de l'automatique est de permettre
le contrôle des \textbf{systèmes dynamiques}
de toutes natures que ce soient: mécanique, chimique, 
électronique, optique, thermique, acoustique\ldots.
Ce contrôle doit dans la pluspart des cas respecter
certaines contraintes en termes de performances (rapidité, précision, stabilité\ldots).

Les systèmes dynamiques de ce cours sont limités 
aux \textbf{systèmes linéaires continus et invariants.}
La \textbf{modélisation} de ces systèmes passe par 
la mise en équation du comportement physique des
systèmes sous forme d'équations différentielles.
Cette étape ne fait pas à proprement parler partie d'un cours d'automatique,
en effet chacunes des disciplines construisent cette modélisation 
en se basant sur les principes et les hypothèses les plus adaptés 
à un problème donné.
La modélisation permet une étude systématique des équations
différentielles en proposant des modèles généraux 
et ce quelque soit la nature du procédé.

L'\textbf{analyse} nous permettra de caractériser et d'identifier 
ces modèles à partir des réponses aux sollicitations et de leurs performances.

Le \textbf{contrôle} est un concept très générale permettant de regrouper
toutes les méthodes et techniques permettant de commander un système dynamique.
Dans ce cours nous présenterons que les principes d'asservissement et de régulation.
Nous verrons comment, il est possible d'élaborer une commande adaptée pour un procédé 
quelconque, notamment lorsque ceux-ci présenterons des défauts de performance.


%\tikzsetnextfilename{diagramme_cours-ap-ext}
\begin{figure}[!h]
\renewcommand\thefigure{A}
\begin{center}
{
\tikzset{external/export=false}
\begin{tikzpicture}
\tikzset{box/.style={rectangle,
                     rounded corners=12pt,
                     minimum width=4.0cm,
                     minimum height=2cm}
}
\draw node[fill=col5!10!white,box] (A) at (0,0) {\scshape Automatique Linéaire};
\draw node[fill=col1!10!white,box] (M) at (-4.5,-2.8) {\scshape Modélisation}; 
\draw node[fill=col1!10!white,box] (N) at ( 0,-2.8) {\scshape Analyse}; 
\draw node[fill=col1!10!white,box] (S) at ( 4.5,-2.8) {\scshape Contrôle}; 
\draw[-latex,ultra thick] (A)--(M.north) ;
\draw[-latex,ultra thick] (A)--(N.north) ;
\draw[-latex,ultra thick] (A)--(S.north) ;
\tikzset{pos1/.style={fill=col1!10!white,
                      yshift=1.75em,
                      text width=4.0cm,
                      minimum height=4.5cm,
                      rounded corners=12pt}
}
\newcommand{\mysize}{\footnotesize}
\newcommand{\mysized}{\normalsize}
\node[below=of M,pos1] {\vbox {\mysize
                                    \begin{itemize}
                                        \item Mise en équation du système
                                        \item Modèles usuels
                                        \item Outils Mathématiques
                                    \end{itemize}}};
\node[below=of N,pos1] {\vbox {\mysize
                                    \begin{itemize}
                                        \item Réponses aux sollicitations
                                        \item Identification
                                        \item Caractérisation
                                        \item Performances
                                    \end{itemize}}};
\node[below=of S,pos1] {\vbox {\mysize
                                    \begin{itemize}
				        \item Consigne
				        \item Commande
					\item Asservissement
					\item Régulation
                                        \item Correction
                                    \end{itemize}}};
\tikzset{pos2/.style={fill=col4!10!white,
                      yshift=-10em,
                      text width=4.0cm,
                      minimum height=4cm,
                      rounded corners=12pt}
}
\node[below=of M,pos2] {\vbox {\mysized
                                    \begin{itemize}
                                        \item \Cref{chap-slci}
                                        \item \Cref{chap-schemabloc}
                                        \item \Cref{chap-model} 
                                        \item \Cref{chap-repreEtat} 
                                    \end{itemize}}};
\node[below=of N,pos2] {\vbox {\mysized
                                    \begin{itemize}
                                        \item \Cref{chap-repfreq}
                                        \item \Cref{chap-perf} 
                                        \item \Cref{chap-stab} 
                                    \end{itemize}}};
\node[below=of S,pos2] {\vbox {\mysized
                                    \begin{itemize}
                                        \item \Cref{chap-asservis} 
                                        \item \Cref{chap-correc} 
                                    \end{itemize}}};
\end{tikzpicture}

}
\end{center}
\caption{Organisation du document.\label{fig-diagramme_cours}}
\end{figure}
