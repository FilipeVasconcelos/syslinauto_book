%%%%%%%%%%%%%%%%%%%%%%%%%%%%%%%%%%%%%%%%%%%%%%%%%%%%%%%%%%%%%%%%%%%%%%%%%%%%%%%%
%%%%%%%%%%%%%%%%%%%%%%%%%%%%%%%%%%%%%%%%%%%%%%%%%%%%%%%%%%%%%%%%%%%%%%%%%%%%%%%%
%%%%%%%%%%%%%%%%%%%%%%%%%%%%%%%%%%%%%%%%%%%%%%%%%%%%%%%%%%%%%%%%%%%%%%%%%%%%%%%%
%%%%%%%%%%%%%%%%%%%%%%%%%%%%%%%%%%%%%%%%%%%%%%%%%%%%%%%%%%%%%%%%%%%%%%%%%%%%%%%%
\chapter{Les nombres complexes\label{annexe-NC}}
%%%%%%%%%%%%%%%%%%%%%%%%%%%%%%%%%%%%%%%%%%%%%%%%%%%%%%%%%%%%%%%%%%%%%%%%%%%%%%%%
%%%%%%%%%%%%%%%%%%%%%%%%%%%%%%%%%%%%%%%%%%%%%%%%%%%%%%%%%%%%%%%%%%%%%%%%%%%%%%%%
%%%%%%%%%%%%%%%%%%%%%%%%%%%%%%%%%%%%%%%%%%%%%%%%%%%%%%%%%%%%%%%%%%%%%%%%%%%%%%%%
%%%%%%%%%%%%%%%%%%%%%%%%%%%%%%%%%%%%%%%%%%%%%%%%%%%%%%%%%%%%%%%%%%%%%%%%%%%%%%%%
%%%%%%%%%%%%%%%%%%%%%%%%%%%%%%%%%%%%%%%%%%%%%%%%%%%%%%%%%%%%%%%%%%%%%%%%%%%%%%%%
\paragraph[Représentation d'un nombre complexe]
          {Représentation géométrique d'un nombre complexe}
%%%%%%%%%%%%%%%%%%%%%%%%%%%%%%%%%%%%%%%%%%%%%%%%%%%%%%%%%%%%%%%%%%%%%%%%%%%%%%%%
Un nombre complexe $z$ est définit par un couple 
de nombre réel $(x,y)$, tel que 
\[
z=x+jy,
\]
où $j$ est le nombre imaginaire pur tel que $j^2=-1$\footnote{En mathématiques 
et en physique, le nombre imaginaire pur est généralement noté $i$. Ici nous 
utilisons la convention des automaticiens et des électroniciens pour ne
pas confondre $i$ avec l'intensité du courant.}.

Un nombre complexe est donc composé d'une partie 
réel $\Re{z}=x$ et d'une partie imaginaire $\Im{z}=y$.

Un nombre complexe peut être représenté géométriquement dans un plan 
(dit complexe), pour lequel l'abscisse et l'ordonné d'un 
point du plan correspondent respectivement 
à la la partie réelle et imaginaire (\Cref{fig-plan_complexe}).
%%%%%%%%%%%%%%%%%%%%%%%%%%%%%%%%%%%%%%%%%%%%%%%%%%%%%%%%%%%%%%%%%%%%%%%%%%%%%%%%
\paragraph{Définition du conjugué d'un nombre complexe}
%%%%%%%%%%%%%%%%%%%%%%%%%%%%%%%%%%%%%%%%%%%%%%%%%%%%%%%%%%%%%%%%%%%%%%%%%%%%%%%%
Le conjugué de $z$ est le nombre noté $\bar{z}$ tel que :
\[
\bar{z}=\Re{z}-j\Im{z}=x-jy
\]
Dans la représentation géométrique le conjugé $\bar{z}$ est le symétrique 
de $z$ par rapport à l'axe des réels (\Cref{fig-plan_complexe}).
%-------------------------------------------------------------------------------
\begin{figure}[!h]
    \centering
    \tikzsetnextfilename{plan_complexe-annexe_cplx-ext}
    \begin{tikzpicture}
\begin{axis}
    [
    axis lines = center,
    minor tick num=1,
    ticks=both,
    xlabel=$\Re{z}$,
    ylabel=$\Im{z}$,
    ymin=-4,
    ymax=+4.9,
    xmin=-5,
    xmax=+4.9
    ]
    \addplot [col4, mark = *] coordinates {( 0, 0)} {};
    \addplot [col4, mark = *] coordinates {( 2, -3)} {};
    \addplot [col4, mark = *] coordinates {( 2, 3)} {};
    \addplot [col4, mark = *] coordinates {( -1, 1)} {};

    \node [below right,col4] at (axis cs:  0, 0) {$0$};
    \node [right, col4]       at (axis cs:  2, -3) {$2-3j$};
    \node [above, col4]       at (axis cs:  2, 3) {$2+3j$};
    \node [left, col4]        at (axis cs:  -1, 1) {$-1+j$};
\end{axis}
\end{tikzpicture}

    \caption{Exemple de représentation géométrique en coordonnées cartésiennes 
             de différents nombres complexes. Les nombres complexes $2+3j$ et 
             $2-3j$ sont conjugués l'un de l'autre. Ces points sont symétriques
             par rapport à l'axe des réel.
             \label{fig-plan_complexe}}
\end{figure}
%-------------------------------------------------------------------------------
%%%%%%%%%%%%%%%%%%%%%%%%%%%%%%%%%%%%%%%%%%%%%%%%%%%%%%%%%%%%%%%%%%%%%%%%%%%%%%%%
\paragraph{Définition du module d'un nombre complexe}
%%%%%%%%%%%%%%%%%%%%%%%%%%%%%%%%%%%%%%%%%%%%%%%%%%%%%%%%%%%%%%%%%%%%%%%%%%%%%%%%
Le module d'un nombre complexe noté $|z|$ est la 
racine carrée de la somme des carrés de $\Re{z}$ et de $\Im{z}$, 
autrement dit, 
\[
|z|=\sqrt{x^2+y^2},
\]
où $x=\Re{z}$ et $y=\Im{z}$ comme définit précedemment.

Dans le plan complexe, le module $|z|$ correspond à la distance à l'origine du 
point correspondant à $z$ dans le plan complexe.
%Le module de $\bar{z}$ est égale au module de $z$.
%%%%%%%%%%%%%%%%%%%%%%%%%%%%%%%%%%%%%%%%%%%%%%%%%%%%%%%%%%%%%%%%%%%%%%%%%%%%%%%%
\paragraph{Propriétés du module}
%%%%%%%%%%%%%%%%%%%%%%%%%%%%%%%%%%%%%%%%%%%%%%%%%%%%%%%%%%%%%%%%%%%%%%%%%%%%%%%%
Soient $z_1$ et $z_2$ deux nombres complexes.
%-------------------------------------------------------------------------------
\begin{itemize}
    \item $|z_1|=0 \Leftrightarrow z_1=0$
    \item $|z_1z_2|=|z_1||z_2|$
    \item $|z^n|=|z|^n$ pour $n\in\mathbb{N}^*$
    \item $\left|\dfrac{z_1}{z_2}\right|=\dfrac{|z_1|}{|z_2|}$ pour $z_2\neq0$
    \item $|z_1+z_2|\le|z_1|+|z_2|$
    \item $|-z|=|z|$;$|\bar{z}|=|z|$
\end{itemize}
%-------------------------------------------------------------------------------
%-------------------------------------------------------------------------------
\begin{figure}[!h]
    \centering
    \tikzsetnextfilename{plan_complexe_2-annexe_cplx-ext}
    \begin{tikzpicture}
\begin{axis}
    [
    axis lines = center,
    minor tick num=1,
    ticks=both,
    xlabel=$\Re{z}$,
    ylabel=$\Im{z}$,
    ymin=-4,
    ymax=+4.9,
    xmin=-5,
    xmax=+4.9
    ]
    \addplot [col4, mark = *] coordinates {( 0, 0)} {};
    \addplot [col4, mark = *] coordinates {( 2, 3)} {};
    \addplot [col4, mark = *] coordinates {( 2, -3)} {};
    \node [below left, col4] at (axis cs:  0, 0) {$0$};
    \node [above, col4]       at (axis cs:  2, 3) {$2+3j$};
    \node [below, col4]       at (axis cs:  2,-3) {$2-3j$};
    \draw[col4] (axis cs:0,0) -- (axis cs:  2, 3) 
    node[midway,yshift=1.1em,xshift=-0.2em] {$|z|$};
    \draw[col4] (axis cs:0,0) -- (axis cs:  2,-3) 
    node[midway,yshift=-1.1em,xshift=-0.2em] {$|\bar{z}|$};
    \draw[col4] (axis cs:1.0,0) arc (0:40:1cm) 
    node[midway,yshift=0.2em,xshift=0.6em] {$\theta$};
    \draw[col4] (axis cs:1.0,0) arc (0:-40:1cm) 
    node[midway,yshift=0.2em,xshift=0.6em] {$-\theta$};
\end{axis}
\end{tikzpicture}

    \caption{Exemple de représentation géométrique en coordonnées polaires 
             d'un nombre complexe. Le nombre complexe $z=2+3j$ s'écrit sous 
             la forme polaire $z=|z|e^{i\theta}$ avec 
             $|z|=\sqrt{x^2+y^2}=\sqrt{13}$ et 
             $\theta=\arccos{\left(\frac{x}{|z|}\right)}=
             \arcsin{\left(\frac{y}{|z|}\right)}=
             \arctan{\left(\frac{y}{x}\right)}$.\label{fig-plan_complexe2}}
\end{figure}
%-------------------------------------------------------------------------------
%%%%%%%%%%%%%%%%%%%%%%%%%%%%%%%%%%%%%%%%%%%%%%%%%%%%%%%%%%%%%%%%%%%%%%%%%%%%%%%%
\paragraph{Définition de l'argument d'un nombre complexe}
%%%%%%%%%%%%%%%%%%%%%%%%%%%%%%%%%%%%%%%%%%%%%%%%%%%%%%%%%%%%%%%%%%%%%%%%%%%%%%%%
L'argument $\arg{(z)}$ d'un nombre complexe $z$ est l'angle qui, dans la 
représentation géométrique, sépare l'axe des réels du vecteur représentatif de 
$z$ (\Cref{fig-plan_complexe2}). Le couple $(|z|,\theta=\arg{(z)})$ sont donc 
les coordonnées polaires de la représentation géométrique d'un nombre complexe.
L'argument est définit à $2\pi$ près. On appelle argument principal celui qui 
est compris entre $[-\pi,\pi]$
%%%%%%%%%%%%%%%%%%%%%%%%%%%%%%%%%%%%%%%%%%%%%%%%%%%%%%%%%%%%%%%%%%%%%%%%%%%%%%%%
\paragraph{Propriétés de l'argument}
%%%%%%%%%%%%%%%%%%%%%%%%%%%%%%%%%%%%%%%%%%%%%%%%%%%%%%%%%%%%%%%%%%%%%%%%%%%%%%%%
Soient $z$, $z_1$ et $z_2$ des nombres complexes.
%-------------------------------------------------------------------------------
\begin{itemize}
    \item $\cos\theta=\dfrac{\Re{z}}{|z|}$; $\sin\theta=\dfrac{\Im{z}}{|z|}$
    \item $\arg(z_1z_2)=\arg(z_1)+\arg(z_2)$; $\arg(\bar{z})=\arg(z)$
    \item $\arg(-z)=\pi+\arg(z)[2\pi]$; 
          $\arg\left(\dfrac{1}{z}\right)=-\arg(z)[2\pi]$
    \item $\arg\left(\dfrac{z_1}{z_2}\right)=\arg(z_1)-\arg(z_2)[2\pi]$ 
          pour $z_2\neq0$
    \item $\arg(z\bar{z})=\arg(z)+\arg(\bar{z})=\arg(z)-\arg(z)=0[2\pi]$
\end{itemize}
%-------------------------------------------------------------------------------
%%%%%%%%%%%%%%%%%%%%%%%%%%%%%%%%%%%%%%%%%%%%%%%%%%%%%%%%%%%%%%%%%%%%%%%%%%%%%%%%
\paragraph{Calcul de l'argument principal d'un nombre complexe}
%%%%%%%%%%%%%%%%%%%%%%%%%%%%%%%%%%%%%%%%%%%%%%%%%%%%%%%%%%%%%%%%%%%%%%%%%%%%%%%%
L'argument étant définit à $2\pi$ près, il est recommandé de donner 
l'argument principale pour des questions d'unicité 
(i.e $\arg{z}\in[-\pi,\pi]$). 
Soit $\phi$ l'argument principale d'un nombre complexe $z=a+ib$, alors 
$\phi$ est définit par :
\[
\phi=
\begin{cases}
    \hphantom{-}\arctan{(b/a)}     & \text{si $a>0$} \\
    \hphantom{-}\arctan{(b/a)}+\pi & \text{si $a<0$ et $b\ge0$} \\
    \hphantom{-}\arctan{(b/a)}-\pi & \text{si $a<0$ et $b<0$} \\
    \hphantom{-} \pi/2             & \text{si $a=0$ et $b>0$} \\
                -\pi/2             & \text{si $a=0$ et $b<0$} \\
    \hphantom{-}0                  & \text{si $a=0$ et $b=0$} \\
\end{cases}
\]
La formule précédente nécessite de distinguer plusieurs cas.
Cependant, de nombreux langages de programmation fournissent
une variante de la fonction arc tangente, qui est souvent appélee
\verb?atan2(b,a)?, et qui traite ces différents cas. 
%-------------------------------------------------------------------------------
\begin{figure}[!ht]
    \centering
    \tikzsetnextfilename{arctangente-annexe_cplx-ext}
    \begin{tikzpicture}
    \begin{axis}[
    axis line style = thick,
    clip=false,
    height=4.5cm,
    width=9.5cm,
    axis x line=center,
    axis y line=center,
    xmin=-7,
    xmax=7,
    ymin=-2,
    ymax=2,
    xlabel={$x$},
    ylabel={$\arctan{(x)}$},
    xlabel style={below right},
    ylabel style={above left},
    ytick={1.5707963267948966},
    yticklabels={$\dfrac{\pi}{2}$},
    extra y ticks={-1.5707963267948966},
    extra y tick labels={$-\dfrac{\pi}{2}$},
    extra y tick style={
          yticklabel style={xshift=0.5ex, anchor=west}
    },
    xtick={-6,-4,-2,0,2,4,6},
    ]
    \addplot [signalb,domain=-7:7]{rad(atan(x))};
    \addplot [col1,thick,dashed,domain=0:7,samples=10]{pi/2};
    \addplot [col1,thick,dashed,domain=-7:0,samples=10]{-pi/2};
    \end{axis}
\end{tikzpicture}

    \caption{Représentation graphique de la fonction arc tangente.
             \label{fig-arctangente}}
\end{figure}
%-------------------------------------------------------------------------------
%%%%%%%%%%%%%%%%%%%%%%%%%%%%%%%%%%%%%%%%%%%%%%%%%%%%%%%%%%%%%%%%%%%%%%%%%%%%%%%%
\paragraph{Forme exponentielle ou polaire d'un nombre complexe}
%%%%%%%%%%%%%%%%%%%%%%%%%%%%%%%%%%%%%%%%%%%%%%%%%%%%%%%%%%%%%%%%%%%%%%%%%%%%%%%%
La formule d'Euler 
\[
e^{i\theta}=\cos\theta+i\sin\theta
\]
permet d'écrire tout nombre complexe sous sa forme exponentielle : 
\[
z=|z|e^{i\theta}
\]
Une conséquence spectaculaire de la formule d'Euler est que
\[
e^{i\pi}=-1.
\]
On notera que $e^{i\theta}$ est un nombre complexe de module 1 admettant 
$\theta$ pour argument. Lorsque $\theta$ varie de $0$ à $2\pi$, l'image du 
nombre complexe $e^{i\theta}$ décrit le cercle unité.
Une autre conséquence est que les fonctions trigononométriques peuvent 
s'exprimer sous forme d'exponentielle complexe:
\begin{align*}
    \sin{\omega t}&=\dfrac{e^{\jw t}-e^{-\jw t}}{2j} \\
    \cos{\omega t}&=\dfrac{e^{\jw t}+e^{-\jw t}}{2}
\end{align*}
%-------------------------------------------------------------------------------
\begin{figure}[!ht]
    \centering
    \tikzsetnextfilename{cercle_trigo-annexe_cplx-ext}
    \begin{tikzpicture}[scale=2.4444,cap=round,>=latex]
    % draw the coordinates
    \draw[->] (-1.7cm,0cm) -- (1.7cm,0cm) node[right,fill=white] {$x$};
    \draw[->] (0cm,-1.7cm) -- (0cm,1.7cm) node[above,fill=white] {$y$};
    % draw each angle in radians
    \foreach \x/\xtext in 
    {
        30/\dfrac{\pi}{6},
        45/\dfrac{\pi}{4},
        60/\dfrac{\pi}{3},
        90/\dfrac{\pi}{2},
        120/\dfrac{2\pi}{3},
        135/\dfrac{3\pi}{4},
        150/\dfrac{5\pi}{6},
        180/\pi,
        210/\dfrac{7\pi}{6},
        225/\dfrac{5\pi}{4},
        240/\dfrac{4\pi}{3},
        270/\dfrac{3\pi}{2},
        300/\dfrac{5\pi}{3},
        315/\dfrac{7\pi}{4},
        330/\dfrac{11\pi}{6},
        360/2\pi}
    \draw (\x:1.2cm) node[fill=white] {\scalebox{.8}{$\xtext$}};
        % draw the horizontal and vertical coordinates
        % the placement is better this way
    \draw (-1.5cm,0cm) node[above=1pt] {\scalebox{.6}{$(-1,0)$}}
          (1.5cm,0cm)  node[above=1pt] {\scalebox{.6}{$(1,0)$}}
          (0cm,-1.5cm) node[fill=white] {\scalebox{.6}{$(0,-1)$}}
          (0cm,1.5cm)  node[fill=white] {\scalebox{.6}{$(0,1)$}};
    \draw[thick] (0cm,0cm) circle(1cm);
    \foreach \x in {0,30,45,60,90,120,135,150,180,210,
                     225,240,270,300,315,330,360} 
     {
        \filldraw[black] (\x:1cm) circle(0.6pt);
     }
     \draw[col4]   (-0.5cm,-0.87cm) rectangle (0.5cm,0.87cm);
     \draw[col3] (-0.71cm,-0.71cm) rectangle (0.71cm,0.71cm);
     \draw[col1]  (-0.87cm,-0.5cm) rectangle (0.866025cm,0.5cm);
     \node[above left,xshift=0.1em,col4] at (0.5,0) 
    {\scalebox{0.6}{$\dfrac{1}{2}$}};
     \node[above left,xshift=0.2em,col3] at (0.707106,0) 
    {\scalebox{0.6}{$\dfrac{\sqrt{2}}{2}$}};
     \node[above left,xshift=0.2em,col1] at (0.866025,0) 
    {\scalebox{0.6}{$\dfrac{\sqrt{3}}{2}$}};
\end{tikzpicture}

    \caption{Quelques points particuliers du cercletrigonométrique 
    ou cercle unité.}
\end{figure}
%-------------------------------------------------------------------------------
%%%%%%%%%%%%%%%%%%%%%%%%%%%%%%%%%%%%%%%%%%%%%%%%%%%%%%%%%%%%%%%%%%%%%%%%%%%%%%%%
%%%%%%%%%%%%%%%%%%%%%%%%%%%%%%%%%%%%%%%%%%%%%%%%%%%%%%%%%%%%%%%%%%%%%%%%%%%%%%%%
%%%%%%%%%%%%%%%%%%%%%%%%%%%%%%%%%%%%%%%%%%%%%%%%%%%%%%%%%%%%%%%%%%%%%%%%%%%%%%%%
%%%%%%%%%%%%%%%%%%%%%%%%%%%%%%%%%%%%%%%%%%%%%%%%%%%%%%%%%%%%%%%%%%%%%%%%%%%%%%%%
%annexe_cplx.tex
