%%%%%%%%%%%%%%%%%%%%%%%%%%%%%%%%%%%%%%%%%%%%%%%%%%%%%%%%%%%%%%%%%%%%%%%%%%%%%%%%
%%%%%%%%%%%%%%%%%%%%%%%%%%%%%%%%%%%%%%%%%%%%%%%%%%%%%%%%%%%%%%%%%%%%%%%%%%%%%%%%
\exercice{L'impulsion de Dirac approchée}
%%%%%%%%%%%%%%%%%%%%%%%%%%%%%%%%%%%%%%%%%%%%%%%%%%%%%%%%%%%%%%%%%%%%%%%%%%%%%%%%
%%%%%%%%%%%%%%%%%%%%%%%%%%%%%%%%%%%%%%%%%%%%%%%%%%%%%%%%%%%%%%%%%%%%%%%%%%%%%%%%
%%%%%%%%%%%%%%%%%%%%%%%%%%%%%%%%%%%%%%%%%%%%%%%%%%%%%%%%%%%%%%%%%%%%%%%%%%%%%%%%
\question{}
%%%%%%%%%%%%%%%%%%%%%%%%%%%%%%%%%%%%%%%%%%%%%%%%%%%%%%%%%%%%%%%%%%%%%%%%%%%%%%%%
%-------------------------------------------------------------------------------
\begin{figure}[!h]
    \centering
    \tikzsetnextfilename{fonction_porte_decomposition-chap-slci-ext}
    \input{tikz/fonction_porte_decomposition-chap-slci.tex}
\end{figure}
%-------------------------------------------------------------------------------
La fonction $\delta_a(t)$ est simplement donnée par 
\[
    \color{col1}p_a(t)
    =\color{col4}\dfrac{1}{a}u(t)\color{col3}-\dfrac{1}{a}u(t-a)
    =\normalcolor\dfrac{1}{a}\left(u(t)-u(t-a)\right)
\] 
où $u(t)$ est la fonction échelon unité.

%%%%%%%%%%%%%%%%%%%%%%%%%%%%%%%%%%%%%%%%%%%%%%%%%%%%%%%%%%%%%%%%%%%%%%%%%%%%%%%%
\question{}
%%%%%%%%%%%%%%%%%%%%%%%%%%%%%%%%%%%%%%%%%%%%%%%%%%%%%%%%%%%%%%%%%%%%%%%%%%%%%%%%
On rappel la transformée de Laplace d'une fonction retardée :
\[
    \laplace{f(t-\tau)} = e^{-\tau p}F(p)
\]
La transformée de Laplace de $\delta_a(t)$ est donc donnée par : 
\[
    \Delta_a(p) = \laplace{\delta_a(t)}
                = \dfrac{1}{a}\left(\laplace{u(t)}-\laplace{u(t-a)}\right)
                = \dfrac{1}{ap}(1-e^{-ap})
\]
%%%%%%%%%%%%%%%%%%%%%%%%%%%%%%%%%%%%%%%%%%%%%%%%%%%%%%%%%%%%%%%%%%%%%%%%%%%%%%%%
\question{}
%%%%%%%%%%%%%%%%%%%%%%%%%%%%%%%%%%%%%%%%%%%%%%%%%%%%%%%%%%%%%%%%%%%%%%%%%%%%%%%%
Le développement limité d'ordre 1 de $e^{x}=1+x+o\left(x\right)$ permet de
déterminer la limite pour $a\rightarrow0$ de la transformée de Laplace de 
$\delta_a(t)$. En effet, 
\[
    \lim_{a\to0} \Delta_a(p) = \dfrac{1}{ap}(1-1+ap)=1.
\]
Puisque la transformée de Laplace d'une impulsion de Dirac est égale à 1, on a
bien vérifié que $\lim_{a\to0} \delta_a(p)=\delta(t)$ où $\delta(t)$ est 
l'impulsion de Dirac.

