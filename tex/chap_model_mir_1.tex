\pagestyle{empty}
\newgeometry{left=1.75cm,right=1.75cm,top=1.75cm,bottom=1.75cm}
%%%%%%%%%%%%%%%%%%%%%%%%%%%%%%%%%%%%%%%%%%%%%%%%%%%%%%%%%%%%%%%%%%%%%%%%%%%%%%%%
%%%%%%%%%%%%%%%%%%%%%%%%%%%%%%%%%%%%%%%%%%%%%%%%%%%%%%%%%%%%%%%%%%%%%%%%%%%%%%%%
%%%%%%%%%%%%%%%%%%%%%%%%%%%%%%%%%%%%%%%%%%%%%%%%%%%%%%%%%%%%%%%%%%%%%%%%%%%%%%%%
\section*{Système du premier ordre}
%%%%%%%%%%%%%%%%%%%%%%%%%%%%%%%%%%%%%%%%%%%%%%%%%%%%%%%%%%%%%%%%%%%%%%%%%%%%%%%%
%%%%%%%%%%%%%%%%%%%%%%%%%%%%%%%%%%%%%%%%%%%%%%%%%%%%%%%%%%%%%%%%%%%%%%%%%%%%%%%%
%%%%%%%%%%%%%%%%%%%%%%%%%%%%%%%%%%%%%%%%%%%%%%%%%%%%%%%%%%%%%%%%%%%%%%%%%%%%%%%%
%%%%%%%%%%%%%%%%%%%%%%%%%%%%%%%%%%%%%%%%%%%%%%%%%%%%%%%%%%%%%%%%%%%%%%%%%%%%%%%%
%%%%%%%%%%%%%%%%%%%%%%%%%%%%%%%%%%%%%%%%%%%%%%%%%%%%%%%%%%%%%%%%%%%%%%%%%%%%%%%%
\subsection*{Définition d'un système du premier ordre}
%%%%%%%%%%%%%%%%%%%%%%%%%%%%%%%%%%%%%%%%%%%%%%%%%%%%%%%%%%%%%%%%%%%%%%%%%%%%%%%%
%%%%%%%%%%%%%%%%%%%%%%%%%%%%%%%%%%%%%%%%%%%%%%%%%%%%%%%%%%%%%%%%%%%%%%%%%%%%%%%%
\index{Système du premier ordre!définition}
Un système du premier ordre est un système régi par une équation
différentielle linéaire à coefficients constants du premier ordre 
(c.-à-d. $n=1$ pour l'\cref{eq-difflci}), de la forme générale :
%-------------------------------------------------------------------------------
\begin{bequation}[ams align]
    \tau\devi{s(t)}{}+s(t)=K e(t)\label{eq-1er}
\end{bequation}
%-------------------------------------------------------------------------------
où \textbf{$K$ est le gain statique} et \textbf{$\tau>0$ la constante de 
temps du système}. La condition sur le signe de $\tau$ sera 
discutée au moment de l'établissement des réponses temporelles.
L'analyse dimentionnelle de cette équation différentielle, nous permet 
de confirmer que $\tau$ à la dimension d'un temps, mais surtout que 
la dimension du gain statique est donnée par le rapport des dimensions de 
la sortie sur l'entrée. Autrement dit, c'est un paramètre sans dimension 
si l'entrée et la sortie sont de même nature.
%%%%%%%%%%%%%%%%%%%%%%%%%%%%%%%%%%%%%%%%%%%%%%%%%%%%%%%%%%%%%%%%%%%%%%%%%%%%%%%%
%%%%%%%%%%%%%%%%%%%%%%%%%%%%%%%%%%%%%%%%%%%%%%%%%%%%%%%%%%%%%%%%%%%%%%%%%%%%%%%%
\subsection*{Fonction de transfert d'un système du premier ordre}
%%%%%%%%%%%%%%%%%%%%%%%%%%%%%%%%%%%%%%%%%%%%%%%%%%%%%%%%%%%%%%%%%%%%%%%%%%%%%%%%
%%%%%%%%%%%%%%%%%%%%%%%%%%%%%%%%%%%%%%%%%%%%%%%%%%%%%%%%%%%%%%%%%%%%%%%%%%%%%%%%
\index{Système du premier ordre!fonction de transfert}
La transformée de Laplace de l'\cref{eq-1er}, dans les conditions de 
Heaviside, nous donne :
\[
\tau pS(p)+S(p)=KE(p)
\]
La fonction de transfert $H(p)$ d'un système du premier ordre est 
donc de la forme:
%-------------------------------------------------------------------------------
\begin{bequation}[ams align]
    H(p)=\dfrac{S(p)}{E(p)}=\dfrac{K}{\tau p + 1}\label{eq-ft1er}
\end{bequation}
%-------------------------------------------------------------------------------
%%%%%%%%%%%%%%%%%%%%%%%%%%%%%%%%%%%%%%%%%%%%%%%%%%%%%%%%%%%%%%%%%%%%%%%%%%%%%%%%
%%%%%%%%%%%%%%%%%%%%%%%%%%%%%%%%%%%%%%%%%%%%%%%%%%%%%%%%%%%%%%%%%%%%%%%%%%%%%%%%
\subsection*{Pôle de la fonction de transfert du premier ordre}
%%%%%%%%%%%%%%%%%%%%%%%%%%%%%%%%%%%%%%%%%%%%%%%%%%%%%%%%%%%%%%%%%%%%%%%%%%%%%%%%
%%%%%%%%%%%%%%%%%%%%%%%%%%%%%%%%%%%%%%%%%%%%%%%%%%%%%%%%%%%%%%%%%%%%%%%%%%%%%%%%
Un système du premier ordre ne possède qu'un seul pôle qui est trivialement 
déterminé par la résolution de l'équation :
\[
\tau p + 1 =0
\]
ce pôle $p_1=-\dfrac{1}{\tau}$ est donc réel négatif pour $\tau>0$.
La fonction de transfert d'un système du premier peut alors s'écrire 
sous la forme factorisée suivante
\[
H(p)=\dfrac{K}{(p-p_1)}=\dfrac{K}{\tau\left(p+\dfrac{1}{\tau}\right)}.
\]
%-------------------------------------------------------------------------------
\begin{figure}[b]
    \centering
    \tikzsetnextfilename{carte_1er-chap_model-ext}
    \begin{tikzpicture}
    \carte[l][-1.75]
    \dpole{-1.5}{0}{1}[col1]
    \node[above,yshift=1em] at (p1) {$\tau$};
\end{tikzpicture}%

\end{figure}
%-------------------------------------------------------------------------------

\clearpage
\thispagestyle{empty}
\captionsetup{width=\linewidth}
%%%%%%%%%%%%%%%%%%%%%%%%%%%%%%%%%%%%%%%%%%%%%%%%%%%%%%%%%%%%%%%%%%%%%%%%%%%%%%%%
%%%%%%%%%%%%%%%%%%%%%%%%%%%%%%%%%%%%%%%%%%%%%%%%%%%%%%%%%%%%%%%%%%%%%%%%%%%%%%%%
\subsection*{Réponse impulsionnelle}
%%%%%%%%%%%%%%%%%%%%%%%%%%%%%%%%%%%%%%%%%%%%%%%%%%%%%%%%%%%%%%%%%%%%%%%%%%%%%%%%
%%%%%%%%%%%%%%%%%%%%%%%%%%%%%%%%%%%%%%%%%%%%%%%%%%%%%%%%%%%%%%%%%%%%%%%%%%%%%%%%
\index{Système du premier ordre!réponse impulsionnelle}
Nous condidérons une excitation impulsionnelle de la forme :
\[
e(t)=E_0\delta(t),
\]
où $\delta(t)$ est l'impulsion de Dirac et $E_0$ est une constante.

La réponse impulsionnelle d'un système du premier ordre est, dans le 
domaine de Laplace, de la forme :
\[
S(p)=H(p)E(p)=\dfrac{KE_0}{(\tau p+1)}.
\]
%\marginnote{$\dfrac{1}{(p+a)}\rightarrow e^{-at}$}[-2em]
La transformée de Laplace inverse de $S(p)$ (c.f ligne 7 de la table des
transformées de Laplace), nous donne la forme générale de la réponse 
impulsionnelle d'un système du premier ordre:
%-------------------------------------------------------------------------------
\begin{bequation}[ams align]
    \laplacei{S(p)}=s(t)=\dfrac{KE_0}{\tau} e^{-t/\tau}\label{eq-1er_imp}.
\end{bequation}
%-------------------------------------------------------------------------------
Cette réponse correspond à une simple exponentielle décroissante pour $\tau>0$.
La~\cref{fig-1er_imp} présente la réponse impulsionnelle d'un système 
du premier ordre pour différentes valeurs de la constante de temps $\tau$.
On observe que pour $t\to\infty$, la valeur de $s(t)$ tend vers 0, ce qui 
est caractéristique d'un système stable. Nous pouvons donc considérer que 
$\tau$ est strictement positif pour une question de stabilité.  

Il est également possible d'observer que la pente à l'origine dépend de la 
constante de temps. La pente à l'origine peut être obtenue directement en 
dérivant la réponse temporelle $s(t)$ 
\[
s'(0)=-\dfrac{KE_0}{\tau^2}
\]      
La pente à l'origine est négative et inversemment proportionnelle 
au carré de la constante de temps du système $\tau$.

Nous constatons que le temps $t_{5\%}$ de 
réponse à 5\% est de l'ordre de 3$\tau$ (c.-à-d. $-\log{5\%}$). Le transitoire 
est lui de l'ordre de $7\tau$ (c'est à dire le temps à partir duquel on 
considère que le signal est nul).
%-------------------------------------------------------------------------------
\begin{figure}[!hb]
    \centering
    \tikzsetnextfilename{1er_imp1-chap_model-ext}
    \input{tikz/1er_imp1-chap_model.tex}
    \hfill
    \tikzsetnextfilename{1er_imp2-chap_model-ext}
    \begin{tikzpicture}
    \begin{axis}
    [   legend style={draw=none},
        axis line style = thick,
        width=0.47\textwidth,
        xmin=0,
        xmax=6,
        ymin=0,
        ymax=2.01,
        xlabel={$t$},
        ylabel={$s(t)$},
        ylabel near ticks, yticklabel pos=right,
        label style={font=\Large},
        legend cell align={left}
    ]
    \addplot[signalb,domain=0:10] {0.5*exp(-x)};
    \addplot[signalr,domain=0:10] {exp(-x)};
    \addplot[signalg,domain=0:10] {2.0*exp(-x)};
    \legend{$K=0.5$,$K=1$,$K=2$}
    \end{axis}
\end{tikzpicture}

    \caption{Réponse impulsionnelle d'un système du premier ordre avec : 
             $E_0=1$ et (gauche) différentes valeurs de la constante 
             de temps $\tau$ pour $K=1$; (droite) différentes valeurs 
             du gain $K$ pour $\tau=1$ (\Cref{eq-1er_imp}). 
             \label{fig-1er_imp}}
\end{figure}
%-------------------------------------------------------------------------------
\clearpage
\thispagestyle{empty}
%%%%%%%%%%%%%%%%%%%%%%%%%%%%%%%%%%%%%%%%%%%%%%%%%%%%%%%%%%%%%%%%%%%%%%%%%%%%%%%%
%%%%%%%%%%%%%%%%%%%%%%%%%%%%%%%%%%%%%%%%%%%%%%%%%%%%%%%%%%%%%%%%%%%%%%%%%%%%%%%%
\subsection*{Réponse indicielle}
%%%%%%%%%%%%%%%%%%%%%%%%%%%%%%%%%%%%%%%%%%%%%%%%%%%%%%%%%%%%%%%%%%%%%%%%%%%%%%%%
%%%%%%%%%%%%%%%%%%%%%%%%%%%%%%%%%%%%%%%%%%%%%%%%%%%%%%%%%%%%%%%%%%%%%%%%%%%%%%%%
\index{Système du premier ordre!réponse indicielle}
Pour déterminer la réponse indicielle, nous considérons une entrée 
$e(t)$ en échelon telle que :
\[
e(t)=E_0\cdot u(t),
\]
où $u(t)$ est l'échelon unitaire et $E_0$ est une constante.

Dans le domaine de Laplace la sortie est donc de la forme :
\[
S(p)=H(p)E(p)=\dfrac{KE_0}{p(1+\tau p)}=\dfrac{KE_0}{\tau p(p+\frac{1}{\tau})}
\]
%\marginnote{$\dfrac{a}{p(p+a)}\rightarrow 1-e^{-at}$}[-2em]
La transformée de Laplace inverse de $S(p)$ (c.f ligne 11 de la table des
transformées de Laplace), nous donne la forme générale de la réponse 
indicielle d'un système du premier ordre:
%-------------------------------------------------------------------------------
\begin{bequation}[ams align]
    \laplacei{S(p)}=s(t)=KE_0\left(1-e^{-t/\tau}\right)\label{eq-1er_ind}
\end{bequation}
%-------------------------------------------------------------------------------
La~\cref{fig-1er_ind} présente cette réponse indicielle pour 
différentes valeurs de la constante de temps $\tau$.
Pour $t\to\infty$, la valeur de $s(t)$ tend vers $KE_0$.
%\footnote{Il est également possible 
%de déterminer cette valeur en appliquant le théorème de la valeur finale 
%sur la fonction dans le domaine de Laplace puisque 
%$pS(p)$ ne possède qu'un seul pôle à partie réelle négative. 
%Ainsi,
%\[
%\lim_{t\to\infty}s(t)=\lim_{p\to0}pS(p)=\lim_{p\to0}
%\dfrac{KE_0}{\tau p+1}=KE_0
%\]}.
La pente à l'origine peut être obtenue directement en dérivant 
la réponse temporelle $s(t)$
\[
s'(0)=\dfrac{KE_0}{\tau}
\]
La pente à l'origine est positive et inversemment proportionnelle 
à la constante de temps du système.

Le tableau sous la figure~\cref{fig-1er_ind} donne quelques valeurs 
particulières de la réponse indicielle. D'après celui-ci, on constate 
que le temps de réponse à 5\% $t_{5\%}$ (temps au bout duquel la 
réponse indicielle atteint 95\% du signal final) est donné par :
\[
t_{5\%}=-\tau\log{0.05}\sim3\tau.
\]
Le temps de montée $t_m$ (temps au bout duquel la réponse de 10\% 
à 90\% du signal final) est donné par :
\[
t_m=-\tau\log{\dfrac{0.1}{0.9}}\sim2.2\tau
\]
%-------------------------------------------------------------------------------
\begin{figure}[!hb]
    \centering
    \tikzsetnextfilename{1er_ind1-chap_model-ext}
    \begin{tikzpicture}
    \begin{axis}
    [%
        legend style={draw=none,font=\scriptsize},
        legend pos=south east,
        axis line style = thick,
        width=0.47\textwidth,
        xmin=0,
        xmax=10,
        ymin=0,
        ymax=1.1,
        xlabel={$t$},
        ylabel={$s(t)$},
        label style={font=\Large},
        legend cell align={left},
    ]%
    \addplot[thick,col5,domain=0:11.5,samples=101] {{1}};
    \addplot[thick,col1,domain=0:11.5,samples=101] {1-exp(-x)};
    \addplot[thick,col2,domain=0:11.5,samples=101] {(1-exp(-x/2))};
    \addplot[thick,col3,domain=0:11.5,samples=101] {(1-exp(-x/3))};
    \addplot[thick,col4,domain=0:11.5,samples=101] {(1-exp(-x/4))};
    \addplot[thick,col6,domain=0:11.5,samples=101] {(1-exp(-x/5))};
    \legend{échelon,$\tau=1$,$\tau=2$,$\tau=3$,$\tau=4$,$\tau=5$}
    \end{axis}%
\end{tikzpicture}%

    \hfill
    \tikzsetnextfilename{1er_ind2-chap_model-ext}
    \input{tikz/1er_ind2-chap_model.tex}
    
    \caption{Réponse indicielle d'un système du premier ordre avec : $E_0=1$ 
             et (gauche) pour différentes valeurs de $\tau$ et avec $K=1$;
	     (droite) pour différentes valeurs du gain $K$ et 
             avec $\tau=1$.
             \label{fig-1er_ind}}
\end{figure}
%-------------------------------------------------------------------------------
