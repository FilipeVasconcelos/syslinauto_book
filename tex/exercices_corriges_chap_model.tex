%%%%%%%%%%%%%%%%%%%%%%%%%%%%%%%%%%%%%%%%%%%%%%%%%%%%%%%%%%%%%%%%%%%%%%%%%%%%%%%%
%%%%%%%%%%%%%%%%%%%%%%%%%%%%%%%%%%%%%%%%%%%%%%%%%%%%%%%%%%%%%%%%%%%%%%%%%%%%%%%%
\exercice{Analyse de la réponse temporelle d'un SLCI~\moyen}
%%%%%%%%%%%%%%%%%%%%%%%%%%%%%%%%%%%%%%%%%%%%%%%%%%%%%%%%%%%%%%%%%%%%%%%%%%%%%%%%
%%%%%%%%%%%%%%%%%%%%%%%%%%%%%%%%%%%%%%%%%%%%%%%%%%%%%%%%%%%%%%%%%%%%%%%%%%%%%%%%
%-------------------------------------------------------------------------------
\begin{center}
    \tikzsetnextfilename{exercice_reponse-2nd_corrige_chap-model-ext}
    \input{tikz/exercice_reponse-2nd_corrige_chap-model.tex}
\end{center}
%-------------------------------------------------------------------------------
%%%%%%%%%%%%%%%%%%%%%%%%%%%%%%%%%%%%%%%%%%%%%%%%%%%%%%%%%%%%%%%%%%%%%%%%%%%%%%%%
\question{\textbf{Déterminer :}}
%%%%%%%%%%%%%%%%%%%%%%%%%%%%%%%%%%%%%%%%%%%%%%%%%%%%%%%%%%%%%%%%%%%%%%%%%%%%%%%%
%-------------------------------------------------------------------------------
\begin{itemize}
    \item Ordre du système : 2nd ordre 
    \item Gain statique $K$ : $K=1.5$ 
    \item Temps de réponse à 5\% : $t_{5\%}\sim\SI{8}{\second}$ 
    \item Dépassement relatif en \% : $D_1\sim\dfrac{1.96-1.5}{1.5}\sim30\%$ 
\end{itemize}
%-------------------------------------------------------------------------------

%%%%%%%%%%%%%%%%%%%%%%%%%%%%%%%%%%%%%%%%%%%%%%%%%%%%%%%%%%%%%%%%%%%%%%%%%%%%%%%%
\question{\textbf{En déduire l'amortissement $\xi$, la pulsation propre 
$\omega_0$ et écrire la fonction de transfert.}}
%%%%%%%%%%%%%%%%%%%%%%%%%%%%%%%%%%%%%%%%%%%%%%%%%%%%%%%%%%%%%%%%%%%%%%%%%%%%%%%%
\`A partir du dépassement : 
\[
\xi=\dfrac{\ln{D}}{\sqrt{\pi^2+(\ln{D})^2}}\sim0.35 
\]

\`A partir du temps du premier dépassement :
$t_1\sim\SI{3.35}{\second}$

\[
\omega_0=\dfrac{\pi}{t_1\sqrt{1-\xi^2}}\sim\SI{1}{\radian\per\sec} 
\]
%%%%%%%%%%%%%%%%%%%%%%%%%%%%%%%%%%%%%%%%%%%%%%%%%%%%%%%%%%%%%%%%%%%%%%%%%%%%%%%%
\question{\textbf{Calculer les pôles du système et les placer les sur une cartes
des pôles (plan complexe).}}
%%%%%%%%%%%%%%%%%%%%%%%%%%%%%%%%%%%%%%%%%%%%%%%%%%%%%%%%%%%%%%%%%%%%%%%%%%%%%%%%
Les pôles sont complexes conjugués et sont donnés par : 
\[
p_{1,2}=-\xi\omega_0\pm j\omega_0\sqrt{1-\xi^2}
\]

$p_1=-0.35+0.93j$ et $p_2=-0.35-0.93j$

%%%%%%%%%%%%%%%%%%%%%%%%%%%%%%%%%%%%%%%%%%%%%%%%%%%%%%%%%%%%%%%%%%%%%%%%%%%%%%%%
%%%%%%%%%%%%%%%%%%%%%%%%%%%%%%%%%%%%%%%%%%%%%%%%%%%%%%%%%%%%%%%%%%%%%%%%%%%%%%%%
\exercice{Détermination de la fonction de transfert~\facile}
%%%%%%%%%%%%%%%%%%%%%%%%%%%%%%%%%%%%%%%%%%%%%%%%%%%%%%%%%%%%%%%%%%%%%%%%%%%%%%%%
%%%%%%%%%%%%%%%%%%%%%%%%%%%%%%%%%%%%%%%%%%%%%%%%%%%%%%%%%%%%%%%%%%%%%%%%%%%%%%%%
On souhaite déterminer les fonctions de transfert donnant lieu à réponse 
indicielle ayant certaines caractéristiques.
\acpl


