\clearpage
\pagestyle{empty}
%%%%%%%%%%%%%%%%%%%%%%%%%%%%%%%%%%%%%%%%%%%%%%%%%%%%%%%%%%%%%%%%%%%%%%%%%%%%%%%%
%%%%%%%%%%%%%%%%%%%%%%%%%%%%%%%%%%%%%%%%%%%%%%%%%%%%%%%%%%%%%%%%%%%%%%%%%%%%%%%%
%%%%%%%%%%%%%%%%%%%%%%%%%%%%%%%%%%%%%%%%%%%%%%%%%%%%%%%%%%%%%%%%%%%%%%%%%%%%%%%%
\section*{Système du second ordre}
%%%%%%%%%%%%%%%%%%%%%%%%%%%%%%%%%%%%%%%%%%%%%%%%%%%%%%%%%%%%%%%%%%%%%%%%%%%%%%%%
%%%%%%%%%%%%%%%%%%%%%%%%%%%%%%%%%%%%%%%%%%%%%%%%%%%%%%%%%%%%%%%%%%%%%%%%%%%%%%%%
%%%%%%%%%%%%%%%%%%%%%%%%%%%%%%%%%%%%%%%%%%%%%%%%%%%%%%%%%%%%%%%%%%%%%%%%%%%%%%%%
%%%%%%%%%%%%%%%%%%%%%%%%%%%%%%%%%%%%%%%%%%%%%%%%%%%%%%%%%%%%%%%%%%%%%%%%%%%%%%%%
%%%%%%%%%%%%%%%%%%%%%%%%%%%%%%%%%%%%%%%%%%%%%%%%%%%%%%%%%%%%%%%%%%%%%%%%%%%%%%%%
\subsection*{Définition d'un système du second ordre}
%%%%%%%%%%%%%%%%%%%%%%%%%%%%%%%%%%%%%%%%%%%%%%%%%%%%%%%%%%%%%%%%%%%%%%%%%%%%%%%%
%%%%%%%%%%%%%%%%%%%%%%%%%%%%%%%%%%%%%%%%%%%%%%%%%%%%%%%%%%%%%%%%%%%%%%%%%%%%%%%%
\index{Système du second ordre!définition}
Un système du second ordre est un système régi par une équation 
différentielle du second ordre de forme générale :
\[
\devi{s(t)}{2}+2\xi\omega_0\devi{s(t)}{}+\omega^2_0s(t)=K\omega^2_0e(t)
\]
où $\xi>0$ est le coefficient d'amortissement, $K$ le gain statique et 
$\omega_0>0$ la pulsation propre du système. Cette pulsation est celle de 
l'oscillateur harmonique équivalent sans amortissement ($\xi=0$).
%%%%%%%%%%%%%%%%%%%%%%%%%%%%%%%%%%%%%%%%%%%%%%%%%%%%%%%%%%%%%%%%%%%%%%%%%%%%%%%%
%%%%%%%%%%%%%%%%%%%%%%%%%%%%%%%%%%%%%%%%%%%%%%%%%%%%%%%%%%%%%%%%%%%%%%%%%%%%%%%%
\subsection*{Fonction de transfert d'un système du second ordre}
%%%%%%%%%%%%%%%%%%%%%%%%%%%%%%%%%%%%%%%%%%%%%%%%%%%%%%%%%%%%%%%%%%%%%%%%%%%%%%%%
%%%%%%%%%%%%%%%%%%%%%%%%%%%%%%%%%%%%%%%%%%%%%%%%%%%%%%%%%%%%%%%%%%%%%%%%%%%%%%%%
\index{Système du second ordre!fonction de transfert}
La transformée de Laplace de l'équation différentielle est, lorsque les CI 
sont toutes nulles :
\[
S(p)\left(p^2+2\xi\omega_0p+\omega^2_0\right)=K\omega^2_0E(p).
\]
La fonction de transfert $H(p)$ de ce système est donc donnée par :
%-------------------------------------------------------------------------------
\begin{bequation}[ams align]
    H(p)=\dfrac{S(p)}{E(p)}=\dfrac{K\omega^2_0}{p^2+2\xi\omega_0 p+\omega^2_0}
    \label{eq-2nd_ft}
\end{bequation}
%-------------------------------------------------------------------------------
La forme suivante, pour laquelle on a factorisée par $\omega^2_0$, est 
également très courante:
\[
H(p)=\dfrac{K}{\left(\dfrac{p}{\omega_0}\right)^2
    +\dfrac{2\xi p}{\omega_0}+1}
\]
%%%%%%%%%%%%%%%%%%%%%%%%%%%%%%%%%%%%%%%%%%%%%%%%%%%%%%%%%%%%%%%%%%%%%%%%%%%%%%%%
%%%%%%%%%%%%%%%%%%%%%%%%%%%%%%%%%%%%%%%%%%%%%%%%%%%%%%%%%%%%%%%%%%%%%%%%%%%%%%%%
\subsection*{Pôles de la fonction de transfert du second ordre}
%%%%%%%%%%%%%%%%%%%%%%%%%%%%%%%%%%%%%%%%%%%%%%%%%%%%%%%%%%%%%%%%%%%%%%%%%%%%%%%%
%%%%%%%%%%%%%%%%%%%%%%%%%%%%%%%%%%%%%%%%%%%%%%%%%%%%%%%%%%%%%%%%%%%%%%%%%%%%%%%%
Les pôles de la fonction de transfert sont donnés par les racines du polynôme :
\[
p^2+2\xi\omega_0p+\omega_0^2 = 0
\]
le discriminant de ce polynôme est :
\[
\Delta=4\xi^2\omega^2_0-4\omega_0^2=4\omega_0^2(\xi^2-1)
\]
Les racines de ce polynôme dépendent donc du signe de $\Delta$ 
et ainsi de la valeur du taux d'amortissement $\xi$ définissant les 
différents régimes d'un système du second ordre :
%-------------------------------------------------------------------------------
\begin{itemize}
    \item Régime apériodique pour $\xi>1$
    \item Régime apériodique critique pour $\xi=1$
    \item Régime pseudo-périodique pour $0<\xi<1$
\end{itemize}
%-------------------------------------------------------------------------------
à noter que le cas $\xi=0$ correspond à un régime périodique associé à 
l'oscillateur harmonique au cas de l'oscillateur harmonique.
Le cas $\xi<0$ correspond à un cas divergent par définition (instable) et 
ne sera donc pas traité.
Le~\cref{tab-poles_2nd} résume les différents types de pôles rencontrées dans 
les différents régimes du système du second ordre.
Quelque soit le régime du système du second ordre, on peut écrire la fonction 
de transfert de la façon suivante en utilsant les pôles appropriés:
\[
H(p)=\dfrac{K\omega^2_0}{(p-p_1)(p-p_2)}
\]
\newpage
\thispagestyle{empty}
%-------------------------------------------------------------------------------
\begin{table}[!hb]
    \ra{1.3}
    \centering
    \setlength{\ltmp}{0.3\textwidth}
    \begin{tabular}{@{}P{\ltmp}P{\ltmp}P{\ltmp}@{}}
    \toprule
    Régime  & Pôles   & Carte des pôles                                    \\
    \midrule
        Régime apériodique\[\xi>1\]                                        &
        Deux pôles réels \[p_{1,2}=-\xi\omega_0\pm\omega_0\sqrt{\xi^2-1}\] & 
    {\tikzset{external/export=false}
     \raisebox{-.5\height}{%\begin{center}%
    \begin{tikzpicture}%
    \draw[very thick,-latex] (0,-1.5) -- (0,2.0) node[left]  {Im};
    \draw[very thick,-latex] (-2.5,0) -- (1.0,0) node[below] {Re};
    \node at (-0.5,0) [thick,cross=5pt,blue]                 {};
    \node at (-0.5,0) [blue,below,yshift=-0.5em]             {$p_1$};
    \node at (-1.5,0) [thick,cross=5pt,blue]                 {};
    \node at (-1.5,0) [blue,below,yshift=-0.5em]             {$p_2$};
    \end{tikzpicture}%
%\end{center}% 
}}                  \\
    \midrule
        Régime apériodique critique \[\xi=1\]                              & 
        Un pôle double réel\[p_1=p_2=-\omega_0\]                           & 
    {\tikzset{external/export=false}
     \raisebox{-.5\height}{\begin{tikzpicture}
    \draw[very thick,-latex] (0,-1.5) -- (0,2.0) node[left]  {Im};
    \draw[very thick,-latex] (-2.5,0) -- (1.0,0) node[below] {Re};
    \draw[red,thick] (0,0) circle [radius=1.5];
    \draw[red,thick] (0,0) -- (-0.75,1.30)
    node[midway,xshift=-1em]                                 {$\omega_0$}; 
    \node at (-1.5,0) [thick,cross=5pt,blue]                 {};
    \node at (-1.5,0) [blue,below,yshift=-0.5em]             {$p_1=p_2$};
\end{tikzpicture} 
}}                 \\
    \midrule
        Régime pseudo-périodique \[0<\xi<1\]                               & 
        Deux pôles complexes conjugués \[p_{1,2}=-\alpha\pm j\omega_d\]         
    avec $\alpha=\xi\omega_0$ et $\omega_d=\omega_0\sqrt{1-\xi^2}$         &
    {\tikzset{external/export=false}
     \raisebox{-.5\height}{\begin{tikzpicture}
    \coordinate (o)  at (0.0,0.0);
    \coordinate (p1) at (-1.0,1.0);
    \coordinate (p2) at (-1.0,-1.0);
    \draw[very thick,-latex] (0,-1.5) -- (0,2.0) coordinate (im2) 
    node(im) [left]                                               {Im};
    \draw[very thick,-latex] (-2.5,0) -- (1.0,0) node(re) [below] {Re};
    \draw[blue,very thick,dotted] (p1) -- (p2) 
    node[blue,midway,xshift=-1em,yshift=-0.8em] {$-\alpha$};
    \draw[very thick] (o) -- (p1);
    \draw[very thick] (o) -- (p2);
    \draw[red,very thick,dotted] (p1) -- (0.0,1.0)   
    node[blue,xshift=1.2em,yshift=0em] {$\omega_d$};
    \draw[red,very thick,dotted] (p2) -- (0.0,-1.0) 
    node[blue,xshift=1.2em,yshift=0em] {$-\omega_d$};
    \draw[red,thick] (0,0) circle [radius=1.41421356237];
    \node at (p1) [thick,cross=5pt,blue] {};
    \node at (p1) [blue,left,xshift=-0.5em] {$p_1$};
    \node at (p2) [thick,cross=5pt,blue]    {};
    \node at (p2) [blue,left,xshift=-0.5em] {$p_2$};
    \pic[draw,
         -latex,
         "$\phi$",
         angle radius=0.5cm,
         angle eccentricity=1.5] 
    {angle = im2--o--p1};
    \pic[draw,
         -latex,
         "$\phi$",
         angle eccentricity=1.5] 
    {angle = p2--o--im};
\end{tikzpicture}  
}}              \\
    \bottomrule
    \end{tabular}
    \caption{Pôles de la fonction de transfert d'un système du second 
             ordre selon le régime associé à l'amortissement.
             \label{tab-poles_2nd}}
\end{table}
%-------------------------------------------------------------------------------
\newpage
%%%%%%%%%%%%%%%%%%%%%%%%%%%%%%%%%%%%%%%%%%%%%%%%%%%%%%%%%%%%%%%%%%%%%%%%%%%%%%%%
%%%%%%%%%%%%%%%%%%%%%%%%%%%%%%%%%%%%%%%%%%%%%%%%%%%%%%%%%%%%%%%%%%%%%%%%%%%%%%%%
\subsection*{Réponse impulsionnelle}
%%%%%%%%%%%%%%%%%%%%%%%%%%%%%%%%%%%%%%%%%%%%%%%%%%%%%%%%%%%%%%%%%%%%%%%%%%%%%%%%
%%%%%%%%%%%%%%%%%%%%%%%%%%%%%%%%%%%%%%%%%%%%%%%%%%%%%%%%%%%%%%%%%%%%%%%%%%%%%%%%
\index{Système du second ordre!réponse impulsionnelle}
La réponse impulsionnelle d'un système du second ordre est, dans le domaine 
de Laplace, donnée par :
\[
S(p)=\dfrac{K\omega_0^2}{p^2+2\xi\omega_0p+\omega_0^2}
\]
où $E(p)=1$ dans le cas d'une impulsion de Dirac unitaire\footnote{Nous avons 
ici posé $E_0=1$ pour alléger la notation.}.
%%%%%%%%%%%%%%%%%%%%%%%%%%%%%%%%%%%%%%%%%%%%%%%%%%%%%%%%%%%%%%%%%%%%%%%%%%%%%%%%
\paragraph{Dans le cas $\xi>1$ (régime apériodique)}
%%%%%%%%%%%%%%%%%%%%%%%%%%%%%%%%%%%%%%%%%%%%%%%%%%%%%%%%%%%%%%%%%%%%%%%%%%%%%%%%
La sortie dans le domaine de Laplace s'écrit :
\[
S(p)=\dfrac{K\omega^2_0}{(p-p_1)(p-p_2)}
\]
%\marginnote{$\dfrac{b-a}{(p+a)(p+b)}\rightarrow e^{-at}-e^{-bt}$}[-1em]
La transformée de Laplace inverse de $S(p)$ (c.f ligne 16 de la table des
transformées de Laplace), nous donne la forme générale de la réponse 
impulsionnelle d'un système du second ordre en régime apériodique:
%-------------------------------------------------------------------------------
\begin{bequation}[ams align]
    s(t)&=\dfrac{K\omega^2_0}{p_1-p_2}\left(e^{p_1t}-e^{p_2t}\right) 
\end{bequation}
%-------------------------------------------------------------------------------
les exponentielles étant sans unité, les pôles sont d'unité 
d'inverse d'un temps, posons donc $p_1=-1/\tau_1$ et $p_2=-1/\tau_2$, 
la réponse devient :
%-------------------------------------------------------------------------------
\begin{bequation}[ams align]
    s(t)&=\dfrac{K}{\tau_1-\tau_2}\left(e^{-\frac{t}{\tau_1}}
     -e^{-\frac{t}{\tau_2}}\right)\label{eq-1-1_2nd}
\end{bequation}
%-------------------------------------------------------------------------------
les paramètres $\tau_1$ et $\tau_2$ peuvent être considérés comme 
les constante de temps de deux systèmes du premier ordre fictifs 
placés en série :
%-------------------------------------------------------------------------------
\begin{center}
    \tikzsetnextfilename{2nd_sb_imp-chap_model-ext}
    \input{tikz/2nd_sb_imp-chap_model.tex}
\end{center}
%-------------------------------------------------------------------------------
où $K_1K_2=K$.
Dans le régime apériodique un système du second ordre sera toujours 
considérer comme la mise en cascade de deux systèmes du premier ordre.
\newpage 
\thispagestyle{empty}
%%%%%%%%%%%%%%%%%%%%%%%%%%%%%%%%%%%%%%%%%%%%%%%%%%%%%%%%%%%%%%%%%%%%%%%%%%%%%%%%
\paragraph{Dans le cas $\xi=1$ (régime apériodique critique)}
%%%%%%%%%%%%%%%%%%%%%%%%%%%%%%%%%%%%%%%%%%%%%%%%%%%%%%%%%%%%%%%%%%%%%%%%%%%%%%%%
La sortie dans le domaine de Laplace s'écrit :
\[
S(p)=\dfrac{K\omega^2_0}{(p-p_1)^2}
\]
La transformée de Laplace inverse de $S(p)$ (c.f ligne 8 de la table des 
transformées de Laplace), nous donne la forme générale de la réponse 
impulsionnelle d'un système du second ordre en régime apériodique critique:
%-------------------------------------------------------------------------------
\begin{bequation}[ams align]
    s(t)=K\omega^2_0te^{p_1t}
\end{bequation}
%-------------------------------------------------------------------------------
posons $p_1=-1/\tau$, la réponse devient :
%-------------------------------------------------------------------------------
\begin{bequation}[ams align]
    s(t)=K\omega^2_0 t e^{-\frac{t}{\tau}}\label{eq-1-2_2nd} 
\end{bequation}
%-------------------------------------------------------------------------------
%%%%%%%%%%%%%%%%%%%%%%%%%%%%%%%%%%%%%%%%%%%%%%%%%%%%%%%%%%%%%%%%%%%%%%%%%%%%%%%%
\paragraph{Dans le cas $0<\xi<1$ (régime pseudo-périodique)}
%%%%%%%%%%%%%%%%%%%%%%%%%%%%%%%%%%%%%%%%%%%%%%%%%%%%%%%%%%%%%%%%%%%%%%%%%%%%%%%%
La sortie dans le domaine de Laplace s'écrit :
\[
S(p)=\dfrac{K\omega^2_0}{(p-p_1)(p-p_2)} = 
\dfrac{\omega^2_0}{(p+\xi\omega_0-j\omega_0\sqrt{1-\xi^2})
(p+\xi\omega_0+j\omega_0\sqrt{1-\xi^2})}
\]
en posant $\alpha=\xi\omega_0$ et $\omega_d=\omega_0\sqrt{1-\xi^2}$, 
la sortie $S(p)$ devient :
\[
S(p)=\dfrac{K\omega^2_0}{(p+\alpha-j\omega_d)(p+\alpha+j\omega_d)} = 
     \dfrac{K\omega^2_0}{(p+\alpha)^2+\omega^2_d}=
     \dfrac{K\omega_d}{1-\xi^2}\cdot\dfrac{\omega_d}{(p+\alpha)^2+\omega^2_d}
\]
La transformée de Laplace inverse de $S(p)$ (c.f ligne 30 de la table des 
transformées de Laplace), nous donne la forme générale de la réponse 
impulsionnelle d'un système du second ordre en régime pseudo-périodique :  
%-------------------------------------------------------------------------------
\begin{bequation}[ams align]
    s(t)=\dfrac{K\omega_d}{1-\xi^2}e^{-\xi\omega_0 t}\sin{\omega_d t}
    \label{eq-1-3_2nd} 
\end{bequation}
%-------------------------------------------------------------------------------
%-------------------------------------------------------------------------------
\begin{figure}[!hb]
    \centering
    \tikzsetnextfilename{2nd_pp_imp-chap_model-ext}
    \resizebox{0.6\linewidth}{!}{
    \pgfmathdeclarefunction{func}{1}{%
    \pgfmathparse{%
    ((1./sqrt(1-#1*#1))*exp(-#1*x)*
    sin(deg(x)*sqrt(1-#1*#1))))
    }%
}

\begin{tikzpicture}
    \tikzstyle{signaltmp}=[signalb,thick,domain=0:20]
    \begin{axis}
    [   legend style={draw=none},
        legend pos=outer north east,
        width=0.65\textwidth,
        axis line style = thick,
        xmin=0,
        xmax=20,
        ymin=-1,
        ymax=1,
        xlabel={$t$},
        ylabel={$s(t)$},
        ytick={-1.5,-1.0,-0.5,0,0.5,1,1.5},
        yticklabels={$-1.5KE_0$,$-KE_0$,$-0.5KE_0$,0,$0.5KE_0$,$KE_0$,$1.5KE_0$},
        label style={font=\Large},
    ]
    % amortissement
    \foreach[count=\i from 0,evaluate=\i as \colfrac using \i*100/9] 
    \z in {0.1,0.2,0.3,0.4,0.5,0.6,0.7,0.8,0.9}% 
    {%
        \edef\temp{\noexpand\addplot[signaltmp,col4!\colfrac!col1] {func(\z))};
        \noexpand\addlegendentry{$\xi=\z$}}
        \temp
    }
    \end{axis}
\end{tikzpicture}

    }
    \caption{Réponse impulsionnelle d'un système du second ordre en régime 
             pseudo-périodique pour différentes valeurs du taux d'amortissement 
             $\xi$ (\Cref{eq-1-3_2nd}) avec $\omega_0=1$.\label{fig-2nd_pp_imp}}
\end{figure}
%-------------------------------------------------------------------------------
\newpage
\thispagestyle{empty}
%%%%%%%%%%%%%%%%%%%%%%%%%%%%%%%%%%%%%%%%%%%%%%%%%%%%%%%%%%%%%%%%%%%%%%%%%%%%%%%%
%%%%%%%%%%%%%%%%%%%%%%%%%%%%%%%%%%%%%%%%%%%%%%%%%%%%%%%%%%%%%%%%%%%%%%%%%%%%%%%%
\subsection*{Réponse indicielle\label{subsubsec-2nd_ind}}
%%%%%%%%%%%%%%%%%%%%%%%%%%%%%%%%%%%%%%%%%%%%%%%%%%%%%%%%%%%%%%%%%%%%%%%%%%%%%%%%
%%%%%%%%%%%%%%%%%%%%%%%%%%%%%%%%%%%%%%%%%%%%%%%%%%%%%%%%%%%%%%%%%%%%%%%%%%%%%%%%
\index{Système du second ordre!réponse indicielle}
La réponse indicielle d'un système du second ordre est, dans le domaine 
de Laplace, donnée par :
\[
S(p)=\dfrac{K\omega_0^2}{p^2+2\xi\omega_0p+\omega_0^2}\cdot\dfrac{E_0}{p}
\]
où $E(p)=\dfrac{E_0}{p}$ est une entrée échelon.

\'Etudions la forme analytique des réponses indicielles pour les différents 
régimes du système du second ordre. 
%%%%%%%%%%%%%%%%%%%%%%%%%%%%%%%%%%%%%%%%%%%%%%%%%%%%%%%%%%%%%%%%%%%%%%%%%%%%%%%%
\paragraph{Dans le cas $\xi>1$ (régime apériodique)} 
%%%%%%%%%%%%%%%%%%%%%%%%%%%%%%%%%%%%%%%%%%%%%%%%%%%%%%%%%%%%%%%%%%%%%%%%%%%%%%%%
La sortie dans le domaine de Laplace s'écrit :
\[
S(p)=\dfrac{K\omega^2_0}{(p-p_1)(p-p_2)}\cdot\dfrac{E_0}{p}
\]
La transformée de Laplace inverse de $S(p)$ (c.f ligne 19 de la table des 
transformées de Laplace), nous donne la forme générale de la 
réponse indicielle d'un système du second ordre en régime apériodique:
%\[
%s(t)=\dfrac{KE_0\omega^2_0}{p_1p_2}\left(1+\dfrac{1}{p_1-p_2}
%(p_2e^{p_1t}-p_1e^{p_2t})\right)
%\]
%et en réarrangant les termes : 
%-------------------------------------------------------------------------------
\begin{bequation}[ams align]
s(t)&=KE_0\left(1+\dfrac{1}{p_1-p_2}\left(p_2e^{p_1t}-p_1e^{p_2t}\right)\right)
\end{bequation}
%-------------------------------------------------------------------------------
posons $p_1=-1/\tau_1$ et $p_2=-1/\tau_2$, la réponse devient :
%-------------------------------------------------------------------------------
\begin{bequation}[ams align]
s(t)=KE_0\left(1+\dfrac{1}{\tau_1-\tau_2}
	 \left(\tau_2e^{-\frac{t}{\tau_2}}-\tau_1e^{-\frac{t}{\tau_1}}\right)
	 \label{eq-2-1_2nd}\right) 
\end{bequation}
%-------------------------------------------------------------------------------
Nous pouvons à nouveau envisager cette réponse comme la réponse 
de deux systèmes du premier ordre en série.
%%%%%%%%%%%%%%%%%%%%%%%%%%%%%%%%%%%%%%%%%%%%%%%%%%%%%%%%%%%%%%%%%%%%%%%%%%%%%%%%
\paragraph{Dans le cas $\xi=1$ (régime apériodique critique)} 
%%%%%%%%%%%%%%%%%%%%%%%%%%%%%%%%%%%%%%%%%%%%%%%%%%%%%%%%%%%%%%%%%%%%%%%%%%%%%%%%
La sortie dans le domaine de Laplace s'écrit :
\[
S(p)=\dfrac{K\omega^2_0}{(p-p_1)^2}\cdot\dfrac{E_0}{p}
\]
La transformée de Laplace inverse de $S(p)$ (c.f ligne 14 de la table des
transformées de Laplace) , nous donne la forme générale de la réponse 
indicielle d'un système du second ordre en régime apériodique critique:
\[
s(t)=\dfrac{KE_0\omega^2_0}{p^2_1}\left(1-(1-p_1t)e^{p_1t}\right)
\]
%-------------------------------------------------------------------------------
\begin{bequation}[ams align]
    s(t)=KE_0\left(1-e^{p_1t}+p_1te^{p_1t}\right)
\end{bequation}
%-------------------------------------------------------------------------------
en posant $p_1=-\dfrac{1}{\tau}$, on obtient:
%-------------------------------------------------------------------------------
\begin{bequation}[ams align]
    s(t)=KE_0\left(1-e^{-\frac{t}{\tau}}-\dfrac{t}{\tau}
    e^{-\frac{t}{\tau}}\right)\label{eq-2-2_2nd} 
\end{bequation}
%-------------------------------------------------------------------------------
\newpage
\thispagestyle{empty}
%-------------------------------------------------------------------------------
%%%%%%%%%%%%%%%%%%%%%%%%%%%%%%%%%%%%%%%%%%%%%%%%%%%%%%%%%%%%%%%%%%%%%%%%%%%%%%%%
\paragraph{Dans le cas $0<\xi<1$ (régime pseudo-périodique)} 
%%%%%%%%%%%%%%%%%%%%%%%%%%%%%%%%%%%%%%%%%%%%%%%%%%%%%%%%%%%%%%%%%%%%%%%%%%%%%%%%
La sortie $S(p)$ dans le domaine de Laplace s'écrit :
%\[
%S(p)=\dfrac{\omega^2_0}{(p-p_1)(p-p_2)}\cdot\dfrac{1}{p} 
%= \dfrac{\omega^2_0}{(p-\xi\omega_0-j\omega_0\sqrt{1-\xi^2})
%(p-\xi\omega_0+j\omega_0\sqrt{1-\xi^2})}\cdot\dfrac{1}{p}
%\]
%en posant $\alpha=\xi\omega_0$ et $\omega_d=\omega_0\sqrt{1-\xi^2}$ ,
%$S(p)$ devient :
%\[
%S(p)=\dfrac{\omega^2_0}{(p-\alpha-j\omega_d)(p-\alpha+j\omega_d)}
%\cdot\dfrac{1}{p} = \dfrac{\omega^2_0}{(p-\alpha)^2+\omega^2_d}
%\cdot\dfrac{1}{p}
%\]
\[
S(p)=\dfrac{K\omega^2_0}{(p+\alpha)^2+\omega^2_d}\cdot\dfrac{E_0}{p}
\]
où l'on a posé $\alpha=\xi\omega_0$ et $\omega_d=\omega_0\sqrt{1-\xi^2}$.

La forme générale de la réponse indicielle d'un système 
du second ordre en régime pseudo-périodique s'écrit :
%-------------------------------------------------------------------------------
\begin{bequation}[ams align]
    s(t) = KE_0 \left( 1 - 
           \dfrac{1}{\sqrt{1-\xi^2}} 
           e^{-\xi\omega_0 t}
           \sin{(\omega_d t+\phi)}\right)\label{eq-2-3_2nd} 
\end{bequation}
%-------------------------------------------------------------------------------
%La valeur finale est obtenue pour 
%\[
%s(\infty)=\lim\limits_{p\to 0}pS(p)
%\]
%-------------------------------------------------------------------------------
\begin{figure}[!t]
    \centering
    \tikzsetnextfilename{2nd_pp-chap_model-ext}
    \resizebox{0.6\linewidth}{!}{
    \input{tikz/2nd_pp-chap_model.tex}
    }
    \caption{Réponse indicielle d'un système du second ordre en régime 
             pseudo-périodique pour différentes valeurs du taux d'amortissement 
             $\xi$ (\Cref{eq-1er_ramp}) avec $\omega_0=1$. \label{fig-2nd_pp}}
\end{figure}
%-------------------------------------------------------------------------------
Il est maintenant possible d'interpréter les différentes grandeurs 
introduites. En effet, cette réponse a la forme d'une sinuso\"ide 
de pulsation $\omega_d$ (dite pseudo-pulsation), de phase $\phi$ et 
amortie par une exponentielle décroissante dépendant de $\xi$.
La~\cref{fig-2nd_pp} présente cette réponse indicielle du régime 
pseudo-périodique pour différentes valeurs du taux d'amortissement pour 
une pulsation propre $\omega_0=1$. Nous constatons que comme attendu, 
l'amplitude des oscillations augmente lorsque le taux d'amortissement diminue.
%\newpage
\thispagestyle{empty}
%%%%%%%%%%%%%%%%%%%%%%%%%%%%%%%%%%%%%%%%%%%%%%%%%%%%%%%%%%%%%%%%%%%%%%%%%%%%%%%%
\paragraph{Dépassement et temps de réponse à 5\%}
%%%%%%%%%%%%%%%%%%%%%%%%%%%%%%%%%%%%%%%%%%%%%%%%%%%%%%%%%%%%%%%%%%%%%%%%%%%%%%%%
Certaines propriétés de la réponse indicielle dans le régime pseudo-périodique 
sont fortement dépendantes du taux d'amortissement. C'est le cas du 
dépassement et du temps de réponse. La~\cref{fig-2nd_depassement_1} présente 
la réponse à un échelon unitaire pour un amortissement de $\xi=0.2$, 
on observe que les dépassements succésifs sont de moins en moins important. 
Pour déterminer la relation entre le dépassement et le taux d'amortissement, 
il nous faut d'abord déterminer le temps du premier maximum $t_1$.
%-------------------------------------------------------------------------------
\begin{figure}[!h]
    \centering
    \tikzsetnextfilename{2nd_depassement_1-chap_model-ext}
    \resizebox{0.5\linewidth}{!}{
    \input{tikz/2nd_depassement_1-chap_model.tex}
    }
    \caption{Définition du dépassement observé dans le cas de la réponse 
             indicielle en régime pseudo-périodique d'un système du second 
             ordre. Les deux enveloppes correspondent aux exponentielles 
             décroissantes $1+e^{-\alpha t}$ et $1-e^{-\alpha t}$. 
             \label{fig-2nd_depassement_1}}
\end{figure}
%-------------------------------------------------------------------------------
%-------------------------------------------------------------------------------
\begin{figure}[!h]
    \centering
    \tikzsetnextfilename{2nd_depassement_2-chap_model-ext}
    \resizebox{0.6\linewidth}{!}{
    \begin{tikzpicture}
    \pgfplotsset{signaltmp/.style={signaln,domain=0.0001:1,
                              unbounded coords=jump}
                }
    \pgfmathsetmacro{\pi}{3.141592653589793}     % dépassement d4 
    \begin{axis}
    [   xmode=log,
        ymode=log,
        width=0.65\textwidth,
        axis line style = thick,
        xmin=0.01,
        xmax=1,
        ymin=0.01,
        ymax=1.0,
        xlabel={$\xi$},
        ylabel={$D_k$},
        label style={font=\Large},
        grid=both,
        grid style={line width=.4pt, draw=black},
        major grid style={line width=.4pt,draw=black},
        legend style={draw=none,font=\normalsize},
        legend pos=outer north east,
        label style={font=\Large},
        legend cell align={left},
    ]
    \addplot[signaltmp,vtcol1] {exp(-(x*\pi)/(sqrt(1-x*x)))};
    \addplot[signaltmp,vtcol2] {exp(-(2*x*\pi)/(sqrt(1-x*x)))};
    \addplot[signaltmp,vtcol3] {exp(-(3*x*\pi)/(sqrt(1-x*x)))};
    \addplot[signaltmp,vtcol4] {exp(-(4*x*\pi)/(sqrt(1-x*x)))};
    \legend{$k=1$,$k=2$,$k=3$,$k=4$}
    \end{axis}
\end{tikzpicture}


    }
    \caption{Variation de la valeur $D_k$ du k-ème dépassement en fonction 
             du taux d'amortissement $\xi$. \label{fig-2nd_depassement_2}}
\end{figure}
%-------------------------------------------------------------------------------
Pour celà il suffit de déterminer le temps pour lequel la dérivée 
du signal $s(t)$ s'annule. On calcul alors un temps $t_1$ à $T_d/2$ 
où $T_d$ est la pseudo-période défini à partir de la 
pseudo-pulsation $\omega_d$. 
On a alors :
%-------------------------------------------------------------------------------
\begin{align*}
T_d=\dfrac{2\pi}{\omega_d}\\
t_1 = \dfrac{\pi}{\omega_d}
\end{align*}
%-------------------------------------------------------------------------------
Formellement, le premier dépassement est défini par :
\[
D_1=\left|\dfrac{s(t_1)-s(\infty)}{s(\infty)-s(0)}\right|
\]
où $s(0)$, $s(\infty)$ et $s(t_1)$ sont respectivement la valeur initiale, 
la valeur finale et la valeur du premier maximum du signal.

La valeur $s(t_1)$ s'obtient en remplaçant la valeur de $t_1$ dans la 
forme analytique de la réponse indicielle du régime 
pseudo-périodique (\Cref{eq-2-3_2nd}) :
%-------------------------------------------------------------------------------
\begin{align*}
    s(t_1) &= KE_0\left(1 - 
    \dfrac{1}{\sqrt{1-\xi^2}} 
    e^{-\alpha t_1}\sin{(\omega_d t_1+\phi)}\right) \\
    s(t_1) &= KE_0\left(1 - \dfrac{1}{\sqrt{1-\xi^2}} 
    e^{-\alpha\pi/\omega_d}\sin{(\pi+\phi)}\right) \\
    s(t_1) &= KE_0\left(1 + e^{-\alpha\pi/\omega_d}\right)
\end{align*}
%-------------------------------------------------------------------------------
Le dépassement est donc donné par l'expression : 
%-------------------------------------------------------------------------------
\begin{bequation}[ams align]
    D=e^{-\dfrac{\xi\pi}{\sqrt{1-\xi^2}}}
\end{bequation}
%-------------------------------------------------------------------------------
et le $k$-ème dépassement $D_k$ est lui donné par :
%-------------------------------------------------------------------------------
\begin{bequation}[ams align]
    D_k=e^{-\dfrac{k\xi\pi}{\sqrt{1-\xi^2}}}
\end{bequation}
%-------------------------------------------------------------------------------
La~\cref{fig-2nd_depassement_2} présente cette relation entre le 
dépassement  et le taux d'amortissement. Il est possible d'utiliser 
cette figure comme un abaque\footnote{Les abaques sont très répandus 
en automatique. Ils permettent de s'affranchir de nombreux claculs 
en lisant des valeurs directement sur un graphique.} 
facilitant le calcul du dépassement connaissant le taux d'amortissement 
et inversement.
\newline
Il n'existe pas de relation analytique simple pour déterminer 
le temps de réponse à 5\% (c.f définition donnée par la~\cref{fig-2nd_t5pc}) 
en fonction du taux d'amortissement. 
La~\cref{fig-2nd_temps_reponse} présente la variation du temps de 
réponse à 5\% réduit à la pulsation (c.-à-d. $\omega_0\cdot t_{5\%}$ ) 
en fonction du taux d'amortissement $\xi$. On observe un minimum du 
temps de réponse pour $\xi\sim 0.7$
%-------------------------------------------------------------------------------
\begin{figure}
\centering
    \tikzsetnextfilename{rapidite_tr_2ndordre-chap_model-ext}
    \input{tikz/rapidite_tr_2ndordre-chap_model.tex}
    \caption{Temps de réponse à 5\% réduit en fonction du taux 
             d'amortissement $\xi$. Le minimum est atteint pour $\xi\sim0.7$ 
             pour lequel $\omega_0\cdot t_{5\%}\sim3$.
             \label{fig-2nd_temps_reponse}}
\end{figure}
%-------------------------------------------------------------------------------
