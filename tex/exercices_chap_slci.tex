%%%%%%%%%%%%%%%%%%%%%%%%%%%%%%%%%%%%%%%%%%%%%%%%%%%%%%%%%%%%%%%%%%%%%%%%%%%%%%%%
%%%%%%%%%%%%%%%%%%%%%%%%%%%%%%%%%%%%%%%%%%%%%%%%%%%%%%%%%%%%%%%%%%%%%%%%%%%%%%%%
\subsection*{Signaux}
%%%%%%%%%%%%%%%%%%%%%%%%%%%%%%%%%%%%%%%%%%%%%%%%%%%%%%%%%%%%%%%%%%%%%%%%%%%%%%%%
%%%%%%%%%%%%%%%%%%%%%%%%%%%%%%%%%%%%%%%%%%%%%%%%%%%%%%%%%%%%%%%%%%%%%%%%%%%%%%%%
%%%%%%%%%%%%%%%%%%%%%%%%%%%%%%%%%%%%%%%%%%%%%%%%%%%%%%%%%%%%%%%%%%%%%%%%%%%%%%%%
%%%%%%%%%%%%%%%%%%%%%%%%%%%%%%%%%%%%%%%%%%%%%%%%%%%%%%%%%%%%%%%%%%%%%%%%%%%%%%%%
\exercice{Décomposition d'un signal~\facile}
%%%%%%%%%%%%%%%%%%%%%%%%%%%%%%%%%%%%%%%%%%%%%%%%%%%%%%%%%%%%%%%%%%%%%%%%%%%%%%%%
%%%%%%%%%%%%%%%%%%%%%%%%%%%%%%%%%%%%%%%%%%%%%%%%%%%%%%%%%%%%%%%%%%%%%%%%%%%%%%%%
Soit le signal d'entrée $e(t)$ définie par le graphe ci-contre.

%%%%%%%%%%%%%%%%%%%%%%%%%%%%%%%%%%%%%%%%%%%%%%%%%%%%%%%%%%%%%%%%%%%%%%%%%%%%%%%%
\question{Décomposer $e(t)$ à l'aide de fonctions échelons.}
%%%%%%%%%%%%%%%%%%%%%%%%%%%%%%%%%%%%%%%%%%%%%%%%%%%%%%%%%%%%%%%%%%%%%%%%%%%%%%%%
%-------------------------------------------------------------------------------
\begin{marginfigure}
    \centering
    \tikzsetnextfilename{fonction_triangle2-chap-slci-ext}
    \begin{tikzpicture}[baseline=0]
   \begin{axis}[
        height=4cm,
        width=6cm,
        axis x line=center,
        axis y line=center,
        xmin=-1,
        xmax=5,
        ymin=-1.5,
        ymax=1.5,
        xlabel={$t$},
        ylabel={$e(t)$},
        xlabel style={below right},
        ylabel style={left},
        yticklabels={-1,1},
        ytick={-1,1},
        y tick label style={left},
        xticklabels={1,2,3,4},
        xtick={1,2,3,4},
        x tick label style={below},
        ]
        \addplot [very thick,col1,domain=-1:0, samples=50]{0.01};
        \addplot [very thick,col1,domain=0:1, samples=50]{x};
        \addplot [very thick,col1,domain=1:3, samples=50]{2-x};
        \addplot [very thick,col1,domain=3:4, samples=50]{x-4};
        \addplot [very thick,col1,domain=4:4.9, samples=50]{0.01};
        \end{axis}
\end{tikzpicture}

\end{marginfigure}
%-------------------------------------------------------------------------------

%%%%%%%%%%%%%%%%%%%%%%%%%%%%%%%%%%%%%%%%%%%%%%%%%%%%%%%%%%%%%%%%%%%%%%%%%%%%%%%%
\question{Donner la transformée de Laplace $E(p)$ de ce signal.}
%%%%%%%%%%%%%%%%%%%%%%%%%%%%%%%%%%%%%%%%%%%%%%%%%%%%%%%%%%%%%%%%%%%%%%%%%%%%%%%%

%%%%%%%%%%%%%%%%%%%%%%%%%%%%%%%%%%%%%%%%%%%%%%%%%%%%%%%%%%%%%%%%%%%%%%%%%%%%%%%%
%%%%%%%%%%%%%%%%%%%%%%%%%%%%%%%%%%%%%%%%%%%%%%%%%%%%%%%%%%%%%%%%%%%%%%%%%%%%%%%%
\exercice{Décomposition d'un signal parabolique~\moyen}
%%%%%%%%%%%%%%%%%%%%%%%%%%%%%%%%%%%%%%%%%%%%%%%%%%%%%%%%%%%%%%%%%%%%%%%%%%%%%%%%
%%%%%%%%%%%%%%%%%%%%%%%%%%%%%%%%%%%%%%%%%%%%%%%%%%%%%%%%%%%%%%%%%%%%%%%%%%%%%%%%

%%%%%%%%%%%%%%%%%%%%%%%%%%%%%%%%%%%%%%%%%%%%%%%%%%%%%%%%%%%%%%%%%%%%%%%%%%%%%%%%
\question{Décomposer $e(t)$ à l'aide de fonctions échelons.}
%%%%%%%%%%%%%%%%%%%%%%%%%%%%%%%%%%%%%%%%%%%%%%%%%%%%%%%%%%%%%%%%%%%%%%%%%%%%%%%%
\[
    e(t)=t^2\left(u(t)-u(t-1)\right)+(t-2)^2\left(u(t-1)-u(t-2)\right)
\]
%-------------------------------------------------------------------------------
\begin{marginfigure}
    \centering
    \tikzsetnextfilename{fonction_parabole-chap-slci-ext}
    \input{tikz/fonction_parabole-chap-slci.tex}
\end{marginfigure}
%-------------------------------------------------------------------------------
%%%%%%%%%%%%%%%%%%%%%%%%%%%%%%%%%%%%%%%%%%%%%%%%%%%%%%%%%%%%%%%%%%%%%%%%%%%%%%%%
\question{Donner la transformée de Laplace $E(p)$ de ce signal.}
%%%%%%%%%%%%%%%%%%%%%%%%%%%%%%%%%%%%%%%%%%%%%%%%%%%%%%%%%%%%%%%%%%%%%%%%%%%%%%%%

%%%%%%%%%%%%%%%%%%%%%%%%%%%%%%%%%%%%%%%%%%%%%%%%%%%%%%%%%%%%%%%%%%%%%%%%%%%%%%%%
%%%%%%%%%%%%%%%%%%%%%%%%%%%%%%%%%%%%%%%%%%%%%%%%%%%%%%%%%%%%%%%%%%%%%%%%%%%%%%%%
\exercice{Décompositon d'un signal et transformée de Laplace~\difficile}
%%%%%%%%%%%%%%%%%%%%%%%%%%%%%%%%%%%%%%%%%%%%%%%%%%%%%%%%%%%%%%%%%%%%%%%%%%%%%%%%
%%%%%%%%%%%%%%%%%%%%%%%%%%%%%%%%%%%%%%%%%%%%%%%%%%%%%%%%%%%%%%%%%%%%%%%%%%%%%%%%

On sollicite un système linéaire continu et invariant par deux signaux $e_1(t)$ 
et $e_2(t)$ (d'amplitude maximale et de durée unité) définits 
par les graphes ci-contre.

%%%%%%%%%%%%%%%%%%%%%%%%%%%%%%%%%%%%%%%%%%%%%%%%%%%%%%%%%%%%%%%%%%%%%%%%%%%%%%%%
\question{Décomposer $e_1(t)$ et $e_2(t)$ à l'aide de la fonction échelon 
unitaire (i.e $u(t)$).}
%%%%%%%%%%%%%%%%%%%%%%%%%%%%%%%%%%%%%%%%%%%%%%%%%%%%%%%%%%%%%%%%%%%%%%%%%%%%%%%%
%-------------------------------------------------------------------------------
\begin{marginfigure}
    \centering
    \tikzsetnextfilename{fonction_e1-chap-slci-ext}
    \input{tikz/fonction_e1-chap-slci.tex}
\end{marginfigure}
%-------------------------------------------------------------------------------


%%%%%%%%%%%%%%%%%%%%%%%%%%%%%%%%%%%%%%%%%%%%%%%%%%%%%%%%%%%%%%%%%%%%%%%%%%%%%%%%
\question{Déterminer les fonctions $E_1(p)$ et $E_2(p)$ les transformées de 
Laplace respectives de $e_1(t)$ et $e_2(t)$.}
%%%%%%%%%%%%%%%%%%%%%%%%%%%%%%%%%%%%%%%%%%%%%%%%%%%%%%%%%%%%%%%%%%%%%%%%%%%%%%%%
%-------------------------------------------------------------------------------
\begin{marginfigure}
    \centering
    \tikzsetnextfilename{fonction_e2-chap-slci-ext}
    \begin{tikzpicture}
    \begin{axis}
    [
        height=4cm,
        width=6cm,
        axis x line=center,
        axis y line=center,
        xmin=-1,
        xmax=1.5,
        ymin=-0.5,
        ymax=1.5,
        xlabel={$t$},
        ylabel={$e_2(t)$},
        xlabel style={below right},
        ylabel style={above left},
        yticklabels={0,1},
        ytick={0,1},
        y tick label style={anchor=east},
        xticklabels={1},
        xtick={1},
        x tick label style={below},
    ]
        \addplot [very thick,col1,domain=-1:0, samples=51]{0};
        \addplot [very thick,col1,domain=0:1, samples=51]{x};
        \addplot [very thick,col1,domain=1:1.45, samples=51]{0};
        \draw[very thick,col1] (axis cs:1,1) -- (axis cs:1,0);
    \end{axis}
\end{tikzpicture}

\end{marginfigure}
%-------------------------------------------------------------------------------

%%%%%%%%%%%%%%%%%%%%%%%%%%%%%%%%%%%%%%%%%%%%%%%%%%%%%%%%%%%%%%%%%%%%%%%%%%%%%%%%
%\question{Généraliser ces entrées pour une amplitude maximale $E_0$ 
%et une durée du signal $\tau$ quelconques.} 
%%%%%%%%%%%%%%%%%%%%%%%%%%%%%%%%%%%%%%%%%%%%%%%%%%%%%%%%%%%%%%%%%%%%%%%%%%%%%%%%

On sollicite maintenant le système par la fonction $e(t)$ constituée 
des signaux $e_1(t)$ et $e_2(t-1)$. 

%-------------------------------------------------------------------------------
%\begin{marginfigure}
%    \centering
%    \tikzsetnextfilename{fonction_e3-chap-slci-ext}
%    \input{tikz/fonction_e3-chap-slci.tex}
%\end{marginfigure}
%-------------------------------------------------------------------------------

%%%%%%%%%%%%%%%%%%%%%%%%%%%%%%%%%%%%%%%%%%%%%%%%%%%%%%%%%%%%%%%%%%%%%%%%%%%%%%%%
\question{Déterminer à partir des résultats précédents la transformée 
de Laplace $E(p)$.}
%%%%%%%%%%%%%%%%%%%%%%%%%%%%%%%%%%%%%%%%%%%%%%%%%%%%%%%%%%%%%%%%%%%%%%%%%%%%%%%%


%%%%%%%%%%%%%%%%%%%%%%%%%%%%%%%%%%%%%%%%%%%%%%%%%%%%%%%%%%%%%%%%%%%%%%%%%%%%%%%%
%%%%%%%%%%%%%%%%%%%%%%%%%%%%%%%%%%%%%%%%%%%%%%%%%%%%%%%%%%%%%%%%%%%%%%%%%%%%%%%%
\exercice{L'impulsion de Dirac approchée~\difficile}
%%%%%%%%%%%%%%%%%%%%%%%%%%%%%%%%%%%%%%%%%%%%%%%%%%%%%%%%%%%%%%%%%%%%%%%%%%%%%%%%
%%%%%%%%%%%%%%%%%%%%%%%%%%%%%%%%%%%%%%%%%%%%%%%%%%%%%%%%%%%%%%%%%%%%%%%%%%%%%%%%
L'impulsion de Dirac peut être approchée par la fonction porte 
$\delta_a(t)$ définie par le graphe ci-contre.

%%%%%%%%%%%%%%%%%%%%%%%%%%%%%%%%%%%%%%%%%%%%%%%%%%%%%%%%%%%%%%%%%%%%%%%%%%%%%%%%
\question{Décomposer $\delta_a(t)$ à l'aide de fonctions échelons.}
%%%%%%%%%%%%%%%%%%%%%%%%%%%%%%%%%%%%%%%%%%%%%%%%%%%%%%%%%%%%%%%%%%%%%%%%%%%%%%%%
%-------------------------------------------------------------------------------
\begin{marginfigure}
    \centering
    \tikzsetnextfilename{fonction_porte-chap-slci-ext}
    \begin{tikzpicture}[baseline=0]
   \begin{axis}[
        height=4cm,
        width=5cm,
        axis x line=center,
        axis y line=center,
        xmin=-1,
        xmax=5,
        ymin=-0.5,
        ymax=2.0,
        xlabel={$t$},
        ylabel={$\delta_a(t)$},
        xlabel style={below right},
        ylabel style={left},
        yticklabels={$\dfrac{1}{a}$},
        ytick={1},
        y tick label style={anchor=east},
        xticklabels={$a$},
        xtick={2},
        x tick label style={anchor=north},
        ]
        \addplot [very thick,col1,const plot] coordinates 
        {(-1,0.01) (0,0.01) (0,1) (2,1)  (2,0.01) (5,0.01) };
        \end{axis}
\end{tikzpicture}

\end{marginfigure}
%-------------------------------------------------------------------------------

%%%%%%%%%%%%%%%%%%%%%%%%%%%%%%%%%%%%%%%%%%%%%%%%%%%%%%%%%%%%%%%%%%%%%%%%%%%%%%%%
\question{Donner sa transformée de Laplace en fonction du paramètre $a$.}
%%%%%%%%%%%%%%%%%%%%%%%%%%%%%%%%%%%%%%%%%%%%%%%%%%%%%%%%%%%%%%%%%%%%%%%%%%%%%%%%

%%%%%%%%%%%%%%%%%%%%%%%%%%%%%%%%%%%%%%%%%%%%%%%%%%%%%%%%%%%%%%%%%%%%%%%%%%%%%%%%
\question{En considérant la limite $a\rightarrow0$, montrer que la transformée
          de Laplace de $\delta_a(t)$ tend bien vers 1.}
%%%%%%%%%%%%%%%%%%%%%%%%%%%%%%%%%%%%%%%%%%%%%%%%%%%%%%%%%%%%%%%%%%%%%%%%%%%%%%%%

%%%%%%%%%%%%%%%%%%%%%%%%%%%%%%%%%%%%%%%%%%%%%%%%%%%%%%%%%%%%%%%%%%%%%%%%%%%%%%%%
%%%%%%%%%%%%%%%%%%%%%%%%%%%%%%%%%%%%%%%%%%%%%%%%%%%%%%%%%%%%%%%%%%%%%%%%%%%%%%%%
\subsection*{\'Equations différentielles}
%%%%%%%%%%%%%%%%%%%%%%%%%%%%%%%%%%%%%%%%%%%%%%%%%%%%%%%%%%%%%%%%%%%%%%%%%%%%%%%%
%%%%%%%%%%%%%%%%%%%%%%%%%%%%%%%%%%%%%%%%%%%%%%%%%%%%%%%%%%%%%%%%%%%%%%%%%%%%%%%%

%%%%%%%%%%%%%%%%%%%%%%%%%%%%%%%%%%%%%%%%%%%%%%%%%%%%%%%%%%%%%%%%%%%%%%%%%%%%%%%%
%%%%%%%%%%%%%%%%%%%%%%%%%%%%%%%%%%%%%%%%%%%%%%%%%%%%%%%%%%%%%%%%%%%%%%%%%%%%%%%%
\exercice{Décharge d'un condensateur}
%%%%%%%%%%%%%%%%%%%%%%%%%%%%%%%%%%%%%%%%%%%%%%%%%%%%%%%%%%%%%%%%%%%%%%%%%%%%%%%%
%%%%%%%%%%%%%%%%%%%%%%%%%%%%%%%%%%%%%%%%%%%%%%%%%%%%%%%%%%%%%%%%%%%%%%%%%%%%%%%%
On reprend la mise en équation de la décharge d'un condensateur 
dans un circuit RC comme présentée dans ce chapitre. L'équation différentielle
de la charge $q(t)$ au borne du condensateur est donnée par :
\[
    RC\devi{q(t)}{}+q(t)=0
\]
Le condensateur est initialement chargé $q(0)=q_0$. On remarquera que le 
produit $RC$ est de dimension d'un temps. On notera cette constante $\tau=RC$.

%%%%%%%%%%%%%%%%%%%%%%%%%%%%%%%%%%%%%%%%%%%%%%%%%%%%%%%%%%%%%%%%%%%%%%%%%%%%%%%%
\question{Déterminer la transformée de Laplace $Q(p)$ de la charge au borne du 
condensateur $q(t)$.}
%%%%%%%%%%%%%%%%%%%%%%%%%%%%%%%%%%%%%%%%%%%%%%%%%%%%%%%%%%%%%%%%%%%%%%%%%%%%%%%%
\[
    \tau(pQ(p)-q_0)+Q(p)=0
\]
\[
    Q(p)(\tau p+1)=q_0\tau
\]
\[
    Q(p)=q_0\dfrac{\tau}{\tau p+1}
\]
%%%%%%%%%%%%%%%%%%%%%%%%%%%%%%%%%%%%%%%%%%%%%%%%%%%%%%%%%%%%%%%%%%%%%%%%%%%%%%%%
\question{Déterminer alors la fonction $q(t)$ solution de l'équation 
différentielle de la décharge d'un condensateur.}
%%%%%%%%%%%%%%%%%%%%%%%%%%%%%%%%%%%%%%%%%%%%%%%%%%%%%%%%%%%%%%%%%%%%%%%%%%%%%%%%
\[
    Q(p)=q_0\dfrac{1}{p+\dfrac{1}{\tau}}
\]
\[
    q(t)=q_0e^{-\dfrac{t}{\tau}}
\]


