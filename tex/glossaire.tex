%%%%%%%%%%%%%%%%%%%%%%%%%%%%%%%%%%%%%%%%%%%%%%%%%%%%%%%%%%%%%%%%%%%%%%%%%%%%%%%%
%%%%%%%%%%%%%%%%%%%%%%%%%%%%%%%%%%%%%%%%%%%%%%%%%%%%%%%%%%%%%%%%%%%%%%%%%%%%%%%%
%%%%%%%%%%%%%%%%%%%%%%%%%%%%%%%%%%%%%%%%%%%%%%%%%%%%%%%%%%%%%%%%%%%%%%%%%%%%%%%%
%%%%%%%%%%%%%%%%%%%%%%%%%%%%%%%%%%%%%%%%%%%%%%%%%%%%%%%%%%%%%%%%%%%%%%%%%%%%%%%%
%glossaire
\newglossaryentry{tempsdereponse}%
{% 
  name={Temps de réponse},% 
  description={Le temps de réponse est le temps écoulé entre le moment de 
  la sollicitation et le moment de sa réponse.
  }
}
\newglossaryentry{signal}%
{% 
  name={Signal},% 
  description={Un signal est la grandeur qui porte l'information. Elle peut être
  de différente nature selon le domaine d'application.
  }
}
\newglossaryentry{fonctiondetranfert}%
{% 
  name={Fonction de Transfert},% 
  description={La fonction de transfert défini la relation
  entre l'entrée et la sortie d'un système linéaire.
  }
}
\newglossaryentry{regulation}%
{% 
  name={Régulation},% 
  description={La régulation est un cas particulier d'asservissement 
  consistant à garder une consigne constante en présence de perturbation.}
}
\newglossaryentry{asservissement}%
{% 
  name={Asservissement},% 
  description={L'asservissment consiste à contrôler un système dynamique
  pour que sa réponse temporelle suive une consigne variable au cours du 
  temps.
  }
}
\newglossaryentry{schemabloc}%
{% 
  name={Schéma-bloc},% 
  description={Un schéma-bloc ou un schéma fonctionnel permet de 
  représenter graphiquement des relations mathématiques entre des grandeurs
  dans les domaines temporel ou de Laplace.
  }
}
\newglossaryentry{precision}%
{% 
  name={Précision},% 
  description={La précision est une performance recherchée par les automaticiens
  pour le système qu'ils souhaitent asservir. 
  Un système automatique est précis si la différence (l'écart) entre la 
  consigne et la sortie est nulle. Il est possible de définir une précision 
  statique (en régime permanent) ou une précision dynamique (l'écart instantané)
  }
}
\newglossaryentry{rapidite}%
{% 
  name={Rapidité},% 
  description={La rapidité est une performance recherchée par les automaticiens 
  pour le système qu'ils souhaitent asservir. 
  La rapidité peut être quantifier par le temps de réponse 
  ou le temps de montée. Un système est d'autant plus rapide que ces temps sont
  brefs.
  }
}
\newglossaryentry{stabilite}%
{% 
  name={Stabilité},% 
  description={Un système est dit stable  
  }
}
\newglossaryentry{depassement}%
{% 
  name={Dépassement},% 
  description={  
  }
}
\newglossaryentry{correcteur}%
{% 
  name={Correcteur},% 
  description={
  }
}
\newglossaryentry{tempsdemontee}%
{% 
  name={Temps de montée},% 
  description={
  }
}
\newglossaryentry{controle}%
{% 
  name={Contrôle},% 
  description={
  }
}
\newglossaryentry{systemelineaire}%
{% 
  name={Système Linéaire},% 
  description={
  }
}
\newglossaryentry{performances}%
{% 
  name={Performances},% 
  description={
  }
}
\newglossaryentry{pole}%
{% 
  name={Pôle},% 
  description={
  }
}
\newglossaryentry{repharmonique}%
{% 
  name={Réponse harmonique},% 
  description={
  }
}
\newglossaryentry{reptemporelle}%
{% 
  name={Réponse temporelle},% 
  description={
  }
}
\newglossaryentry{regtransitoire}%
{% 
  name={Régime transitoire},% 
  description={
  }
}
\newglossaryentry{regpermanent}%
{% 
  name={Régime permanent},% 
  description={
  }
}
\newglossaryentry{regaperiodique}%
{% 
  name={Régime apériodique},% 
  description={
  }
}
\newglossaryentry{regacritique}%
{% 
  name={Régime critique},% 
  description={
  }
}
\newglossaryentry{regpseudoperiodique}%
{% 
  name={Régime pseudo-périodique},% 
  description={
  }
}
\newglossaryentry{diagbode}%
{% 
  name={Diagramme de Bode},% 
  description={
  }
}
\newglossaryentry{diagnyquist}%
{% 
  name={Diagramme de Nyquist},% 
  description={
  }
}
\newglossaryentry{diagblack}%
{% 
  name={Diagramme de Blach-Nichols},% 
  description={
  }
}
%%%%%%%%%%%%%%%%%%%%%%%%%%%%%%%%%%%%%%%%%%%%%%%%%%%%%%%%%%%%%%%%%%%%%%%%%%%%%%%%
%%%%%%%%%%%%%%%%%%%%%%%%%%%%%%%%%%%%%%%%%%%%%%%%%%%%%%%%%%%%%%%%%%%%%%%%%%%%%%%%
%%%%%%%%%%%%%%%%%%%%%%%%%%%%%%%%%%%%%%%%%%%%%%%%%%%%%%%%%%%%%%%%%%%%%%%%%%%%%%%%
%%%%%%%%%%%%%%%%%%%%%%%%%%%%%%%%%%%%%%%%%%%%%%%%%%%%%%%%%%%%%%%%%%%%%%%%%%%%%%%%
%glossaire.tex
