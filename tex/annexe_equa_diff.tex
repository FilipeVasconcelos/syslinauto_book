\chapter{Équations différentielles à coefficients constants\label{annexe-eqndiff}}
\chaptermark{Équations différentielles}

\section{Définition}
La forme la plus générale d'une équation différentielles à coefficients constants est donnée
par :
\begin{align}
\sum_{i=0}^{n}a_i\devi{s(t)}{i}=\sum_{i=0}^{m}b_i\devi{e(t)}{i}
\end{align}


\section{Résolution équation différentielle du premier ordre}

La forme générale d'une équation différentielle du premier ordre est
donnée par : 
$$
a_1\devi{s(t)}{}+a_0s(t)=e(t)
$$
Il est toujours possible de simplifier une telle équation sous la forme :
$$
\devi{s(t)}{}+\dfrac{a_0}{a_1}s(t)=e(t)
$$


\subsection{Sans second membre}
$$
a_1\devi{s(t)}{}+a_0s(t)=0
$$

La solution générale de l'équation sans second membre est de la forme :
$$
s(t) = C e^{-\frac{a_0}{a_1}t}\,\,\,\,\text{avec}\,\,\,\,C\in\mathbb{R}
$$
