%%%%%%%%%%%%%%%%%%%%%%%%%%%%%%%%%%%%%%%%%%%%%%%%%%%%%%%%%%%%%%%%%%%%%%%%%%%%%%%%
%%%%%%%%%%%%%%%%%%%%%%%%%%%%%%%%%%%%%%%%%%%%%%%%%%%%%%%%%%%%%%%%%%%%%%%%%%%%%%%%
\exercice{Étude et conception de correcteurs AP, PI et PID\difficile}
%%%%%%%%%%%%%%%%%%%%%%%%%%%%%%%%%%%%%%%%%%%%%%%%%%%%%%%%%%%%%%%%%%%%%%%%%%%%%%%%
%%%%%%%%%%%%%%%%%%%%%%%%%%%%%%%%%%%%%%%%%%%%%%%%%%%%%%%%%%%%%%%%%%%%%%%%%%%%%%%%
L'objectif de cet exercice, (adapté de~\cite{Bourles}), est de déterminer les 
paramètres de correcteurs à avance de phase (AV), Proportionnel et Intégral (PI) 
et Proportionnel, Intégral et Dérivé (PID) afin d'améliorer les performances 
d'un système en boucle fermée.

On se placera dans le cas d'un système régi en boucle ouverte par la fonction 
de transfert
\[
    H(p)=\dfrac{5p+10}{p^3+3p^2+4p+2}
\]
et placé dans une boucle de contre-réaction avec un correcteur en série $C(p)$.
%-------------------------------------------------------------------------------
\begin{center}
    \tikzsetnextfilename{corrbruni-exercices-chap_correction-ext}
    \input{tikz/corrbruni-exercices-chap_correction.tex}
\end{center}
%-------------------------------------------------------------------------------
On étudiera en détail trois correcteurs:
%-------------------------------------------------------------------------------
\begin{itemize}
    \item Avance de phase: 
\[
    C(p)=C_{AV}(p)=K_d\dfrac{1+\alpha\tau_d p}{1+\tau_d p}
\]
    avec $K_d$ le gain proportionnel, $\alpha>1$ et $\tau_d$ les paramètres 
    du correcteur.
    \item Proportionnel et Intégral : 
\[ 
        C(p)=C_{PI}(p)=K_i\dfrac{1+\tau_i p}{\tau_i p}
\]
    avec $K_i$ le gain proportionnel et $\tau_i$ la constante de temps du 
    terme intégral.
    \item Proportionnel, Intégral et Dérivé : 
\[
    C(p)=C_{PID}(p)=K_pC_{AV}(p)C_{PI}(p)
\]
    avec $K_p$ le gain proportionnel du correcteur PID
\end{itemize}
%-------------------------------------------------------------------------------
Le correcteur PID est la mise en série des deux premiers correcteurs 
(avec des paramètres qui leurs sont propres).

On pourra utiliser Scilab pour les calculs et les tracés des réponses 
harmoniques et temporelles.
\clearpage
%%%%%%%%%%%%%%%%%%%%%%%%%%%%%%%%%%%%%%%%%%%%%%%%%%%%%%%%%%%%%%%%%%%%%%%%%%%%%%%%
%%%%%%%%%%%%%%%%%%%%%%%%%%%%%%%%%%%%%%%%%%%%%%%%%%%%%%%%%%%%%%%%%%%%%%%%%%%%%%%%
\subsection*{Caractérisation du système non corrigée}
%%%%%%%%%%%%%%%%%%%%%%%%%%%%%%%%%%%%%%%%%%%%%%%%%%%%%%%%%%%%%%%%%%%%%%%%%%%%%%%%
%%%%%%%%%%%%%%%%%%%%%%%%%%%%%%%%%%%%%%%%%%%%%%%%%%%%%%%%%%%%%%%%%%%%%%%%%%%%%%%%
%%%%%%%%%%%%%%%%%%%%%%%%%%%%%%%%%%%%%%%%%%%%%%%%%%%%%%%%%%%%%%%%%%%%%%%%%%%%%%%%
\question{Tracer le diagramme de Bode de la boucle ouverte non corrigée}
%%%%%%%%%%%%%%%%%%%%%%%%%%%%%%%%%%%%%%%%%%%%%%%%%%%%%%%%%%%%%%%%%%%%%%%%%%%%%%%%
%%%%%%%%%%%%%%%%%%%%%%%%%%%%%%%%%%%%%%%%%%%%%%%%%%%%%%%%%%%%%%%%%%%%%%%%%%%%%%%%
\question{Déterminer la pulsation $\omega_{\SI{0}{\dB}}$, telle que le gain 
          de la fonction de transfert en boucle ouverte soit égale à 
          \SI{0}{\dB}. Remarque: on pourra l'établir précisément par le calcul
          ou par la mesure sur le diagramme de Bode précédent.}
%%%%%%%%%%%%%%%%%%%%%%%%%%%%%%%%%%%%%%%%%%%%%%%%%%%%%%%%%%%%%%%%%%%%%%%%%%%%%%%%
%%%%%%%%%%%%%%%%%%%%%%%%%%%%%%%%%%%%%%%%%%%%%%%%%%%%%%%%%%%%%%%%%%%%%%%%%%%%%%%%
\question{Déterminer si le système est stable en boucle fermée par la méthode 
          de votre choix. Si oui, 
          déterminer la marge de phase $M_\phi$.}
%%%%%%%%%%%%%%%%%%%%%%%%%%%%%%%%%%%%%%%%%%%%%%%%%%%%%%%%%%%%%%%%%%%%%%%%%%%%%%%%
%%%%%%%%%%%%%%%%%%%%%%%%%%%%%%%%%%%%%%%%%%%%%%%%%%%%%%%%%%%%%%%%%%%%%%%%%%%%%%%%
\question{Tracer la réponse indicielle de la boucle fermée et 
          déterminer la valeur finale de la boucle fermée. On pourra donner
          certaines grandeurs caractérisant sa rapidité, son dépassement}
%%%%%%%%%%%%%%%%%%%%%%%%%%%%%%%%%%%%%%%%%%%%%%%%%%%%%%%%%%%%%%%%%%%%%%%%%%%%%%%%
%%%%%%%%%%%%%%%%%%%%%%%%%%%%%%%%%%%%%%%%%%%%%%%%%%%%%%%%%%%%%%%%%%%%%%%%%%%%%%%%
%%%%%%%%%%%%%%%%%%%%%%%%%%%%%%%%%%%%%%%%%%%%%%%%%%%%%%%%%%%%%%%%%%%%%%%%%%%%%%%%
\subsection*{Régulation du correcteur à avance de phase}
%%%%%%%%%%%%%%%%%%%%%%%%%%%%%%%%%%%%%%%%%%%%%%%%%%%%%%%%%%%%%%%%%%%%%%%%%%%%%%%%
%%%%%%%%%%%%%%%%%%%%%%%%%%%%%%%%%%%%%%%%%%%%%%%%%%%%%%%%%%%%%%%%%%%%%%%%%%%%%%%%
On souhaite corriger le système en rapidité et en robustesse (augmenter la 
marge de stabilité). Pour augmenter la bande passante en boucle fermée, on 
décide de fixer la pulsation de coupure $\omega_0$ à 
\SI{4}{\radian\per\second} et on cherche à obtenir parallèlement une marge 
de phase $M_\phi$ de \SI{60}{\degree}.
%%%%%%%%%%%%%%%%%%%%%%%%%%%%%%%%%%%%%%%%%%%%%%%%%%%%%%%%%%%%%%%%%%%%%%%%%%%%%%%%
\question{À partir du diagramme de Bode de la boucle ouverte non corrigée, 
          déterminer la phase $\phi_d$ et le gain $G_d$ devant être apportés 
          par le correcteur à avance de phase à la pulsation de coupure que 
          l'on s'est donnée.}
%%%%%%%%%%%%%%%%%%%%%%%%%%%%%%%%%%%%%%%%%%%%%%%%%%%%%%%%%%%%%%%%%%%%%%%%%%%%%%%%
%%%%%%%%%%%%%%%%%%%%%%%%%%%%%%%%%%%%%%%%%%%%%%%%%%%%%%%%%%%%%%%%%%%%%%%%%%%%%%%%
\question{À partir des relations suivantes, déterminer les paramètres 
$(\alpha,\tau_d,K_d)$ du correcteur à avance de phase}
%%%%%%%%%%%%%%%%%%%%%%%%%%%%%%%%%%%%%%%%%%%%%%%%%%%%%%%%%%%%%%%%%%%%%%%%%%%%%%%%
\[
    \alpha=\dfrac{1+\sin\phi_d}{1-\sin\phi_d}\quad\quad
    \tau_d=\dfrac{1}{\omega_0\sqrt\alpha}\quad\quad~
    K_d=\dfrac{G_d}{\sqrt\alpha}
\]
%%%%%%%%%%%%%%%%%%%%%%%%%%%%%%%%%%%%%%%%%%%%%%%%%%%%%%%%%%%%%%%%%%%%%%%%%%%%%%%%
\question{Donner la forme numérique de la fonction de transfert du correcteur 
          à avance de phase}
%%%%%%%%%%%%%%%%%%%%%%%%%%%%%%%%%%%%%%%%%%%%%%%%%%%%%%%%%%%%%%%%%%%%%%%%%%%%%%%%
%%%%%%%%%%%%%%%%%%%%%%%%%%%%%%%%%%%%%%%%%%%%%%%%%%%%%%%%%%%%%%%%%%%%%%%%%%%%%%%%
\question{Tracer le diagramme de Bode de la boucle ouverte corrigée par 
          ce correcteur}
%%%%%%%%%%%%%%%%%%%%%%%%%%%%%%%%%%%%%%%%%%%%%%%%%%%%%%%%%%%%%%%%%%%%%%%%%%%%%%%%
%%%%%%%%%%%%%%%%%%%%%%%%%%%%%%%%%%%%%%%%%%%%%%%%%%%%%%%%%%%%%%%%%%%%%%%%%%%%%%%%
%%%%%%%%%%%%%%%%%%%%%%%%%%%%%%%%%%%%%%%%%%%%%%%%%%%%%%%%%%%%%%%%%%%%%%%%%%%%%%%%
\subsection*{Régulation du correcteur PI}
%%%%%%%%%%%%%%%%%%%%%%%%%%%%%%%%%%%%%%%%%%%%%%%%%%%%%%%%%%%%%%%%%%%%%%%%%%%%%%%%
%%%%%%%%%%%%%%%%%%%%%%%%%%%%%%%%%%%%%%%%%%%%%%%%%%%%%%%%%%%%%%%%%%%%%%%%%%%%%%%%
On souhaite maintenant rendre le système précis à l'aide de l'effet intégral
Cependant, on veut également garantir de nouveau une marge de phase $M_\phi$ de 
\SI{60}{\degree}. Pour maintenir le système stable, il faut donc centrer le 
correcteur PI à une pulsation $\omega_0<\omega_{\SI{0}{\dB}}$
%%%%%%%%%%%%%%%%%%%%%%%%%%%%%%%%%%%%%%%%%%%%%%%%%%%%%%%%%%%%%%%%%%%%%%%%%%%%%%%%
\question{Déterminer l'intervalle des pulsations pour laquelle la marge de 
          phase $M_\phi$ de \SI{60}{\degree} est accessible}
%%%%%%%%%%%%%%%%%%%%%%%%%%%%%%%%%%%%%%%%%%%%%%%%%%%%%%%%%%%%%%%%%%%%%%%%%%%%%%%%
On choisit arbitrairement une pulsation $\omega_0=\SI{1}{\radian\per\second}$ 
compris dans l'intervalle précédent.
%%%%%%%%%%%%%%%%%%%%%%%%%%%%%%%%%%%%%%%%%%%%%%%%%%%%%%%%%%%%%%%%%%%%%%%%%%%%%%%%
\question{Déterminer le gain $G_i$ et la phase $\phi_i$ devant être
          apporté pour maintenir une marge de phase de \SI{60}{\degree}
          à cette pulsation de coupure.}
%%%%%%%%%%%%%%%%%%%%%%%%%%%%%%%%%%%%%%%%%%%%%%%%%%%%%%%%%%%%%%%%%%%%%%%%%%%%%%%%
%%%%%%%%%%%%%%%%%%%%%%%%%%%%%%%%%%%%%%%%%%%%%%%%%%%%%%%%%%%%%%%%%%%%%%%%%%%%%%%%
\question{Déterminer les paramètres $\tau_i$ et $K_i$ à partir des relations 
          du correcteur PI suivantes:}
%%%%%%%%%%%%%%%%%%%%%%%%%%%%%%%%%%%%%%%%%%%%%%%%%%%%%%%%%%%%%%%%%%%%%%%%%%%%%%%%
\[
    \tau_i=-\dfrac{1}{\omega_0\tan\phi_i}\quad\quad~
    K_i=\dfrac{G_i\tau_i\omega_0}{\sqrt{1+\tau^2_i\omega^2_0}}
\]
%%%%%%%%%%%%%%%%%%%%%%%%%%%%%%%%%%%%%%%%%%%%%%%%%%%%%%%%%%%%%%%%%%%%%%%%%%%%%%%%
\question{Donner la forme numérique de la fonction de transfert du 
          régulateur PI ainsi obtenu}
%%%%%%%%%%%%%%%%%%%%%%%%%%%%%%%%%%%%%%%%%%%%%%%%%%%%%%%%%%%%%%%%%%%%%%%%%%%%%%%%
%%%%%%%%%%%%%%%%%%%%%%%%%%%%%%%%%%%%%%%%%%%%%%%%%%%%%%%%%%%%%%%%%%%%%%%%%%%%%%%%
\question{Tracer le diagramme de Bode de la boucle ouverte corrigée par ce 
          correcteur}
%%%%%%%%%%%%%%%%%%%%%%%%%%%%%%%%%%%%%%%%%%%%%%%%%%%%%%%%%%%%%%%%%%%%%%%%%%%%%%%%
%%%%%%%%%%%%%%%%%%%%%%%%%%%%%%%%%%%%%%%%%%%%%%%%%%%%%%%%%%%%%%%%%%%%%%%%%%%%%%%%
%%%%%%%%%%%%%%%%%%%%%%%%%%%%%%%%%%%%%%%%%%%%%%%%%%%%%%%%%%%%%%%%%%%%%%%%%%%%%%%%
\subsection*{Régulation du correcteur PID}
%%%%%%%%%%%%%%%%%%%%%%%%%%%%%%%%%%%%%%%%%%%%%%%%%%%%%%%%%%%%%%%%%%%%%%%%%%%%%%%%
%%%%%%%%%%%%%%%%%%%%%%%%%%%%%%%%%%%%%%%%%%%%%%%%%%%%%%%%%%%%%%%%%%%%%%%%%%%%%%%%
Nous allons dans un premier temps présenter la méthodologie pour déterminer les 
paramètres du correcteur PID. La principale contrainte est de maintenir une 
marge de phase de \SI{60}{\degree} à une pulsation de coupure $\omega_0$ de 
\SI{4}{\radian\per\second}.
La méthodologie est la suivante :
%-------------------------------------------------------------------------------
\begin{itemize}
    \item Le correcteur à avance de phase amène un déphasage $\phi_d$ compris 
          en \SI{0}{\degree} et \SI{90}{\degree}. 
          Pour ce déphasage, on détermine les paramètres $\alpha$, $\tau_d$ et 
          $K_d$ pour que le gain de ce correcteur soit égal à \SI{0}{\dB} 
          à la pulsation de coupure $\omega_c$.
    \item Le correcteur PI en série amène un déphasage négatif $\phi_i$ compris 
          en \SI{0}{\degree} et \SI{90}{\degree}.
          Pour ce déphasage on détermine les paramètres $\tau_i$, $K_i$ pour 
          que le gain de ce correcteur soit égal à \SI{0}{\dB} à la 
          pulsation de coupure $\omega_0$.
    \item Pour déterminer $\phi_d$ et $\phi_i$. On utilise la relation 
        \[
            \arg{H(j\omega_0)} + \phi_d - \phi_i = -\SI{180}{\degree} + M_\phi
        \]
          avec la contrainte suivante $\phi_d + \phi_i = \SI{90}{\degree}$
    \item On détermine finalement le paramètre $K_p$ telle que le gain de la 
          boucle ouverte corrigée soit égale \SI{0}{\dB} 
          à la pulsation de coupure $\omega_0$.
\end{itemize}
%-------------------------------------------------------------------------------
%%%%%%%%%%%%%%%%%%%%%%%%%%%%%%%%%%%%%%%%%%%%%%%%%%%%%%%%%%%%%%%%%%%%%%%%%%%%%%%%
\question{En utilisant cette approche, déterminer les paramètres des 
          régulateurs composant ce PID}
%%%%%%%%%%%%%%%%%%%%%%%%%%%%%%%%%%%%%%%%%%%%%%%%%%%%%%%%%%%%%%%%%%%%%%%%%%%%%%%%
%%%%%%%%%%%%%%%%%%%%%%%%%%%%%%%%%%%%%%%%%%%%%%%%%%%%%%%%%%%%%%%%%%%%%%%%%%%%%%%%
\question{Donner la forme numérique de la fonction de transfert du correcteur}
%%%%%%%%%%%%%%%%%%%%%%%%%%%%%%%%%%%%%%%%%%%%%%%%%%%%%%%%%%%%%%%%%%%%%%%%%%%%%%%%
%%%%%%%%%%%%%%%%%%%%%%%%%%%%%%%%%%%%%%%%%%%%%%%%%%%%%%%%%%%%%%%%%%%%%%%%%%%%%%%%
\question{Tracer le diagramme de Bode de la boucle ouverte corrigée}
%%%%%%%%%%%%%%%%%%%%%%%%%%%%%%%%%%%%%%%%%%%%%%%%%%%%%%%%%%%%%%%%%%%%%%%%%%%%%%%%
%%%%%%%%%%%%%%%%%%%%%%%%%%%%%%%%%%%%%%%%%%%%%%%%%%%%%%%%%%%%%%%%%%%%%%%%%%%%%%%%
%%%%%%%%%%%%%%%%%%%%%%%%%%%%%%%%%%%%%%%%%%%%%%%%%%%%%%%%%%%%%%%%%%%%%%%%%%%%%%%%
\subsection*{Réponses temporelles en boucle fermée}
%%%%%%%%%%%%%%%%%%%%%%%%%%%%%%%%%%%%%%%%%%%%%%%%%%%%%%%%%%%%%%%%%%%%%%%%%%%%%%%%
%%%%%%%%%%%%%%%%%%%%%%%%%%%%%%%%%%%%%%%%%%%%%%%%%%%%%%%%%%%%%%%%%%%%%%%%%%%%%%%%
%%%%%%%%%%%%%%%%%%%%%%%%%%%%%%%%%%%%%%%%%%%%%%%%%%%%%%%%%%%%%%%%%%%%%%%%%%%%%%%%
\question{Comparer sur une figure les diagrammes de Bode de ces boucles ouvertes
          corrigées et non corrigées}
%%%%%%%%%%%%%%%%%%%%%%%%%%%%%%%%%%%%%%%%%%%%%%%%%%%%%%%%%%%%%%%%%%%%%%%%%%%%%%%%
%%%%%%%%%%%%%%%%%%%%%%%%%%%%%%%%%%%%%%%%%%%%%%%%%%%%%%%%%%%%%%%%%%%%%%%%%%%%%%%%
\question{Tracer les réponses indicielles en boucle fermée de tous ces 
          correcteurs. Conclure ses les performances corrigées.}
%%%%%%%%%%%%%%%%%%%%%%%%%%%%%%%%%%%%%%%%%%%%%%%%%%%%%%%%%%%%%%%%%%%%%%%%%%%%%%%%
