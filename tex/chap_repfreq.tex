%%%%%%%%%%%%%%%%%%%%%%%%%%%%%%%%%%%%%%%%%%%%%%%%%%%%%%%%%%%%%%%%%%%%%%%%%%%%%%%%
%%%%%%%%%%%%%%%%%%%%%%%%%%%%%%%%%%%%%%%%%%%%%%%%%%%%%%%%%%%%%%%%%%%%%%%%%%%%%%%%
%%%%%%%%%%%%%%%%%%%%%%%%%%%%%%%%%%%%%%%%%%%%%%%%%%%%%%%%%%%%%%%%%%%%%%%%%%%%%%%%
%%%%%%%%%%%%%%%%%%%%%%%%%%%%%%%%%%%%%%%%%%%%%%%%%%%%%%%%%%%%%%%%%%%%%%%%%%%%%%%%
\chapter[Analyse fréquentielle]
        {Analyse fréquentielle et représentation graphique\label{chap-anafreq}}
%%%%%%%%%%%%%%%%%%%%%%%%%%%%%%%%%%%%%%%%%%%%%%%%%%%%%%%%%%%%%%%%%%%%%%%%%%%%%%%%
%%%%%%%%%%%%%%%%%%%%%%%%%%%%%%%%%%%%%%%%%%%%%%%%%%%%%%%%%%%%%%%%%%%%%%%%%%%%%%%%
%%%%%%%%%%%%%%%%%%%%%%%%%%%%%%%%%%%%%%%%%%%%%%%%%%%%%%%%%%%%%%%%%%%%%%%%%%%%%%%%
%%%%%%%%%%%%%%%%%%%%%%%%%%%%%%%%%%%%%%%%%%%%%%%%%%%%%%%%%%%%%%%%%%%%%%%%%%%%%%%%
\minitoc
\newpage
%%%%%%%%%%%%%%%%%%%%%%%%%%%%%%%%%%%%%%%%%%%%%%%%%%%%%%%%%%%%%%%%%%%%%%%%%%%%%%%%
%%%%%%%%%%%%%%%%%%%%%%%%%%%%%%%%%%%%%%%%%%%%%%%%%%%%%%%%%%%%%%%%%%%%%%%%%%%%%%%%
%%%%%%%%%%%%%%%%%%%%%%%%%%%%%%%%%%%%%%%%%%%%%%%%%%%%%%%%%%%%%%%%%%%%%%%%%%%%%%%%
%\section{Introduction}
%%%%%%%%%%%%%%%%%%%%%%%%%%%%%%%%%%%%%%%%%%%%%%%%%%%%%%%%%%%%%%%%%%%%%%%%%%%%%%%%
%%%%%%%%%%%%%%%%%%%%%%%%%%%%%%%%%%%%%%%%%%%%%%%%%%%%%%%%%%%%%%%%%%%%%%%%%%%%%%%%
%%%%%%%%%%%%%%%%%%%%%%%%%%%%%%%%%%%%%%%%%%%%%%%%%%%%%%%%%%%%%%%%%%%%%%%%%%%%%%%%

Dans ce chapitre, nous allons établir la forme de la réponse d'un \gls{slci}~à 
une entrée sinuso\"idale, dite \textbf{réponse harmonique} en régime permanent.
Nous présenterons ensuite en détail les différentes représentations graphiques 
qui constitueront l'\textbf{analyse fréquentielle} de cette réponse harmonique.
Nous verrons en fin de chapitre une étude du transitoire dans des cas usuels.


%%%%%%%%%%%%%%%%%%%%%%%%%%%%%%%%%%%%%%%%%%%%%%%%%%%%%%%%%%%%%%%%%%%%%%%%%%%%%%%%
%%%%%%%%%%%%%%%%%%%%%%%%%%%%%%%%%%%%%%%%%%%%%%%%%%%%%%%%%%%%%%%%%%%%%%%%%%%%%%%%
%%%%%%%%%%%%%%%%%%%%%%%%%%%%%%%%%%%%%%%%%%%%%%%%%%%%%%%%%%%%%%%%%%%%%%%%%%%%%%%%
\section{Réponse harmonique}
%%%%%%%%%%%%%%%%%%%%%%%%%%%%%%%%%%%%%%%%%%%%%%%%%%%%%%%%%%%%%%%%%%%%%%%%%%%%%%%%
%%%%%%%%%%%%%%%%%%%%%%%%%%%%%%%%%%%%%%%%%%%%%%%%%%%%%%%%%%%%%%%%%%%%%%%%%%%%%%%%
%%%%%%%%%%%%%%%%%%%%%%%%%%%%%%%%%%%%%%%%%%%%%%%%%%%%%%%%%%%%%%%%%%%%%%%%%%%%%%%%


%L'excitation d'un \SLCI~par une entrée sinuso\"idale donne lieu, en régime 
%permanent, à une réponse harmonique dépendant de la fréquence d'excitation. 
%Nous allons ici établir la forme de cette réponse.

Soit un \gls{slci}~défini par une fonction de transfert $H(p)$ auquel on 
applique une entrée sinuso\"idale $e(t)$ tel que :
$$
e(t)=E_0\sin\omega t 
$$
d'amplitude $E_0$ et de pulsation $\omega$\footnote{Strictement, $\omega$ est 
une pulsation en unité \si{\radian\per\second}, la fréquence associée étant 
$f=\omega/2\pi$, en \si{\per\second} ou \si{\hertz}. Cependant, par abus de 
langage, il est courant de se référer en terme de fréquence en parlant de la 
pulsation $\omega$. Nous prendrons cependant soin d'utiliser la bonne forme 
dans nos applications numériques.}. Dans le domaine de Laplace, la sortie $S(p)$
est de la forme :
$$
S(p)=H(p)E(p)
$$
où $E(p)$ est la transformée de Laplace d'un sinus (c.f ligne 23 du tableau de 
l'\cref{annexe-lap}), on obtient alors :
$$
S(p)=H(p)\dfrac{E_0\omega}{p^2+\omega^2}
$$

Les pôles de la fonction de transfert $H(p)$ donnent lieu au 
régime transitoire alors que les pôles de l'excitation donnent 
lieu au régime permanent. 
Les deux pôles de l'excitation sont $p_{1,2}=\pm\jw$. La forme factorisée 
s'écrit alors:
$$
S(p)=H(p)\dfrac{E_0\omega}{(p+\jw)(p-\jw)}
$$

En régime permanent, la décomposition de $S(p)$ en éléments simples s'écrit :
$$
S(p)=\dfrac{A}{p-\jw} + \dfrac{B}{p+\jw}
$$

où les coefficients s'obtiennent par évaluation :
\begin{align*}
    A&=(p-\jw)S(p)\bigg|_{p=\hphantom{-}\jw}=
       \dfrac{E_0\omega}{p+\jw}H(p)\bigg|_{p=\hphantom{-}\jw}=
       \hphantom{-}\dfrac{E_0}{2j}H(\jw)\\
    B&=(p+\jw)S(p)\bigg|_{p=-\jw}=
       \dfrac{E_0\omega}{p-\jw}H(p)\bigg|_{p=-\jw}=
       -\dfrac{E_0}{2j}H(-\jw)
\end{align*}

nous obtenons donc :
$$
S(p)=\dfrac{E_0}{2j}\left(\dfrac{H(\jw)}{p-\jw}-\dfrac{H(-\jw)}{p+\jw} \right)
$$

La transformée de Laplace inverse de la sortie $S(p)$ permet d'obtenir la 
réponse temporelle 
$$
s(t)=\dfrac{E_0}{2j}\left(H(\jw)e^{\jw t}-H(-\jw)e^{-\jw t}\right)
$$

En écrivant le nombre complexe $H(\jw)$ sous sa forme exponentielle 
(\Cref{annexe-NC}) :
\begin{align*}
    H(\jw)  &= |H(\jw)| e^{j\phi} \\
    H(-\jw) &= |H(\jw)| e^{-j\phi}
\end{align*}
où $|H(\jw)|$ et $\phi$ sont respectivement le module et l'argument du nombre
complexe $H(\jw)$ 
et où l'on considère de plus que $H(-\jw)$ est égale à son conjugué (i.e 
$H(-\jw)=\overline{H(\jw)}$).

La réponse temporelle peut alors s'écrire sous la forme 
\begin{align*}
s(t)=E_0|H(\jw)|\left(\dfrac{e^{j(\omega t+\phi)}
    -e^{-j(\omega t+\phi)}}{2j}\right)
\end{align*}
où l'on reconnaît la forme exponentielle de la fonction sinus qui nous permet
d'écrire :
\begin{bequation}[ams align]
    s(t)=E_0|H(\jw)|\sin{(\omega t+\phi)}\label{eq-rh}
\end{bequation}

Cette relation exprime que \textbf{l'excitation d'un {\protect{\gls{slci}}}
~par une entrée sinuso\"idale donne lieu, en régime permanent, à une réponse 
harmonique dépendant de la fréquence d'excitation dont le gain en amplitude et
la phase sont respectivement donné par le module et l'argument de la fonction
de transfert du système.}

\`A noter que $H(\jw)$ correspond au rapport de la sortie sur l'entrée,
ainsi le gain $|H(\jw)|$ et la phase peuvent être définits à partir de la 
sortie et de l'entrée du signal,
\begin{align*}
    H(\jw) &= \dfrac{S(\jw)}{E(\jw)} \\
    |H(\jw)|&= \dfrac{|S(\jw)|}{|E(\jw)|} \\
    \arg{H(\jw)}&=\arg{S(\jw)}-\arg{E(\jw)}
\end{align*}

Le gain $|H(\jw)|$ est une fonction réelle de $\omega$ de ce fait nous 
utiliserons par la suite $G(\omega)$ pour noter plus explicitement cette 
dépendance. La phase est également une fonction de la pulsation d'excitation, 
nous la noterons donc $\phi(\omega)$ par la suite.


%%%%%%%%%%%%%%%%%%%%%%%%%%%%%%%%%%%%%%%%%%%%%%%%%%%%%%%%%%%%%%%%%%%%%%%%%%%%%%%%
%%%%%%%%%%%%%%%%%%%%%%%%%%%%%%%%%%%%%%%%%%%%%%%%%%%%%%%%%%%%%%%%%%%%%%%%%%%%%%%%
\subsection{Exemple de réponse harmonique dans le domaine temporel}
%%%%%%%%%%%%%%%%%%%%%%%%%%%%%%%%%%%%%%%%%%%%%%%%%%%%%%%%%%%%%%%%%%%%%%%%%%%%%%%%
%%%%%%%%%%%%%%%%%%%%%%%%%%%%%%%%%%%%%%%%%%%%%%%%%%%%%%%%%%%%%%%%%%%%%%%%%%%%%%%%

\index{Système du premier ordre!réponse harmonique dans le domaine temporel}
Considérons un \gls{slci}~définit par une fonction de transfert $H(p)$ du 
premier ordre (\Cref{eq-ft1er}) de forme canonique:
$$
H(p)=\dfrac{1}{1+p}
$$
avec $K=$1, $\tau=\SI{1}{\second}$.

Comme nous venons de le montrer la réponse harmonique est complétement 
déterminée par la connaissance du module et de l'argument du nombre complexe
$H(\jw)$. Le module donnant accès au rapport du gain en amplitude de la sortie 
sur l'entrée et l'argument à la différence de phase entre la sortie et l'entrée.

Calculons donc ces deux quantités pour notre fonction de transfert du premier 
ordre:
\begin{align*}
    G(\omega)   &=|H(\jw)|               =\left|\dfrac{1}{1+\jtw}\right|\\
    \phi(\omega)&=\arg\left(H(\jw)\right)=-\arctan(\omega\tau)
\end{align*}
Le~\cref{tab-1ertemp} présente le module et l'argument pour quelques valeurs 
particulières de $\omega$ 
($\omega=0.1$, $1$ et $\SI{10}{\radian\per\second}$).
\begin{table}
    \ra{1.3}
    \centering
    \setlength{\ltmp}{2.0cm}
    \begin{tabular}{P{\ltmp}P{\ltmp}P{\ltmp}P{\ltmp}}
        \toprule
        $\omega\si{[\radian\per\second]}$&$\omega=0.1$&$\omega=1$&$\omega=10$\\
        \midrule
        $G(\omega)$         & 0.99       & 0.70       & 0.1                  \\
        \midrule
        $\phi(\omega)$      &-5.7\degree &-45\degree  &-84.3\degree          \\
        \bottomrule
    \end{tabular}
\caption{Quelques valeurs particulières du gain et de la phase de la 
        fonction de transfert du premier ordre, pour $K=1$ et 
        $\tau=\SI{1}{\second}$\label{tab-1ertemp}.}
\end{table}
D'après ces valeurs, nous constatons que le rapport des amplitudes décroit et
que le déphasage augmente lorsque la pulsation de l'excitation augmente.

La~\cref{fig-repham} présente la forme des réponses temporelles de ce système 
pour les données calculées du gain et de la phase de la fonction de transfert 
considérée. Cette représentation graphique montre ses limites, en effet quand
est-il de toutes les autres valeurs de la pulsation ? 

Nous allons maintenant généraliser cette analyse sans pour autant avoir à 
tracer la réponse temporelle pour toutes les pulsations que l'on souhaite 
étudier.

\begin{figure}[!h]
\begin{center}
\tikzsetnextfilename{repham_1-chap3_ext}
    \begin{tikzpicture}
        \begin{axis}[
        ticks=none,
        axis line style = thick,
        height=4cm,
        width=12cm,
        axis x line=center,
        axis y line=center,
        xmin=-2,
        xmax=12,
        ymin=-1.25,
        ymax=1.25,
        xlabel={$t$},
        ylabel={\textcolor{red}{$s(t)$} \textcolor{blue}{$e(t)$}},
        xlabel style={below right},
        ylabel style={right},
        ]
\addplot[thick,color=blue,domain=0:11.5,samples=101]
            {sin(deg(x))};
\addplot[thick,color=red,domain=0:11.5,samples=101]
            {0.9950371902099892*sin(deg(x)-5.7)};
\node[left] at (axis cs:0,0.5)  {$\omega=0.1$};
        \end{axis}
    \end{tikzpicture}
\tikzsetnextfilename{repham_2-chap3_ext}
    \begin{tikzpicture}
        \begin{axis}[
        ticks=none,
        axis line style = thick,
        height=4cm,
        width=12cm,
        axis x line=center,
        axis y line=center,
        xmin=-2,
        xmax=12,
        ymin=-1.25,
        ymax=1.25,
        xlabel={$t$},
        xlabel style={below right},
        ylabel={\hphantom{\textcolor{red}{$s(t)$} \textcolor{blue}{$e(t)$}}},
        ylabel style={right},
        ]
            \addplot[thick,color=blue,domain=0:11.5,samples=101]
            {sin(deg(x))};
            \addplot[thick,color=red,domain=0:11.5,samples=101]
            {0.707106*sin(deg(x)-45)};
            \node[left] at (axis cs:0,0.5) {$\omega=1$};
        \end{axis}
    \end{tikzpicture}
\tikzsetnextfilename{repham_3-chap3_ext}
    \begin{tikzpicture}
        \begin{axis}[
        ticks=none,
        axis line style = thick,
        height=4cm,
        width=12cm,
        axis x line=center,
        axis y line=center,
        xmin=-2,
        xmax=12,
        ymin=-1.25,
        ymax=1.25,
        xlabel={$t$},
        xlabel style={below right},
        ylabel={\hphantom{\textcolor{red}{$s(t)$} \textcolor{blue}{$e(t)$}}},
        ylabel style={right},
        ]
      \addplot[thick,color=blue,domain=0:11.5,samples=101]
            {sin(deg(x))};
      \addplot[thick,color=red,domain=0:11.5,samples=101]
          {0.2*sin(deg(x)-84.2894)};
      \node[left] at (axis cs:0,0.5)  {$\omega=10$};
      \node[left] at (axis cs:11.5,0.5)  {\textcolor{red}{$s(t)\times 2$}};
      \end{axis}
    \end{tikzpicture}
\caption{Réponse harmonique (en régime permanent) (\Cref{eq-rh}) d'un système 
         du premier ordre pour différentes pulsations d'excitation de la forme 
         $e(t)=\sin{\omega t}$, (données du~\cref{tab-1ertemp}). Cette figure
         permet d'observer l'augmentation du déphasage et la diminution de 
         l'amplitude lorsque la fréquence d'excitations augmente. (bleu) 
         excitation $e(t)$ (rouge) sortie $s(t)$.\label{fig-repham}}
\end{center}
\end{figure}

%>>> np.arctan(0.1)
%0.09966865249116204
%>>> np.arctan(1)
%0.7853981633974483
%>>> np.arctan(10)
%1.4711276743037347

%>>> np.arctan(0.1)*180/np.pi
%5.710593137499643
%>>> np.arctan(1)*180/np.pi
%45.0
%>>> np.arctan(10)*180/np.pi
%84.28940686250037
%>>> np.arctan(100)*180/np.pi
%89.42706130231653

%>>> abs(1./complex(1,0.1))
%0.9950371902099892
%>>> abs(1./complex(1,1))
%0.7071067811865476
%>>> abs(1./complex(1,10))
%0.09950371902099892




\newpage
%%%%%%%%%%%%%%%%%%%%%%%%%%%%%%%%%%%%%%%%%%%%%%%%%%%%%%%%%%%%%%%%%%%%%%%%%%%%%%%%
%%%%%%%%%%%%%%%%%%%%%%%%%%%%%%%%%%%%%%%%%%%%%%%%%%%%%%%%%%%%%%%%%%%%%%%%%%%%%%%%
%%%%%%%%%%%%%%%%%%%%%%%%%%%%%%%%%%%%%%%%%%%%%%%%%%%%%%%%%%%%%%%%%%%%%%%%%%%%%%%%
\section{Représentation graphique de la réponse harmonique}
%%%%%%%%%%%%%%%%%%%%%%%%%%%%%%%%%%%%%%%%%%%%%%%%%%%%%%%%%%%%%%%%%%%%%%%%%%%%%%%%
%%%%%%%%%%%%%%%%%%%%%%%%%%%%%%%%%%%%%%%%%%%%%%%%%%%%%%%%%%%%%%%%%%%%%%%%%%%%%%%%
%%%%%%%%%%%%%%%%%%%%%%%%%%%%%%%%%%%%%%%%%%%%%%%%%%%%%%%%%%%%%%%%%%%%%%%%%%%%%%%%


Comme nous venons de le voir, il est possible d'étudier
la réponse harmonique (en régime permanent) d'un \gls{slci}~dans le domaine 
temporel et observer la variation d'amplitude et du 
déphasage qui dépend de la pulsation d'excitation. Ces variations 
d'amplitude et de phase sont totalement déterminées par la 
connaissance du module et de l'argument du nombre complexe $H(\jw)$, 
c'est ce qui constitue l'analyse fréquentielle des \gls{slci}.

Dans cette partie, nous présenterons trois types de représentations graphiques, 
notamment :
\begin{itemize}
    \item le diagramme de Bode,
    \item le diagramme de Nyquist,
    \item et le diagramme de Black-Nichols\footnote{
    \index{Nichols, Nathaniel}Nathaniel B. Nichols, (1914–1997) 
           ingénieur américain.}. 
\end{itemize}
Nous étudierons en détail les diagrammes des modèles usuels que nous 
avons déjà rencontrés au chapitre précédent (\Cref{chap-model})

%Remarquons que la représentation 
Le terme \textbf{lieu de transfert} est communément utilisé pour parler du 
points de coordonnées $(\omega, \phi(\omega),G(\omega))$.


%%%%%%%%%%%%%%%%%%%%%%%%%%%%%%%%%%%%%%%%%%%%%%%%%%%%%%%%%%%%%%%%%%%%%%%%%%%%%%%%
%%%%%%%%%%%%%%%%%%%%%%%%%%%%%%%%%%%%%%%%%%%%%%%%%%%%%%%%%%%%%%%%%%%%%%%%%%%%%%%%
\subsection{Diagramme de Bode}
%%%%%%%%%%%%%%%%%%%%%%%%%%%%%%%%%%%%%%%%%%%%%%%%%%%%%%%%%%%%%%%%%%%%%%%%%%%%%%%%
%%%%%%%%%%%%%%%%%%%%%%%%%%%%%%%%%%%%%%%%%%%%%%%%%%%%%%%%%%%%%%%%%%%%%%%%%%%%%%%%

Un diagramme de Bode\footnote{\index{Bode, Hendrik}Hendrik Wade Bode 
(1905-1982), ingénieur, chercheur et inventeur américain} permet de 
représenter le comportement fréquentielle d'un système quelconque en fonction 
de la fréquence  d'excitation en entrée. Il se compose de deux graphiques :
\begin{itemize}
    \item[i)] le tracé du gain en décibel en fonction de la pulsation $\omega$:
        \begin{bequation}[ams align] 
        G_{dB}(\omega)=20\log{G(\omega)}=20\log{|H(\jw)|} 
        \end{bequation}
    \item[ii)] le tracé de la phase en fonction de la pulsation $\omega$ :
        \begin{bequation}[ams align] 
        \phi(\omega)=\arg{H(\jw)}
        \end{bequation}
\end{itemize}
L'axe des pulsations étant généralement représenté par une échelle 
logarithmique pour permmettre la représentation de la réponse harmonique sur 
une large plage de valeurs en pulsation (\Cref{annexe-log}). Le calcul de la
phase passe lui par la détermination de l'argument principale 
(\Cref{annexe-NC}).

\begin{figure}[!h]
%\begin{center}
\centering
\tikzsetnextfilename{sche_bode-chap3_ext}
\begin{tikzpicture}
    \begin{axis}[
        name=axx,
        ticks=none,
        axis line style = thick,
        xmode=log,
        enlargelimits=false,
        height=4.cm,
        width=9cm,
        axis x line=center,
        axis y line=left,
        xmin=1e-2,
        xmax=1e2,
        ymin=-40,
        ymax=10,
        xlabel={$\log\omega$},
        xlabel style={below right},
        ylabel={$G_{dB}(\omega)$},
        ylabel style={left,rotate=-90,yshift=2.25em},
        clip=false,
        ]
\addplot[ultra thick,color=black,domain=1e-2:1e2, samples=101]
        {-10*log10(1+x*x)};
\draw[blue,dashed] (1e-2,{-10*log10(1+100)}) node[left] {$G_{dB}(\omega_2)$} --
        (1e1,{-10*log10(1+100)}) ; 
\draw[red ,dashed] (1e-2,{-10*log10(1+4)})   node[left] {$G_{dB}(\omega_1)$} --
        (2e0,{-10*log10(1+4)}) ; 
        \addplot [red, mark = *]  coordinates {(2e0,{-10*log10(1+4)})} {};
        \addplot [blue, mark = *] coordinates {(1e1,{-10*log10(1+100)})} {};
        \end{axis}
%\end{tikzpicture}
%\begin{tikzpicture}
        \begin{axis}[
        at={(axx.below south west)},yshift=-0.2cm,anchor=north west,
        ticks=none,
        axis line style = thick,
        xmode=log,
        enlargelimits=false,
        height=4.cm,
        width=9cm,
        axis x line=center,
        axis y line=left,
        xmin=1e-2,
        xmax=1e2,
        ymin=-90,
        ymax=20,
        xlabel={$\log\omega$},
        xlabel style={below right},
        ylabel={$\hphantom{_DB}\phi(\omega)$},
        ylabel style={left,rotate=-90,yshift=2.25em},
        clip=false,
        ]
\addplot[ultra thick,color=black,domain=1e-2:1e2, samples=101]{-atan2(x,1)};
\draw[blue,dashed] (1e1,{-atan2(1e1,1)})  -- 
            node[left,yshift=0.5em] {$\omega_2$} (1e1,77) ; 
\draw[red,dashed]  (2e0,{-atan2(2e0,1)})  -- 
            node[left,yshift=-0.5em] {$\omega_1$} (2e0,100) ; 
\draw[blue,dashed] (1e-2,{-atan2(1e1,1)}) node[left] {$\phi(\omega_2)$} -- 
            (1e1,{-atan2(1e1,1)}) ; 
\draw[red,dashed]  (1e-2,{-atan2(2e0,1)}) node[left] {$\phi(\omega_1)$} -- 
            (2e0,{-atan2(2e0,1)}) ; 
        \addplot [red, mark = *]  coordinates {(2e0,{-atan2(2e0,1)})} {};
        \addplot [blue, mark = *] coordinates {(1e1,{-atan2(1e1,1)})} {};
        \end{axis}
\end{tikzpicture}
%\end{center}
\caption{Représentation schématique d'un diagramme de Bode. Le gain en 
         décibel et la phase associé à une fonction de transfert sont 
         représentés en fonction de la pulsation (à l'échelle log) sur 
         deux repères distincts.\label{fig-sche_bode}}
\end{figure}

La principale propriété du diagramme de Bode est de permettre de 
simplifier un grand nombre calcul. En effet, dans le cas par exemple où 
deux systèmes $H_1$ et $H_2$ sont mis en série,
$$
H(\jw)=H_1(\jw)H_2(\jw),
$$
Le diagramme de Bode de $H(\jw)$ est la somme de deux diagrammes indépendants:
$$
\mathrm{Bode}(total)=\mathrm{Bode}(1)+\mathrm{Bode}(2)
$$
%\newpage
%%%%%%%%%%%%%%%%%%%%%%%%%%%%%%%%%%%%%%%%%%%%%%%%%%%%%%%%%%%%%%%%%%%%%%%%%%%%%%%%
%%%%%%%%%%%%%%%%%%%%%%%%%%%%%%%%%%%%%%%%%%%%%%%%%%%%%%%%%%%%%%%%%%%%%%%%%%%%%%%%
\subsection{Diagramme de Nyquist}
%%%%%%%%%%%%%%%%%%%%%%%%%%%%%%%%%%%%%%%%%%%%%%%%%%%%%%%%%%%%%%%%%%%%%%%%%%%%%%%%
%%%%%%%%%%%%%%%%%%%%%%%%%%%%%%%%%%%%%%%%%%%%%%%%%%%%%%%%%%%%%%%%%%%%%%%%%%%%%%%%


Un diagramme de Nyquist\footnote{\index{Nyquist, Harry}Harry 
Nyquist (1889-1976), électronicien, ingénieur américain.} 
présente la partie imaginaire et la partie réelle de $H(\jw)$ pour différentes 
valeurs paramétrées de $\omega$. Il a l'avantage de combiner les deux 
graphiques du diagramme de Bode en un seul. En effet, la phase et l'amplitude 
d'un point dans le plan complexe peut être déterminé graphiquement par 
respectivement l'angle avec l'axe des réels et la distance à l'origine 
(\Cref{annexe-NC}). Cette représentation graphique est communément appelée 
\textbf{le lieu de Nyquist}. Le lieu de Nyquist complet est le tracé théorique 
des parties réel et imaginaire de $H(\jw)$, en considérant les pulsations 
négatives, c'est à dire entre $\omega\rightarrow-\infty$ et 
$\omega\rightarrow+\infty$. 

\begin{figure}[!h]
\begin{center}
\tikzsetnextfilename{sche_nyquist_1-chap3_ext}
\begin{tikzpicture}
\begin{axis}
    [
    height=6.5cm,
    width=0.45\textwidth,
    axis lines = center,
    ticks=none,
    axis line style = thick,
    enlargelimits=false,
    xlabel=$\Re{H(\jw)}$,
    ylabel=$\Im{H(\jw)}$,
    xlabel style={below right},
    ylabel style={left},
    ymin=-10,
    ymax=+10,
    xmin=-10,
    xmax=+10
    ]     
    \def\xu{6.65043990901}
    \def\yu{3.24525022405}
    \def\xd{-2.6946926406}
    \def\yd{5.13601319826}
    \addplot[black, mark = *] coordinates {( 0, 0)} {};
    \node[below] at (axis cs:  7.853981633974483, 0) {\scriptsize$\omega=0$};
    \addplot[black, mark = *] coordinates {(7.853981633974483, 0)} {};
    \node[below] at (axis cs:  -2.2, 0) {\scriptsize$\omega\rightarrow\infty$};
    \addplot[red, mark = *] coordinates {(\xu,\yu)} {};
    \node[above right, red] at (axis cs:  \xu,\yu) {\scriptsize$\omega_1$};
    \draw[red,thick] (axis cs:0,0) -- (axis cs: \xu,\yu) 
        node[midway,yshift=1.1em,xshift=0.6em] {\scriptsize$G(\omega_1)$};
    \draw[red,thick] (axis cs:2.6,0) arc (0:18:1cm) 
        node[midway,yshift=0.23em,xshift=1.6em] {\scriptsize$\phi(\omega_1)$};
    \addplot[blue, mark = *] coordinates {( \xd,\yd)} {};
    \node[above left, blue]  at (axis cs:  \xd,\yd) {\scriptsize$\omega_2$};
    \draw[blue,thick] (axis cs:0,0) -- (axis cs:\xd,\yd) 
        node[midway,yshift=-0.5em,xshift=-1em] {\scriptsize$G(\omega_2)$};
    \draw[blue,thick] (axis cs:1.9,0) arc (0:119:1.9)    
        node[midway,yshift=-1.7em,xshift=-1.85em] {\scriptsize$\phi(\omega_2)$};
    \addplot[postaction={decorate, decoration={markings,
              mark=at position 0.105 with {\arrow[rotate=-180]{latex};},
              mark=at position 0.31 with  {\arrow[rotate=-180]{latex};},
              mark=at position 0.51 with  {\arrow[rotate=-180]{latex};},
              mark=at position 0.71 with  {\arrow[rotate=-180]{latex};},
              mark=at position 0.9 with   {\arrow[rotate=-180]{latex};}
              }},thick,domain=0:7.853981633974483,samples=100]
              ({x*sin(deg(x))},{x*cos(deg(x))});
\end{axis}
\end{tikzpicture}%
\hspace{0cm}
\tikzsetnextfilename{sche_nyquist_2-chap3_ext}
\begin{tikzpicture}
\begin{axis}
    [
    height=6.5cm,
    %width=8cm,
    width=0.45\textwidth,
    axis lines = center,
    ticks=none,
    axis line style = thick,
    enlargelimits=false,
    xlabel=$\Re{H(\jw)}$,
    ylabel=$\Im{H(\jw)}$,
    xlabel style={below right},
    ylabel style={left},
    ymin=-10,
    ymax=+10,
    xmin=-10,
    xmax=+10
    ]     
    \def\xu{6.65043990901}
    \def\yu{3.24525022405}
    \def\xd{-2.6946926406}
    \def\yd{5.13601319826}
    \addplot [postaction={decorate, decoration={markings,
            mark=at position 0.105 with {\arrow[rotate=-180]{latex};},
            mark=at position 0.31 with  {\arrow[rotate=-180]{latex};},
            mark=at position 0.51 with  {\arrow[rotate=-180]{latex};},
            mark=at position 0.71 with  {\arrow[rotate=-180]{latex};},
            mark=at position 0.9 with   {\arrow[rotate=-180]{latex};}
             }},thick,domain=0:7.853981633974483,samples=100]
             ({x*sin(deg(x))},{x*cos(deg(x))});
    \addplot [postaction={decorate, decoration={markings,
            mark=at position 0.105 with {\arrow{latex};},
            mark=at position 0.31 with  {\arrow{latex};},
            mark=at position 0.51 with  {\arrow{latex};},
            mark=at position 0.71 with  {\arrow{latex};},
            mark=at position 0.9 with   {\arrow{latex};}
      }},dashed,thick,domain=0:7.853981633974483,samples=100]
      ({x*sin(deg(x))},{-x*cos(deg(x))});
\end{axis}
\end{tikzpicture}
\end{center}
    \caption{(gauche) Représentation schématique d'un diagramme de Nyquist. 
             Le nombre complexe $H(\jw)$ est représenté dans le plan 
             complexe pour différentes valeurs de la pulsation $\omega$ de 
             0 à $\infty$. (droite) Représentation schématique du lieu 
             complet de Nyquist, symmétrique par rapport à l'axe des réels.
             \label{fig-sche_nyquist}}
\end{figure}



%%%%%%%%%%%%%%%%%%%%%%%%%%%%%%%%%%%%%%%%%%%%%%%%%%%%%%%%%%%%%%%%%%%%%%%%%%%%%%%%
%%%%%%%%%%%%%%%%%%%%%%%%%%%%%%%%%%%%%%%%%%%%%%%%%%%%%%%%%%%%%%%%%%%%%%%%%%%%%%%%
\subsection{Diagramme de Black-Nichols}
%%%%%%%%%%%%%%%%%%%%%%%%%%%%%%%%%%%%%%%%%%%%%%%%%%%%%%%%%%%%%%%%%%%%%%%%%%%%%%%%
%%%%%%%%%%%%%%%%%%%%%%%%%%%%%%%%%%%%%%%%%%%%%%%%%%%%%%%%%%%%%%%%%%%%%%%%%%%%%%%%

Le diagramme de Black-Nichols\footnote{Il également simplement appelé diagramme 
de Black.} consiste à tracer le gain en décibel $G_{dB}(\omega)$ en fonction de 
la phase, paramétré par la pulsation $\omega$. À l'instar du diagramme de 
Nyquist, le diagramme de Black à l'avantage de combiner les deux graphiques 
du diagramme de Bode. La diagramme de Black est habituellement utilisé dans 
l'étude des systèmes asservis (\Cref{chap-asservis}) pour déterminer le lieu 
de transfert dans le plan de Black d'un système en boucle fermée (FTBF) à 
partir de la connaissance du lieu de transfert dans le plan de Black de la 
Fonction de Transfert en Boucle ouverte (FTBO).
%(moins important dans une premiere découverte des \SLCI.
%Il pourra être introduit séparemment dans le cas des systèmes asservis.)

\begin{figure}[!h]
\begin{center}
\tikzsetnextfilename{sche_black-chap3_ext}
\begin{tikzpicture}
\begin{axis}
    [
    clip=false,
    height=8cm,
    width=8cm,
    ticks=none,
    axis lines = center,
    axis line style = thick,
    enlargelimits=false,
    ylabel=$G(\omega)$,
    xlabel=$\phi(\omega)$,
    xlabel style={right},
    ylabel style={above},
    xticklabel style={above,yshift=0.3em},
    yticklabel style={right,xshift=0.3em},
    ymin=-60,
    ymax=+10,
    xmin=-100,
    xmax=+20
    ]     
    \addplot[thick,domain=0:10,samples=100] 
    ({-atan2(x,1)},{-10*log10(1+x*x)});
    \addplot[thick,domain=10:100,samples=100,-latex] 
    ({-atan2(x,1)},{-10*log10(1+x*x)});
    \addplot[thick,domain=100:1000,samples=100] 
    ({-atan2(x,1)},{-10*log10(1+x*x)});
    \addplot[black, mark = *] coordinates {(0,0)} {};                    
    \addplot[black, mark = *,blue] coordinates 
    {({-atan2(10,1)},{-10*log10(1+10*10)} )} {};                    
    \draw[thick,blue,dashed] (axis cs: {-atan2(10,1)},{-10*log10(1+10*10)}) --
    (axis cs: {-atan2(10,1)},0) node[above] {$\phi(\omega_2)$};
    \draw[thick,blue,dashed] (axis cs: {-atan2(10,1)},{-10*log10(1+10*10)}) --
    (axis cs: 0,{-10*log10(1+10*10)}) node[right] {$G_{dB}(\omega_2)$};
    \addplot [black, mark = *,red] coordinates 
    {({-atan2(2,1)},{-10*log10(1+4)} )} {};
    \draw[thick,red,dashed] (axis cs: {-atan2(2,1)},{-10*log10(5)}) -- 
                            (axis cs: {-atan2(2,1)},0) 
                            node[above] {$\phi(\omega_1)$};
    \draw[thick,red,dashed] (axis cs: {-atan2(2,1)},{-10*log10(5)}) --
                             (axis cs: 0,{-10*log10(5)}) 
                            node[right] {$G_{dB}(\omega_1)$};
    \node [above right]  at (axis cs: 0 , 0)   {$\omega=0$}; 
    \node [above right]  at (axis cs:-90, -60) {$\omega\rightarrow+\infty$}; 

\end{axis}
\end{tikzpicture}
\end{center}
\caption{Représentation schématique d'un diagramme de Black. Le gain et 
         la phase de la fonction de transfert $H(\jw)$ sont représentés sur 
         le lieu de Black pour différentes valeurs de la pulsation $\omega$ 
         de 0 à $\infty$.\label{fig-sche_black}}
\end{figure}


\newpage
%%%%%%%%%%%%%%%%%%%%%%%%%%%%%%%%%%%%%%%%%%%%%%%%%%%%%%%%%%%%%%%%%%%%%%%%%%%%%%%%
%%%%%%%%%%%%%%%%%%%%%%%%%%%%%%%%%%%%%%%%%%%%%%%%%%%%%%%%%%%%%%%%%%%%%%%%%%%%%%%%
%%%%%%%%%%%%%%%%%%%%%%%%%%%%%%%%%%%%%%%%%%%%%%%%%%%%%%%%%%%%%%%%%%%%%%%%%%%%%%%%
\section{Analyse fréquentielle des modèles usuels}
%%%%%%%%%%%%%%%%%%%%%%%%%%%%%%%%%%%%%%%%%%%%%%%%%%%%%%%%%%%%%%%%%%%%%%%%%%%%%%%%
%%%%%%%%%%%%%%%%%%%%%%%%%%%%%%%%%%%%%%%%%%%%%%%%%%%%%%%%%%%%%%%%%%%%%%%%%%%%%%%%
%%%%%%%%%%%%%%%%%%%%%%%%%%%%%%%%%%%%%%%%%%%%%%%%%%%%%%%%%%%%%%%%%%%%%%%%%%%%%%%%


Nous allons ici présenter la forme canonique des diagrammes fréquentielles 
(Bode, Nyquist et Black-Nichols) pour les modèles usuels rencontrées dans 
l'étude des \gls{slci}. Les diagrammes de Bode restent l'outil principale et 
fera l'objet d'une présentation plus détaillée.
%%%%%%%%%%%%%%%%%%%%%%%%%%%%%%%%%%%%%%%%%%%%%%%%%%%%%%%%%%%%%%%%%%%%%%%%%%%%%%%%
%%%%%%%%%%%%%%%%%%%%%%%%%%%%%%%%%%%%%%%%%%%%%%%%%%%%%%%%%%%%%%%%%%%%%%%%%%%%%%%%
\subsection{Diagrammes de Bode : méthodologie générale}
%%%%%%%%%%%%%%%%%%%%%%%%%%%%%%%%%%%%%%%%%%%%%%%%%%%%%%%%%%%%%%%%%%%%%%%%%%%%%%%%
%%%%%%%%%%%%%%%%%%%%%%%%%%%%%%%%%%%%%%%%%%%%%%%%%%%%%%%%%%%%%%%%%%%%%%%%%%%%%%%%
Pour chacuns des modèles usuels, nous appliquerons la procédure suivante :
\begin{itemize}
    \item Définir la fonction de transfert $H(p)$ du modèle pour $p=\jw$
    \item \'Etablir la fonction du gain $G(\omega)$ à partir du 
          module de $|H(\jw)|$
    \item \'Etablir la fonction de la phase $\phi(\omega)$ à partir de 
          l'argument principale de $|H(\jw)|$.
          L'argument principale est définit à l'\Cref{annexe-NC}.
    \item Si les fonctions $G(\omega)$ et $\phi(\omega)$ ne sont pas de 
          simples constantes, réalisér une étude
          asymptotique pour $\omega\rightarrow 0$ et 
          $\omega\rightarrow +\infty$.
    \item Tracer le diagramme de Bode \textbf{réel} et le diagramme 
          de Bode \textbf{asymptotique}.
\end{itemize}

\newpage

%%%%%%%%%%%%%%%%%%%%%%%%%%%%%%%%%%%%%%%%%%%%%%%%%%%%%%%%%%%%%%%%%%%%%%%%%%%%%%%%
\subsubsection{Diagramme de Bode d'un gain pur}
%%%%%%%%%%%%%%%%%%%%%%%%%%%%%%%%%%%%%%%%%%%%%%%%%%%%%%%%%%%%%%%%%%%%%%%%%%%%%%%%

La fonction de transfert d'un gain pur est de la forme $H(\jw)=K$,
le gain est donc simplement donné par 
$$
G(\omega)=|H(\jw)|=K
$$ 
d'où le gain $G_{dB}$ en décibel :
\begin{bequation}[ams align*]
G_{dB}(\omega)= 20\log{K}
\end{bequation} ce qui correspond à une constante en gain 
(\Cref{fig-bode_gain}) et la phase s'obtient à partir de l'argument 
principale du nombre complexe $H(\jw)$:
\begin{bequation}[ams align*]
\phi(\omega) = 0
\end{bequation}

\begin{figure}[!htb]
\centering
\tikzsetnextfilename{bode_gain_1-chap3_ext}
\begin{tikzpicture}[trim axis left]
\begin{axis}[
    ticklabel style = {font=\footnotesize},
    width=0.9\textwidth,
    height=0.25\textheight,
    ylabel={Gain (\si{\decibel})},
    xtick={1e-3,1e-2,1e-1,1,1e1,1e2,1e3}, 
    ytick={-60,-40,-20,0,20,40,60}, 
    xticklabels={$10^{-3}$,$10^{-2}$,$10^{-1}$,
                 $10^{0}$,$10^{1}$,$10^{2}$,$10^{3}$},
    yticklabels={-60,-40,-20,0,20,40,60}, 
    xmode=log,ymode=normal,
    xmin=1e-3, xmax=1e3,
    ymin=-60, ymax=60,
    grid=both,
    major grid style={black!40}
]
    \addplot[ultra thick,black,domain=1e-3:1e3,samples=101]{20*log10(0.1)}; 
    \addplot[ultra thick,blue ,domain=1e-3:1e3,samples=101]{20*log10(1.0)}; 
    \addplot[ultra thick,red  ,domain=1e-3:1e3,samples=101]{20*log10(10)}; 
\end{axis}
\end{tikzpicture}

\tikzsetnextfilename{bode_gain_2-chap3_ext}
\begin{tikzpicture}[trim axis left]
\begin{axis}[
    ticklabel style = {font=\footnotesize},
    width=0.9\textwidth,
    height=0.25\textheight,
    xlabel={Pulsation (\si{\radian\per\second})},
    ylabel={Phase (deg)},
    xtick={1e-3,1e-2,1e-1,1,1e1,1e2,1e3}, 
    ytick={-90,-45,0,45,90}, 
    yticklabels={-90,-45,0,45,90},
    xticklabels={$10^{-3}$,$10^{-2}$,$10^{-1}$,
                 $10^{0}$,$10^{1}$,$10^{2}$,$10^{3}$},
    xmode=log,ymode=normal,
    xmin=1e-3, xmax=1e3,
    ymin=-90, ymax=90,
    grid=both,
    major grid style={black!40}
]
    \addplot[ultra thick, blue,domain=1e-3:1e3, samples=101] {0}; 
\end{axis}
\end{tikzpicture}
    \caption{Diagramme de Bode d'un gain pur avec (noir) 
             $K=0.1$, (bleu) $K=1$ et (rouge) $K=10$. Remarquons que la 
             phase reste inchangée lorsque le gain statique $K$ varie et que 
             seul le gain $G_{dB}(\omega)$ est modifié. \label{fig-bode_gain}}
\end{figure}
\newpage


%%%%%%%%%%%%%%%%%%%%%%%%%%%%%%%%%%%%%%%%%%%%%%%%%%%%%%%%%%%%%%%%%%%%%%%%%%%%%%%%
\subsubsection{Diagramme de Bode d'un intégrateur pur}
%%%%%%%%%%%%%%%%%%%%%%%%%%%%%%%%%%%%%%%%%%%%%%%%%%%%%%%%%%%%%%%%%%%%%%%%%%%%%%%%


La fonction de transfert d'un intégrateur pur est de la forme 
$H(\jw)=\frac{K}{\jw}$, le gain est donc simplement donné par 
$$
G(\omega)=|H(\jw)|=\frac{K}{\omega}
$$ 
d'où le gain $G_{dB}$ en décibel :
\begin{bequation}[ams align*]
G_{dB}(\omega)= 20\log{K} - 20\log{\omega}
\end{bequation} ce qui correspond à une pente de -20dB/décade 
(\Cref{fig-bode_int}) et la phase s'obtient à partir de l'argument principale 
du nombre complexe $H(\jw)$:
\begin{bequation}[ams align*]
\phi(\omega) = -\dfrac{\pi}{2}
\end{bequation}

\begin{figure}[!htb]
\centering
\tikzsetnextfilename{bode_integ_1-chap3_ext}
\begin{tikzpicture}[trim axis left]
\begin{axis}[
    ticklabel style = {font=\footnotesize},
    width=0.9\textwidth,
    height=0.25\textheight,
    ylabel={Gain (dB)},
    xtick={1e-3,1e-2,1e-1,1,1e1,1e2,1e3}, 
    ytick={-60,-40,-20,0,20,40,60}, 
    xticklabels={$10^{-3}$,$10^{-2}$,$10^{-1}$,
                 $10^{0}$,$10^{1}$,$10^{2}$,$10^{3}$},
    yticklabels={-60,-40,-20,0,20,40,60}, 
    xmode=log,ymode=normal,
    xmin=1e-3, xmax=1e3,
    ymin=-60, ymax=60,
    grid=both,
    major grid style={black!40}
]
    \addplot[ultra thick,black,domain=1e-3:1e3, samples=101]
    {20*log10(0.1)-20*log10(x)}; 
    \addplot[ultra thick,blue,domain=1e-3:1e3, samples=101]
    {-20*log10(x)}; 
    \addplot[ultra thick,red,domain=1e-3:1e3, samples=101]
    {20*log10(10)-20*log10(x)}; 
\end{axis}
\end{tikzpicture}

\tikzsetnextfilename{bode_integ_2-chap3_ext}
\begin{tikzpicture}[trim axis left]
\begin{axis}[
    ticklabel style = {font=\footnotesize},
    width=0.9\textwidth,
    height=0.25\textheight,
    xlabel={Pulsation (rad/s)},
    ylabel={Phase (deg)},
    xtick={1e-3,1e-2,1e-1,1,1e1,1e2,1e3}, 
    ytick={-180,-135,-90,-45,0}, 
    yticklabels={-180,-135,-90,-45,0},
    xticklabels={$10^{-3}$,$10^{-2}$,$10^{-1}$,
                 $10^{0}$,$10^{1}$,$10^{2}$,$10^{3}$},
    xmode=log,ymode=normal,
    xmin=1e-3, xmax=1e3,
    ymin=-180, ymax=0,
    grid=both,
    major grid style={black!40}
]
    \addplot[ultra thick, blue,domain=1e-3:1e3, samples=101] {-atan(1000000)};
\end{axis}
\end{tikzpicture}
    \caption{Diagramme de Bode d'un intégrateur pur avec (noir) 
             $K=0.1$, (bleu) $K=1$ et (rouge) $K=10$. Remarquons 
             que le gain s'annule pour $\omega=K$ et que la phase reste 
             inchangée.\label{fig-bode_int}}
\end{figure}


%%%%%%%%%%%%%%%%%%%%%%%%%%%%%%%%%%%%%%%%%%%%%%%%%%%%%%%%%%%%%%%%%%%%%%%%%%%%%%%%
\subsubsection{Diagramme de Bode d'un dérivateur pur}
%%%%%%%%%%%%%%%%%%%%%%%%%%%%%%%%%%%%%%%%%%%%%%%%%%%%%%%%%%%%%%%%%%%%%%%%%%%%%%%%


La fonction de transfert d'un dérivateur pur est de la forme $H(\jw)=K\jw$,
le gain est donc simplement donné par 
$$
G(\omega)=|H(\jw)|=K\jw
$$ 
d'où le gain $G_{dB}$ en décibel :
\begin{bequation}[ams align*]
G_{dB}(\omega)= 20\log{K} + 20\log{\omega}
\end{bequation}
ce qui correspond à une pente de +20dB/décade (\Cref{fig-bode_deriv}) et
la phase s'écrit simplement 
\begin{bequation}[ams align*]
\phi(\omega) = \dfrac{\pi}{2}
\end{bequation}

\begin{figure}[!htb]
\centering
\tikzsetnextfilename{bode_deriv_1-chap3_ext}
\begin{tikzpicture}[trim axis left]
\begin{axis}[
    ticklabel style = {font=\footnotesize},
    width=0.9\textwidth,
    height=0.25\textheight,
    ylabel={Gain (dB)},
    xtick={1e-3,1e-2,1e-1,1,1e1,1e2,1e3}, 
    ytick={-60,-40,-20,0,20,40,60}, 
    xticklabels={$10^{-3}$,$10^{-2}$,$10^{-1}$,
                 $10^{0}$,$10^{1}$,$10^{2}$,$10^{3}$},
    yticklabels={-60,-40,-20,0,20,40,60}, 
    xmode=log,ymode=normal,
    xmin=1e-3, xmax=1e3,
    ymin=-60, ymax=60,
    grid=both,
    major grid style={black!40}
]
    \addplot[ultra thick, black,domain=1e-3:1e3, samples=101]
    {20*log10(0.1)+20*log10(x)}; 
    \addplot[ultra thick, blue,domain=1e-3:1e3, samples=101] {20*log10(x)};
    \addplot[ultra thick, red,domain=1e-3:1e3, samples=101] 
    {20*log10(10)+20*log10(x)};
\end{axis}
\end{tikzpicture}

\tikzsetnextfilename{bode_deriv_2-chap3_ext}
\begin{tikzpicture}[trim axis left]
\begin{axis}[
    ticklabel style = {font=\footnotesize},
    width=0.9\textwidth,
    height=0.25\textheight,
    xlabel={Pulsation (rad/s)},
    ylabel={Phase (deg)},
    xtick={1e-3,1e-2,1e-1,1,1e1,1e2,1e3}, 
    ytick={0,45,90,135,180}, 
    yticklabels={0,45,90,135,180}, 
    xticklabels={$10^{-3}$,$10^{-2}$,$10^{-1}$,
                 $10^{0}$,$10^{1}$,$10^{2}$,$10^{3}$},
    xmode=log,ymode=normal,
    xmin=1e-3, xmax=1e3,
    ymin=0, ymax=180,
    grid=both,
    major grid style={black!40}
]
    \addplot[ultra thick, blue,domain=1e-3:1e3, samples=101] {atan(1000000)};
\end{axis}
\end{tikzpicture}
\caption{Diagramme de Bode d'un dérivateur pur 
         avec (noir) $K=0.1$, (bleu) $K=1$ et (rouge) $K=10$. Remarquons que 
         le gain s'annule pour $\omega=\frac{1}{K}$ et que la phase reste 
         inchangée.\label{fig-bode_deriv}}
\end{figure}



%%%%%%%%%%%%%%%%%%%%%%%%%%%%%%%%%%%%%%%%%%%%%%%%%%%%%%%%%%%%%%%%%%%%%%%%%%%%%%%%
\subsubsection{Diagramme de Bode d'un système à retard pur}
%%%%%%%%%%%%%%%%%%%%%%%%%%%%%%%%%%%%%%%%%%%%%%%%%%%%%%%%%%%%%%%%%%%%%%%%%%%%%%%%
\index{Retard pur!diagramme de Bode}
La fonction de transfert d'un retard pur est de la forme $H(\jw)=e^{-\jtw}$,
le gain est donc simplement donné par
$$
G(\omega)=|H(\jw)|=1
$$
d'où le gain $G_{dB}$ en décibel :
\begin{bequation}[ams align*]
    G_{dB}(\omega)= \SI{0}{\decibel}
\end{bequation}
et la phase s'écrit simplement
\begin{bequation}[ams align*]
    \phi(\omega) = -\tau\omega
\end{bequation}


\begin{figure}[!htb]
\centering
\tikzsetnextfilename{bode_retard_1-chap3_ext}
\begin{tikzpicture}[trim axis left]
\begin{axis}[
    ticklabel style = {font=\footnotesize},
    width=0.9\textwidth,
    height=0.25\textheight,
    ylabel={Gain (dB)},
    xtick={1e-3,1e-2,1e-1,1,1e1,1e2,1e3},
    ytick={-60,-40,-20,0,20,40,60},
    xticklabels={$10^{-3}$,$10^{-2}$,$10^{-1}$,
                 $10^{0}$,$10^{1}$,$10^{2}$,$10^{3}$},
    yticklabels={-60,-40,-20,0,20,40,60},
    xmode=log,ymode=normal,
    xmin=1e-2, xmax=1e3,
    ymin=-60, ymax=40,
    grid=both,
    major grid style={black!40}
]
    \addplot[ultra thick, blue,domain=1e-2:1e3, samples=101] {0};
\end{axis}
\end{tikzpicture}

\tikzsetnextfilename{bode_retard_2-chap3_ext}
\begin{tikzpicture}[trim axis left]
\begin{axis}[
    ticklabel style = {font=\footnotesize},
    width=0.9\textwidth,
    height=0.25\textheight,
    xlabel={Pulsation (rad/s)},
    ylabel={Phase (deg)},
    xtick={1e-1,1,1e1,1e2,1e3},
    ytick={-1000,-750,-500,-250,0},
    yticklabels={-1000,-750,-500,-250,0},
    xticklabels={$10^{-1}$,$10^{0}$,$10^{1}$,$10^{2}$,$10^{3}$},
    xmode=log,ymode=normal,
    xmin=1e-1, xmax=1e3,
    ymin=-1000, ymax=0,
    grid=both,
    major grid style={black!40}
]
    \addplot[ultra thick, blue,domain=1e-1:1e3, samples=201] {-x};
\end{axis}
\end{tikzpicture}
    \caption{Diagramme de Bode d'un retard pur 
             avec $\tau=1$. Remarquons que le gain est constant pour toutes 
             pulsations et le déphasage est monotone décroissant en fonction 
             de la pulsation\label{fig-bode_retard_1}.}
\end{figure}





%%%%%%%%%%%%%%%%%%%%%%%%%%%%%%%%%%%%%%%%%%%%%%%%%%%%%%%%%%%%%%%%%%%%%%%%%%%%%%%%
\subsubsection{Diagramme de Bode d'un système du premier ordre}
%%%%%%%%%%%%%%%%%%%%%%%%%%%%%%%%%%%%%%%%%%%%%%%%%%%%%%%%%%%%%%%%%%%%%%%%%%%%%%%%
\index{Système du premier ordre!diagramme de Bode}
Un système du premier ordre présente une fonction de transfert de la forme:
\begin{align}
H(\jw)=\dfrac{K}{1+j\tau\omega }\label{eq-1er_ftjw}
\end{align}

Le module de cette fonction de transfert $G(\omega) =|H(\jw)|$ s'écrit :
$$G(\omega)=\dfrac{K}{\sqrt{1+\tau^2\omega^2}}$$
Le gain en dB s'obtient alors par :
\begin{bequation}[ams align]
    G_{dB}(\omega)=20\log{K}-20\log{\sqrt{1+\tau^2\omega^2}}\label{eq-gain_1er}
\end{bequation}
et la phase est simplement donné par la fonction tangente réciproque:
\begin{bequation}[ams align]
    \phi(\omega)=\arg{H(\jw)}=-\arctan{(\tau\omega)}\label{eq-phase_1er} 
\end{bequation}
Ce sont ces deux fonctions de la fréquence que nous traçons sur un diagramme
de Bode. Elles sont représentés sur les~\cref{fig-bode_1er_1,fig-bode_1er_2}, 
pour respectivement différentes valeurs du gain statique $K$ et du temps 
caractéristique $\tau$.
\newline

Il est cependant recommandé de déterminer les asymptotes 
de ces deux fonctions à basse et haute fréquence. 
Pour celà, nous introduisons une \textbf{fréquence de cassure} 
$\omega_c=\dfrac{1}{\tau}$ qui délimite ces deux domaines.
\`A cette fréquence, le gain en décibel est de $G_{dB}(\omega_c)=20\log{K}-3$ 
et la phase $\phi(\omega)=\arctan{(1)}=\dfrac{\pi}{4}$.
Le gain de -3dB est la valeur approximative de $20\log{\sqrt{2}}$, communément
utilisée pour définir la \textbf{fréquence de coupure}.

\`A basse fréquence, c'est à dire lorsque $\tau\omega\ll1$ ou 
encore $\omega\ll\omega_0$, le gain et la phase se comporte comme, 
\begin{bequation}[ams align*]
    G_{dB}(\omega)&\sim20\log{K} \\
    \phi(\omega)&\sim0\si{\degree}.
\end{bequation} 

\`A haute fréquence, c'est à dire lorsque $\tau\omega\gg1$ ou 
encore $\omega\gg\omega_0$, le gain et la phase se comporte comme,
\begin{bequation}[ams align*]
    G_{dB}(\omega)&\sim20\log{K}-20\log{\frac{\omega}{\omega_0}} \\
    \phi(\omega)&\sim-\dfrac{\pi}{2}.
\end{bequation} 
La~\cref{fig-bode_1er_3} présente sur un même diagramme de Bode, les courbes 
réels et les courbes asymptotiques.

\begin{figure}[!t]
\centering
\tikzsetnextfilename{bode_1er_1-chap3_ext}
\begin{tikzpicture}[trim axis left]
\begin{axis}[
    ticklabel style = {font=\footnotesize},
    width=0.9\textwidth,
    height=0.22\textheight,
    ylabel={Gain (dB)},
    xtick={1e-3,1e-2,1e-1,1,1e1,1e2,1e3}, 
    ytick={-60,-40,-20,0,20,40,60}, 
    xticklabels={$10^{-3}$,$10^{-2}$,$10^{-1}$,
                 $10^{0}$,$10^{1}$,$10^{2}$,$10^{3}$},
    yticklabels={-60,-40,-20,0,20,40,60}, 
    xmode=log,ymode=normal,
    xmin=1e-3, xmax=1e3,
    ymin=-60, ymax=40,
    grid=both,
    major grid style={black!40}
]
    \addplot[ultra thick, black,domain=1e-3:1e3, samples=101]
    {20*log10(0.1)-20*log10(sqrt(1+x*x))}; 
    \addplot[ultra thick, blue,domain=1e-3:1e3, samples=101]
    {-20*log10(sqrt(1+x*x))}; 
    \addplot[ultra thick, red,domain=1e-3:1e3, samples=101]
    {20*log10(10)-20*log10(sqrt(1+x*x))}; 
\end{axis}
\end{tikzpicture}

\tikzsetnextfilename{bode_1er_2-chap3_ext}
\begin{tikzpicture}[trim axis left]
\begin{axis}[
    ticklabel style = {font=\footnotesize},
    width=0.9\textwidth,
    height=0.22\textheight,
    xlabel={Pulsation (rad/s)},
    ylabel={Phase (deg)},
    xtick={1e-3,1e-2,1e-1,1,1e1,1e2,1e3}, 
    ytick={-90,-45,0}, 
    yticklabels={-90,-45,0},
    xticklabels={$10^{-3}$,$10^{-2}$,$10^{-1}$,
                 $10^{0}$,$10^{1}$,$10^{2}$,$10^{3}$},
    xmode=log,ymode=normal,
    xmin=1e-3, xmax=1e3,
    ymin=-90, ymax=0,
    grid=both,
    major grid style={black!40}
]
    \addplot[ultra thick, blue,domain=1e-3:1e3, samples=101] {-atan2(x,1)}; 
\end{axis}
\end{tikzpicture}
    \caption{Diagramme de Bode d'un système du premier ordre 
             (\Cref{eq-1er_ftjw}) avec (noir) $K=0.1$ (bleu) $K=1$ et 
             (rouge) $K=10$. L'effet du gain $K$ est de décaler verticalement 
             la courbe de gain.\label{fig-bode_1er_1}}
\end{figure}
\begin{figure}[!b]
\centering
\tikzsetnextfilename{bode_1er_3-chap3_ext}
\begin{tikzpicture}[trim axis left]
\begin{axis}[
    ticklabel style = {font=\footnotesize},
    width=0.9\textwidth,
    height=0.22\textheight,
    ylabel={Gain (dB)},
    xtick={1e-3,1e-2,1e-1,1,1e1,1e2,1e3}, 
    ytick={-60,-40,-20,0,20,40,60}, 
    xticklabels={$10^{-3}$,$10^{-2}$,$10^{-1}$,
                 $10^{0}$,$10^{1}$,$10^{2}$,$10^{3}$},
    yticklabels={-60,-40,-20,0,20,40,60}, 
    xmode=log,ymode=normal,
    xmin=1e-3, xmax=1e3,
    ymin=-60, ymax=20,
    grid=both,
    major grid style={black!40}
]
    \addplot[ultra thick, black,domain=1e-3:1e3, samples=101] 
    {-20*log10(sqrt(1+10*x*x))}; 
    \addplot[ultra thick, blue,domain=1e-3:1e3, samples=101] 
    {-20*log10(sqrt(1+x*x))}; 
    \addplot[ultra thick, red,domain=1e-3:1e3, samples=101] 
    {-20*log10(sqrt(1+0.1*x*x))}; 
\end{axis}
\end{tikzpicture}

\tikzsetnextfilename{bode_1er_4-chap3_ext}
\begin{tikzpicture}[trim axis left]
\begin{axis}[
    ticklabel style = {font=\footnotesize},
    width=0.9\textwidth,
    height=0.22\textheight,
    xlabel={Pulsation (rad/s)},
    ylabel={Phase (deg)},
    xtick={1e-3,1e-2,1e-1,1,1e1,1e2,1e3}, 
    ytick={-90,-45,0}, 
    yticklabels={-90,-45,0},
    xticklabels={$10^{-3}$,$10^{-2}$,$10^{-1}$,
                 $10^{0}$,$10^{1}$,$10^{2}$,$10^{3}$},
    xmode=log,ymode=normal,
    xmin=1e-3, xmax=1e3,
    ymin=-90, ymax=0,
    grid=both,
    major grid style={black!40}
]
    \addplot[ultra thick, black,domain=1e-3:1e3, samples=101] 
    {-atan2(10*x,1)}; 
    \addplot[ultra thick, blue,domain=1e-3:1e3, samples=101] 
    {-atan2(x,1)}; 
    \addplot[ultra thick, red,domain=1e-3:1e3, samples=101] 
    {-atan2(0.1*x,1)}; 
\end{axis}
\end{tikzpicture}
    \caption{Diagramme de Bode d'un système du premier ordre 
             (\Cref{eq-1er_ftjw}) avec (noir) $\tau=10$ (bleu) $\tau=1$ et 
             (rouge) $\tau=0.1$. L'effet du temps caractéristique $\tau$ est 
             de décaler horizontalement la courbe de phase.
             \label{fig-bode_1er_2}}
\end{figure}
\afterpage{\clearpage}
\begin{figure}[!t]
\centering
\tikzsetnextfilename{bode_1er_5-chap3_ext}
\begin{tikzpicture}[trim axis left]
\begin{axis}[
    ticklabel style = {font=\footnotesize},
    width=0.9\textwidth,
    height=0.22\textheight,
    ylabel={Gain (dB)},
    xtick={1e-1,1,1e1}, 
    ytick={-10,-8,-6,-4,-2,0,2}, 
    xticklabels={$10^{-1}$,$10^{0}$,$10^{1}$},
    yticklabels={-10,-8,-6,-4,-2,0,2}, 
    xmode=log,ymode=normal,
    xmin=1e-1, xmax=1e1,
    ymin=-10, ymax=3,
    grid=both,
    major grid style={black!40}
]
    \addplot[ultra thick,blue,domain=1e-3:1e3, samples=101] 
    {-20*log10(sqrt(1+x*x))};
    \addplot[line width=2pt,red,dashed,domain=1e-1:1e0, samples=101] {0};
    \addplot[line width=2pt,red,dashed,domain=1e0:1e1, samples=101] 
    {-20*log10(x)};
\end{axis}
\end{tikzpicture}

\tikzsetnextfilename{bode_1er_6-chap3_ext}
\begin{tikzpicture}[trim axis left]
\begin{axis}[
    ticklabel style = {font=\footnotesize},
    width=0.9\textwidth,
    height=0.22\textheight,
    xlabel={Pulsation (rad/s)},
    ylabel={Phase (deg)},
    xtick={1e-1,1,1e1}, 
    ytick={-90,-45,0}, 
    yticklabels={-90,-45,0},
    xticklabels={$10^{-1}$,$10^{0}$,$10^{1}$},
    xmode=log,ymode=normal,
    xmin=1e-1, xmax=1e1,
    ymin=-90, ymax=0,
    grid=both,
    major grid style={black!40}
]
    \addplot[ultra thick,blue,domain=1e-3:1e3, samples=101] {-atan2(x,1)};
    \addplot[line width=2pt,red,dashed,domain=1e-1:1e0,samples=101] {0};
    \addplot[line width=2pt,red,dashed,domain=1e0:1e1,samples=101] {-90};
    \draw[line width=2pt,red,dashed] (axis cs:1,0) -- (axis cs:1,-90);
\end{axis}
\end{tikzpicture}
\caption{Diagramme de Bode d'un système du premier ordre 
         (\Cref{eq-1er_ftjw}) (i.e $K=1$, $\tau=1$ et $\omega_c=1$) avec 
         (bleu) le diagramme réel et (rouge) le diagramme asymptotique. On 
         vérifie que les valeurs asymptotiques sont de bonnes approximations 
         à basse et haute fréquence. Il est également possible de lire un 
         gain de \SI{-3}{\dB} et une phase de -45\si{\degree} à la fréquence 
         de coupure.\label{fig-bode_1er_3}}
\end{figure}
%\caption{Diagramme de Bode d'un système du premier ordre 
%$H(p)=(1-\tau p)^{-1}$\label{fig-bode_1er_ms}}
%\caption{Diagramme de Bode d'un système du premier ordre passe-bas 
%$H(p)=(1+\tau p)^{-1}$.\label{fig-bode_1er_ps}}
%\caption{Diagramme de Bode d'un système du premier ordre (passe-haut).
%\label{fig-bode_int}}

%%%%%%%%%%%%%%%%%%%%%%%%%%%%%%%%%%%%%%%%%%%%%%%%%%%%%%%%%%%%%%%%%%%%%%%%%%%%%%%%
\subsubsection{Diagramme de Bode de deux systèmes du premier ordre en série }
%%%%%%%%%%%%%%%%%%%%%%%%%%%%%%%%%%%%%%%%%%%%%%%%%%%%%%%%%%%%%%%%%%%%%%%%%%%%%%%%
La fonction de transfert globale de deux systèmes du premier ordre en série 
s'écrit:
\begin{align}
H(\jw)=\dfrac{K_1K_2}{(1+j\tau_1\omega)(1+j\tau_2\omega)}\label{eq-1er_serie}
\end{align}
On utilise la propriété du logarithme pour écrire le gain globale 
$G_{dB}(\omega)$ comme une somme de gain de deux systèmes du premier ordre, 
soit
$$
G_{dB}(\omega) = G_{dB1}(\omega) + G_{dB2}(\omega)
$$
De même pour la phase:
$$
\phi(\omega)= \phi_1(\omega) + \phi_2(\omega)
$$
En reprenant les~\cref{eq-gain_1er,eq-phase_1er} on établit facilement que,
\begin{bequation}[ams align*]
G_{dB}(\omega) = 20\log{K_1K_2}-20\log{\sqrt{1+\tau_1^2\omega^2}}
                -20\log{\sqrt{1+\tau_2^2\omega^2}}
\end{bequation}
et
\begin{bequation}[ams align*]
\phi(\omega)=-\arctan{\tau_1\omega}-\arctan{\tau_2\omega}
\end{bequation}

L'étude asymptotique se fait en considérant deux fréquences de coupures 
$\omega_{c1}=\frac{1}{\tau_1}$ et $\omega_{c2}=\frac{1}{\tau_2}$.
Supposons d'abord, sans perte de généralité, que $\omega_{c2}>\omega_{c1}$ et 
considérons 
les trois domaines de fréquence ainsi définits selon que 
$\omega\ll\omega_{c1}$, $\omega_{c1}<\omega<\omega_{c1}$ ou 
$\omega\gg\omega_{c2}$

%%%%%%%%%%%%%%%%%%%%%%%%%%%%%%%%%%%%%%%%%%%%%%%%%%%%%%%%%%%%%%%%%%%%%%%%%%%%%%%%
\paragraph{Pour $\omega\ll\omega_{c1}$}
%%%%%%%%%%%%%%%%%%%%%%%%%%%%%%%%%%%%%%%%%%%%%%%%%%%%%%%%%%%%%%%%%%%%%%%%%%%%%%%%
\begin{bequation}[ams align*]
G_{dB}(\omega)&\sim20\log{K_1K_2}\\
    \phi(\omega)&\sim0\si{\degree}
\end{bequation}

%%%%%%%%%%%%%%%%%%%%%%%%%%%%%%%%%%%%%%%%%%%%%%%%%%%%%%%%%%%%%%%%%%%%%%%%%%%%%%%%
\paragraph{Pour $\omega_{c1}<\omega<\omega_{c1}$}
%%%%%%%%%%%%%%%%%%%%%%%%%%%%%%%%%%%%%%%%%%%%%%%%%%%%%%%%%%%%%%%%%%%%%%%%%%%%%%%%
\begin{bequation}[ams align*]
G_{dB}(\omega)&\sim20\log{K_1K_2}-20\log{\dfrac{\omega}
                                               {\omega_{c1}\omega_{c2}}}\\
\phi(\omega)&\sim-90\si{\degree}
\end{bequation}
%%%%%%%%%%%%%%%%%%%%%%%%%%%%%%%%%%%%%%%%%%%%%%%%%%%%%%%%%%%%%%%%%%%%%%%%%%%%%%%%
\paragraph{Pour $\omega\gg\omega_{c2}$}
%%%%%%%%%%%%%%%%%%%%%%%%%%%%%%%%%%%%%%%%%%%%%%%%%%%%%%%%%%%%%%%%%%%%%%%%%%%%%%%%
\begin{bequation}[ams align*]
G_{dB}(\omega)&\sim20\log{K_1K_2}-40\log{\dfrac{\omega}
                                               {\omega_{c1}\omega_{c2}}}\\
\phi(\omega)&\sim-180\si{\degree}
\end{bequation}

La~\cref{fig-bode_1er_serie} présente le diagramme de Bode réel et 
asymptotique de deux systèmes du premier ordre en cascade. On remarquera 
que l'approximation asymptotique est suffisante pour décrire le gain de ce 
genre de système. En marquant la discontinuité dans le graphe de la phase, 
on distingue plus facilement les différentes zones et les changements de 
pente du gain. Pour la phase, il suffit de déterminer sa valeur pour 
quelques valeurs particulières de la pulsation. 

Comme nous l'avons déjà rencontré, l'étude de deux systèmes du premier 
ordre en série correspond à l'étude d'un système du second ordre en 
régime apériodique.

\begin{figure}[!t]
\centering
\tikzsetnextfilename{bode_1er_serie_2nd_1-chap3_ext}
\begin{tikzpicture}[trim axis left]
\begin{axis}[
    ticklabel style = {font=\footnotesize},
    width=0.9\textwidth,
    height=0.25\textheight,
    ylabel={Gain (dB)},
    xtick={1e-3,1e-2,1e-1,1,1e1,1e2,1e3}, 
    ytick={-80,-60,-40,-20,0,20,40,60}, 
    xticklabels={$10^{-3}$,$10^{-2}$,$10^{-1}$,
                 $10^{0}$,$10^{1}$,$10^{2}$,$10^{3}$},
    yticklabels={-80,-60,-40,-20,0,20,40,60}, 
    xmode=log,ymode=normal,
    xmin=1e-3, xmax=1e3,
    ymin=-120, ymax=20,
    grid=both,
    major grid style={black!40}
]
    \addplot[ultra thick, blue,domain=1e-3:1e3, samples=101]
    {-20*log10(sqrt(1+100*x*x))-20*log10(sqrt(1+0.01*x*x))}; 
    \addplot[line width=2pt,red,dashed,domain=1e-3:1e-1, samples=101]{0};
    \addplot[line width=2pt,red,dashed,domain=1e-1:1e1, samples=101]
    {-20*log10(x/0.1)};
    \addplot[line width=2pt,red,dashed,domain=1e1:1e3, samples=101]
    {-20*log10(x/0.1)-20*log10(x/10)};
    \node[right,xshift=1em,yshift=-0.1em] at (axis cs:0.1,-70)
    {{\large \textbf{-20dB/décade}}};
    \node[right,xshift=1em,yshift=-0.1em] at (axis cs:10,-10)
    {{\large \textbf{-40dB/décade}}};
\end{axis}
\end{tikzpicture}

\tikzsetnextfilename{bode_1er_serie_2nd_2-chap3_ext}
\begin{tikzpicture}[trim axis left]
\begin{axis}[
    ticklabel style = {font=\footnotesize},
    width=0.9\textwidth,
    height=0.25\textheight,
    xlabel={Pulsation (rad/s)},
    ylabel={Phase (deg)},
    xtick={1e-3,1e-2,1e-1,1,1e1,1e2,1e3}, 
    ytick={-180,-135,-90,-45,0}, 
    yticklabels={-180,-135,-90,-45,0},
    xticklabels={$10^{-3}$,$10^{-2}$,$10^{-1}$,
                 $10^{0}$,$10^{1}$,$10^{2}$,$10^{3}$},
    xmode=log,ymode=normal,
    xmin=1e-3, xmax=1e3,
    ymin=-180, ymax=0,
    grid=both,
    major grid style={black!40}
]
    \addplot[ultra thick, blue,domain=1e-3:1e3, samples=101]
    {-atan2(x,1)-atan2(x,0.1)}; 
    \addplot[line width=2pt,red,dashed,domain=1e-3:1e-1, samples=101]{0};
    \addplot[line width=2pt,red,dashed,domain=1e-1:1e1, samples=101] {-90};
    \addplot[line width=2pt,red,dashed,domain=1e1:1e3, samples=101] {-180};
    \draw[line width=2pt,red,dashed] (axis cs:0.1,0) -- (axis cs:0.1,-90);
    \draw[line width=2pt,red,dashed] (axis cs:10,-90) -- (axis cs:10,-180);
\end{axis}
\end{tikzpicture}
    \caption{Diagramme de Bode de systèmes du premier ordre en série 
    (\Cref{eq-1er_serie}) avec $\tau_1=10$ et $\tau_2=0.1$ (bleu) le 
    diagramme réel et (rouge) le diagramme asymptotique.
    \label{fig-bode_1er_serie}}
\end{figure}

\newpage
%%%%%%%%%%%%%%%%%%%%%%%%%%%%%%%%%%%%%%%%%%%%%%%%%%%%%%%%%%%%%%%%%%%%%%%%%%%%%%%%
\subsubsection{Diagramme de Bode d'un système second d'ordre}
%%%%%%%%%%%%%%%%%%%%%%%%%%%%%%%%%%%%%%%%%%%%%%%%%%%%%%%%%%%%%%%%%%%%%%%%%%%%%%%%
La fonction de transfert d'un système du second ordre (\Cref{eq-2nd_ft}) 
est donnée par :
\begin{align}
H(\jw)=\dfrac{K\omega^2_0}{(\omega^2_0-\omega^2)+j2\xi\omega_0\omega}
\label{eq-2nd_ftjw}
\end{align}
Le gain s'obtient en calculant le module de ce nombre complexe :
$$
G(\omega)=\dfrac{K\omega^2_0}{\sqrt{(\omega^2_0-\omega^2)^2
         +(2\xi\omega_0\omega)^2}}
$$
Le gain en décibel s'écrit alors :
\begin{bequation}[ams align*]
G_{db}(\omega)=20\log{K\omega_0^2}-20\log{\sqrt{(\omega^2_0-\omega^2)^2
              +(2\xi\omega_0\omega)^2}}
\end{bequation}
et la phase par l'argument princiale:
\begin{bequation}[ams align*]
\phi(\omega)=
\begin{cases}
    -\arctan{\left(\dfrac{2\xi\omega_0\omega}{\omega_0^2-\omega^2}\right)}     
    &\,\,\,\,\text{si $\omega^2<\omega^2_0$}\\
    -\arctan{\left(\dfrac{2\xi\omega_0\omega}{\omega_0^2-\omega^2}\right)}+\pi 
    &\,\,\,\,\text{si $\omega^2>\omega^2_0$}\\
    -\dfrac{\pi}{2}                                                            
    &\,\,\,\,\text{si $\omega^2=\omega^2_0$}
\end{cases}
\end{bequation}

Comme précdemment, il est recommandé d'étudier les valeurs asymptotiques 
du gain et de la phase.
%%%%%%%%%%%%%%%%%%%%%%%%%%%%%%%%%%%%%%%%%%%%%%%%%%%%%%%%%%%%%%%%%%%%%%%%%%%%%%%%
\paragraph{Pour $\omega \ll\omega_0$}
%%%%%%%%%%%%%%%%%%%%%%%%%%%%%%%%%%%%%%%%%%%%%%%%%%%%%%%%%%%%%%%%%%%%%%%%%%%%%%%%
\begin{bequation}[ams align*]
G_{dB}(\omega)&\sim20\log{K}\\
\phi(\omega)&\sim0\si{\degree}
\end{bequation}

%%%%%%%%%%%%%%%%%%%%%%%%%%%%%%%%%%%%%%%%%%%%%%%%%%%%%%%%%%%%%%%%%%%%%%%%%%%%%%%%
\paragraph{Pour $\omega \gg\omega_0$}
%%%%%%%%%%%%%%%%%%%%%%%%%%%%%%%%%%%%%%%%%%%%%%%%%%%%%%%%%%%%%%%%%%%%%%%%%%%%%%%%
\begin{bequation}[ams align*]
G_{dB}(\omega)&\sim20\log{K\omega^2_0}-40\log\omega\\
    \phi(\omega)&\sim-180\si{\degree}
\end{bequation}
La~\cref{fig-bode_2nd_1} présente le diagramme de Bode associé à ces deux 
fonctions pour $\xi=1$, ainsi que le diagramme de Bode asymptotique. 
La~\cref{fig-bode_2nd_2} présente l'effet du taux d'amortissement $\xi$ sur 
le diagramme de Bode. Il est possible d'observer une augmentation de la valeur 
maximum du gain proche de la fréquence de coupure.
C'est ce phénomène de résonance que nous allons discuter dans la 
prochaine partie.


\begin{figure}[!t]
\centering
\tikzsetnextfilename{bode_2nd_1-chap3_ext}
\begin{tikzpicture}[trim axis left]
\begin{axis}[
    ticklabel style = {font=\footnotesize},
    width=0.9\textwidth,
    height=0.22\textheight,
    ylabel={Gain (dB)},
    xtick={1e-3,1e-2,1e-1,1,1e1,1e2,1e3}, 
    ytick={-120,-100,-80,-60,-40,-20,0,20,40,60}, 
    xticklabels={$10^{-3}$,$10^{-2}$,$10^{-1}$,
                 $10^{0}$,$10^{1}$,$10^{2}$,$10^{3}$},
    yticklabels={-120,-100,-80,-60,-40,-20,0,20,40,60}, 
    xmode=log,ymode=normal,
    xmin=1e-3, xmax=1e3,
    ymin=-100, ymax=20,
    grid=both,
    major grid style={black!40},
    cycle list name=color list,
]
    \addplot[line width=2pt,red,dashed,domain=1e-3:1e0, samples=101]{0};
    \addplot[line width=2pt,red,dashed,domain=1e0:1e3, samples=101] 
    {-40*log10(x)};
    \foreach \a in {1.0}
        \addplot[thick,domain=1e-3:1e3, blue,samples=101] 
        {-20*log10(sqrt( (1-x*x)^2 +(2*\a*x)^2 )  )}; 
\end{axis}
\end{tikzpicture}

\tikzsetnextfilename{bode_2nd_2-chap3_ext}
\begin{tikzpicture}[trim axis left]
\begin{axis}[
    legend style={draw=none},    
    legend pos=outer north east, 
    ticklabel style = {font=\footnotesize},
    width=0.9\textwidth,
    height=0.22\textheight,
    xlabel={Pulsation (rad/s)},
    ylabel={Phase (deg)},
    xtick={1e-3,1e-2,1e-1,1,1e1,1e2,1e3}, 
    ytick={-180,-135,-90,-45,0}, 
    yticklabels={-180,-135,-90,-45,0},
    xticklabels={$10^{-3}$,$10^{-2}$,$10^{-1}$,
                 $10^{0}$,$10^{1}$,$10^{2}$,$10^{3}$},
    xmode=log,ymode=normal,
    xmin=1e-3, xmax=1e3,
    ymin=-180, ymax=0,
    grid=both,
    major grid style={black!40},
    cycle list name=color list,
]
    \addplot[line width=2pt,red,dashed,domain=1e-3:1e0, samples=101] {0};
    \addplot[line width=2pt,red,dashed,domain=1e0:1e3, samples=101] {-180};
    \draw[line width=2pt,red,dashed] (axis cs:1,0) -- (axis cs:1,-180);
    \foreach \a in {1.0}
        \addplot+[thick,blue,domain=1e-3:1e3, samples=101] 
        {-atan2(2*\a*x,(1-x*x))}; 
\end{axis}
\end{tikzpicture}
    \caption{Diagramme de Bode d'une fonction de transfert second ordre 
    (\Cref{eq-2nd_ftjw}) avec $K=1$, $\omega_0=1$ et $\xi=1$
    \label{fig-bode_2nd_1}}
\end{figure}

\begin{figure}[!t]
\centering
\tikzsetnextfilename{bode_2nd_3-chap3_ext}
    \newcommand{\LHS}[2][1.5em]{\hspace{#1}\mathllap{#2}}
\begin{tikzpicture}
\begin{axis}[
    name=ax1,
    ticklabel style = {font=\footnotesize},
    width=0.9\textwidth,
    height=0.25\textheight,
    ylabel={Gain (dB)},
    xtick={1e-1,1,1e1},
    ytick={-120,-100,-80,-60,-40,-20,0,20,40,60},
    xticklabels={$10^{-1}$,$10^{0}$,$10^{1}$},
    yticklabels={-120,-100,-80,-60,-40,-20,0,20,40,60},
    xmode=log,ymode=normal,
    xmin=1e-1, xmax=1e1,
    ymin=-40, ymax=30,
    grid=both,
    major grid style={black!40},
    cycle list name=fmvcolist,
]
    \foreach \a in {0.02,0.1,0.2,0.3,0.4,0.5,0.6}
    \addplot+[very thick,domain=1e-1:1e1, samples=201] 
    {-20*log10(sqrt((1-x*x)^2+(2*\a*x)^2))};
\end{axis}
\begin{axis}[
    at={(ax1.south west)},
    yshift=-12em,
%    xshift=-14em,
    legend style={draw=none,yshift=1em},
    legend pos=outer north east,
    ticklabel style = {font=\footnotesize},
    width=0.9\textwidth,
    height=0.25\textheight,
    xlabel={Pulsation (rad/s)},
    ylabel={Phase (degré)},
    xtick={1e-1,1,1e1},
    ytick={-180,-135,-90,-45,0},
    yticklabels={-180,-135,-90,-45,0},
    xticklabels={$10^{-1}$,$10^{0}$,$10^{1}$},
    xmode=log,ymode=normal,
    xmin=1e-1, xmax=1e1,
    ymin=-180, ymax=0,
    grid=both,
    major grid style={black!40},
    cycle list name=fmvcolist,
]
    \foreach \a in {0.02,0.1,0.2,0.3,0.4,0.5,0.6}
    \addplot+[very thick,domain=1e-1:1e1, samples=201] {-atan2(2*\a*x,(1-x*x))};

    \legend{$\LHS{\xi}=0.02$,$\LHS{\xi}=0.1$,$\LHS{\xi}=0.2$, 
            $\LHS{\xi}=0.3$, $\LHS{\xi}=0.4$,$\LHS{\xi}=0.5$, 
            $\LHS{\xi}=0.6$, $\LHS{\xi}=0.7$, $\LHS{\xi}=0.8$, $\LHS{\xi}=0.9$}
\end{axis}
\end{tikzpicture}


    \caption{Diagramme de Bode d'une fonction de transfert du second ordre 
             (\Cref{eq-2nd_ftjw}) pour différentes valeurs de $\xi$ avec 
             $K=1$ et $\omega_0=1$\label{fig-bode_2nd_2}}
\end{figure}
\afterpage{\clearpage}

%%%%%%%%%%%%%%%%%%%%%%%%%%%%%%%%%%%%%%%%%%%%%%%%%%%%%%%%%%%%%%%%%%%%%%%%%%%%%%%%
\paragraph{Phénomène de résonance}
%%%%%%%%%%%%%%%%%%%%%%%%%%%%%%%%%%%%%%%%%%%%%%%%%%%%%%%%%%%%%%%%%%%%%%%%%%%%%%%%
Le gain d'un système du second ordre présente un maximum pour certaines valeurs 
du taux d'amortissement $\xi$. Nous allons établir en détail les différentes 
grandeurs caractéristiques de ce phénomène de résonance. 
L'approche suivante s'inspire en partie de~\cite{laroche}.

Partons du gain naturel $G(\omega)$ d'un système du second ordre pour lequel,
$$
G(\omega)=\dfrac{K\omega^2_0}{\sqrt{(\omega^2_0-\omega^2)^2
+(2\xi\omega_0\omega)^2}}
$$
on pose $X=\omega^2$, et on porte le gain au carré pour éliminer la 
racine carrée. 
On obtient alors,
$$
(G(\omega))^2=\dfrac{K^2\omega^4_0}{(\omega^2_0-X)^2+(2\xi\omega_0)^2X}
$$
Le numérateur étant constant, le gain présentera un maximum si le 
dénominateur présente un minimum. Notons $D(X)$, ce dénominateur qui 
s'écrit:
$$
D(X)=(\omega^2_0-X)^2+(2\xi\omega_0)^2X
$$
Calculons, la dérivée par rapport à $X$,
$$
\dfrac{\mathrm{d}D(X)}{\mathrm{d}X}=-2(\omega^2_0-X)+(2\xi\omega_0)^2
$$
qui s'annule pour 
$$
X=X_0=\omega^2_0(1-2\xi^2).
$$
La dérivée seconde étant positive, le dénominateur $D(X)$ présente un 
minimum en $X_0$. Puisque $X>0$ et $\omega^2_0>0$ alors la condition sur le 
taux d'amortissement est 
\begin{bequation}[ams align]
    \xi<\dfrac{\sqrt{2}}{2}
\end{bequation}
La \textbf{pulsation de résonance} est donc définit par : 
\begin{bequation}[ams align]
\omega_r=\omega_0\sqrt{1-2\xi^2}.
\end{bequation}
La valeur du gain maximal est obtenue à la pulsation de résonance, 
$$
G(\omega_r)=\dfrac{K}{2\xi\sqrt{1-\xi^2}},
$$
ce qui permet de définir le \textbf{facteur de surtension} $Q$ qui est le 
rapport entre le maximum atteint par le gain et la valeur de l'asymptote 
à basse fréquence, d'où 
\begin{bequation}[ams align]
    Q=\dfrac{1}{2\xi\sqrt{1-\xi^2}}
\end{bequation}

D'après ces dernières expressions, on observe qu'à la limite $\xi\to0$, 
la pulsation de résonance $\omega_r$ tend vers $\omega_0$, et le gain 
maximal tend lui vers l'infini.
La pulsation $\omega_0$ est donc la valeur pour lequel le phénomène de 
résonance est le plus intense.
La~\cref{fig-gain_2nd} présente la position du gain  maximum à la pulsation 
de résonance pour différentes valeurs du taux d'amortissement.

\begin{figure}[!h]
    \centering
\tikzsetnextfilename{gain_resonance-chap3_ext}
    \begin{tikzpicture}
    \pgfplotscreateplotcyclelist{mycolorlist}{%
            blue\\%
            red\\%
            brown!60!black\\%
            black\\%
            green!60!black\\%
            red!60!yellow\\
            }
    \begin{axis}
    [   ticklabel style = {font=\normalsize},
        legend style={draw=none},
        legend pos=outer north east,
        legend cell align={left},
        ylabel={Gain (dB)},
        xlabel={Pulsation (rad/s)},
        xmode=normal,ymode=normal,
        xmin=0.0, xmax=2,
        ymin=-8, ymax=10,
        major grid style={black!40},
        cycle list name=mycolorlist,
    ]
    \foreach \a in {0.2,0.3,0.4,0.5,0.6,0.707} 
    \addplot+[thick,domain=0:2,samples=201] 
    {-20*log10(sqrt((1-x*x)^2 +(2*\a*x)^2))};

    \addplot[dashed,domain=0.1:5,samples=201] {0};
    \def\a{1.0}
    \addplot[dashed,domain=0:2,samples=201] 
    {-20*log10(sqrt((1-x*x)^2 +(2*\a*x)^2))};
    \coordinate (P) at 
    (axis cs:0.75,{-20*log10(sqrt((1-0.75*0.75)^2 +(2*\a*0.75)^2 ))});
    \node[left] (a) at (axis cs:0.5,-5) {$\xi=1$};
    \draw [thick] (a.east) -- (P);
    \addplot[mark size=1.75pt,black,fill=black,mark=*,only marks] 
    coordinates {
            (0.959166304663,8.13608784305)
            (0.905538513814,4.84656106912)
            (0.824621125124,2.69540739954)
            (0.707106781187,1.24938736608)
            (0.529150262213,0.354575339209)
    };
    \draw[dashed] (axis cs:1,-10) -- (axis cs:1,10);
    \legend{$\xi=0.2$,$\xi=0.3$,$\xi=0.4$,$\xi=0.5$,$\xi=0.6$,$\xi=\sqrt{2}/2$}
    \end{axis}
\end{tikzpicture}


    \caption{\'Evolution du gain en décibel en fonction de la pulsation 
             pour différentes valeurs du taux d'amortissement du régime 
             pseudo-périodique. Le gain maximal à la pulsation de résonance 
             $\omega_r$ est représenté par une pastille noir sur chacune des 
             courbes pour $\xi<\sqrt{2}/2$. On remarquera l'utilisation 
             exceptionnelle d'une échelle linéaire pour les pulsations.
             \label{fig-gain_2nd}}
\end{figure}
\afterpage{\clearpage}
\newpage

%%%%%%%%%%%%%%%%%%%%%%%%%%%%%%%%%%%%%%%%%%%%%%%%%%%%%%%%%%%%%%%%%%%%%%%%%%%%%%%%
\subsubsection{Diagramme de Bode d'un système d'ordre quelconque}
%%%%%%%%%%%%%%%%%%%%%%%%%%%%%%%%%%%%%%%%%%%%%%%%%%%%%%%%%%%%%%%%%%%%%%%%%%%%%%%%
Dans le cas d'un système d'ordre supérieur à deux, nous allons utiliser 
les propriétés d'additivité des diagrammes de Bode, en décomposant la fonction 
de transfert en différents modèles simples.

Il est notamment toujours possible d'écrire une fonction de transfert 
(~\cref{chap-model}) sous la forme d'un produit de gains purs, 
d'intégrateurs, de dérivateurs, de systèmes du premier et du second ordre: 
\begin{bequation}[ams align]
H(p)= K_0p^{\alpha}\prod_{i} (1+\tau_ip)^{n_i}\prod_{j} 
      (1+2\xi_j\tau_jp+\tau_jp^2)^{n_j}
\end{bequation}
où les exposants $\alpha$, $n_i$ et $n_j$ peuvent être positifs et négatifs. 

Nous listons ci-dessous l'effet sur le gain et la phase d'un diagramme de 
Bode pour chacuns de ces élements selon le signe des exposants $\alpha$, 
$n_i$, et $n_j$.

\begin{itemize}
    \item le terme $K_0$ (i.e gain pur) provoque:
    \begin{itemize}
        \item gain  : $+20\log{K_0}$
        \item phase : rien 
    \end{itemize}
    \item le terme $K_0p^{\alpha}$ (i.e intégrateur si $\alpha<0$ ou 
          dérivateur si $\alpha>0$) provoque :
        \begin{itemize}
            \item gain  : pente de 20$\alpha$ dB/décade 
            \item phase : 90$\alpha$\si{\degree}
        \end{itemize}
    \item un terme $\dfrac{1}{(1+\tau_ip)}$ (i.e premier ordre au 
          dénominateur si $n_i=-1$) provoque, en $\omega=\frac{1}{\tau_i}$
        \begin{itemize}
            \item gain  : une rupture de pente de -20 dB/décade 
            \item phase : un saut de -90\si{\degree}
        \end{itemize}
    \item un terme $(1+\tau_ip)$ (i.e premier ordre au numérateur si $n_i=1$) 
          provoque, en $\omega=\frac{1}{\tau_i}$
        \begin{itemize}
            \item gain  : une rupture de pente de +20 dB/décade 
            \item phase : un saut de +90\si{\degree}
        \end{itemize}
    \item un terme $\dfrac{1}{(1+2\xi_j\tau_jp+\tau_jp^2)}$ 
          (i.e second ordre au dénominateur si $n_j=-1$) provoque, 
          en $\omega=\frac{1}{\tau_j}$
        \begin{itemize}
            \item gain  : une rupture de pente de -40 dB/décade 
            \item phase : un saut de -180\si{\degree}
        \end{itemize}
    \item un terme $(1+2\xi_j\tau_jp+\tau_jp^2)$ (i.e second ordre au 
          numérateur si $n_j=-1$) provoque, en $\omega=\frac{1}{\tau_j}$
        \begin{itemize}
            \item gain  : une rupture de pente de +40 dB/décade 
            \item phase : un saut de +180\si{\degree}
        \end{itemize}
\end{itemize}

%%%%%%%%%%%%%%%%%%%%%%%%%%%%%%%%%%%%%%%%%%%%%%%%%%%%%%%%%%%%%%%%%%%%%%%%%%%%%%%%
%%%%%%%%%%%%%%%%%%%%%%%%%%%%%%%%%%%%%%%%%%%%%%%%%%%%%%%%%%%%%%%%%%%%%%%%%%%%%%%%
\subsection*{Exemple}
%%%%%%%%%%%%%%%%%%%%%%%%%%%%%%%%%%%%%%%%%%%%%%%%%%%%%%%%%%%%%%%%%%%%%%%%%%%%%%%%
%%%%%%%%%%%%%%%%%%%%%%%%%%%%%%%%%%%%%%%%%%%%%%%%%%%%%%%%%%%%%%%%%%%%%%%%%%%%%%%%

Soit la fonction de transfert $H(p)$ telle que 

\begin{align}
    H(p) = \dfrac{100(p+1)^2}{(100p+1)(10p+1)(0.01p+1)}\label{eq-ft_qq}
\end{align}
La première étape consiste à ordonner les temps caractéristiques par ordre 
décroissant cela nous permettra d'obtenir les pulsations propres par ordre 
croissant. Ensuite, il faut identifier les différents modèles.
Pour cet exemple, nous identifions :
\begin{itemize}
    \item un gain pur $K_0=100$
    \item un second ordre double au numérateur de temps caractéristique 
          $\tau=1$
    \item trois premier ordre au dénominateur de temps caractéristique 
          $\tau=\{0.01,10,100\}$
\end{itemize}

Enfin, nous regroupons dans un tableau l'effet sur le gain et la phase pour 
chaque domaines en pulsations compris entre les différentes pulsations 
caractéristiques.
On adopte la notation suivante : $\tau_1=100$, $\tau_2=10$, $\tau_3=1$ 
et $\tau_4=0.01$, avec $\omega_i=1/\tau_i$, on obtient alors:
$\omega_1=0.01$, $\omega_2=0.1$, $\omega_3=1$ et $\omega_4=100$.

\begin{table}[!h]
    \ra{1.3}
    \centering
    \resizebox{\linewidth}{!}{
    \begin{tabular}{@{}P{1cm}P{2.5cm}P{2.5cm}P{2.5cm}P{2.5cm}P{2.5cm}@{}}
    \toprule
    & $\omega\ll\omega_1$ 
    & $\omega_1<\omega<\omega_2$ 
    & $\omega_2<\omega<\omega_3$  
    & $\omega_3<\omega<\omega_4$ 
    & $\omega\gg\omega_4$ \\[0em]
    \midrule
    $G_{dB}(\omega)$ (pente) & 0(40dB)       &-20dB/décade    & -20dB/décade 
                             & +40dB/décade & -20dB/décade \\[0em]
        $\phi(\omega)$       & 0\si{\degree} &-90\si{\degree} & -90\si{\degree} 
                             & +180\degree  & -90\degree   \\[0em]
    \midrule
    $G_{dB}(\omega)$ total   & 0(40dB)       &-20dB/décade    & -40dB/décade    
                             & 0(-20dB)     & -20dB/décade \\[0em]
    \midrule
    $\phi(\omega)$   total   & 0\degree      &-90\degree      & -180\degree     
                             & 0            & -90\degree   \\[0em]
    \bottomrule
    \end{tabular}}
\end{table}


\begin{figure}[!t]
\centering
\tikzsetnextfilename{bode_exemple_1-chap3_ext}
\begin{tikzpicture}[trim axis left]
\begin{axis}[
    ticklabel style = {font=\footnotesize},
    width=0.9\textwidth,
    height=0.25\textheight,
    ylabel={Gain (dB)},
    xtick={1e-4,1e-3,1e-2,1e-1,1,1e1,1e2,1e3,1e4,1e5}, 
    ytick={-80,-60,-40,-20,0,20,40,60}, 
    xticklabels={$10^{-4}$,$10^{-3}$,$10^{-2}$,$10^{-1}$,
                 $10^{0}$,$10^{1}$,$10^{2}$,$10^{3}$,$10^{4}$,$10^{5}$},
    yticklabels={-80,-60,-40,-20,0,20,40,60}, 
    xmode=log,ymode=normal,
    xmin=1e-4, xmax=1e5,
    ymin=-80, ymax=60,
    grid=both,
    major grid style={black!40}
    ]
    \addplot[ultra thick, blue,domain=1e-4:1e5, samples=201] 
    {40+20*log10(1+x*x)-10*log10(1+10000*x*x)-10*log10(1+100*x*x)-
     10*log10(1+0.0001*x*x)}; 
    \addplot[line width=2pt,red,dashed,domain=1e-4:1e-2, samples=101]
    {40};
    \addplot[line width=2pt,red,dashed,domain=1e-2:1e-1, samples=101]
    {-20*log10(x)};
    \addplot[line width=2pt,red,dashed,domain=1e-1:1e0 , samples=101]
    {-40*log10(x)-20*log10(10)};
    \addplot[line width=2pt,red,dashed,domain=1e0:1e2  , samples=101]
    {-20};
    \addplot[line width=2pt,red,dashed,domain=1e2:1e5  , samples=101]
    {-20*log10(x)+20*log10(10)};
\end{axis}
\end{tikzpicture}

\tikzsetnextfilename{bode_exemple_2-chap3_ext}
\begin{tikzpicture}[trim axis left]
\begin{axis}
    [
    ticklabel style = {font=\footnotesize},
    width=0.9\textwidth,
    height=0.25\textheight,
    xlabel={Pulsation (rad/s)},
    ylabel={Phase (deg)},
    xtick={1e-4,1e-3,1e-2,1e-1,1,1e1,1e2,1e3,1e4,1e5}, 
    ytick={-180,-135,-90,-45,0}, 
    yticklabels={-180,-135,-90,-45,0},
    xticklabels={$10^{-4}$,$10^{-3}$,$10^{-2}$,$10^{-1}$,
                 $10^{0}$,$10^{1}$,$10^{2}$,$10^{3}$,$10^{4}$,$10^{5}$},
    xmode=log,ymode=normal,
    xmin=1e-4, xmax=1e5,
    ymin=-180, ymax=0,
    grid=both,
    major grid style={black!40}
    ]
    \addplot[ultra thick, blue,domain=1e-4:1e5, samples=201] 
    {2*atan2(x,1)-atan2(100*x,1)-atan2(10*x,1)-atan2(0.01*x,1)}; 
    \addplot[line width=2pt,red,dashed,domain=1e-4:1e-2, samples=101] {0};
    \addplot[line width=2pt,red,dashed,domain=1e-2:1e-1, samples=101] {-90};
    \addplot[line width=2pt,red,dashed,domain=1e-1:1e0 , samples=101] {-180};
    \addplot[line width=2pt,red,dashed,domain=1e0:1e2  , samples=101] {0};
    \addplot[line width=2pt,red,dashed,domain=1e2:1e5  , samples=101] {-90};
    \draw[line width=2pt,red,dashed] (axis cs:0.01,0)  -- (axis cs:0.01,-90);
    \draw[line width=2pt,red,dashed] (axis cs:0.1,-90) -- (axis cs:0.1,-180);
    \draw[line width=2pt,red,dashed] (axis cs:1,-180)  -- (axis cs:1,0);
    \draw[line width=2pt,red,dashed] (axis cs:100,0)   -- (axis cs:100,-90);
\end{axis}
\end{tikzpicture}
    \caption{Diagramme de Bode du système d'ordre quelconque de 
             l'\cref{eq-ft_qq} (bleu) diagramme de Bode réel et (rouge) 
             diagramme de Bode asymptotique.\label{fig-bode_qq}}
\end{figure}

Il est également possible de déterminer la forme analytique du gain et 
de la phase.
\begin{bequation}[ams align]
G_{dB}(\omega)=40+20\log{(1+\tau_3^2\omega^2)}
                 -10\log{(1+\tau_1^2\omega^2)
                 (1+\tau_2^2\omega^2)(1+\tau_4^2\omega^2)}
\end{bequation}
et
\begin{bequation}[ams align]
\phi(\omega)=2\arctan{\tau_3\omega}-\arctan{\tau_1\omega}
             -\arctan{\tau_2\omega}-\arctan{\tau_4\omega} 
\end{bequation}

\clearpage
%%%%%%%%%%%%%%%%%%%%%%%%%%%%%%%%%%%%%%%%%%%%%%%%%%%%%%%%%%%%%%%%%%%%%%%%%%%%%%%%
%%%%%%%%%%%%%%%%%%%%%%%%%%%%%%%%%%%%%%%%%%%%%%%%%%%%%%%%%%%%%%%%%%%%%%%%%%%%%%%%
\subsection{Diagrammes de Nyquist: méthodologie générale}
%%%%%%%%%%%%%%%%%%%%%%%%%%%%%%%%%%%%%%%%%%%%%%%%%%%%%%%%%%%%%%%%%%%%%%%%%%%%%%%%
%%%%%%%%%%%%%%%%%%%%%%%%%%%%%%%%%%%%%%%%%%%%%%%%%%%%%%%%%%%%%%%%%%%%%%%%%%%%%%%%
Pour chacuns des modèles usuels, nous appliquerons la procédure suivante :
\begin{itemize}
    \item Définir la fonction de transfert $H(p)$ du modèle pour $p=\jw$
    \item \'Etablir la partie réelle et imaginaire du nombre complexe $H(\jw)$
    \item Tracer le lieu de Nyquist point par point, pour différentes valeurs de
        $\omega$ de 0 à $+\infty$, c'est à dire $\Re{H(\jw)}$ 
          et $\Im{H(\jw)}$ dans le plan complexe.
\end{itemize}
Dans chacuns des exemples suivants nous reproduisons le lieu de Nyquist 
complet le domaine des pulsations négatives étant représenté en pointillé. 
Dans la pratique, il suffit de tracer le symétrique par rapport à 
l'axe des réels et d'inverser le sens de la flêche pour obtenir le sens de la 
pulsation de $-\infty\to 0$. 
\newpage

%%%%%%%%%%%%%%%%%%%%%%%%%%%%%%%%%%%%%%%%%%%%%%%%%%%%%%%%%%%%%%%%%%%%%%%%%%%%%%%%
\subsubsection{Diagramme de Nyquist d'un gain pur}
%%%%%%%%%%%%%%%%%%%%%%%%%%%%%%%%%%%%%%%%%%%%%%%%%%%%%%%%%%%%%%%%%%%%%%%%%%%%%%%%

Le diagramme de Nyquist d'un gain pur est trivial. En effet le nombre complexe 
$H(\jw)$ étant égal à une constante réel $K$, le diagramme de Nyquist ce 
limite à un point sur l'axe des réels quelque soit la valeur de $\omega$.
Ce qui est en accord avec le fait qu'un gain pur présente un déphasage nul.
\begin{align*}
    \Re{H(\jw)}&=K\\
    \Im{H(\jw)}&=0\\
\end{align*}

\begin{figure}[!h]
\begin{center}
\tikzsetnextfilename{nyquist_gainpur-chap3_ext}
\begin{tikzpicture}
\begin{axis}
    [
    axis line style = thick,
    xlabel=$\Re{H(\jw)}$,
    ylabel=$\Im{H(\jw)}$,
    ymin=-10,
    ymax=+10,
    xmin=-10,
    xmax=+10,
    grid=both,
    ]     
    \node [below]       at (axis cs:  5, 0)   {$\forall\omega$};
    \node [above,blue]       at (axis cs:  5, 0)   {$K$};
    \addplot [blue, mark = *] coordinates {(5, 0)} {};
\end{axis}
\end{tikzpicture}
\end{center}
\caption{Diagramme de Nyquist d'un gain pur. Le nombre complexe $H(\jw)$ est 
         représenté par un point sur l'axe des réels à la valeur $K$ 
         (ici $K=5$). \label{fig-nyquist_1}}
\end{figure}


\newpage
%%%%%%%%%%%%%%%%%%%%%%%%%%%%%%%%%%%%%%%%%%%%%%%%%%%%%%%%%%%%%%%%%%%%%%%%%%%%%%%%
\subsubsection{Diagramme de Nyquist d'un intégrateur pur}
%%%%%%%%%%%%%%%%%%%%%%%%%%%%%%%%%%%%%%%%%%%%%%%%%%%%%%%%%%%%%%%%%%%%%%%%%%%%%%%%

Le diagramme de Nyquist d'un intégrateur pur est également trivial, puisque 
le nombre complexe $H(\jw)=\dfrac{K}{\jw}$ est un nombre imaginaire pur. 
Cependant il dépend de la pulsation $\omega$.
\begin{align*}
    \Re{H(\jw)}&=0\\
    \Im{H(\jw)}&=\dfrac{-K}{\omega}
\end{align*}
\begin{figure}[!h]
\begin{center}
\tikzsetnextfilename{nyquist_integ-chap3_ext}
\begin{tikzpicture}
\begin{axis}
    [
    axis line style = thick,
    xlabel=$\Re{H(\jw)}$,
    ylabel=$\Im{H(\jw)}$,
    ymin=-10,
    ymax=+10,
    xmin=-10,
    xmax=+10,
    grid=both,
    ] 
    \node [above right] at (axis cs:  0, -10)     {$\omega=0^+$};
    \node [below right] at (axis cs:  0,  10)     {$\omega=0^-$};
    \node [below right]       at (axis cs:  0, 0) {$\omega\rightarrow+\infty$};
    \node [above right]       at (axis cs:  0, 0) {$\omega\rightarrow-\infty$};
\addplot [blue, mark = *] coordinates {(0, 0)}{};
\addplot[blue,thick,domain=0.01:0.2,samples=50,-{Latex[length=3mm]}]
    (0,{-1/x});
\addplot[blue,thick,domain=0.1:100,samples=50]
    (0,{-1/x});
\addplot[dashed,blue,thick,domain=-100:-0.2,-{Latex[length=3mm]},samples=100]
    (0,{-1/x});
\addplot[dashed,blue,thick,domain=-0.2:-0.01,samples=100](0,{-1/x});
\end{axis}
\end{tikzpicture}
\end{center}
\caption{Diagramme de Nyquist d'un intégrateur pur. Le lieu de Nyquist 
         est représenté par une demi droite sur l'axe des nombres 
         imaginaires purs négatifs.\label{fig-nyquist_2}}
\end{figure}

\newpage
%%%%%%%%%%%%%%%%%%%%%%%%%%%%%%%%%%%%%%%%%%%%%%%%%%%%%%%%%%%%%%%%%%%%%%%%%%%%%%%%
\subsubsection{Diagramme de Nyquist d'un dérivateur pur}
%%%%%%%%%%%%%%%%%%%%%%%%%%%%%%%%%%%%%%%%%%%%%%%%%%%%%%%%%%%%%%%%%%%%%%%%%%%%%%%%

Le diagramme de Nyquist d'un dérivateur pur est également représentatif d'un 
nombre complexe $H(\jw)=K\jw$ imaginaire pur. Les parties réelles et imaginaire
de ce nombre complexe sont :
\begin{align*}
    \Re{H(\jw)}&=0\\
    \Im{H(\jw)}&=K\omega
\end{align*}
\begin{figure}[!h]
\begin{center}
\tikzsetnextfilename{nyquist_deriv-chap3_ext}
\begin{tikzpicture}
\begin{axis}
    [
    axis line style = thick,
    xlabel=$\Re{H(\jw)}$,
    ylabel=$\Im{H(\jw)}$,
    ymin=-10,
    ymax=+10,
    xmin=-10,
    xmax=+10,
    grid=both,
    ]     
    \node [below right] at (axis cs:  0, 10)  {$\omega\rightarrow\infty$};
    \node [above right] at (axis cs:  0,-10)  {$\omega\rightarrow-\infty$};
    \node [below right]       at (axis cs:  0, 0)   {$\omega=0$};
    \addplot [blue, mark = *] coordinates {(0, 0)} {};
    \addplot [blue,thick,domain=0:5,samples=50,-{Latex[length=3mm]}](0,{x});
    \addplot [blue,thick,domain=4:10,samples=50](0,{x});
    \addplot [dashed,blue,thick,domain=-10:-5,samples=50,-{Latex[length=3mm]}]
    (0,{x});
    \addplot [dashed,blue,thick,domain=-6:0,samples=50]
    (0,{x}); 
\end{axis}
\end{tikzpicture}
\end{center}
\caption{Diagramme de Nyquist d'un dérivateur pur. Le lieu de Nyquist 
         est représenté par une demi droite sur l'axe des nombres 
         imaginaires purs positifs.\label{fig-nyquist_3}}
\end{figure}

\newpage
%%%%%%%%%%%%%%%%%%%%%%%%%%%%%%%%%%%%%%%%%%%%%%%%%%%%%%%%%%%%%%%%%%%%%%%%%%%%%%%%
\subsubsection{Diagramme de Nyquist d'un retard pur}
%%%%%%%%%%%%%%%%%%%%%%%%%%%%%%%%%%%%%%%%%%%%%%%%%%%%%%%%%%%%%%%%%%%%%%%%%%%%%%%%

\index{Retard pur!diagramme de Nyquist}
La fonction de transfert d'un retard pur s'écrit :
$$
H(\jw)=e^{-j\tau\omega}
$$
Les parties réelles et imaginaires sont simplement donnés par :
\begin{align*}
    \Re{H(\jw)}&=\hphantom{-}\cos{\tau\omega}\\
    \Im{H(\jw)}&=-\sin{\tau\omega}
\end{align*}

Ces coordonnées dans le plan complexe sont celles du cercle unité 
centré sur l'origine. Le lieu de transfert (c.a.d $\omega\to\infty$) 
est la rotation infinie sur ce cercle. La \og vitesse angulaire\fg dépend 
de $\tau$.

\begin{figure}[!h]
\begin{center}
\tikzsetnextfilename{nyquist_retard-chap3_ext}
\begin{tikzpicture}
\begin{axis}
    [
    axis line style = thick,
    xlabel=$\Re{H(\jw)}$,
    ylabel=$\Im{H(\jw)}$,
    ymin=-1.5,
    ymax=+1.5,
    xmin=-1.5,
    xmax=+1.5,
    grid=both,
    ]     
\node[right]       at (axis cs:  1, 0)   {\scriptsize$\omega=0$};
\addplot[blue, mark = *] coordinates {(1, 0)} {};
\addplot[blue,thick,domain=pi/2:-pi/2-0.2,samples=50,-{Latex[length=3mm]}]
    ({cos(deg(x))},{sin(deg(x)}); 
\addplot [blue,thick,domain=-pi/2:-3*pi/2-0.2,samples=50,-{Latex[length=3mm]}]
    ({cos(deg(x))},{sin(deg(x)}); 
\end{axis}
\end{tikzpicture}
\end{center}
\caption{Diagramme de Nyquist d'un retard pur. Le lieu de Nyquist 
         est représenté par le cercle unité dans le plan complexe.
         \label{fig-nyquist_4}}
\end{figure}

On remarquera cependant que ce modèle est fondamentalement 
\textbf{instable} puisque le module $|H(\jw)|$ ne s'annule pas 
lorsque $\omega\to\infty$, de ce fait nous le trouverons jamais seul.

\newpage
%%%%%%%%%%%%%%%%%%%%%%%%%%%%%%%%%%%%%%%%%%%%%%%%%%%%%%%%%%%%%%%%%%%%%%%%%%%%%%%%
\subsubsection{Diagramme de Nyquist d'un système du premier ordre}
%%%%%%%%%%%%%%%%%%%%%%%%%%%%%%%%%%%%%%%%%%%%%%%%%%%%%%%%%%%%%%%%%%%%%%%%%%%%%%%%

\index{Système du premier ordre!diagramme de Nyquist}
La fonction de transfert d'un système du premier ordre s'écrit:
$$
H(\jw)=\dfrac{K}{1+j\tau\omega}
$$
Les parties réelle et imaginaire de cette fonction de transfert sont données 
par:
\begin{align*}
    \Re{H(\jw)}&=\dfrac{K}{1+\tau^2\omega^2}\\
    \Im{H(\jw)}&=-\dfrac{K\tau\omega}{1+\tau^2\omega^2}\\
\end{align*}
Nous avons regroupé dans le~\cref{tab-nyquist-vp_1er} quelques valeurs 
particulières de $\Re{H(\jw)}$ et $\Im{H(\jw)}$ pour quelques valeurs 
de $\omega$.

Le lieu complet de Nyquist d'un système du premier ordre à la forme d'un 
cercle, nous allons établir ses caractéristiques\cite{9782729860127}.

Posons tout d'abord, 
$$
X=\Re{H(\jw)}=\dfrac{K}{1+\tau^2\omega^2}
$$
on peut écrire,
$$
\tau^2\omega^2=\dfrac{K}{X}-1
$$
En posant maintenant, 
$$
Y=\Im{H(\jw)}=\dfrac{\tau\omega}{1+\tau^2\omega}=-\tau\omega X
$$
on obtient une relation entre $Y$ et $X$:
$$
Y^2=\left(\dfrac{K}{X}-1\right)X^2
$$
on reconnaît alors l'équation d'un cercle de centre $(K/2,0)$ et 
de rayon $K/2$
$$
\left(X-\dfrac{K}{2}\right)^2+Y^2=\left(\dfrac{K}{2}\right)^2
$$
\begin{table}
    \ra{1.3}
    \centering
    \begin{tabular}{P{2.0cm}P{2.0cm}P{2.0cm}P{2.0cm}}
        \toprule
        & $\omega=0$  & $\omega\to\infty$ & $\omega=\dfrac{1}{\tau}$ \\[0em]
        \midrule
        $\Re{H(\jw)}$ & $K$               & 0               & $K/2$  \\[0em]
        \midrule
        $\Im{H(\jw)}$ & 0                 & 0               & $-K/2$ \\[0em]
        \bottomrule
    \end{tabular}
    \caption{Quelques valeurs particulières de $\Re{H(\jw)}$ et $\Im{H(\jw)}$
    selon $\omega$ pour un système du premier ordre\label{tab-nyquist-vp_1er}.}
\end{table}

\begin{figure}[!h]
\begin{center}
\tikzsetnextfilename{nyquist_1er-chap3_ext}
\begin{tikzpicture}
\begin{axis}
    [
    axis line style = thick,
    xlabel=$\Re{H(\jw)}$,
    ylabel=$\Im{H(\jw)}$,
    ymin=-0.75,
    ymax=+0.75,
    xmin=-0.3,
    xmax=+1.3,
    clip=false,
    grid=both
    ]
    \addplot [blue,thick,domain=0:1.1,samples=100,-{Latex[length=3mm]}]
    ({1/(1+x*x)},{-x/(1+x*x)}); 
    \addplot [blue,thick,domain=0.5:5,samples=100]({1/(1+x*x)},{-x/(1+x*x)});
    \addplot [blue,thick,domain=5:10,samples=100]({1/(1+x*x)},{-x/(1+x*x)});
    \addplot [blue,thick,domain=10:100,samples=100]({1/(1+x*x)},{-x/(1+x*x)});
    \addplot [dashed,blue,thick,domain=-100:-10,samples=100]
    ({1/(1+x*x)},{-x/(1+x*x)});
    \addplot [dashed,blue,thick,domain=-10:-5,samples=100]
    ({1/(1+x*x)},{-x/(1+x*x)});
    \addplot [dashed,blue,thick,domain=-5:-0.5,samples=100,-{Latex[length=3mm]}]
    ({1/(1+x*x)},{-x/(1+x*x)}); 
    \addplot [dashed,blue,thick,domain=-0.5:0,samples=100]
    ({1/(1+x*x)},{-x/(1+x*x)});

    \node [above left,xshift=0.2em] 
    at (axis cs:  0, 0)  {\scriptsize$\omega\rightarrow-\infty$};         
    \node [below left,xshift=0.2em] 
    at (axis cs:  0, 0)  {\scriptsize$\omega\rightarrow+\infty$};         
    \node [right]       at (axis cs:  1, 0) {\scriptsize$\omega=0$};
    \addplot[blue, mark = *] coordinates {(0, 0)} {};
    \addplot[blue, mark = *] coordinates {(1, 0)} {};
%    \draw[blue,ultra thick,-latex] (axis cs:  0,  0 ) -- (axis cs:  0, 5 );
%    \draw[blue,ultra thick] (axis cs:  0,  4 ) -- (axis cs:  0, 9 );
\end{axis}
\end{tikzpicture}
\end{center}
\caption{Diagramme de Nyquist d'un système du premier ordre. 
         Avec $K=1$ et $\tau=1$. Le lieu de Nyquist 
         est représenté par un demi cercle idans le plan des nombres 
         imaginaires négatifs. Le lieu de Nyquist complet 
         correspond à un cercle de rayon $K/2$ et de centre $(K/2,0)$ 
         \label{fig-nyquist_1er}}
\end{figure}


\newpage

%%%%%%%%%%%%%%%%%%%%%%%%%%%%%%%%%%%%%%%%%%%%%%%%%%%%%%%%%%%%%%%%%%%%%%%%%%%%%%%%
\subsubsection{Diagramme de Nyquist d'un système du second ordre}
%%%%%%%%%%%%%%%%%%%%%%%%%%%%%%%%%%%%%%%%%%%%%%%%%%%%%%%%%%%%%%%%%%%%%%%%%%%%%%%%

\index{Système du second ordre!diagramme de Nyquist}
La fonction de transfert d'un système du second ordre s'écrit:
$$
H(\jw)=\dfrac{K\omega_0^2}{(\omega_0^2-\omega^2)+j2\xi\omega_0\omega}
$$
Les parties réel et imaginaire de cette fonction de transfert sont données par:
\begin{align*}
    \Re{H(\jw)}&=
    \dfrac{K\omega_0^2(\omega_0^2-\omega^2)}{(\omega_0^2-\omega^2)^2
    +4\xi^2\omega_0^2\omega^2}\\
    \Im{H(\jw)}&=\dfrac{-2\xi\omega_0^2\omega^2}{(\omega_0^2-\omega^2)^2
    +4\xi^2\omega_0^2\omega^2}\\
\end{align*}
\begin{figure}[!h]
\begin{center}
\tikzsetnextfilename{nyquist_2nd_1-chap3_ext}
\begin{tikzpicture}
\begin{axis}
    [
    axis line style = thick,
    xlabel=$\Re{H(\jw)}$,
    ylabel=$\Im{H(\jw)}$,
    ymin=-1.5,
    ymax=+1.5,
    xmin=-0.5,
    xmax=+1.3,
    clip=false,
    grid=both
    ]
    \tikzstyle{lieua}=[blue,thick,samples=150,-{Latex[length=3mm]}]
    \tikzstyle{lieu}=[blue,thick,samples=150]
    \tikzstyle{lieua_c}=[dashed,blue,thick,samples=150,-{Latex[length=3mm]}]
    \tikzstyle{lieu_c}=[dashed,blue,thick,samples=150]
    
    \def\xi{0.5}
\addplot[lieua,domain=0:1]
    ({(1-x^2)/((1-x^2)^2+4*\xi^2*x^2)},{-(2*\xi*x)/((1-x^2)^2+4*\xi^2*x^2)}); 
\addplot[lieu, domain=0.99:5]
    ({(1-x^2)/((1-x^2)^2+4*\xi^2*x^2)},{-(2*\xi*x)/((1-x^2)^2+4*\xi^2*x^2)}); 
\addplot[lieu,domain=5:10]
    ({(1-x^2)/((1-x^2)^2+4*\xi^2*x^2)},{-(2*\xi*x)/((1-x^2)^2+4*\xi^2*x^2)}); 
\addplot[lieu,domain=10:100]
    ({(1-x^2)/((1-x^2)^2+4*\xi^2*x^2)},{-(2*\xi*x)/((1-x^2)^2+4*\xi^2*x^2)}); 

\addplot[lieu_c,domain=-100:-10]
    ({(1-x^2)/((1-x^2)^2+4*\xi^2*x^2)},{-(2*\xi*x)/((1-x^2)^2+4*\xi^2*x^2)}); 
\addplot[lieu_c,domain=-10:-5]
    ({(1-x^2)/((1-x^2)^2+4*\xi^2*x^2)},{-(2*\xi*x)/((1-x^2)^2+4*\xi^2*x^2)}); 
\addplot[lieua_c,domain=-5:-0.99]
    ({(1-x^2)/((1-x^2)^2+4*\xi^2*x^2)},{-(2*\xi*x)/((1-x^2)^2+4*\xi^2*x^2)}); 
\addplot[lieu_c,domain=-0.99:0]
    ({(1-x^2)/((1-x^2)^2+4*\xi^2*x^2)},{-(2*\xi*x)/((1-x^2)^2+4*\xi^2*x^2)});

\node[above right] at (axis cs:  0, 0){\scriptsize$\omega\rightarrow-\infty$};
\node[right]       at (axis cs:  1, 0){\scriptsize$\omega=0$};       
\node[below right] at (axis cs:  0, 0){\scriptsize$\omega\rightarrow+\infty$};
\addplot[blue, mark = *] coordinates {(0, 0)} {};                  
\addplot[blue, mark = *] coordinates {(1, 0)} {};                  
\end{axis}
\end{tikzpicture}
\end{center}
\caption{Diagramme de Nyquist d'un système du second ordre. 
         Avec $K=1$ et $\tau=1$. Le lieu de Nyquist est représenté par 
         une demi cardio\"iode dans le plan des nombres imaginaires 
         négatifs.\label{fig-nyquist_2nd_1}}
\end{figure}
\begin{figure}[!h]
\begin{center}
\tikzsetnextfilename{nyquist_2nd_2-chap3_ext}
\begin{tikzpicture}
        \pgfplotscreateplotcyclelist{mycolorlist}{%
            blue\\%
            blue\\%
            blue\\%
            blue\\%
            blue\\%
            blue\\%
            blue\\%
            blue\\%
            red\\%
            red\\%
            red\\%
            red\\%
            red\\%
            red\\%
            red\\%
            red\\%
            brown!60!black\\%
            brown!60!black\\%
            brown!60!black\\%
            brown!60!black\\%
            brown!60!black\\%
            brown!60!black\\%
            brown!60!black\\%
            brown!60!black\\%
            black\\%
            black\\%
            black\\%
            black\\%
            black\\%
            black\\%
            black\\%
            black\\%
            }
\begin{axis}
    [
    name=nyq1,
    axis line style = thick,
    xlabel=$\Re{H(\jw)}$,
    ylabel=$\Im{H(\jw)}$,
    ymin=-2,
    ymax=+2,
    xmin=-0.8,
    xmax=+1.3,
    grid=both,
    cycle list name=mycolorlist,
    legend style={draw=none,yshift=1em},
    legend pos=outer north east, 
    ]
    \tikzstyle{lieua}=[thick,samples=100,-{Latex[length=3mm]}]
    \tikzstyle{lieu}=[thick,samples=100]
    \tikzstyle{lieua_c}=[dashed,thick,samples=100,-{Latex[length=3mm]}]
    \tikzstyle{lieu_c}=[dashed,thick,samples=100]
    \foreach \xi in {0.4,0.6,0.8,1.0} 
    {
    \addplot+[lieua,domain=0:1]  
    ({(1-x^2)/((1-x^2)^2+4*\xi^2*x^2)},{-(2*\xi*x)/((1-x^2)^2+4*\xi^2*x^2)});
    \addplot+[lieu, domain=0.99:5] 
    ({(1-x^2)/((1-x^2)^2+4*\xi^2*x^2)},{-(2*\xi*x)/((1-x^2)^2+4*\xi^2*x^2)});
    \addplot+[lieu,domain=5:10]    
    ({(1-x^2)/((1-x^2)^2+4*\xi^2*x^2)},{-(2*\xi*x)/((1-x^2)^2+4*\xi^2*x^2)});
    \addplot+[lieu,domain=10:100] 
    ({(1-x^2)/((1-x^2)^2+4*\xi^2*x^2)},{-(2*\xi*x)/((1-x^2)^2+4*\xi^2*x^2)});
    \addplot+[lieu_c,domain=-100:-10]
    ({(1-x^2)/((1-x^2)^2+4*\xi^2*x^2)},{-(2*\xi*x)/((1-x^2)^2+4*\xi^2*x^2)});
    \addplot+[lieu_c,domain=-10:-5] 
    ({(1-x^2)/((1-x^2)^2+4*\xi^2*x^2)},{-(2*\xi*x)/((1-x^2)^2+4*\xi^2*x^2)});
    \addplot+[lieua_c,domain=-5:-0.99]
    ({(1-x^2)/((1-x^2)^2+4*\xi^2*x^2)},{-(2*\xi*x)/((1-x^2)^2+4*\xi^2*x^2)});
    \addplot+[lieu_c,domain=-0.99:0] 
    ({(1-x^2)/((1-x^2)^2+4*\xi^2*x^2)},{-(2*\xi*x)/((1-x^2)^2+4*\xi^2*x^2)});
    }
\legend{,$\xi=0.4$,,,,,,,,$\xi=0.6$,,,,,,,,$\xi=0.8$,,,,,,,,$\xi=1.0$,,,,,,,} 
\end{axis}
        \pgfplotscreateplotcyclelist{mycolorlist}{%
            blue\\%
            blue\\%
            blue\\%
            blue\\%
            blue\\%
            blue\\%
            red\\%
            red\\%
            red\\%
            red\\%
            red\\%
            red\\%
            brown!60!black\\%
            brown!60!black\\%
            brown!60!black\\%
            brown!60!black\\%
            brown!60!black\\%
            brown!60!black\\%
            black\\%
            black\\%
            black\\%
            black\\%
            black\\%
            black\\%
            }
\begin{axis}
    [
    at={(nyq1.below south west)},yshift=-0.2cm,anchor=north west,
    axis line style = thick,
    xlabel=$\Re{H(\jw)}$,
    ylabel=$\Im{H(\jw)}$,
    ymin=-15,
    ymax=+15,
    xmin=-6,
    xmax=+6,
    grid=both,
    cycle list name=mycolorlist,
    legend style={draw=none,yshift=1em},
    legend pos=outer north east,
    ]
    \tikzstyle{lieua}=[thick,samples=200,-{Latex[length=3mm]}]
    \tikzstyle{lieu}=[thick,samples=200]
    \tikzstyle{lieua_c}=[dashed,thick,samples=200,-{Latex[length=3mm]}]
    \tikzstyle{lieu_c}=[dashed,thick,samples=200]
    \foreach \xi in {0.05,0.07,0.1,0.2} 
    {
    \addplot+[lieu,domain=0:0.5]
    ({(1-x^2)/((1-x^2)^2+4*\xi^2*x^2)},{-(2*\xi*x)/((1-x^2)^2+4*\xi^2*x^2)});
    \addplot+[lieua, domain=0.5:1]
    ({(1-x^2)/((1-x^2)^2+4*\xi^2*x^2)},{-(2*\xi*x)/((1-x^2)^2+4*\xi^2*x^2)});
    \addplot+[lieu, domain=0.99:2]
    ({(1-x^2)/((1-x^2)^2+4*\xi^2*x^2)},{-(2*\xi*x)/((1-x^2)^2+4*\xi^2*x^2)});
    
    \addplot+[lieua_c,domain=-2:-1.0] 
    ({(1-x^2)/((1-x^2)^2+4*\xi^2*x^2)},{-(2*\xi*x)/((1-x^2)^2+4*\xi^2*x^2)});
    \addplot+[lieu_c,domain=-1.0:-0.5]
    ({(1-x^2)/((1-x^2)^2+4*\xi^2*x^2)},{-(2*\xi*x)/((1-x^2)^2+4*\xi^2*x^2)});
    \addplot+[lieu_c,domain=-0.5:0] 
    ({(1-x^2)/((1-x^2)^2+4*\xi^2*x^2)},{-(2*\xi*x)/((1-x^2)^2+4*\xi^2*x^2)});
    }
    \legend{$\xi=0.05$,,,,,,$\xi=0.07$,,,,,,$\xi=0.1$,,,,,,$\xi=0.2$,,,,,,} 
\end{axis}
\end{tikzpicture}
\end{center}
\caption{Diagramme de Nyquist d'un système du second ordre pour différentes 
         valeurs du taux d'amortissement $\xi$. Avec $K=1$ et $\tau=1$. Le 
         lieu de Nyquist est représenté par une demi cardio\"iode dans le 
         plan des imaginaires négatifs.\label{fig-nyquist_2nd_2}}
\end{figure}

\newpage
%%%%%%%%%%%%%%%%%%%%%%%%%%%%%%%%%%%%%%%%%%%%%%%%%%%%%%%%%%%%%%%%%%%%%%%%%%%%%%%%
\subsubsection{Effet d'un retard sur le diagramme de Nyquist}
%%%%%%%%%%%%%%%%%%%%%%%%%%%%%%%%%%%%%%%%%%%%%%%%%%%%%%%%%%%%%%%%%%%%%%%%%%%%%%%%
\index{Retard pur!effet d'un retard sur le diagramme de Nyquist}
La fonction de transfert $H_R(\jw)$ d'un retard est donnée par la relation :
$$
H_R(\jw)=e^{-j\tau_1\omega}=\cos{\tau_1\omega}-j\sin{\tau_1\omega}
$$
avec $\tau_1$ le retard. 
\'Etudions l'effet de ce retard sur le diagramme de Nyquist d'un système 
du premier ordre $H(\jw)$. La fonction de transfert modifié est:
$$
H(\jw)=\dfrac{K}{1+j\tau\omega}H_R(\jw)=
\dfrac{K}{1+j\tau\omega}\left(\cos{\tau\omega}-j\sin{\tau\omega}\right)
$$
Les parties réels et imaginaire de la fonction de transfert sont:
\begin{align*}
\Re{H(\jw)}&=
\dfrac{K\left(\cos{\tau_1\omega-\tau\omega\sin{\tau_1\omega}}\right)}
      {1+\tau^2\omega^2}\\
\Im{H(\jw)}&=
\dfrac{-K\left(\tau\omega\cos{\tau_1\omega+\sin{\tau_1\omega}}\right)}
{1+\tau^2\omega^2}\\
\end{align*}

\begin{figure}[!h]
\begin{center}
\tikzsetnextfilename{nyquist_effet_retard_1-chap3_ext}
\begin{tikzpicture}
\begin{axis}
    [
    axis line style = thick,
    xlabel=$\Re{H(\jw)}$,
    ylabel=$\Im{H(\jw)}$,
    ymin=-1,
    ymax=1,
    xmin=-0.8,
    xmax=+1.3,
    grid=both,
    ]
    \def\t{1.0}
    \def\tu{1.1}
    \addplot[blue,thick,domain=0:1,samples=200] 
    ({(cos(deg(\tu*x))-\t*x*sin(deg(\tu*x)))/(1+\t^2*x^2)},
     {-(\t*x*cos(deg(\tu*x))+sin(deg(\tu*x)))/(1+\t^2*x^2)});
    \addplot[blue,thick,domain=1:10,samples=200] 
    ({(cos(deg(\tu*x))-\t*x*sin(deg(\tu*x)))/(1+\t^2*x^2)},
     {-(\t*x*cos(deg(\tu*x))+sin(deg(\tu*x)))/(1+\t^2*x^2)});
    \addplot[blue,thick,domain=10:20,samples=200]
    ({(cos(deg(\tu*x))-\t*x*sin(deg(\tu*x)))/(1+\t^2*x^2)},
     {-(\t*x*cos(deg(\tu*x))+sin(deg(\tu*x)))/(1+\t^2*x^2)});
    \addplot[blue,thick,domain=20:50,samples=200]
    ({(cos(deg(\tu*x))-\t*x*sin(deg(\tu*x)))/(1+\t^2*x^2)},
     {-(\t*x*cos(deg(\tu*x))+sin(deg(\tu*x)))/(1+\t^2*x^2)});
    \addplot[blue,thick,domain=50:100,samples=200]
    ({(cos(deg(\tu*x))-\t*x*sin(deg(\tu*x)))/(1+\t^2*x^2)},
     {-(\t*x*cos(deg(\tu*x))+sin(deg(\tu*x)))/(1+\t^2*x^2)});
    \node[above]       at (axis cs:  1, 0)   {$\omega=0$};
    \addplot[blue, mark = *] coordinates {(1, 0)} {};
    \node[below right] (A) at (axis cs:  0.5, 0.5) 
    {$\omega\rightarrow+\infty$};
    \node (B) at (axis cs:  0, 0)  {};
    \draw[->] (A) -- (B);
\end{axis}
\end{tikzpicture}
\end{center}
\caption{Effet d'un retard sur le diagramme de Nyquist d'un système 
         du premier ordre. Avec $K=1$, $\tau=1$ et $\tau_1=2$. Le lieu 
         de Nyquist est représenté par une spirale.
         \label{fig-nyquist_effet_retard_1}}
\end{figure}

\begin{figure}[!h]
\begin{center}
\tikzsetnextfilename{nyquist_effet_retard_2-chap3_ext}
\begin{tikzpicture}
\pgfplotscreateplotcyclelist{mycolorlist}{%
            blue\\%
            blue\\%
            red\\%
            red\\%
            black\\%
            black\\%
            brown!60!black\\%
            brown!60!black\\%
            green!60!black\\%
            green!60!black\\%
            red!60!yellow\\
            red!60!yellow\\
        }
\begin{axis}
    [
    axis line style = thick,
    xlabel=$\Re{H(\jw)}$,
    ylabel=$\Im{H(\jw)}$,
    ymin=-1,
    ymax=1,
    xmin=-0.8,
    xmax=+1.3,
    grid=both,
    cycle list name=mycolorlist,
    ]
    \def\t{1.0}
    \def\tu{0.5}

    \foreach \tu in {0.5,1.0,2.0}
    {
    \addplot+[thick,domain=0:1,samples=200]
    ({(cos(deg(\tu*x))-\t*x*sin(deg(\tu*x)))/(1+\t^2*x^2)},
     {-(\t*x*cos(deg(\tu*x))+sin(deg(\tu*x)))/(1+\t^2*x^2)});
    \addplot+[thick,domain=1:{10/\tu^2},samples=200] 
    ({(cos(deg(\tu*x))-\t*x*sin(deg(\tu*x)))/(1+\t^2*x^2)},
     {-(\t*x*cos(deg(\tu*x))+sin(deg(\tu*x)))/(1+\t^2*x^2)});
}
\end{axis}
\end{tikzpicture}
\end{center}
    \caption{Effet d'un retard sur le diagramme de Nyquist d'un système du 
             premier ordre pour différentes valeurs de retard. (bleu) 
             $\tau_1=0.5$, (rouge) $\tau_1=1.0$ et (noir) $\tau_1=2.0$.
             Avec $K=1$, $\tau=1$ et $\tau_1=2$. Le lieu de Nyquist est 
             représenté par une spirale. Par souci de clarté, nous n'avons 
             ici représenté que l'intervalle $\omega\in[0,\dfrac{10}{\tau_1}]$ 
             \label{fig-nyquist_effet_retard_2}}
\end{figure}


%%%%%%%%%%%%%%%%%%%%%%%%%%%%%%%%%%%%%%%%%%%%%%%%%%%%%%%%%%%%%%%%%%%%%%%%%%%%%%%%
%%%%%%%%%%%%%%%%%%%%%%%%%%%%%%%%%%%%%%%%%%%%%%%%%%%%%%%%%%%%%%%%%%%%%%%%%%%%%%%%
\subsection{Diagrammes de Black: méthodologie générale}
%%%%%%%%%%%%%%%%%%%%%%%%%%%%%%%%%%%%%%%%%%%%%%%%%%%%%%%%%%%%%%%%%%%%%%%%%%%%%%%%
%%%%%%%%%%%%%%%%%%%%%%%%%%%%%%%%%%%%%%%%%%%%%%%%%%%%%%%%%%%%%%%%%%%%%%%%%%%%%%%%


%%%%%%%%%%%%%%%%%%%%%%%%%%%%%%%%%%%%%%%%%%%%%%%%%%%%%%%%%%%%%%%%%%%%%%%%%%%%%%%%
%%%%%%%%%%%%%%%%%%%%%%%%%%%%%%%%%%%%%%%%%%%%%%%%%%%%%%%%%%%%%%%%%%%%%%%%%%%%%%%%
%%%%%%%%%%%%%%%%%%%%%%%%%%%%%%%%%%%%%%%%%%%%%%%%%%%%%%%%%%%%%%%%%%%%%%%%%%%%%%%%
\section{Etude du transitoire de la réponse harmonique}
%%%%%%%%%%%%%%%%%%%%%%%%%%%%%%%%%%%%%%%%%%%%%%%%%%%%%%%%%%%%%%%%%%%%%%%%%%%%%%%%
%%%%%%%%%%%%%%%%%%%%%%%%%%%%%%%%%%%%%%%%%%%%%%%%%%%%%%%%%%%%%%%%%%%%%%%%%%%%%%%%
%%%%%%%%%%%%%%%%%%%%%%%%%%%%%%%%%%%%%%%%%%%%%%%%%%%%%%%%%%%%%%%%%%%%%%%%%%%%%%%%

\acplhp
%c.f pdf/[AnaHSLCI][CO]Analyse_harmonique_des_SLCI.pdf
%%%%%%%%%%%%%%%%%%%%%%%%%%%%%%%%%%%%%%%%%%%%%%%%%%%%%%%%%%%%%%%%%%%%%%%%%%%%%%%%
%%%%%%%%%%%%%%%%%%%%%%%%%%%%%%%%%%%%%%%%%%%%%%%%%%%%%%%%%%%%%%%%%%%%%%%%%%%%%%%%
\subsection{Exemple d'un système du premier ordre}
%%%%%%%%%%%%%%%%%%%%%%%%%%%%%%%%%%%%%%%%%%%%%%%%%%%%%%%%%%%%%%%%%%%%%%%%%%%%%%%%
%%%%%%%%%%%%%%%%%%%%%%%%%%%%%%%%%%%%%%%%%%%%%%%%%%%%%%%%%%%%%%%%%%%%%%%%%%%%%%%%
\acplhp
%%%%%%%%%%%%%%%%%%%%%%%%%%%%%%%%%%%%%%%%%%%%%%%%%%%%%%%%%%%%%%%%%%%%%%%%%%%%%%%%
%%%%%%%%%%%%%%%%%%%%%%%%%%%%%%%%%%%%%%%%%%%%%%%%%%%%%%%%%%%%%%%%%%%%%%%%%%%%%%%%
\subsection{Exemple d'un système du second ordre}
%%%%%%%%%%%%%%%%%%%%%%%%%%%%%%%%%%%%%%%%%%%%%%%%%%%%%%%%%%%%%%%%%%%%%%%%%%%%%%%%
%%%%%%%%%%%%%%%%%%%%%%%%%%%%%%%%%%%%%%%%%%%%%%%%%%%%%%%%%%%%%%%%%%%%%%%%%%%%%%%%
\acplhp

%\newpage
%%%%%%%%%%%%%%%%%%%%%%%%%%%%%%%%%%%%%%%%%%%%%%%%%%%%%%%%%%%%%%%%%%%%%%%%%%%%%%%%
%%%%%%%%%%%%%%%%%%%%%%%%%%%%%%%%%%%%%%%%%%%%%%%%%%%%%%%%%%%%%%%%%%%%%%%%%%%%%%%%
%%%%%%%%%%%%%%%%%%%%%%%%%%%%%%%%%%%%%%%%%%%%%%%%%%%%%%%%%%%%%%%%%%%%%%%%%%%%%%%%
%\section*{Exercices du chapitre}
%%%%%%%%%%%%%%%%%%%%%%%%%%%%%%%%%%%%%%%%%%%%%%%%%%%%%%%%%%%%%%%%%%%%%%%%%%%%%%%%
%%%%%%%%%%%%%%%%%%%%%%%%%%%%%%%%%%%%%%%%%%%%%%%%%%%%%%%%%%%%%%%%%%%%%%%%%%%%%%%%
%%%%%%%%%%%%%%%%%%%%%%%%%%%%%%%%%%%%%%%%%%%%%%%%%%%%%%%%%%%%%%%%%%%%%%%%%%%%%%%%
%\exercice{}
%\newpage
%%%%%%%%%%%%%%%%%%%%%%%%%%%%%%%%%%%%%%%%%%%%%%%%%%%%%%%%%%%%%%%%%%%%%%%%%%%%%%%%
%%%%%%%%%%%%%%%%%%%%%%%%%%%%%%%%%%%%%%%%%%%%%%%%%%%%%%%%%%%%%%%%%%%%%%%%%%%%%%%%
%%%%%%%%%%%%%%%%%%%%%%%%%%%%%%%%%%%%%%%%%%%%%%%%%%%%%%%%%%%%%%%%%%%%%%%%%%%%%%%%
%\section*{Corrigé des exercices}
%%%%%%%%%%%%%%%%%%%%%%%%%%%%%%%%%%%%%%%%%%%%%%%%%%%%%%%%%%%%%%%%%%%%%%%%%%%%%%%%
%%%%%%%%%%%%%%%%%%%%%%%%%%%%%%%%%%%%%%%%%%%%%%%%%%%%%%%%%%%%%%%%%%%%%%%%%%%%%%%%
%%%%%%%%%%%%%%%%%%%%%%%%%%%%%%%%%%%%%%%%%%%%%%%%%%%%%%%%%%%%%%%%%%%%%%%%%%%%%%%%

%%%%%%%%%%%%%%%%%%%%%%%%%%%%%%%%%%%%%%%%%%%%%%%%%%%%%%%%%%%%%%%%%%%%%%%%%%%%%%%%
%%%%%%%%%%%%%%%%%%%%%%%%%%%%%%%%%%%%%%%%%%%%%%%%%%%%%%%%%%%%%%%%%%%%%%%%%%%%%%%%
%%%%%%%%%%%%%%%%%%%%%%%%%%%%%%%%%%%%%%%%%%%%%%%%%%%%%%%%%%%%%%%%%%%%%%%%%%%%%%%%
%%%%%%%%%%%%%%%%%%%%%%%%%%%%%%%%%%%%%%%%%%%%%%%%%%%%%%%%%%%%%%%%%%%%%%%%%%%%%%%%
%chap_repfreq.tex
