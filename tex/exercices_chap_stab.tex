%%%%%%%%%%%%%%%%%%%%%%%%%%%%%%%%%%%%%%%%%%%%%%%%%%%%%%%%%%%%%%%%%%%%%%%%%%%%%%%%
%%%%%%%%%%%%%%%%%%%%%%%%%%%%%%%%%%%%%%%%%%%%%%%%%%%%%%%%%%%%%%%%%%%%%%%%%%%%%%%%
\exercice{Stabilité d'un système du second ordre~\facile}
%%%%%%%%%%%%%%%%%%%%%%%%%%%%%%%%%%%%%%%%%%%%%%%%%%%%%%%%%%%%%%%%%%%%%%%%%%%%%%%%
%%%%%%%%%%%%%%%%%%%%%%%%%%%%%%%%%%%%%%%%%%%%%%%%%%%%%%%%%%%%%%%%%%%%%%%%%%%%%%%%
Un système est décrit par le schéma bloc suivant :
%-------------------------------------------------------------------------------
\begin{center}
    \tikzsetnextfilename{sb_ex1-exercices-chap_stab-ext}
    \begin{tikzpicture}
    \sbStyleBloc{very thick}
    \sbEntree{E}
    \sbComp{a}{E}
    \sbRelier[$E(p)$]{E}{a}
    \sbBloc{b}{$C(p)$}{a}
    \sbRelier{a}{b}
    \sbBlocL{c}{$H(p)$}{b}
    \sbSortie[4]{S}{c}
    \sbRelier{c}{S}
    \sbRenvoi{c-S}{a}{}
    \sbNomLien[0.8]{S}{$S(p)$}
\end{tikzpicture}

\end{center}
%-------------------------------------------------------------------------------
avec :
\[
H(p)=\dfrac{K}{\frac{p^2}{\omega^2_0}+\frac{2\xi}{\omega_0}p+1}
\]
%%%%%%%%%%%%%%%%%%%%%%%%%%%%%%%%%%%%%%%%%%%%%%%%%%%%%%%%%%%%%%%%%%%%%%%%%%%%%%%%
\question{\`A partir du critère de Routh, déterminer les conditions de 
stabilité sur $K$, $\xi$ et  $\omega_0$ pour deux types de correcteurs : }
%%%%%%%%%%%%%%%%%%%%%%%%%%%%%%%%%%%%%%%%%%%%%%%%%%%%%%%%%%%%%%%%%%%%%%%%%%%%%%%%
%-------------------------------------------------------------------------------
\begin{itemize}
    \item P (proportionnel) : $C(p)=1$
    \item I (intégrateur) : $C(p)=\dfrac{1}{p}$
\end{itemize}
%-------------------------------------------------------------------------------
%%%%%%%%%%%%%%%%%%%%%%%%%%%%%%%%%%%%%%%%%%%%%%%%%%%%%%%%%%%%%%%%%%%%%%%%%%%%%%%%
%%%%%%%%%%%%%%%%%%%%%%%%%%%%%%%%%%%%%%%%%%%%%%%%%%%%%%%%%%%%%%%%%%%%%%%%%%%%%%%%
\exercice{Application du critère de Routh~\difficile}
%%%%%%%%%%%%%%%%%%%%%%%%%%%%%%%%%%%%%%%%%%%%%%%%%%%%%%%%%%%%%%%%%%%%%%%%%%%%%%%%
%%%%%%%%%%%%%%%%%%%%%%%%%%%%%%%%%%%%%%%%%%%%%%%%%%%%%%%%%%%%%%%%%%%%%%%%%%%%%%%%
%%%%%%%%%%%%%%%%%%%%%%%%%%%%%%%%%%%%%%%%%%%%%%%%%%%%%%%%%%%%%%%%%%%%%%%%%%%%%%%%
\question{Déterminer la stabilité des systèmes définis par les équations 
caractéristiques suivantes par le critère de Routh :}
%%%%%%%%%%%%%%%%%%%%%%%%%%%%%%%%%%%%%%%%%%%%%%%%%%%%%%%%%%%%%%%%%%%%%%%%%%%%%%%%
%-------------------------------------------------------------------------------
\begin{itemize}
    \item[a)] $D(p)=p^5+2p^4+3p^3+4p^2+3p+1$
    \item[b)] $D(p)=p^4+p^3+3p^2+p+1$
    \item[c)] $D(p)=p^5+2p^4+3p^3+6p^2+5p+3$
    \item[d)] $D(p)=p^5+7p^4+6p^3+42p^2+8p+56$
\end{itemize}
%-------------------------------------------------------------------------------
\clearpage
%%%%%%%%%%%%%%%%%%%%%%%%%%%%%%%%%%%%%%%%%%%%%%%%%%%%%%%%%%%%%%%%%%%%%%%%%%%%%%%%
%%%%%%%%%%%%%%%%%%%%%%%%%%%%%%%%%%%%%%%%%%%%%%%%%%%%%%%%%%%%%%%%%%%%%%%%%%%%%%%%
\exercice{Critère de Nyquist~\difficile}
%%%%%%%%%%%%%%%%%%%%%%%%%%%%%%%%%%%%%%%%%%%%%%%%%%%%%%%%%%%%%%%%%%%%%%%%%%%%%%%%
%%%%%%%%%%%%%%%%%%%%%%%%%%%%%%%%%%%%%%%%%%%%%%%%%%%%%%%%%%%%%%%%%%%%%%%%%%%%%%%%
% c.f matlab script matlab/chap_stab/ex3.m 
Un système en boucle ouverte présente deux pôles instables (à parties réelles 
positives). On cherche à montrer que ce système peut être stable en 
modifiant la valeur du gain $K$ en boucle ouverte. La figure ci-dessous 
représente la réponse harmonique du système en boucle ouverte que l'on souhaite
asservir à l'aide d'une boucle de contre-réaction pour $K=1$.
%-------------------------------------------------------------------------------
\begin{center}
\includegraphics[width=0.8\textwidth]
                {exercice_nyquist_chap_stab_ex3_enonce.eps}
\end{center}
%-------------------------------------------------------------------------------
%%%%%%%%%%%%%%%%%%%%%%%%%%%%%%%%%%%%%%%%%%%%%%%%%%%%%%%%%%%%%%%%%%%%%%%%%%%%%%%%
\question{D'après le critère de Nyquist, déterminer les zones du plan complexe 
qui permettraient au système d'être stable en boucle fermée.}
%%%%%%%%%%%%%%%%%%%%%%%%%%%%%%%%%%%%%%%%%%%%%%%%%%%%%%%%%%%%%%%%%%%%%%%%%%%%%%%%
%%%%%%%%%%%%%%%%%%%%%%%%%%%%%%%%%%%%%%%%%%%%%%%%%%%%%%%%%%%%%%%%%%%%%%%%%%%%%%%%
\question{Déterminer alors la condition sur $K$ pour ce système soit stable en 
boucle fermée.} 
%%%%%%%%%%%%%%%%%%%%%%%%%%%%%%%%%%%%%%%%%%%%%%%%%%%%%%%%%%%%%%%%%%%%%%%%%%%%%%%%
