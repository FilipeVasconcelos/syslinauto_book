%%%%%%%%%%%%%%%%%%%%%%%%%%%%%%%%%%%%%%%%%%%%%%%%%%%%%%%%%%%%%%%%%%%%%%%%%%%%%%%%
%%%%%%%%%%%%%%%%%%%%%%%%%%%%%%%%%%%%%%%%%%%%%%%%%%%%%%%%%%%%%%%%%%%%%%%%%%%%%%%%
\exercice{Régulation de la température d'une enceinte}
%%%%%%%%%%%%%%%%%%%%%%%%%%%%%%%%%%%%%%%%%%%%%%%%%%%%%%%%%%%%%%%%%%%%%%%%%%%%%%%%
%%%%%%%%%%%%%%%%%%%%%%%%%%%%%%%%%%%%%%%%%%%%%%%%%%%%%%%%%%%%%%%%%%%%%%%%%%%%%%%%
On considère un système de chauffage qui doit faire passer la température 
d'une enceinte de $\theta_a$  (température ambiante) à $\theta_e$ (température 
de l'enceinte). 
La puissance totale $P_1$ fournie à la resistance de chauffage 
est proportionnelle de gain $A=\SI{200}{\watt\per\volt}$ à la tension de 
commande $u_c$. La puissance $P_2$ permettant de chauffer l'enceinte est 
telle que 
\[
    P_2=MC\devi{\theta_e-\theta_a}{}
\]
où $M$ est la masse du matériau de l'enceinte et $C$ sa capacité calorifique.
La puissance perdue $P_3$ par l'enceinte est telle que :
\[
    P_3=\dfrac{\theta_e-\theta_a}{R}
\]
où $R$ est la résistance thermique de l'enceinte.
Le bilan de puissance s'exprime donc par la relation $P_2=P_1-P_3$.

%%%%%%%%%%%%%%%%%%%%%%%%%%%%%%%%%%%%%%%%%%%%%%%%%%%%%%%%%%%%%%%%%%%%%%%%%%%%%%%%
\question{}
%%%%%%%%%%%%%%%%%%%%%%%%%%%%%%%%%%%%%%%%%%%%%%%%%%%%%%%%%%%%%%%%%%%%%%%%%%%%%%%%
En posant $P_1=Au_c$, $\Delta\theta(t)=\theta_e(t)-\theta_a(t)$ et en utilisant
la relation $P_2=P_1-P_3$, on obtient 
\[
    MC\devi{\Delta\theta(t)}{}+\dfrac{1}{R}\Delta\theta(t)=Au_c(t)
\]
sous forme canonique on obtient :
\[
    MCR\devi{\Delta\theta(t)}{}+\Delta\theta(t)=ARu_c(t)
\]

%%%%%%%%%%%%%%%%%%%%%%%%%%%%%%%%%%%%%%%%%%%%%%%%%%%%%%%%%%%%%%%%%%%%%%%%%%%%%%%%
\question{}
%%%%%%%%%%%%%%%%%%%%%%%%%%%%%%%%%%%%%%%%%%%%%%%%%%%%%%%%%%%%%%%%%%%%%%%%%%%%%%%%
La transformée de Laplace de l'équation différentielle précédente, nous donne:
\[
    (MCRp+1)\Delta\theta(p)=AR U_c(p)
\]
On détermine la fonction de transfert en boucle ouverte $H(p)$ telle que :
\[
    H(p)=\dfrac{AR}{MCRp+1} U_c(p)
\]
On identifie donc les paramètres du premier ordre :
\[
    K=AR
\]
et
\[
    \tau=MCR
\]
%%%%%%%%%%%%%%%%%%%%%%%%%%%%%%%%%%%%%%%%%%%%%%%%%%%%%%%%%%%%%%%%%%%%%%%%%%%%%%%%
\question{}
%%%%%%%%%%%%%%%%%%%%%%%%%%%%%%%%%%%%%%%%%%%%%%%%%%%%%%%%%%%%%%%%%%%%%%%%%%%%%%%%
Le temps de réponse à 5\% ($\sim 3\tau$) est de \SI{180}{\second}. 
On a donc $\tau=\SI{60}{\second}$. 
La valeur finale de la réponse indicielle d'un système du premier ordre est 
$KE_0$ où $K$ est le gain statique et $E_0$ la valeur de l'amplitude de 
l'échelon. Ici cette valeur finale est de \SI{60}{\celsius} (c'est à dire 
$\SI{80}{\celsius} -\SI{20}{\celsius}$) pour un $E_0=\SI{5}{\volt}$
Donc $K=\SI{12}{\celsius\per\volt}$

On écrit $H(p)$ sous la forme numérique :
\[
    H(p)=\dfrac{12}{1+60p}
\]
%%%%%%%%%%%%%%%%%%%%%%%%%%%%%%%%%%%%%%%%%%%%%%%%%%%%%%%%%%%%%%%%%%%%%%%%%%%%%%%%
\question{}
%%%%%%%%%%%%%%%%%%%%%%%%%%%%%%%%%%%%%%%%%%%%%%%%%%%%%%%%%%%%%%%%%%%%%%%%%%%%%%%%
\[
    G(p)=\dfrac{H(p)}{1+B(p)H(p)}
\]
\[
    G(p)=\dfrac{K}{\tau p+1+BK}=\dfrac{\dfrac{K}{1+BK}}{\dfrac{\tau}{1+BK}p+1}
\]
On identifie les paramètres du premier ordre de la fonction de transfert en 
boucle fermée :
\[
    K_{BF}=\dfrac{K}{1+BK}=\SI{7.5}{\celsius\per\volt}
\]
et
\[
    \tau_{BF}=\dfrac{\tau}{1+BK}=\SI{37.5}{\second}
\]
%%%%%%%%%%%%%%%%%%%%%%%%%%%%%%%%%%%%%%%%%%%%%%%%%%%%%%%%%%%%%%%%%%%%%%%%%%%%%%%%
\question{}
%%%%%%%%%%%%%%%%%%%%%%%%%%%%%%%%%%%%%%%%%%%%%%%%%%%%%%%%%%%%%%%%%%%%%%%%%%%%%%%%
Pour atteindre une différence de température de \SI{60}{\celsius}, il faut une 
tension de commande de $u_c=\dfrac{60}{7.5}=\SI{8}{\volt}$
Le temps de réponse à 5\% est d'environ $3\tau_{BF}=\SI{112.5}{\second}$.
%%%%%%%%%%%%%%%%%%%%%%%%%%%%%%%%%%%%%%%%%%%%%%%%%%%%%%%%%%%%%%%%%%%%%%%%%%%%%%%%
\question{}
%%%%%%%%%%%%%%%%%%%%%%%%%%%%%%%%%%%%%%%%%%%%%%%%%%%%%%%%%%%%%%%%%%%%%%%%%%%%%%%%
\[
    u_c(\infty)-u_r(\infty)=\SI{8}{\volt}-0.05\SI{60}{\celsius}=\SI{5}{\volt}
\]
