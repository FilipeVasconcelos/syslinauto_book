\newcommand{\bdM}{\boldsymbol{M}}
\newcommand{\bdI}{\boldsymbol{I}}
\newcommand{\bdA}{\boldsymbol{A}}
\newcommand{\bdB}{\boldsymbol{B}}
\newcommand{\bdC}{\boldsymbol{C}}
%%%%%%%%%%%%%%%%%%%%%%%%%%%%%%%%%%%%%%%%%%%%%%%%%%%%%%%%%%%%%%%%%%%%%%%%%%%%%%%%
\exercice{Détermination de la fonction de transfert à partir de la 
représentation d'état~\moyen}
%%%%%%%%%%%%%%%%%%%%%%%%%%%%%%%%%%%%%%%%%%%%%%%%%%%%%%%%%%%%%%%%%%%%%%%%%%%%%%%%
%%%%%%%%%%%%%%%%%%%%%%%%%%%%%%%%%%%%%%%%%%%%%%%%%%%%%%%%%%%%%%%%%%%%%%%%%%%%%%%%
\question{\textbf{Déterminer la fonction de transfert associée à la représentation
d'état suivante:}
\[
    \boldsymbol{A}=
    \begin{pmatrix} 
        3 &  1\\
        0 & -2 
    \end{pmatrix}\quad\quad
    \boldsymbol{B}=
    \begin{pmatrix} 
        0\\
        1
    \end{pmatrix}\quad\quad
    \boldsymbol{C}=
    \begin{pmatrix} 
        1 & 1
    \end{pmatrix}\quad\quad
    D=0
\]
}
%%%%%%%%%%%%%%%%%%%%%%%%%%%%%%%%%%%%%%%%%%%%%%%%%%%%%%%%%%%%%%%%%%%%%%%%%%%%%%%%
Déterminons d'abord $\bdM=(p\bdI-\bdA)$, $\Delta=\det{\bdM}$ et $\bdM^{-1}$
\[
    \bdM=
    \begin{pmatrix} 
        p-3 & -1\\
        0 & p+2 
    \end{pmatrix}
\]
\[
    \Delta=
    \begin{vmatrix} 
        p-3 & -1\\
        0 & p+2 
    \end{vmatrix}=(p-3)(p+2)=p^2-p-6
\]
\[
    \bdM^{-1}=
    \dfrac{1}{\Delta}
    \begin{pmatrix} 
        p+2 &  1\\
        0 & p-3 
    \end{pmatrix}
\]
Nous pouvons maintenant calculer la fonction de transfert $H(p)$ associé à
cette représentation d'état.
\begin{align*}
    H(p)&=\bdC\bdM^{-1}\bdB\\
    H(p)&=\dfrac{1}{\Delta}\bdC\begin{pmatrix}1\\p-3\end{pmatrix}\\
    H(p)&=\dfrac{1}{\Delta}(p-2)\\
    H(p)&=\dfrac{p-2}{(p-3)(p+2)}
\end{align*}
%%%%%%%%%%%%%%%%%%%%%%%%%%%%%%%%%%%%%%%%%%%%%%%%%%%%%%%%%%%%%%%%%%%%%%%%%%%%%%%%
\question{\textbf{Déterminer la fonction de transfert associée à la représentation
d'état suivante:}
\[
    \boldsymbol{A}=
    \begin{pmatrix} 
        -2 &  0\\
        -1 &  -1 
    \end{pmatrix}\quad\quad
    \boldsymbol{B}=
    \begin{pmatrix} 
        1\\
        2
    \end{pmatrix}\quad\quad
    \boldsymbol{C}=
    \begin{pmatrix} 
        1 & 0
    \end{pmatrix}\quad\quad
    D=2
\]
}
%%%%%%%%%%%%%%%%%%%%%%%%%%%%%%%%%%%%%%%%%%%%%%%%%%%%%%%%%%%%%%%%%%%%%%%%%%%%%%%%
Déterminons d'abord $\bdM=(p\bdI-\bdA)$, $\Delta=\det{\bdM}$ et $\bdM^{-1}$
\[
    \bdM=
    \begin{pmatrix} 
        p+2 & 0\\
        1 & p+1 
    \end{pmatrix}
\]
\[
    \Delta=
    \begin{vmatrix} 
        p+2 & 0\\
        1 & p+1 
    \end{vmatrix}=(p+2)(p+1)=p^2+3p+2
\]
\[
    \bdM^{-1}=
    \dfrac{1}{\Delta}
    \begin{pmatrix} 
        p+1 & 0\\
        -1 & p+2 
    \end{pmatrix}
\]
Nous pouvons maintenant calculer la fonction de transfert $H(p)$ associé à
cette représentation d'état.
\begin{align*}
    H(p)&=\bdC\bdM^{-1}\bdB+D\\
    H(p)&=\dfrac{1}{\Delta}\bdC\begin{pmatrix}p+1\\2p+3\end{pmatrix}+D\\
    H(p)&=\dfrac{1}{\Delta}(p+1)+2\\
    H(p)&=\dfrac{(p+1)+2(p+1)(p+2)}{(p+2)(p+1)}\\
    H(p)&=\dfrac{2p+5}{p+2}
\end{align*}

