\chapter{Transformation de Laplace\label{annexe-lap}}
\chaptermark{Transformation de Laplace}


\section{Définitions} 

Soit f une fonction de la variable réelle t définie sur $\mathbb{R}$ 
et supossée  nulle pour $t<0$, on appelle transformée de Laplace de $f$, la fonction $F$ définie par :
$$
F(p) = \int_0^\infty e^{-pt} f(t) \mathrm{d}t
$$
avec $p\in\mathbb{C}$. 

En automatique, on n'utilise que la transformée de Laplace restreinte qui 
ne s'applique qu'aux fonctions causales.
Pour transformer une fonction quelconque en fonction causale, 
on la combine avec la fonction de Heaviside $u(t)$ qui est telle que :

\begin{center}
\begin{tikzpicture}
    \draw[-{Stealth[scale=1.3,angle'=45]},] (-2,0) -- (4.,0) ;
    \draw[-{Stealth[scale=1.3,angle'=45]}] (0,-1) -- (0,2.5)  ;


\draw[line width =2pt,domain=-1.5:0,smooth,variable=\x,blue] plot ({\x},{0});
\draw[line width =2pt,domain=0:3.5,smooth,variable=\x,blue] plot ({\x},{1});
\node[left,blue] at (2.5,1.4)  (ft)  {\large $u(t)$};

\node[below] at (3.5,0)  (t)     {\large $t$};

    \node[below] at (-5.0,1.5) {$u(t)=\begin{cases}0 \qquad \forall t<0\\1 \qquad \forall t\geq 0\end{cases} $};
\end{tikzpicture}
\end{center}

On note $F(p)=\laplace{f(t)}$, la transformée de Laplace de $f(t)$ et on dit de $F(p)$ qu'elle est l'image de $f(t)$
dans le domaine de Laplace
\footnote{Plusieurs termes sont utilisés dans la littérature. On parle 
de domaine complexe, domaine fréquentielle ou de domaine symbolique. On choisit dans ce document 
de ne parler que du domaine de Laplace} et on notera $\laplacei{F(p)}$ la transformée de Laplace inverse.
%\newpage
\section{Propriétés} 
\begin{itemize}
%\item \emph{unicité} : a une fonction $f(t)$ il correspond une unique transformée de Laplace 	
%\item \emph{addition} : 
%$$ 
%\mathcal{L}[f(t)+g(t)] = F(p) + G(p)
%$$ 
\item \emph{linéarité} :
$$ 
\laplace{af(t)+bg(t)} = aF(p) + bG(p)
$$ 
\item \emph{dilatation du temps} : 
$$
\laplace{f(kt)}=\dfrac{1}{k}F\left(\dfrac{p}{k}\right)
$$
\item \emph{produit de convolution} : 
$$
\laplace{f(t)*g(t)} = F(p)G(p)
$$
\item \emph{dérivation} : 
\begin{align*}
\laplace{\devi{f(t)}{}}&=pF(p) - f(0^+) \\
\laplace{\devi{f(t)}{2}}&=p^2F(p) - pf(0^+) - f'(0^+)\\
\laplace{\devi{f(t)}{n}}&=p^nF(p) 
\end{align*}
si toutes les conditions initiales sont nulles.
\item \emph{intégration} :
$$
\laplace{\int_0^t f(u)du} = \dfrac{F(p)}{p} + \dfrac{g(0^+)}{p}
$$
avec :
$g(t)=\int_0^t f(u)du$
\item \emph{théorème du retard en (t)} :
$$
\laplace{f(t-\tau)}=e^{-\tau p}F(p)
$$
\item \emph{théorème du retard en (p)} :
$$
\laplacei{F(p+a)}=e^{-at}f(t)
$$
\item \emph{théorème de la valeur initiale} :
$$
\lim\limits_{t \to 0} f(t)=\lim\limits_{p \to \infty}\, p F(p)
$$
\item \emph{théorème de la valeur finale} :
$$
\lim\limits_{t \to \infty} f(t)=\lim\limits_{p \to 0}\,p F(p)
$$
\item \emph{transformée de Laplace d'une fonction périodique et $f(t)$ de période $T$} :
$$
        F(p) = \dfrac{F_0(p)}{1-e^{-Tp}}
$$
où $F_0(p)$ est la transformée de Laplace du motif $f_0(t)$ égal à 
$f(t)$ sur le segment $[0,T]$ et nul partout ailleurs.
\end{itemize}
\section{Tableau des transformées de Laplace}
\begin{table}[H]
%\resizebox{0.5\textwidth}{!}{
\begin{tabular}{|M{0.5cm}|M{3.9cm}|M{8.9cm}|N}
\hline
& $F(p)$ & $f(t)=\laplacei{F(p)}$ &\\[20pt]
\hline
1 & 1 & $\delta(t)$ &\\[20pt]
\hline
2 & $e^{-\tau p}$ & $\delta(t-\tau)$ &\\[20pt]
\hline
3 & $\dfrac{1}{p}$ & 1 &\\[20pt]
\hline
4 & $\dfrac{1}{p^2}$ & t &\\[20pt]
\hline
5 & $\dfrac{1}{p^3}$ & $\dfrac{1}{2}t^2$ &\\[20pt]
\hline
6 & $\dfrac{1}{p^n}$ & $\dfrac{1}{(n-1)!}t^{n-1}$ &\\[20pt]
\hline
7 & $\dfrac{1}{p+a}$ & $e^{-at}$ &\\[20pt]
\hline
8 & $\dfrac{1}{(p+a)^2}$ & $te^{-at}$&\\[20pt]
\hline
9 & $\dfrac{1}{(p+a)^3}$ & $\dfrac{1}{2}t^2e^{-at}$&\\[20pt]
\hline
10 & $\dfrac{1}{(p+a)^n}$ & $\dfrac{1}{(n-1)!}t^{n-1}e^{-at}$&\\[20pt]
\hline
11 & $\dfrac{a}{p(p+a)}$ & $1-e^{-at}$ &\\[20pt]
\hline
12 & $\dfrac{a}{p^2(p+a)}$ & $\dfrac{1}{a}\left[at-\left(1-e^{-at}\right)\right]$ &\\[20pt]
\hline
13 & $\dfrac{p}{(p+a)^2}$ & $(1-at)e^{-at}$&\\[20pt]
\hline
14 & $\dfrac{a^2}{p(p+a)^2}$ & $1-(1+at)e^{-at}$&\\[20pt]
\hline
15 & $\dfrac{a^2(p+z)}{p(p+a)^2}$ & $z-\left(z+a(z-a)t\right)e^{-at}$&\\[20pt]
\hline
16 & $\dfrac{b-a}{(p+a)(p+b)}$ & $e^{-at}-e^{-bt}$&\\[20pt]
\hline
\end{tabular}
%}
$\,$
\clearpage
\captionof{table}{Table de transformées de Laplace d'après\cite{Ostertag}}
\end{table}
\newpage
\begin{table}[!h]
\resizebox{\textwidth}{!}{
\begin{tabular}{|M{0.5cm}|M{3.9cm}|M{8.9cm}|N}
\hline
   & $F(p)$                         & $f(t)=\laplacei{F(p)}$             &\\[20pt]
\hline
17 & $\dfrac{(b-a)p}{(p+a)(p+b)}$ & $-ae^{-at}+be^{-bt}$&\\[20pt]
\hline
18 & $\dfrac{(b-a)(p+z)}{(p+a)(p+b)}$ & $(z-a)e^{-at}-(z-b)e^{-bt}$&\\[20pt]
\hline
19 & $\dfrac{ab}{p(p+a)(p+b)}$      & $1+\dfrac{be^{-at}-ae^{-bt}}{a-b}$ &\\[20pt]
\hline
20 & $\dfrac{ab(p+z)}{p(p+a)(p+b)}$ & $z+\dfrac{b(z-a)e^{-at}-a(z-b)e^{-bt}}{a-b}$ &\\[20pt]
\hline
21 & $\dfrac{1}{(p+a)(p+b)(p+c)}$   & $\dfrac{e^{-at}}{(b-a)(c-a)}+\dfrac{e^{-bt}}{(c-b)(a-b)}+\dfrac{e^{-ct}}{(a-c)(b-c)}$ &\\[20pt]
\hline
22 & $\dfrac{p+z}{(p+a)(p+b)(p+c)}$& $\dfrac{(z-a)e^{-at}}{(b-a)(c-a)}+\dfrac{(z-b)e^{-bt}}{(c-b)(a-b)}+\dfrac{(z-c)e^{-ct}}{(a-c)(b-c)}$ &\\[20pt]
\hline
23 & $\dfrac{\omega}{p^2+\omega^2}$ & $\sin\omega t$ &\\[20pt]
\hline
24 & $\dfrac{p}{p^2+\omega^2}$ & $\cos\omega t$ &\\[20pt]
\hline
25 & $\dfrac{p+z}{p^2+\omega^2}$ & $\sqrt{\dfrac{z^2+\omega^2}{\omega^2}}\sin{(\omega t+\phi)}$ avec $\phi=\arctan{\dfrac{\omega}{z}}$ &\\[20pt]
\hline
26 & $\dfrac{\omega^2}{p(p^2+\omega)^2}$ & $1-\cos\omega t$&\\[20pt]
\hline
27 & $\dfrac{\omega^2(p+z)}{p(p^2+\omega)^2}$ & $z-\sqrt{\dfrac{z^2+\omega^2}{\omega^2}}\cos{(\omega t+\phi)}$ avec $\phi=\arctan{\dfrac{\omega}{z}}$&\\[20pt]
\hline
28 & $\dfrac{\omega}{p^2-\omega^2}$ & $\sinh{\omega t}$&\\[20pt]
\hline
29 & $\dfrac{p}{p^2-\omega^2}$ & $\cosh{\omega t}$&\\[20pt]
\hline
30 & $\dfrac{\omega}{(p+a)^2+\omega^2}$ & $e^{-at}\sin{\omega t}$ &\\[20pt]
\hline
31 & $\dfrac{p+a}{(p+a)^2+\omega^2}$ & $e^{-at}\cos{\omega t}$ &\\[20pt]
\hline
32 & $\dfrac{p+z}{(p+a)^2+\omega^2}$ & $\sqrt{\dfrac{(z-a)^2+\omega^2}{\omega^2}}e^{-at}\sin{(\omega t+\phi)}$ avec $\phi=\arctan{\dfrac{\omega}{z-a}}$&\\[20pt]
\hline
33 & $\dfrac{\omega^2}{p^2+2\xi\omega p +\omega^2}$ & $\dfrac{\omega}{\sqrt{1-\xi^2}}e^{-\xi\omega t}\sin{\omega\sqrt{1-\xi^2} t}$&\\[20pt]
\hline
34 & $\dfrac{\omega^2}{p(p^2+2\xi\omega p +\omega^2)}$ & $1-\dfrac{1}{\sqrt{1-\xi^2}}e^{-\xi\omega t}\sin{\omega\sqrt{1-\xi^2}t+\phi}$ avec $\phi=\arccos{\xi}$&\\[20pt]
\hline
\end{tabular}
}
\captionof{table}{(suite) Table de transformées de Laplace d'après~\cite{Ostertag}}
\end{table}
