%%%%%%%%%%%%%%%%%%%%%%%%%%%%%%%%%%%%%%%%%%%%%%%%%%%%%%%%%%%%%%%%%%%%%%%%%%%%%%%%
\exercice{Détermination de la fonction de transfert à partir de la 
représentation d'état~\moyen}
%%%%%%%%%%%%%%%%%%%%%%%%%%%%%%%%%%%%%%%%%%%%%%%%%%%%%%%%%%%%%%%%%%%%%%%%%%%%%%%%
%%%%%%%%%%%%%%%%%%%%%%%%%%%%%%%%%%%%%%%%%%%%%%%%%%%%%%%%%%%%%%%%%%%%%%%%%%%%%%%%
\question{Déterminer la fonction de transfert associée à la représentation
d'état suivante:
\[
    \boldsymbol{A}=
    \begin{pmatrix} 
        3 &  1\\
        0 & -2 
    \end{pmatrix}\quad\quad
    \boldsymbol{B}=
    \begin{pmatrix} 
        0\\
        1
    \end{pmatrix}\quad\quad
    \boldsymbol{C}=
    \begin{pmatrix} 
        1 & 1
    \end{pmatrix}\quad\quad
    D=0
\]
}
%%%%%%%%%%%%%%%%%%%%%%%%%%%%%%%%%%%%%%%%%%%%%%%%%%%%%%%%%%%%%%%%%%%%%%%%%%%%%%%%
%%%%%%%%%%%%%%%%%%%%%%%%%%%%%%%%%%%%%%%%%%%%%%%%%%%%%%%%%%%%%%%%%%%%%%%%%%%%%%%%
\question{Déterminer la fonction de transfert associée à la représentation
d'état suivante:
\[
    \boldsymbol{A}=
    \begin{pmatrix} 
        -2 &  0\\
        -1 &  -1 
    \end{pmatrix}\quad\quad
    \boldsymbol{B}=
    \begin{pmatrix} 
        1\\
        2
    \end{pmatrix}\quad\quad
    \boldsymbol{C}=
    \begin{pmatrix} 
        1 & 0
    \end{pmatrix}\quad\quad
    D=2
\]
}
%%%%%%%%%%%%%%%%%%%%%%%%%%%%%%%%%%%%%%%%%%%%%%%%%%%%%%%%%%%%%%%%%%%%%%%%%%%%%%%%
\exercice{Représentation d'état d'un système hydraulique~\moyen}
%%%%%%%%%%%%%%%%%%%%%%%%%%%%%%%%%%%%%%%%%%%%%%%%%%%%%%%%%%%%%%%%%%%%%%%%%%%%%%%%
%TODO : nous reprenons ici l'exercice de Bachelier p122
%on le portera également dans les autres chapitres
