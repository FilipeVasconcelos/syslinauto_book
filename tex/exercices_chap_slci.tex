%%%%%%%%%%%%%%%%%%%%%%%%%%%%%%%%%%%%%%%%%%%%%%%%%%%%%%%%%%%%%%%%%%%%%%%%%%%%%%%%
%%%%%%%%%%%%%%%%%%%%%%%%%%%%%%%%%%%%%%%%%%%%%%%%%%%%%%%%%%%%%%%%%%%%%%%%%%%%%%%%
\subsection*{Signaux}
%%%%%%%%%%%%%%%%%%%%%%%%%%%%%%%%%%%%%%%%%%%%%%%%%%%%%%%%%%%%%%%%%%%%%%%%%%%%%%%%
%%%%%%%%%%%%%%%%%%%%%%%%%%%%%%%%%%%%%%%%%%%%%%%%%%%%%%%%%%%%%%%%%%%%%%%%%%%%%%%%
%%%%%%%%%%%%%%%%%%%%%%%%%%%%%%%%%%%%%%%%%%%%%%%%%%%%%%%%%%%%%%%%%%%%%%%%%%%%%%%%
%%%%%%%%%%%%%%%%%%%%%%%%%%%%%%%%%%%%%%%%%%%%%%%%%%%%%%%%%%%%%%%%%%%%%%%%%%%%%%%%
\exercice{Décomposition d'un signal~\facile}
%%%%%%%%%%%%%%%%%%%%%%%%%%%%%%%%%%%%%%%%%%%%%%%%%%%%%%%%%%%%%%%%%%%%%%%%%%%%%%%%
%%%%%%%%%%%%%%%%%%%%%%%%%%%%%%%%%%%%%%%%%%%%%%%%%%%%%%%%%%%%%%%%%%%%%%%%%%%%%%%%
Soit le signal d'entrée $e(t)$ défini par le graphe ci-contre.

%%%%%%%%%%%%%%%%%%%%%%%%%%%%%%%%%%%%%%%%%%%%%%%%%%%%%%%%%%%%%%%%%%%%%%%%%%%%%%%%
\question{Décomposer $e(t)$ à l'aide de fonctions échelons.}
%%%%%%%%%%%%%%%%%%%%%%%%%%%%%%%%%%%%%%%%%%%%%%%%%%%%%%%%%%%%%%%%%%%%%%%%%%%%%%%%
%-------------------------------------------------------------------------------
\begin{marginfigure}
    \centering
    \tikzsetnextfilename{fonction_triangle2-chap-slci-ext}
    \begin{tikzpicture}[baseline=0]
   \begin{axis}[
        height=4cm,
        width=6cm,
        axis x line=center,
        axis y line=center,
        xmin=-1,
        xmax=5,
        ymin=-1.5,
        ymax=1.5,
        xlabel={$t$},
        ylabel={$e(t)$},
        xlabel style={below right},
        ylabel style={left},
        yticklabels={-1,1},
        ytick={-1,1},
        y tick label style={left},
        xticklabels={1,2,3,4},
        xtick={1,2,3,4},
        x tick label style={below},
        ]
        \addplot [very thick,col1,domain=-1:0, samples=50]{0.01};
        \addplot [very thick,col1,domain=0:1, samples=50]{x};
        \addplot [very thick,col1,domain=1:3, samples=50]{2-x};
        \addplot [very thick,col1,domain=3:4, samples=50]{x-4};
        \addplot [very thick,col1,domain=4:4.9, samples=50]{0.01};
        \end{axis}
\end{tikzpicture}

\end{marginfigure}
%-------------------------------------------------------------------------------
%%%%%%%%%%%%%%%%%%%%%%%%%%%%%%%%%%%%%%%%%%%%%%%%%%%%%%%%%%%%%%%%%%%%%%%%%%%%%%%%
\question{Donner la transformée de Laplace $E(p)$ de ce signal.}
%%%%%%%%%%%%%%%%%%%%%%%%%%%%%%%%%%%%%%%%%%%%%%%%%%%%%%%%%%%%%%%%%%%%%%%%%%%%%%%%

%%%%%%%%%%%%%%%%%%%%%%%%%%%%%%%%%%%%%%%%%%%%%%%%%%%%%%%%%%%%%%%%%%%%%%%%%%%%%%%%
%%%%%%%%%%%%%%%%%%%%%%%%%%%%%%%%%%%%%%%%%%%%%%%%%%%%%%%%%%%%%%%%%%%%%%%%%%%%%%%%
\exercice{Décomposition d'un signal parabolique~\moyen}
%%%%%%%%%%%%%%%%%%%%%%%%%%%%%%%%%%%%%%%%%%%%%%%%%%%%%%%%%%%%%%%%%%%%%%%%%%%%%%%%
%%%%%%%%%%%%%%%%%%%%%%%%%%%%%%%%%%%%%%%%%%%%%%%%%%%%%%%%%%%%%%%%%%%%%%%%%%%%%%%%
Soit le signal d'entrée $e(t)$ défini par le graphe ci-contre.

%%%%%%%%%%%%%%%%%%%%%%%%%%%%%%%%%%%%%%%%%%%%%%%%%%%%%%%%%%%%%%%%%%%%%%%%%%%%%%%%
\question{Décomposer $e(t)$ à l'aide de fonctions échelons.}
%%%%%%%%%%%%%%%%%%%%%%%%%%%%%%%%%%%%%%%%%%%%%%%%%%%%%%%%%%%%%%%%%%%%%%%%%%%%%%%%
%-------------------------------------------------------------------------------
\begin{marginfigure}
    \centering
    \tikzsetnextfilename{fonction_parabole-chap-slci-ext}
    \input{tikz/fonction_parabole-chap-slci.tex}
\end{marginfigure}
%-------------------------------------------------------------------------------
%%%%%%%%%%%%%%%%%%%%%%%%%%%%%%%%%%%%%%%%%%%%%%%%%%%%%%%%%%%%%%%%%%%%%%%%%%%%%%%%
\question{Donner la transformée de Laplace $E(p)$ de ce signal.}
%%%%%%%%%%%%%%%%%%%%%%%%%%%%%%%%%%%%%%%%%%%%%%%%%%%%%%%%%%%%%%%%%%%%%%%%%%%%%%%%

%%%%%%%%%%%%%%%%%%%%%%%%%%%%%%%%%%%%%%%%%%%%%%%%%%%%%%%%%%%%%%%%%%%%%%%%%%%%%%%%
%%%%%%%%%%%%%%%%%%%%%%%%%%%%%%%%%%%%%%%%%%%%%%%%%%%%%%%%%%%%%%%%%%%%%%%%%%%%%%%%
\exercice{Décompositon d'un signal et transformée de Laplace~\difficile}
%%%%%%%%%%%%%%%%%%%%%%%%%%%%%%%%%%%%%%%%%%%%%%%%%%%%%%%%%%%%%%%%%%%%%%%%%%%%%%%%
%%%%%%%%%%%%%%%%%%%%%%%%%%%%%%%%%%%%%%%%%%%%%%%%%%%%%%%%%%%%%%%%%%%%%%%%%%%%%%%%
On sollicite un système linéaire continu et invariant par deux signaux $e_1(t)$ 
et $e_2(t)$ (d'amplitude maximale et de durée unité) définits 
par les graphes ci-contre.

%%%%%%%%%%%%%%%%%%%%%%%%%%%%%%%%%%%%%%%%%%%%%%%%%%%%%%%%%%%%%%%%%%%%%%%%%%%%%%%%
\question{Décomposer $e_1(t)$ et $e_2(t)$ à l'aide de la fonction échelon 
unitaire (c.-à-d. $u(t)$).}
%%%%%%%%%%%%%%%%%%%%%%%%%%%%%%%%%%%%%%%%%%%%%%%%%%%%%%%%%%%%%%%%%%%%%%%%%%%%%%%%
%-------------------------------------------------------------------------------
\begin{marginfigure}
    \centering
    \tikzsetnextfilename{fonction_e1-chap-slci-ext}
    \input{tikz/fonction_e1-chap-slci.tex}
\end{marginfigure}
%-------------------------------------------------------------------------------
%%%%%%%%%%%%%%%%%%%%%%%%%%%%%%%%%%%%%%%%%%%%%%%%%%%%%%%%%%%%%%%%%%%%%%%%%%%%%%%%
\question{Déterminer les fonctions $E_1(p)$ et $E_2(p)$ les transformées de 
Laplace respectives de $e_1(t)$ et $e_2(t)$.}
%%%%%%%%%%%%%%%%%%%%%%%%%%%%%%%%%%%%%%%%%%%%%%%%%%%%%%%%%%%%%%%%%%%%%%%%%%%%%%%%
%-------------------------------------------------------------------------------
\begin{marginfigure}
    \centering
    \tikzsetnextfilename{fonction_e2-chap-slci-ext}
    \begin{tikzpicture}
    \begin{axis}
    [
        height=4cm,
        width=6cm,
        axis x line=center,
        axis y line=center,
        xmin=-1,
        xmax=1.5,
        ymin=-0.5,
        ymax=1.5,
        xlabel={$t$},
        ylabel={$e_2(t)$},
        xlabel style={below right},
        ylabel style={above left},
        yticklabels={0,1},
        ytick={0,1},
        y tick label style={anchor=east},
        xticklabels={1},
        xtick={1},
        x tick label style={below},
    ]
        \addplot [very thick,col1,domain=-1:0, samples=51]{0};
        \addplot [very thick,col1,domain=0:1, samples=51]{x};
        \addplot [very thick,col1,domain=1:1.45, samples=51]{0};
        \draw[very thick,col1] (axis cs:1,1) -- (axis cs:1,0);
    \end{axis}
\end{tikzpicture}

\end{marginfigure}
%-------------------------------------------------------------------------------
%%%%%%%%%%%%%%%%%%%%%%%%%%%%%%%%%%%%%%%%%%%%%%%%%%%%%%%%%%%%%%%%%%%%%%%%%%%%%%%%
%\question{Généraliser ces entrées pour une amplitude maximale $E_0$ 
%et une durée du signal $\tau$ quelconques.} 
%%%%%%%%%%%%%%%%%%%%%%%%%%%%%%%%%%%%%%%%%%%%%%%%%%%%%%%%%%%%%%%%%%%%%%%%%%%%%%%%
On sollicite maintenant le système par la fonction $e(t)$ constituée 
des signaux $e_1(t)$ et $e_2(t-1)$. 

%-------------------------------------------------------------------------------
%\begin{marginfigure}
%    \centering
%    \tikzsetnextfilename{fonction_e3-chap-slci-ext}
%    \input{tikz/fonction_e3-chap-slci.tex}
%\end{marginfigure}
%-------------------------------------------------------------------------------
%%%%%%%%%%%%%%%%%%%%%%%%%%%%%%%%%%%%%%%%%%%%%%%%%%%%%%%%%%%%%%%%%%%%%%%%%%%%%%%%
\question{Déterminer à partir des résultats précédents la transformée 
de Laplace $E(p)$.}
%%%%%%%%%%%%%%%%%%%%%%%%%%%%%%%%%%%%%%%%%%%%%%%%%%%%%%%%%%%%%%%%%%%%%%%%%%%%%%%%
\clearpage
%%%%%%%%%%%%%%%%%%%%%%%%%%%%%%%%%%%%%%%%%%%%%%%%%%%%%%%%%%%%%%%%%%%%%%%%%%%%%%%%
%%%%%%%%%%%%%%%%%%%%%%%%%%%%%%%%%%%%%%%%%%%%%%%%%%%%%%%%%%%%%%%%%%%%%%%%%%%%%%%%
\exercice{L'impulsion de Dirac approchée~\difficile}
%%%%%%%%%%%%%%%%%%%%%%%%%%%%%%%%%%%%%%%%%%%%%%%%%%%%%%%%%%%%%%%%%%%%%%%%%%%%%%%%
%%%%%%%%%%%%%%%%%%%%%%%%%%%%%%%%%%%%%%%%%%%%%%%%%%%%%%%%%%%%%%%%%%%%%%%%%%%%%%%%
L'impulsion de Dirac peut être approchée par la fonction porte 
$\delta_a(t)$ définie par le graphe ci-contre.

%%%%%%%%%%%%%%%%%%%%%%%%%%%%%%%%%%%%%%%%%%%%%%%%%%%%%%%%%%%%%%%%%%%%%%%%%%%%%%%%
\question{Décomposer $\delta_a(t)$ à l'aide de fonctions échelons.}
%%%%%%%%%%%%%%%%%%%%%%%%%%%%%%%%%%%%%%%%%%%%%%%%%%%%%%%%%%%%%%%%%%%%%%%%%%%%%%%%
%-------------------------------------------------------------------------------
\begin{marginfigure}
    \centering
    \tikzsetnextfilename{fonction_porte-chap-slci-ext}
    \begin{tikzpicture}[baseline=0]
   \begin{axis}[
        height=4cm,
        width=5cm,
        axis x line=center,
        axis y line=center,
        xmin=-1,
        xmax=5,
        ymin=-0.5,
        ymax=2.0,
        xlabel={$t$},
        ylabel={$\delta_a(t)$},
        xlabel style={below right},
        ylabel style={left},
        yticklabels={$\dfrac{1}{a}$},
        ytick={1},
        y tick label style={anchor=east},
        xticklabels={$a$},
        xtick={2},
        x tick label style={anchor=north},
        ]
        \addplot [very thick,col1,const plot] coordinates 
        {(-1,0.01) (0,0.01) (0,1) (2,1)  (2,0.01) (5,0.01) };
        \end{axis}
\end{tikzpicture}

\end{marginfigure}
%-------------------------------------------------------------------------------
%%%%%%%%%%%%%%%%%%%%%%%%%%%%%%%%%%%%%%%%%%%%%%%%%%%%%%%%%%%%%%%%%%%%%%%%%%%%%%%%
\question{Donner sa transformée de Laplace en fonction du paramètre $a$.}
%%%%%%%%%%%%%%%%%%%%%%%%%%%%%%%%%%%%%%%%%%%%%%%%%%%%%%%%%%%%%%%%%%%%%%%%%%%%%%%%
%%%%%%%%%%%%%%%%%%%%%%%%%%%%%%%%%%%%%%%%%%%%%%%%%%%%%%%%%%%%%%%%%%%%%%%%%%%%%%%%
\question{En considérant la limite $a\rightarrow0$, montrer que la transformée
          de Laplace de $\delta_a(t)$ tend bien vers 1.}
%%%%%%%%%%%%%%%%%%%%%%%%%%%%%%%%%%%%%%%%%%%%%%%%%%%%%%%%%%%%%%%%%%%%%%%%%%%%%%%%

%%%%%%%%%%%%%%%%%%%%%%%%%%%%%%%%%%%%%%%%%%%%%%%%%%%%%%%%%%%%%%%%%%%%%%%%%%%%%%%%
%%%%%%%%%%%%%%%%%%%%%%%%%%%%%%%%%%%%%%%%%%%%%%%%%%%%%%%%%%%%%%%%%%%%%%%%%%%%%%%%
\subsection*{\'Equations différentielles}
%%%%%%%%%%%%%%%%%%%%%%%%%%%%%%%%%%%%%%%%%%%%%%%%%%%%%%%%%%%%%%%%%%%%%%%%%%%%%%%%
%%%%%%%%%%%%%%%%%%%%%%%%%%%%%%%%%%%%%%%%%%%%%%%%%%%%%%%%%%%%%%%%%%%%%%%%%%%%%%%%

%%%%%%%%%%%%%%%%%%%%%%%%%%%%%%%%%%%%%%%%%%%%%%%%%%%%%%%%%%%%%%%%%%%%%%%%%%%%%%%%
%%%%%%%%%%%%%%%%%%%%%%%%%%%%%%%%%%%%%%%%%%%%%%%%%%%%%%%%%%%%%%%%%%%%%%%%%%%%%%%%
\exercice{Décharge d'un condensateur~\facile}
%%%%%%%%%%%%%%%%%%%%%%%%%%%%%%%%%%%%%%%%%%%%%%%%%%%%%%%%%%%%%%%%%%%%%%%%%%%%%%%%
%%%%%%%%%%%%%%%%%%%%%%%%%%%%%%%%%%%%%%%%%%%%%%%%%%%%%%%%%%%%%%%%%%%%%%%%%%%%%%%%
On reprend la mise en équation de la décharge d'un condensateur 
dans un circuit RC comme présentée dans ce chapitre 
(c.f \cref{para-decharge}). L'équation différentielle
de la charge $q(t)$ au borne du condensateur est donnée par :
\[
    RC\devi{q(t)}{}+q(t)=0
\]
Le condensateur est initialement chargé $q(0)=q_0$. On remarquera que le 
produit $RC$ est de dimension d'un temps. On notera cette constante $\tau=RC$.
%-------------------------------------------------------------------------------
\begin{marginfigure}
    \centering
    \tikzsetnextfilename{decharge_condensateur-chap_slci-ext}
    %               K           
%     A ------B    C---------D
%     |                      |
%     |                      |
% C1--E--C2                --J-R2 
%                          |   |
% C3--F--C4               R1-I--
%     |                      |
%     |                      |
%     G----------------------H
%
%
\begin{tikzpicture}

    \pgfmathsetmacro{\pC}{0.75}   % distance de C(1,2,3,4) en x relativement E/F
    \pgfmathsetmacro{\eC}{0.5}    % ecartement du condensateur E-F
    \pgfmathsetmacro{\lR}{1.25}   % longueur de la resistance
    \pgfmathsetmacro{\pR}{0.25}   % position de R1 / I
    \pgfmathsetmacro{\eR}{2*\pR}
    \pgfmathsetmacro{\ab}{1.5}
    \pgfmathsetmacro{\cb}{1.0}
    \pgfmathsetmacro{\gh}{2*\ab+\cb}
    \pgfmathsetmacro{\fg}{1}
    \pgfmathsetmacro{\tmpu}{0.5*\eC}
    \pgfmathsetmacro{\tmpd}{-0.5*\lR}
    \pgfmathsetmacro{\dj}{\fg+\tmpu+\tmpd}

    \coordinate (A)  at (0,0);
    \coordinate (B)  at ($(A)+(\ab,0)$);
    \coordinate (C)  at ($(B)+(\cb,0)$);
    \coordinate (D)  at ($(C)+(\ab,0)$);
    \coordinate (E)  at ($(A)+(0,-\fg)$);
    \coordinate (F)  at ($(E)+(0,-\eC)$);
    \coordinate (C1) at ($(E)+(-\pC,0)$);
    \coordinate (C2) at ($(E)+(\pC,0)$);
    \coordinate (C3) at ($(F)+(-\pC,0)$);
    \coordinate (C4) at ($(F)+(\pC,0)$);
    \coordinate (G)  at ($(F)+(0,-\fg)$);
    \coordinate (H)  at ($(G)+(\gh,0)$);
    \coordinate (I)  at ($(H)+(0,\dj)$);
    \coordinate (J)  at ($(I)+(0,\lR)$);
    \coordinate (K)  at ($(C)+(-0.75,0.5)$);
    \coordinate (R1) at ($(I)+(-\pR,0)$);
    \coordinate (R2) at ($(R1)+(\eR,\lR)$);

    %\node[circle,fill=white,draw=black,ultra thick] at (B) {};
    \draw[thick] (A)  -- (B);
    \draw[thick] (C)  -- (D);
    \draw[thick] (A)  -- (E);
    \draw[very thick,col1] (C1) -- (C2) 
    node[col1,xshift=1em,yshift=-0.5em] {$C$};
    \draw[very thick,col1] (C3) -- (C4);
    \draw[thick] (F)  -- (G);
    \draw[thick] (G)  -- (H);
    \draw[thick] (I)  -- (H);
    \draw[thick] (J)  -- (D);
    \draw[thick] (K)  -- (C);
    \draw[very thick,col4] (R1) rectangle (R2) 
    node[yshift=-1.5em,xshift=0.75em] {$R$};
    \draw[draw=black,fill=white] (B) circle (2pt);
    \draw[draw=black,fill=white] (C) circle (2pt);
\end{tikzpicture}

\end{marginfigure}
%-------------------------------------------------------------------------------

%%%%%%%%%%%%%%%%%%%%%%%%%%%%%%%%%%%%%%%%%%%%%%%%%%%%%%%%%%%%%%%%%%%%%%%%%%%%%%%%
\question{Déterminer la transformée de Laplace $Q(p)$ de la charge au borne du 
condensateur $q(t)$.}
%%%%%%%%%%%%%%%%%%%%%%%%%%%%%%%%%%%%%%%%%%%%%%%%%%%%%%%%%%%%%%%%%%%%%%%%%%%%%%%%
%%%%%%%%%%%%%%%%%%%%%%%%%%%%%%%%%%%%%%%%%%%%%%%%%%%%%%%%%%%%%%%%%%%%%%%%%%%%%%%%
\question{Déterminer alors la fonction $q(t)$ solution de l'équation 
différentielle de la décharge d'un condensateur.}
%%%%%%%%%%%%%%%%%%%%%%%%%%%%%%%%%%%%%%%%%%%%%%%%%%%%%%%%%%%%%%%%%%%%%%%%%%%%%%%%
\clearpage
%%%%%%%%%%%%%%%%%%%%%%%%%%%%%%%%%%%%%%%%%%%%%%%%%%%%%%%%%%%%%%%%%%%%%%%%%%%%%%%%
%%%%%%%%%%%%%%%%%%%%%%%%%%%%%%%%%%%%%%%%%%%%%%%%%%%%%%%%%%%%%%%%%%%%%%%%%%%%%%%%
\exercice{Système masse-ressort~\difficile}
%%%%%%%%%%%%%%%%%%%%%%%%%%%%%%%%%%%%%%%%%%%%%%%%%%%%%%%%%%%%%%%%%%%%%%%%%%%%%%%%
%%%%%%%%%%%%%%%%%%%%%%%%%%%%%%%%%%%%%%%%%%%%%%%%%%%%%%%%%%%%%%%%%%%%%%%%%%%%%%%%
On reprend ici, la mise en équation du système mécanique masse-ressort 
présentée dans ce chapitre (c.f \cref{para-masse_ressort}). L'équation 
différentielle de la position de la masse est donnée par 
\[
m\devi{x(t)}{2}+b\devi{x(t)}{}+kx(t)=f(t)
\]
où $m$ est la masse, $b$ le coefficient d'amortissement visqueux et $k$ la
constante de raideur du ressort. Ce système peut être considéré comme
un système dont l'entrée est la force $f(t)$ et la sortie est la position
$x(t)$ de la masse.
%-------------------------------------------------------------------------------
\begin{marginfigure}
    \centering
    \tikzsetnextfilename{masse_ressort-chap_slci-ext}
    \input{tikz/masse_ressort-chap_slci.tex}
\end{marginfigure}
%-------------------------------------------------------------------------------
On se place dans le cas de conditions initiales nulles. 

%%%%%%%%%%%%%%%%%%%%%%%%%%%%%%%%%%%%%%%%%%%%%%%%%%%%%%%%%%%%%%%%%%%%%%%%%%%%%%%%
\question{Déterminer la transformée de Laplace $X(p)$ solution algébrique 
de la transformée de Laplace de l'équation différentielle de ce système.}
%%%%%%%%%%%%%%%%%%%%%%%%%%%%%%%%%%%%%%%%%%%%%%%%%%%%%%%%%%%%%%%%%%%%%%%%%%%%%%%%
%%%%%%%%%%%%%%%%%%%%%%%%%%%%%%%%%%%%%%%%%%%%%%%%%%%%%%%%%%%%%%%%%%%%%%%%%%%%%%%%
\question{Indentifier la fonction de transfert $H(p)$ du système et 
donner sa forme canonique}
%%%%%%%%%%%%%%%%%%%%%%%%%%%%%%%%%%%%%%%%%%%%%%%%%%%%%%%%%%%%%%%%%%%%%%%%%%%%%%%%
%%%%%%%%%%%%%%%%%%%%%%%%%%%%%%%%%%%%%%%%%%%%%%%%%%%%%%%%%%%%%%%%%%%%%%%%%%%%%%%%
\question{Indentifier la solution algébrique $X(p)$ pour une cette 
solicitation.}
%%%%%%%%%%%%%%%%%%%%%%%%%%%%%%%%%%%%%%%%%%%%%%%%%%%%%%%%%%%%%%%%%%%%%%%%%%%%%%%%
%%%%%%%%%%%%%%%%%%%%%%%%%%%%%%%%%%%%%%%%%%%%%%%%%%%%%%%%%%%%%%%%%%%%%%%%%%%%%%%%
\question{Identifier les paramètres $K$, $\xi$ et $\omega$ en fonction des 
paramètres du problème.}
%%%%%%%%%%%%%%%%%%%%%%%%%%%%%%%%%%%%%%%%%%%%%%%%%%%%%%%%%%%%%%%%%%%%%%%%%%%%%%%%
En comparant, la solution algébrique précédente à la forme 
\[
    \dfrac{K}{\dfrac{1}{\omega^2}p^2+\dfrac{2\xi}{\omega} p+1}.
\] 
%\newpage
%\restoregeometry
%\captionsetup{width=0.9\linewidth}
En utilisant, la transformée de Laplace inverse suivante : 
\[
    \laplacei{\dfrac{1}{\dfrac{1}{\omega^2}p^2+\dfrac{2\xi}{\omega} p+1}}=
    \dfrac{\omega}{\sqrt{1-\xi^2}}
    e^{-\xi\omega t}\sin{\omega\sqrt{1-\xi^2} t}
\quad\quad \text{pour}\quad\xi<1 
\]
%%%%%%%%%%%%%%%%%%%%%%%%%%%%%%%%%%%%%%%%%%%%%%%%%%%%%%%%%%%%%%%%%%%%%%%%%%%%%%%%
\question{Déterminer $x(t)$ la solution temporelle de l'équation différentielle 
pour une telle sollicitation.}
%%%%%%%%%%%%%%%%%%%%%%%%%%%%%%%%%%%%%%%%%%%%%%%%%%%%%%%%%%%%%%%%%%%%%%%%%%%%%%%%
$\,$\newline
On réalise une experience avec une masse $m=\SI{0.5}{\kilogram}$, 
un ressort de constante de raideur $k=\SI{10}{\newton\per\meter}$ et 
un amortissement visqueux $b=\SI{0.894}{\kilogram\per\second}$. 
%%%%%%%%%%%%%%%%%%%%%%%%%%%%%%%%%%%%%%%%%%%%%%%%%%%%%%%%%%%%%%%%%%%%%%%%%%%%%%%%
\question{Déterminer les valeurs numériques des paramètres $\xi$ et $\omega$}
%%%%%%%%%%%%%%%%%%%%%%%%%%%%%%%%%%%%%%%%%%%%%%%%%%%%%%%%%%%%%%%%%%%%%%%%%%%%%%%%
%%%%%%%%%%%%%%%%%%%%%%%%%%%%%%%%%%%%%%%%%%%%%%%%%%%%%%%%%%%%%%%%%%%%%%%%%%%%%%%%
\question{Tracer la réponse $x(t)$ et déterminer le temps pour que la masse 
retourne à son position d'équilibre}
%%%%%%%%%%%%%%%%%%%%%%%%%%%%%%%%%%%%%%%%%%%%%%%%%%%%%%%%%%%%%%%%%%%%%%%%%%%%%%%%

%%%%%%%%%%%%%%%%%%%%%%%%%%%%%%%%%%%%%%%%%%%%%%%%%%%%%%%%%%%%%%%%%%%%%%%%%%%%%%%%
%%%%%%%%%%%%%%%%%%%%%%%%%%%%%%%%%%%%%%%%%%%%%%%%%%%%%%%%%%%%%%%%%%%%%%%%%%%%%%%%
\subsection*{Fonction de transfert}
%%%%%%%%%%%%%%%%%%%%%%%%%%%%%%%%%%%%%%%%%%%%%%%%%%%%%%%%%%%%%%%%%%%%%%%%%%%%%%%%
%%%%%%%%%%%%%%%%%%%%%%%%%%%%%%%%%%%%%%%%%%%%%%%%%%%%%%%%%%%%%%%%%%%%%%%%%%%%%%%%

%%%%%%%%%%%%%%%%%%%%%%%%%%%%%%%%%%%%%%%%%%%%%%%%%%%%%%%%%%%%%%%%%%%%%%%%%%%%%%%%
%%%%%%%%%%%%%%%%%%%%%%%%%%%%%%%%%%%%%%%%%%%%%%%%%%%%%%%%%%%%%%%%%%%%%%%%%%%%%%%%
\exercice{Forme canonique d'une fonction de transfert du premier ordre~\facile}
%%%%%%%%%%%%%%%%%%%%%%%%%%%%%%%%%%%%%%%%%%%%%%%%%%%%%%%%%%%%%%%%%%%%%%%%%%%%%%%%
%%%%%%%%%%%%%%%%%%%%%%%%%%%%%%%%%%%%%%%%%%%%%%%%%%%%%%%%%%%%%%%%%%%%%%%%%%%%%%%%

Soit un système régi par l'équation différentielle suivante 
\[
    a_1\devi{s(t)}{} + a_0 s(t) = b_0 e(t)
\]

%%%%%%%%%%%%%%%%%%%%%%%%%%%%%%%%%%%%%%%%%%%%%%%%%%%%%%%%%%%%%%%%%%%%%%%%%%%%%%%%
\question{Déterminer la forme canonique de fonction de transfert 
$H(p)$ associée à cette équation différentielle}
%%%%%%%%%%%%%%%%%%%%%%%%%%%%%%%%%%%%%%%%%%%%%%%%%%%%%%%%%%%%%%%%%%%%%%%%%%%%%%%%
%%%%%%%%%%%%%%%%%%%%%%%%%%%%%%%%%%%%%%%%%%%%%%%%%%%%%%%%%%%%%%%%%%%%%%%%%%%%%%%%
\question{Déterminer les paramètres $K$ et $\tau$ en fonction des coefficients
$a_i$ et $b_i$ de tels sorte que $H(p)$ s'écrivent sous la forme}
%%%%%%%%%%%%%%%%%%%%%%%%%%%%%%%%%%%%%%%%%%%%%%%%%%%%%%%%%%%%%%%%%%%%%%%%%%%%%%%%
%%%%%%%%%%%%%%%%%%%%%%%%%%%%%%%%%%%%%%%%%%%%%%%%%%%%%%%%%%%%%%%%%%%%%%%%%%%%%%%%
\question{Déterminer les dimensions de ces paramètres quelques soient celles 
des signaux d'entrée $e(t)$ et de sortie $s(t)$}
%%%%%%%%%%%%%%%%%%%%%%%%%%%%%%%%%%%%%%%%%%%%%%%%%%%%%%%%%%%%%%%%%%%%%%%%%%%%%%%%

%%%%%%%%%%%%%%%%%%%%%%%%%%%%%%%%%%%%%%%%%%%%%%%%%%%%%%%%%%%%%%%%%%%%%%%%%%%%%%%%
%%%%%%%%%%%%%%%%%%%%%%%%%%%%%%%%%%%%%%%%%%%%%%%%%%%%%%%%%%%%%%%%%%%%%%%%%%%%%%%%
\exercice{Carte des pôles d'une fonction de transfert~\moyen}
%%%%%%%%%%%%%%%%%%%%%%%%%%%%%%%%%%%%%%%%%%%%%%%%%%%%%%%%%%%%%%%%%%%%%%%%%%%%%%%%
%%%%%%%%%%%%%%%%%%%%%%%%%%%%%%%%%%%%%%%%%%%%%%%%%%%%%%%%%%%%%%%%%%%%%%%%%%%%%%%%
Soit la carte des pôles et zéros ci-contre 
%%%%%%%%%%%%%%%%%%%%%%%%%%%%%%%%%%%%%%%%%%%%%%%%%%%%%%%%%%%%%%%%%%%%%%%%%%%%%%%%
\question{Donner les zéros et les pôles de la fonction de transfert.}
%%%%%%%%%%%%%%%%%%%%%%%%%%%%%%%%%%%%%%%%%%%%%%%%%%%%%%%%%%%%%%%%%%%%%%%%%%%%%%%%
%%%%%%%%%%%%%%%%%%%%%%%%%%%%%%%%%%%%%%%%%%%%%%%%%%%%%%%%%%%%%%%%%%%%%%%%%%%%%%%%
\question{\'Ecrire la fonction de transfert $H(p)$ sachant que le gain de 
sa forme factorisée vaut 3.}
%%%%%%%%%%%%%%%%%%%%%%%%%%%%%%%%%%%%%%%%%%%%%%%%%%%%%%%%%%%%%%%%%%%%%%%%%%%%%%%%
%-------------------------------------------------------------------------------
\begin{marginfigure}
    \centering
    \tikzsetnextfilename{carte-exo-chap_slci-ext}
\begin{tikzpicture}
    \begin{axis}
    [   scaled y ticks = false,
        axis line style = thick,
        axis x line=center,
        axis y line=center,
        width=\linewidth,
        height=6cm,
        xmin=-6,
        xmax=1,
        ymin=-2.5,
        ymax=2.5,
        xlabel={$\boldsymbol{\Re{p}}$},
        ylabel={$\boldsymbol{\Im{p}}$},
        xlabel style={below right},
        ylabel style={above right},
        xticklabels={-5,-3,-1},
        xtick={-5,-3,-1},
        yticklabels={-2,-1,0,1,2},
        ytick={-2,-1,0,1,2},
        y tick label style={anchor=west,xshift=0.5em}
    ]
    \addplot[mark=o,utb,only marks,mark size=5pt] coordinates{(-5,1) (-5,-1)};
    \addplot[mark=x,utb,only marks,mark size=5pt] coordinates{(-1,2) (-1,-2)};
    \addplot[mark=x,utb,mark size=5pt] coordinates{ (-3,0) };
    \end{axis}
\end{tikzpicture}
\end{marginfigure}
%-------------------------------------------------------------------------------
%%%%%%%%%%%%%%%%%%%%%%%%%%%%%%%%%%%%%%%%%%%%%%%%%%%%%%%%%%%%%%%%%%%%%%%%%%%%%%%%
\question{Déterminer la forme canonique de $H(p)$ et déduire le gain statique, 
la classe et l'ordre de ce système.}
%%%%%%%%%%%%%%%%%%%%%%%%%%%%%%%%%%%%%%%%%%%%%%%%%%%%%%%%%%%%%%%%%%%%%%%%%%%%%%%%
