%%%%%%%%%%%%%%%%%%%%%%%%%%%%%%%%%%%%%%%%%%%%%%%%%%%%%%%%%%%%%%%%%%%%%%%%%%%%%%%%
%%%%%%%%%%%%%%%%%%%%%%%%%%%%%%%%%%%%%%%%%%%%%%%%%%%%%%%%%%%%%%%%%%%%%%%%%%%%%%%%
\exercice{Précision d'un système du second ordre~\facile}
%%%%%%%%%%%%%%%%%%%%%%%%%%%%%%%%%%%%%%%%%%%%%%%%%%%%%%%%%%%%%%%%%%%%%%%%%%%%%%%%
%%%%%%%%%%%%%%%%%%%%%%%%%%%%%%%%%%%%%%%%%%%%%%%%%%%%%%%%%%%%%%%%%%%%%%%%%%%%%%%%
Soit un système du second ordre dont la fonction de transfert $H(p)$ est donnée
par la forme canonique suivante:
\[
H(p)=\dfrac{K\omega_0^2}{p^2+2\xi\omega_0p+\omega_0^2}
\]
%%%%%%%%%%%%%%%%%%%%%%%%%%%%%%%%%%%%%%%%%%%%%%%%%%%%%%%%%%%%%%%%%%%%%%%%%%%%%%%%
\question{\textbf{Déterminer la forme générale de l'erreur statique de la 
réponse indicielle d'un système du second ordre}}
%%%%%%%%%%%%%%%%%%%%%%%%%%%%%%%%%%%%%%%%%%%%%%%%%%%%%%%%%%%%%%%%%%%%%%%%%%%%%%%%
%%%%%%%%%%%%%%%%%%%%%%%%%%%%%%%%%%%%%%%%%%%%%%%%%%%%%%%%%%%%%%%%%%%%%%%%%%%%%%%%
\question{\textbf{Déterminer la forme générale de l'erreur statique de la 
réponse indicielle de ce système placé dans une boucle de contre réaction}}
%%%%%%%%%%%%%%%%%%%%%%%%%%%%%%%%%%%%%%%%%%%%%%%%%%%%%%%%%%%%%%%%%%%%%%%%%%%%%%%%
%%%%%%%%%%%%%%%%%%%%%%%%%%%%%%%%%%%%%%%%%%%%%%%%%%%%%%%%%%%%%%%%%%%%%%%%%%%%%%%%
\question{\textbf{Déterminer la forme générale de l'erreur statique de la 
réponse à une rampe d'un système du second ordre}}
%%%%%%%%%%%%%%%%%%%%%%%%%%%%%%%%%%%%%%%%%%%%%%%%%%%%%%%%%%%%%%%%%%%%%%%%%%%%%%%%
%%%%%%%%%%%%%%%%%%%%%%%%%%%%%%%%%%%%%%%%%%%%%%%%%%%%%%%%%%%%%%%%%%%%%%%%%%%%%%%%
%%%%%%%%%%%%%%%%%%%%%%%%%%%%%%%%%%%%%%%%%%%%%%%%%%%%%%%%%%%%%%%%%%%%%%%%%%%%%%%%
\exercice{Pôles dominants}
%%%%%%%%%%%%%%%%%%%%%%%%%%%%%%%%%%%%%%%%%%%%%%%%%%%%%%%%%%%%%%%%%%%%%%%%%%%%%%%%
%%%%%%%%%%%%%%%%%%%%%%%%%%%%%%%%%%%%%%%%%%%%%%%%%%%%%%%%%%%%%%%%%%%%%%%%%%%%%%%%
%%%%%%%%%%%%%%%%%%%%%%%%%%%%%%%%%%%%%%%%%%%%%%%%%%%%%%%%%%%%%%%%%%%%%%%%%%%%%%%%
\question{\textbf{Déterminer la décomposition en éléments simples.}}
%%%%%%%%%%%%%%%%%%%%%%%%%%%%%%%%%%%%%%%%%%%%%%%%%%%%%%%%%%%%%%%%%%%%%%%%%%%%%%%%
La décomposition en éléments simples est donnée par :
\[
H(p)=\dfrac{A}{(1+0.1p)}+\dfrac{B}{(1+0.1p)^2}+\dfrac{C}{(1+0.05p)}
\]
Il nous faut maintenant identifier les coefficients $A$,$B$ et $C$.
Le coefficient $C$ est obtenu $H(p)$ par $(1+0.05p)$ et en évaluant $p$ en $-20$:

\[
C=(1+0.05p)H(p)\Big|_{p=-20}=5
\]

Par identification des numérateurs :
\[
A(1+0.1p)(1+0.05p)+B(1+0.05p)+5(1+0.1p)^2=5
\]
il apparait évident que le monome d'ordre 0 nous donne l'expression:
\[
A+B=0
\]
pour le l'ordre 2, nous avons :
\[
0.005A+0.05=0
\]
soit alors $A=-10$ et $B=10$

La décomposition en éléments simples s'écrit donc :
\[
H(p)=\dfrac{-10}{(1+0.1p)}+\dfrac{10}{(1+0.1p)^2}+\dfrac{5}{(1+0.05p)}
\]
%%%%%%%%%%%%%%%%%%%%%%%%%%%%%%%%%%%%%%%%%%%%%%%%%%%%%%%%%%%%%%%%%%%%%%%%%%%%%%%%
\question{\textbf{Identifier les pôles de chacunes des contributions de 
cette décomposition et les pôles dominants.}}
%%%%%%%%%%%%%%%%%%%%%%%%%%%%%%%%%%%%%%%%%%%%%%%%%%%%%%%%%%%%%%%%%%%%%%%%%%%%%%%%
On pose
\begin{align*}
    H(p)&= {\color{col1} H_1(p)}+
           {\color{col2} H_2(p)}+
           {\color{col4} H_3(p)}\\
    H(p)&= {\color{col1} \dfrac{-10}{(1+0.1p)}}+
           {\color{col2} \dfrac{10}{(1+0.1p)^2}}+
           {\color{col4} \dfrac{5}{(1+0.05p)}}\\
\end{align*}
Le premier terme est un système du premier ordre de temps de réponse à 5\% de
$3\tau_1=\SI{0.3}{\second}$.
Le second terme est un système du second ordre: avec $\xi=1$ et $\omega_0=10$
de temps de réponse à 5\% de $\SI{0.5}{\second}$ (lecture sur l'abaque) 

Le dernier terme est un premier ordre de temps de réponse à 5\% de 
$3\tau_3=\SI{0.15}{\second}$.

Les pôles dominants sont les pôles du premier et second termes de partie réelle -10.
%-------------------------------------------------------------------------------
\begin{center}
    \tikzsetnextfilename{carte-exercices_corriges-chap_perf-ext}
    \begin{tikzpicture}
    \carte[l]
    \dpole{-1}{0}{A}[col1]
    \dpole{-1}{0}{B}[col2][above]
    \dpole{-2}{0}{C}[col4]
\end{tikzpicture}

\end{center}
%-------------------------------------------------------------------------------
%%%%%%%%%%%%%%%%%%%%%%%%%%%%%%%%%%%%%%%%%%%%%%%%%%%%%%%%%%%%%%%%%%%%%%%%%%%%%%%%
\question{\textbf{Tracer la réponse indicielle de chacunes des 
contributions. Conclure.}}
%%%%%%%%%%%%%%%%%%%%%%%%%%%%%%%%%%%%%%%%%%%%%%%%%%%%%%%%%%%%%%%%%%%%%%%%%%%%%%%%
On représente ci-dessous,
le réponse indicielle $s_i(t)$ pour chacunes des contributions
$H_i(p)$ ainsi que la réponse totale $s(t)$
%-------------------------------------------------------------------------------
\begin{center}
    \tikzsetnextfilename{reponses_indicielles-exercices_corriges-chap_perf-ext}
    \begin{tikzpicture}[shift={(5,0)}]
    \begin{axis}
    [   legend style={draw=none,font=\normalsize},
        legend pos=south east,
        axis line style = thick,
        width=0.6\textwidth,
        xmin=0,
        xmax=0.9,
        ymin=0,
        ymax=11,
        xlabel={$t$},
        ylabel={$s(t)$},
        label style={font=\Large},
        legend cell align={left},
    ]%
    \addplot[signaln,domain=0:0.9] {5*(1-exp(-20*x))-
                                   10*(1-exp(-10*x))
                                  +10*(1-exp(-10*x)
                                  -10*x*exp(-10*x))};
    \addplot[signalr,domain=0:0.9] {10*(1-exp(-10*x))};
    \addplot[signalo,domain=0:0.9] {10*(1-exp(-10*x)-10*x*exp(-10*x))};
    \addplot[signalb,domain=0:0.9] {5*(1-exp(-20*x))};
    \legend{$s(t)$,$-s_1(t)$,$s_2(t)$,$s_3(t)$}
    \end{axis}%
\end{tikzpicture}

\end{center}
%-------------------------------------------------------------------------------
On conclue bien sur la domination de la réponse $s_2(t)$ sur la réponse totale.
\clearpage
%%%%%%%%%%%%%%%%%%%%%%%%%%%%%%%%%%%%%%%%%%%%%%%%%%%%%%%%%%%%%%%%%%%%%%%%%%%%%%%
%%%%%%%%%%%%%%%%%%%%%%%%%%%%%%%%%%%%%%%%%%%%%%%%%%%%%%%%%%%%%%%%%%%%%%%%%%%%%%%
\exercice{Temps de montée et bande passante}
%%%%%%%%%%%%%%%%%%%%%%%%%%%%%%%%%%%%%%%%%%%%%%%%%%%%%%%%%%%%%%%%%%%%%%%%%%%%%%%
%%%%%%%%%%%%%%%%%%%%%%%%%%%%%%%%%%%%%%%%%%%%%%%%%%%%%%%%%%%%%%%%%%%%%%%%%%%%%%%
Soit un système en boucle ouverte de fonction de transfert en 
boucle ouverte $H(p)$ définie  par :
\[
H(p)=\dfrac{K}{p(p+1)}
\]
On place ce simple dans une boucle de contre réaction unitaire.
%%%%%%%%%%%%%%%%%%%%%%%%%%%%%%%%%%%%%%%%%%%%%%%%%%%%%%%%%%%%%%%%%%%%%%%%%%%%%%%
\question{\textbf{Pour les valeurs de $K=0.1$, $K=1$ et $K=10$. 
Déterminer le temps de montée du système en boucle fermée.}}
%%%%%%%%%%%%%%%%%%%%%%%%%%%%%%%%%%%%%%%%%%%%%%%%%%%%%%%%%%%%%%%%%%%%%%%%%%%%%%%
La fonction de transfert en boucle fermée est donnée par :
\[
H_{BF}(p)=\dfrac{K}{p(p+1)+K}=\dfrac{K}{p^2+p+K}
\]
par identification avec la fonction de transfert canonique d'un système 
du second ordre, on détermine :
%-------------------------------------------------------------------------------
\begin{align*}
K_{BF}=1\\
\omega_{0,BF}=\sqrt{K}\\
\xi_{BF}=\dfrac{1}{2\sqrt{K}}\\
\end{align*}
%-------------------------------------------------------------------------------
Nous déterminons les temps de montée à la valeur finale à l'aide 
de l'abaque approprié. Nous regroupons les données dans le tableau suivant :
%-------------------------------------------------------------------------------
\begin{table}[!hb]
\centering
\setlength{\ltmp}{0.15\textwidth}
\begin{tabular}{@{}P{\ltmp}P{\ltmp}P{\ltmp}P{\ltmp}P{\ltmp}@{}}
    \toprule
    $K$ & $\xi_{BF}$ & $\omega_{0,BF}$ & $t_M\cdot\omega_{0,BF}$ 
        & $t_M$ (\si{\second}) \\
    \midrule
    0.1 & $\sim$1.6  & $\sim$0.316 & -    & - \\       
    1   & 0.5        & 1.0         & 2.5  & 2.5 \\       
    10  & $\sim$0.16 & $\sim$3.16  & 1.75 & $\sim$0.55 \\       
    \bottomrule
\end{tabular}    
\end{table}
%-------------------------------------------------------------------------------
%-------------------------------------------------------------------------------
\begin{center}
    \tikzsetnextfilename{temps_montee-exercices_corriges-chap_perf-ext}
    \begin{tikzpicture}
\pgfmathdeclarefunction{func2}{1}{%
    \pgfmathparse{%
        (3.14159265-rad(atan(sqrt(1-#1*#1)/#1)))/sqrt(1-#1*#1)
    }%
}
    \begin{axis}
    [   legend style={draw=none,font=\normalsize},
        legend pos=outer north east,
        axis line style = thick,
        width=0.55\textwidth,
        xmin=1e-1,
        xmax=1e0,
        ymin=1e0,
        ymax=1e2,
        xlabel={$\xi$},
        ylabel={$t_{M}\cdot\omega_0$},
        label style={font=\Large},
        legend cell align={left},
        xmode=log,
        ymode=log,
        grid=both,
        grid style={line width=.4pt, draw=black},
        major grid style={line width=.4pt,draw=black},
        clip mode=individual,
    ]%
        \addplot[vtr,domain=0.1:1,samples=500]   {func2(x)};
        \draw [dutb] (axis cs:1e-1,2.49) -- (axis cs:0.5,2.49);
        \draw [dutb] (axis cs:0.5,1)   -- (axis cs:0.5,2.49);
        \node[left,col1] at (axis cs:1e-1,2.49) {$t_{M}\cdot\omega_0\sim2.5$} ;
        \node[below,col1] at (axis cs:0.5,1) {$\xi\sim0.5$};
        \draw [dutr] (axis cs:1e-1,1.75) -- (axis cs:0.16,1.75);
        \draw [dutr] (axis cs:0.16,1)   -- (axis cs:0.16,1.75);
        \node[left,col4] at (axis cs:1e-1,1.75) {$t_{M}\cdot\omega_0\sim1.75$} ;
        \node[below,col4] at (axis cs:0.16,1) {$\xi\sim0.16$};
    \end{axis}
\end{tikzpicture}

\end{center}
%-------------------------------------------------------------------------------
\clearpage
%%%%%%%%%%%%%%%%%%%%%%%%%%%%%%%%%%%%%%%%%%%%%%%%%%%%%%%%%%%%%%%%%%%%%%%%%%%%%%%%
\question{\textbf{Tracer les diagrammes de Bode des système en boucle 
ouverte et en boucle fermée pour $K=0.1$, $K=1$ et $K=10$.}}
%%%%%%%%%%%%%%%%%%%%%%%%%%%%%%%%%%%%%%%%%%%%%%%%%%%%%%%%%%%%%%%%%%%%%%%%%%%%%%%%
%%%%%%%%%%%%%%%%%%%%%%%%%%%%%%%%%%%%%%%%%%%%%%%%%%%%%%%%%%%%%%%%%%%%%%%%%%%%%%%%
\question{\textbf{Pour chacunes de ces figures déterminer la bande passante 
à \SI{3}{\decibel} en boucle fermée ainsi que la valeur de $\omega_{0,dB}$ 
en boucle ouverte (c'est à dire la valeur de la pulsation pour laquelle 
le gain est nulle en boucle ouverte).}}
%%%%%%%%%%%%%%%%%%%%%%%%%%%%%%%%%%%%%%%%%%%%%%%%%%%%%%%%%%%%%%%%%%%%%%%%%%%%%%%%
Pour $K=0.1$
%-------------------------------------------------------------------------------
\begin{center}
    \tikzsetnextfilename{bodeBP_K0.1-exercices_corriges-chap_perf-ext}
    \begin{tikzpicture}[trim axis left]
\begin{axis}
[
    ticklabel style = {font=\footnotesize},
    width=0.9\textwidth,
    height=0.25\textheight,
    grid=both,
    major grid style={black!40},
    xmode=log,ymode=normal,ylabel={Gain(\si{\decibel})},
    xtick={1e-2,1e-1,1e0,1e1,1e2},
    xticklabels={$10^{-2}$,$10^{-1}$,$10^{0}$,$10^{1}$,$10^{2}$},
    ytick={-60,-50,-40,-30,-20,-10,0,10,20,30,40},
    yticklabels={-60,,-40,,-20,,0,,20,,40},
    xmin=1e-2,xmax=1e2,
    ymin=-60,ymax=40,
    clip=false
]
\pgfmathsetmacro{\kk}{0.1}
\addplot[signalb,domain=1e-2:1e1] {20*log10(\kk)-10*log10((\kk-x*x)^2+x*x)};
\addplot[signalo,domain=1e-2:1e1] {20*log10(\kk)-20*log10(x)-10*log10(1+x*x)};
\draw[dutb] (axis cs:1e-2,-3) node[left] 
                              {\SI{-3}{\dB} }-- (axis cs:1e-1,-3);
\draw[dutb] (axis cs:1e-1,-60) node[below,xshift=2em,yshift=-1em] 
                               {$\omega_{-3dB,BF}$}-- (axis cs:1e-1,-3) ;
\draw[duto] (axis cs:1e-1,-60) node[below,xshift=-2em,yshift=-1em] 
                               {$\omega_{0,BO}=$}-- (axis cs:1e-1,-3) ;
\end{axis}
\end{tikzpicture}

\end{center}
%-------------------------------------------------------------------------------
Pour $K=1$
%-------------------------------------------------------------------------------
\begin{center}
    \tikzsetnextfilename{bodeBP_K1-exercices_corriges-chap_perf-ext}
    \input{tikz/bodeBP_K1-exercices_corriges-chap_perf.tex}
\end{center}
%-------------------------------------------------------------------------------
Pour $K=10$
%-------------------------------------------------------------------------------
\begin{center}
    \tikzsetnextfilename{bodeBP_K10-exercices_corriges-chap_perf-ext}
    \input{tikz/bodeBP_K10-exercices_corriges-chap_perf.tex}
\end{center}
%-------------------------------------------------------------------------------
\clearpage
%%%%%%%%%%%%%%%%%%%%%%%%%%%%%%%%%%%%%%%%%%%%%%%%%%%%%%%%%%%%%%%%%%%%%%%%%%%%%%%%
\question{\textbf{D'après les résultats quelle lien existe t-il entre la 
bande passante et la rapidité. Et comment l'approximer à partir de la 
réponse harmonique en boucle ouverte.}}
%%%%%%%%%%%%%%%%%%%%%%%%%%%%%%%%%%%%%%%%%%%%%%%%%%%%%%%%%%%%%%%%%%%%%%%%%%%%%%%%
Il aparrait que la bande passante à \SI{3}{\decibel} est liée à la rapidité. 
Plus cette bande passante est grande plus faible est le temps de montée (donc plus rapide).
Il apparait également que la bande passante peut être approximée par la valeur de la pulsation
donnant un gain de \SI{0}{\dB} en boucle ouverte. On peut alors dire :
\[
\omega_{-3dB,BF}\sim\omega_{0,BO}
\]


