%%%%%%%%%%%%%%%%%%%%%%%%%%%%%%%%%%%%%%%%%%%%%%%%%%%%%%%%%%%%%%%%%%%%%%%%%%%%%%%%
%%%%%%%%%%%%%%%%%%%%%%%%%%%%%%%%%%%%%%%%%%%%%%%%%%%%%%%%%%%%%%%%%%%%%%%%%%%%%%%%
\exercice{Décomposition d'un signal}
%%%%%%%%%%%%%%%%%%%%%%%%%%%%%%%%%%%%%%%%%%%%%%%%%%%%%%%%%%%%%%%%%%%%%%%%%%%%%%%%
%%%%%%%%%%%%%%%%%%%%%%%%%%%%%%%%%%%%%%%%%%%%%%%%%%%%%%%%%%%%%%%%%%%%%%%%%%%%%%%%

%%%%%%%%%%%%%%%%%%%%%%%%%%%%%%%%%%%%%%%%%%%%%%%%%%%%%%%%%%%%%%%%%%%%%%%%%%%%%%%%
\question{}
%%%%%%%%%%%%%%%%%%%%%%%%%%%%%%%%%%%%%%%%%%%%%%%%%%%%%%%%%%%%%%%%%%%%%%%%%%%%%%%%
Le signal $e(t)$ est défini par :
\[
e(t)=\begin{cases}
    0&\quad t<0 \\
    t&\quad t\in[0,1]\\
    2-t&\quad t\in[1,3]\\
    t-4&\quad t\in[3,4]\\
    0&\quad t>4
     \end{cases}
\]
On peut alors décomposer $e(t)$ en somme d'échelon :
\[
    e(t)=t    \left(u(t)-u(t-1)  \right)+(2-t)\left(u(t-1)-u(t-3)\right)+
         (t-4)\left(u(t-3)-u(t-4)\right) 
\]
En regroupant les termes, on obtient alors :
\[
    e(t)=tu(t)+2(1-t)u(t-1)+2(t-3)u(t-3)-(t-4)u(t-4)
\]
%%%%%%%%%%%%%%%%%%%%%%%%%%%%%%%%%%%%%%%%%%%%%%%%%%%%%%%%%%%%%%%%%%%%%%%%%%%%%%%%
\question{}
%%%%%%%%%%%%%%%%%%%%%%%%%%%%%%%%%%%%%%%%%%%%%%%%%%%%%%%%%%%%%%%%%%%%%%%%%%%%%%%%
La transformée de Laplace $E(p)$ est simplement donnée par la transformée de
chacun des termes.
\[
    E(p)=\dfrac{1}{p^2}+\dfrac{2e^{-p}}{p^2}\left(p-1\right)
                       +\dfrac{2e^{-3p}}{p^2}\left(1-3p\right)
                       -\dfrac{e^{-4p}}{p^2}\left(1-4p\right)
\]
ou encore
\[
    E(p)=\dfrac{1}{p^2}\left(1+2e^{-p}(p-1)+2e^{-3p}(1-3p)-e^{-4p}(1-4p)\right)
\]
%%%%%%%%%%%%%%%%%%%%%%%%%%%%%%%%%%%%%%%%%%%%%%%%%%%%%%%%%%%%%%%%%%%%%%%%%%%%%%%%
%%%%%%%%%%%%%%%%%%%%%%%%%%%%%%%%%%%%%%%%%%%%%%%%%%%%%%%%%%%%%%%%%%%%%%%%%%%%%%%%
\exercice{Décomposition d'un signal parabolique~\moyen}
%%%%%%%%%%%%%%%%%%%%%%%%%%%%%%%%%%%%%%%%%%%%%%%%%%%%%%%%%%%%%%%%%%%%%%%%%%%%%%%%
%%%%%%%%%%%%%%%%%%%%%%%%%%%%%%%%%%%%%%%%%%%%%%%%%%%%%%%%%%%%%%%%%%%%%%%%%%%%%%%%

%%%%%%%%%%%%%%%%%%%%%%%%%%%%%%%%%%%%%%%%%%%%%%%%%%%%%%%%%%%%%%%%%%%%%%%%%%%%%%%%
\question{}
%%%%%%%%%%%%%%%%%%%%%%%%%%%%%%%%%%%%%%%%%%%%%%%%%%%%%%%%%%%%%%%%%%%%%%%%%%%%%%%%
\[
    e(t)=t^2\left(u(t)-u(t-1)\right)+(t-2)^2\left(u(t-1)-u(t-2)\right)
\]
%%%%%%%%%%%%%%%%%%%%%%%%%%%%%%%%%%%%%%%%%%%%%%%%%%%%%%%%%%%%%%%%%%%%%%%%%%%%%%%%
\question{Donner la transformée de Laplace $E(p)$ de ce signal.}
%%%%%%%%%%%%%%%%%%%%%%%%%%%%%%%%%%%%%%%%%%%%%%%%%%%%%%%%%%%%%%%%%%%%%%%%%%%%%%%%

%%%%%%%%%%%%%%%%%%%%%%%%%%%%%%%%%%%%%%%%%%%%%%%%%%%%%%%%%%%%%%%%%%%%%%%%%%%%%%%%
%%%%%%%%%%%%%%%%%%%%%%%%%%%%%%%%%%%%%%%%%%%%%%%%%%%%%%%%%%%%%%%%%%%%%%%%%%%%%%%%
\exercice{Décompositon d'un signal et transformée de Laplace}
%%%%%%%%%%%%%%%%%%%%%%%%%%%%%%%%%%%%%%%%%%%%%%%%%%%%%%%%%%%%%%%%%%%%%%%%%%%%%%%%
%%%%%%%%%%%%%%%%%%%%%%%%%%%%%%%%%%%%%%%%%%%%%%%%%%%%%%%%%%%%%%%%%%%%%%%%%%%%%%%%

On sollicite un système linéaire continu et invariant par deux signaux $e_1(t)$ 
et $e_2(t)$ (d'amplitude maximale et de durée unité) définits 
par les graphes ci-contre.

%%%%%%%%%%%%%%%%%%%%%%%%%%%%%%%%%%%%%%%%%%%%%%%%%%%%%%%%%%%%%%%%%%%%%%%%%%%%%%%%
\question{}
%%%%%%%%%%%%%%%%%%%%%%%%%%%%%%%%%%%%%%%%%%%%%%%%%%%%%%%%%%%%%%%%%%%%%%%%%%%%%%%%
Décomposons $e_1(t)$ et $e_2(t)$ par morceaux. On a alors :
\[
e_1(t)=
\begin{cases}
    0&\quad t<0\\
    1-t&\quad t\in[0,1]\\
    0&\quad t>1
\end{cases}
\]
et
\[
e_2(t)=
\begin{cases}
    0&\quad t<0\\
    t&\quad t\in[0,1]\\
    0&\quad t>1
\end{cases}
\]
On peut alors écrire ces signaux comme la somme des signaux usuels comme:
\[
    e_1(t)=(1-t)\left(u(t)-u(t-1)\right)=u(t)-r(t)-u(t-1)+r(t-1)
\]
et
\[
    e_2(t)=t\left(u(t)-u(t-1)\right)=r(t)-r(t-1)
\]
où $u(t)$ est la fonction échelon et $r(t)$ la fonction rampe.

%%%%%%%%%%%%%%%%%%%%%%%%%%%%%%%%%%%%%%%%%%%%%%%%%%%%%%%%%%%%%%%%%%%%%%%%%%%%%%%%
\question{}
%%%%%%%%%%%%%%%%%%%%%%%%%%%%%%%%%%%%%%%%%%%%%%%%%%%%%%%%%%%%%%%%%%%%%%%%%%%%%%%%
Les transformées sont simplement données par la propriété des fonctions 
retardés et des signaux rampes. On a notamment,
\[
    E_1(p)=\dfrac{(p-1)}{p^2}\left(1-e^{-p}\right)
\]
et
\[
    E_2(p)=\dfrac{1}{p^2}\left(1-e^{-p}\right)
\]

%%%%%%%%%%%%%%%%%%%%%%%%%%%%%%%%%%%%%%%%%%%%%%%%%%%%%%%%%%%%%%%%%%%%%%%%%%%%%%%%
%\question{} 
%%%%%%%%%%%%%%%%%%%%%%%%%%%%%%%%%%%%%%%%%%%%%%%%%%%%%%%%%%%%%%%%%%%%%%%%%%%%%%%%
%Pour une amplitude maximale de $E_0$ et une durée de signal $\tau$, les
%signaux temporels deviennent :
%\[
%    e_1(t)=\dfrac{E_0}{\tau}(1-t)\left(u(t)-u(t-\tau)\right)
%\]
%\[
%    e_2(t)=\dfrac{E_0}{\tau}t\left(u(t)-u(t-\tau)\right)
%\]

%%%%%%%%%%%%%%%%%%%%%%%%%%%%%%%%%%%%%%%%%%%%%%%%%%%%%%%%%%%%%%%%%%%%%%%%%%%%%%%%
\question{}
%%%%%%%%%%%%%%%%%%%%%%%%%%%%%%%%%%%%%%%%%%%%%%%%%%%%%%%%%%%%%%%%%%%%%%%%%%%%%%%%
On sollicite le système par le signal $e(t)$ défini par :
\[
    e(t)=e_1(t)+e_2(t-1)
\]
La transformée de Laplace $E(p)$ est simplement donnée par la somme,
\[
    E(p)=E_1(p)+e^{-p} E_2(p)
        =\dfrac{(p-1)}{p^2}\left(1-e^{-p}\right)+
         \dfrac{e^{-p}}{p^2}\left(1-e^{-p}\right)
        =\dfrac{1-e^{-p}}{p^2}\left(e^{-p}+p-1\right)
\]
%%%%%%%%%%%%%%%%%%%%%%%%%%%%%%%%%%%%%%%%%%%%%%%%%%%%%%%%%%%%%%%%%%%%%%%%%%%%%%%%
%%%%%%%%%%%%%%%%%%%%%%%%%%%%%%%%%%%%%%%%%%%%%%%%%%%%%%%%%%%%%%%%%%%%%%%%%%%%%%%%
\exercice{L'impulsion de Dirac approchée}
%%%%%%%%%%%%%%%%%%%%%%%%%%%%%%%%%%%%%%%%%%%%%%%%%%%%%%%%%%%%%%%%%%%%%%%%%%%%%%%%
%%%%%%%%%%%%%%%%%%%%%%%%%%%%%%%%%%%%%%%%%%%%%%%%%%%%%%%%%%%%%%%%%%%%%%%%%%%%%%%%
%%%%%%%%%%%%%%%%%%%%%%%%%%%%%%%%%%%%%%%%%%%%%%%%%%%%%%%%%%%%%%%%%%%%%%%%%%%%%%%%
\question{}
%%%%%%%%%%%%%%%%%%%%%%%%%%%%%%%%%%%%%%%%%%%%%%%%%%%%%%%%%%%%%%%%%%%%%%%%%%%%%%%%
%-------------------------------------------------------------------------------
\begin{figure}[!h]
    \centering
    \tikzsetnextfilename{fonction_porte_decomposition-chap-slci-ext}
    \input{tikz/fonction_porte_decomposition-chap-slci.tex}
\end{figure}
%-------------------------------------------------------------------------------
La fonction $\delta_a(t)$ est simplement donnée par 
\[
    \color{col1}p_a(t)
    =\color{col4}\dfrac{1}{a}u(t)\color{col3}-\dfrac{1}{a}u(t-a)
    =\normalcolor\dfrac{1}{a}\left(u(t)-u(t-a)\right)
\] 
où $u(t)$ est la fonction échelon unité.

%%%%%%%%%%%%%%%%%%%%%%%%%%%%%%%%%%%%%%%%%%%%%%%%%%%%%%%%%%%%%%%%%%%%%%%%%%%%%%%%
\question{}
%%%%%%%%%%%%%%%%%%%%%%%%%%%%%%%%%%%%%%%%%%%%%%%%%%%%%%%%%%%%%%%%%%%%%%%%%%%%%%%%
On rappel la transformée de Laplace d'une fonction retardée :
\[
    \laplace{f(t-\tau)} = e^{-\tau p}F(p)
\]
La transformée de Laplace de $\delta_a(t)$ est donc donnée par : 
\[
    \Delta_a(p) = \laplace{\delta_a(t)}
                = \dfrac{1}{a}\left(\laplace{u(t)}-\laplace{u(t-a)}\right)
                = \dfrac{1}{ap}(1-e^{-ap})
\]
%%%%%%%%%%%%%%%%%%%%%%%%%%%%%%%%%%%%%%%%%%%%%%%%%%%%%%%%%%%%%%%%%%%%%%%%%%%%%%%%
\question{}
%%%%%%%%%%%%%%%%%%%%%%%%%%%%%%%%%%%%%%%%%%%%%%%%%%%%%%%%%%%%%%%%%%%%%%%%%%%%%%%%
Le développement limité d'ordre 1 de $e^{x}=1+x+o\left(x\right)$ permet de
déterminer la limite pour $a\rightarrow0$ de la transformée de Laplace de 
$\delta_a(t)$. En effet, 
\[
    \lim_{a\to0} \Delta_a(p) = \dfrac{1}{ap}(1-1+ap)=1.
\]
Puisque la transformée de Laplace d'une impulsion de Dirac est égale à 1, on a
bien vérifié que $\lim_{a\to0} \delta_a(p)=\delta(t)$ où $\delta(t)$ est 
l'impulsion de Dirac.

%%%%%%%%%%%%%%%%%%%%%%%%%%%%%%%%%%%%%%%%%%%%%%%%%%%%%%%%%%%%%%%%%%%%%%%%%%%%%%%%
%%%%%%%%%%%%%%%%%%%%%%%%%%%%%%%%%%%%%%%%%%%%%%%%%%%%%%%%%%%%%%%%%%%%%%%%%%%%%%%%
\exercice{Décharge d'un condensateur}
%%%%%%%%%%%%%%%%%%%%%%%%%%%%%%%%%%%%%%%%%%%%%%%%%%%%%%%%%%%%%%%%%%%%%%%%%%%%%%%%
%%%%%%%%%%%%%%%%%%%%%%%%%%%%%%%%%%%%%%%%%%%%%%%%%%%%%%%%%%%%%%%%%%%%%%%%%%%%%%%%

%%%%%%%%%%%%%%%%%%%%%%%%%%%%%%%%%%%%%%%%%%%%%%%%%%%%%%%%%%%%%%%%%%%%%%%%%%%%%%%%
\question{}
%%%%%%%%%%%%%%%%%%%%%%%%%%%%%%%%%%%%%%%%%%%%%%%%%%%%%%%%%%%%%%%%%%%%%%%%%%%%%%%%
\[
    \tau(pQ(p)-q_0)+Q(p)=0
\]
\[
    Q(p)(\tau p+1)=q_0\tau
\]
\[
    Q(p)=q_0\dfrac{\tau}{\tau p+1}
\]
%%%%%%%%%%%%%%%%%%%%%%%%%%%%%%%%%%%%%%%%%%%%%%%%%%%%%%%%%%%%%%%%%%%%%%%%%%%%%%%%
\question{}
%%%%%%%%%%%%%%%%%%%%%%%%%%%%%%%%%%%%%%%%%%%%%%%%%%%%%%%%%%%%%%%%%%%%%%%%%%%%%%%%
\[
    Q(p)=q_0\dfrac{1}{p+\dfrac{1}{\tau}}
\]
\[
    q(t)=q_0e^{-\dfrac{t}{\tau}}
\]

%-------------------------------------------------------------------------------
\begin{figure}[!h]
    \centering
    \tikzsetnextfilename{decharge_condensateur_solution-chap-slci-ext}
    \begin{tikzpicture}
    \begin{axis}
    [
        height=6cm,
        width=9cm,
        axis x line=center,
        axis y line=center,
        xmin=-1,
        xmax=6,
        ymin=-0.25,
        ymax=1.5,
        xlabel={$t$},
        ylabel={$q(t)$},
        xlabel style={below right},
        ylabel style={above left},
        yticklabels={0},
        ytick={0},
        y tick label style={anchor=east},
        xticklabels={0,1,2,3,4,5},
        xtick={0,1,2,3,4,5},
        x tick label style={below},
    ]
        \addplot [signalb,domain=-1:0]{1};
        \addplot [signalb,domain=0:5.9,samples=200]{exp(-x)};
        \node[above left] at (axis cs:0,1) {$q_0$};
    \end{axis}
\end{tikzpicture}

\end{figure}
%-------------------------------------------------------------------------------


