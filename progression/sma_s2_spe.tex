\documentclass[a4paper,11pt]{article}
\usepackage[utf8]{inputenc} 
\usepackage[T1]{fontenc}
\usepackage{lmodern}
\usepackage[french]{babel}
\usepackage{verbatim}
\usepackage{geometry}
\geometry{% siehe geometry.pdf (Figure 1)
          top=20mm,
          bottom=25mm,
          inner=25mm,
          outer=30mm,
          showframe=false, % For debugging: try true and see the layout frames
         }

\title{Systèmes Mécaniques et Automatiques 4}

\date{Septembre 2021}
\author{ESME Sudria Bordeaux-Lille-Lyon-Paris}
\begin{document}
\maketitle

\section*{Contenu général du cours}

\subsection*{Enseignement synchrone}

\noindent \textbf{12 séances de Cours :}
\begin{itemize}
    \item Asservissement                                 \hfill (2 sem.)
    \item Performances des systèmes asservis             \hfill (2 sem.)
    \item Etude de la stabilité des systèmes asservis    \hfill (3 sem.)
    \item Correction des systèmes asservis               \hfill (4 sem.)
    \item Résumé du module et Initiation à la représentation d'état  \hfill (1 sem.)
\end{itemize}
$\,$\newline
\noindent \textbf{12 séances de TD :}
\begin{itemize}
    \item TD 13 -- 
    \item TD 14 -- 
    \item TD 15 -- 
    \item TD 16 -- 
    \item TD 17 -- 
    \item TD 18 -- 
    \item TD 19 -- 
    \item TD 20 -- 
    \item TD 21 -- 
    \item TD 22 -- 
    \item TD 23 -- 
    \item TD 24 -- 
\end{itemize}


\subsection*{Enseignement asynchrone}

Deux devoirs (rapport TP) seront à rendre sur la plateforme Moodle pendant le semestre :
\begin{itemize}
    \item TP 1 --
    \item TP 2 -- 
\end{itemize}

Vous trouverez plus de détails sur les consignes dans la section concernée. 

\section*{Objectifs du cours}

À  l'issue  de  ce  cours  les  élèves  pourront  formaliser,  d'un  point  de  vue  mathématique,  
la dynamique d'un système complexe linéaire, continu et invariant. Cette formalisation se décline selon 
les points suivants :  

\begin{itemize}
    \item \'Ecriture  de  la  dynamique  de  chaque  sous-système  dans  le  formalisme  du  
          problème différentiel de Cauchy (système d'équations différentielles et conditions limites).
    \item Réduction du problème différentiel en termes de fonction de transfert : relation entrée/sortie 
          dans le domaine de Laplace.
    \item Assemblage des fonctions de transfert grâce à l'algèbre des schémas à bloc. 
    \item Détermination de la sortie du système, assemblé dans le domaine temporel, en fonction 
          de l'entrée (réponse dans le domaine de Laplace).
    \item Transformation de la solution dans le domaine temporel (réponse temporelle).
    \item Modélisation des systèmes modèles du 1er et second ordre 
    \item Représentation de la réponse fréquentielle des systèmes modèles
    \item Modélisation numérique de la réponse temporelle des systèmes modèles (SCILAB)
\end{itemize}

\section*{Programme détaillé}
\begin{enumerate}
    \item \textbf{Introduction à l'automatique}
        \begin{itemize}
            \item Définition d'un système
            \item Définition des différents types de signaux usuels
            \item Caractérisation d'un signal temporel (valeur finale, temps de réponse).
            \item Décomposition de signaux en signaux élémentaires 
        \end{itemize}
    \item \textbf{Systèmes linéaires continus et invariants (SLCI)}
        \begin{itemize}
            \item Différentes propriétés des systèmes. (Linéaire, continu, invariant et causal)
            \item Equations différentielles à coefficients constants.
            \item Résolution classique d'une équation diff.
            \item Définition et propriétés de la Transformée de Laplace
            \item Transformée des signaux usuels
            \item Résolution d'une équation différentielle par la transformée de Laplace.
            \item Définition et propriétés de la fonction de transfert.
            \item Représentation de la fonction de transfert (algébrique/graphique)
        \end{itemize}
    \item \textbf{Schémas fonctionnels (schéma bloc)}
        \begin{itemize} 
            \item Eléments de base d'un schéma bloc (bloc, flèche, comparateur, point de jonction)
            \item Transformation des schémas blocs (Réduction, Manipulation)
        \end{itemize}
    \item \textbf{Modélisation des SLCI}
        \begin{itemize}
            \item Système du premier ordre
            \item Système du deuxième ordre
        \end{itemize}
    \item \textbf{Analyse fréquentielle et représentation graphique de la réponse harmonique.}
        \begin{itemize}
            \item Définition de la réponse harmonique
            \item Représentation graphique
            \item Diagramme de Bode
            \item Diagramme de Nyquist
            \item Diagramme de Black-Nichols
        \end{itemize}
\end{enumerate}

% ===============================================================================
%  Progression du cours de SMA (IngeSpe) 
% ===============================================================================
%  Résumé du premier semestre :
%    12 séances de Cours :
%        - Introduction                                   (1 semaine )
%        - Systèmes Linéaires Continus et Invariants      (3 semaines)
%        - Schéma-fonctionnel                             (1 semaine )
%        - Modélisation des SLCI ( réponses temporelles ) (3 semaines)
%        - Réponses harmoniques                           (4 semaines) 
%    12 séances de TD/TP
%             
% ===============================================================================
%  Semaine 1 
% ===============================================================================
% -------------------------------------------------------------------------------
%  Cours 1  
% -------------------------------------------------------------------------------
%  Chapitre Avant-propos et 1 : Introduction à l'automatique.     
%            - Modalité de validation du module.
%            - Petit historique (Clepsydre, Machine de Watt)  
%            - Un exemple de la vie courante (temperature d'une douche :
%              régulation de la température pour introduire: 
%                - un bouton (une seule sortie)
%                - deux robinets (humain/manuel) 
%                - la régulation (thermostat/automatique)
%                - les performances. 
%            - Schéma bloc simple (entrée/système/sortie)
%            - Objectifs de l'année : 
%                - Modélisation    |
%                - Analyse         |  des systèmes dynamiques 
%                - Contrôle        | 
%            - Définition simple d'un système (dynamique) :
%                - frontiere (définition structurelle)
%                - modifie les flux de MEI (définition fonctionnelle)
%            - Définition signaux (Les systèmes seront traités au prochain cours)
%            - Différents types de signaux :
%                - Impulsion de Dirac
%                - Échelon
%                - Rampe
%                - Exponentielle décroissante
%                - Periodique (ex. Sinusoide )
%                - Pseudo-périodique (ex. Sinusoide amortie)
%            - Discuter des caractéristiques d'un signal 
%              (valeure finale, temps de réponse à 5\%, dépassement, 
%              oscillation, temps de montée ) qui seront traités dans 
%              le premier TD )
% -------------------------------------------------------------------------------
%  TD 1 
% -------------------------------------------------------------------------------
%            - Décomposition de signaux en signaux élémentaires 
%            - Détermination graphique de la valeure finale du temps de 
%              réponse à 5\%, du temps de montée, du dépassement.
% ===============================================================================
%  Semaine 2 
% ===============================================================================
% -------------------------------------------------------------------------------
%  Cours 2  
% -------------------------------------------------------------------------------
%  Chapitre 1 : Systèmes linéaires continus et invariants
%
%            - Différentes propriétés systèmes : 
%                - linéaire
%                - continue
%                - invariant
%                - causal   
%            - SLCI 
%            - Equations différentielles à coefficient constants
%            - Résolution classique  ( pour sentir la solution )
%            - Discuter un exemple du premier ordre :
%                - chute libre
%                - décharge d'un condensateur
% -------------------------------------------------------------------------------
%  TD 2  
% -------------------------------------------------------------------------------
%            - Résolution classique d'équations différentielles
%              à coefficient constants :
%                - 1er ordre / 2nd ordre 
%                    - sans second membre
%                    - avec second membre
% ===============================================================================
%  Semaine 3
% ===============================================================================
% -------------------------------------------------------------------------------
%  Cours 3 
% -------------------------------------------------------------------------------
%  Chapitre 1 : Systèmes linéaires continus et invariants
%
%            - Présentation de la méthodologie de résolution des équations 
%              différentielles. Introduction de la transformée de Laplace schématique (domaine temporelle/de Laplace).
%            - Transformée de Laplace.
%                - Définition
%            - Propriétés de la transformée de Laplace : 
%                - Linéarité 
%                - retard en t
%                - retard en p
%                - dérivation (exemple d'une petite équa. diff. au passage)
%                - intégration 
%                - Théorème de la valeur initiale
%                - Théorème de la valeur finale
%            - Transformée des signaux usuels :
%                - Dirac
%                - Echelon 
%                - Rampe 
%                - exponentielle décroissante 
%                - Cosinus/Sinus
%            - Distribution des tables de transformées        
%            - Résolution d'une équation différentielle
% -------------------------------------------------------------------------------
%  TD 3
% -------------------------------------------------------------------------------
%            - Transformée de Laplace :
%                - Résolution d'équations différentielles
%                - Valeure initiale/Finale
%                - Transformée de Laplace d'une fonction périodique  
% ===============================================================================
%  Semaine 4 
% ===============================================================================
% -------------------------------------------------------------------------------
%  Cours 4  
% -------------------------------------------------------------------------------
%  Chapitre 1 : Systèmes linéaires continus et invariants
%
%            - Fonctions de transfert :
%                - Définition
%                - Remarque par rapport au lien entre la FT et la réponse
%                  impulsionnelle
%                - Fraction rationnelle
%                - Pôles / Zéros ( + petit exemple)
%                - Représentation en bloc de la fonction de transfert 
%                  (la semaine prochaine nous présenterons l'algèbre de bloc)
%            - Représentation de la fonction de transfert algébrique/graphique :
%                - Forme canonique  ( + exemple )
%                - Forme factorisée ( + exemple )
%                - Carte des pôles et zéros d'une fonction de transfert. 
% -------------------------------------------------------------------------------
%  TD 4 
% -------------------------------------------------------------------------------
%            - Exercices sur les fonctions de transfert :
%                - Détermination d'une fonction de transfert simple
%                - Cartes des pôles
%                - Système du second ordre
%                - Fonction de transfert d'un système masse-ressort/Dipôle électrique
% ===============================================================================
%  Semaine 5 
% ===============================================================================
% -------------------------------------------------------------------------------
%  Cours 5 
% -------------------------------------------------------------------------------
%  Chapitre 2 : Schémas fonctionnels et graphe de fluence
%
%            - Eléments de base des schémas fonctionnels 
%                - Bloc
%                - Flèche
%                - Comparateur/Sommateur
%                - Point de jonction/prélèvement
%            - Transformation des schémas fonctionnels
%                - Réduction : 
%                    - en série
%                    - en parallèle
%                    - boucle de contre réaction unitaire
%                    - boucle de contre réaction
%                - Manipulation :
%                    - Déplacement d'un comparateur vers la gauche 
%                    - Déplacement d'un point de prélèvement vers la droite
%            - Cas des entrées multiples
%                - Entrée/Perturbation
%                - Principe de superposition
%            - Méthodologie générale pour la réduction de schéma-bloc de grande 
%              taille  
%            - Mentionner les graphes de fluence du poly (hors programme).
% -------------------------------------------------------------------------------
%  TD 5 
% -------------------------------------------------------------------------------
%            - Réduction de schéma bloc 
%            - Schéma-fonctionnels MCC  
% ===============================================================================
%  Semaine 6 
% ===============================================================================
% -------------------------------------------------------------------------------
%  Cours 6  
% -------------------------------------------------------------------------------
%  Chapitre 3 : Modélisation des SLCI et leurs réponses temporelles
%            - Système du premier ordre :
%                - Définition :
%                    - équation différentielle canonique
%                    - définition des paramètres (K,tau)
%                - Fonction de transfert 
%                - Pôle de la fonction de transfert 
%                - Réponses temporelles :
%                    - réponse impulsionnelle (pente à l'origine)
%                    - réponse indicielle (pente à l'origine, t5%, valeur finale)
%                    - réponse à une rampe
%                - Résumé des caractéristiques d'un système du premier ordre :
%                    - réponse à 5%
%                    - pente à l'origine 
%                - Exemple de système du 1er ordre :
%                    -  Circuit RC
% -------------------------------------------------------------------------------
%  TP 1
% -------------------------------------------------------------------------------
%            - Découverte de Scilab
%            - Modélisation d'un système du premier ordre  
%            - Proposer un exercice noté.
% ===============================================================================
%  Semaine 7 
% ===============================================================================
% -------------------------------------------------------------------------------
%  Cours 7  
% -------------------------------------------------------------------------------
%  Chapitre 3 : Modélisation des SLCI et leurs réponses temporelles
%
%            - Système du second ordre (partie 1):
%                - Définition 
%                - Fonction de transfert 
%                - Pôle de la fonction de transfert 
%                - Réponses temporelles :
%                    - impulsionnelle
%                    - indicielle 
% -------------------------------------------------------------------------------
%  TP 2 
% -------------------------------------------------------------------------------
%            - Modélisation d'un système du second ordre   
% ===============================================================================
%  Semaine 8 Semaine du DS
% ===============================================================================
% -------------------------------------------------------------------------------
%  Cours 8  
% -------------------------------------------------------------------------------
%  Chapitre 3 : Modélisation des SLCI et leurs réponses temporelles
% -------------------------------------------------------------------------------
%            - Système du second ordre (partie 2):
%            c.f Whiteboard cours 8
%
% -------------------------------------------------------------------------------
%  TD 6 
% -------------------------------------------------------------------------------
%            - Modélisation d'un système du second ordre 
%            - Lecture des abaques
% 
% ===============================================================================
%  Semaine 9  
% ===============================================================================
% -------------------------------------------------------------------------------
%  Cours 9  
% -------------------------------------------------------------------------------
%  Chapitre 4 : Analyse fréquentielle et représentation graphique 
%               de la réponse harmonique.
%            - Définition de la réponse harmonique / exemple de 
%              réponse harmonique
%            - Représentation graphique de la réponse harmonique 
%              (schématiquement) :
%                - Diagramme de Bode
%                - Diagramme de Black-Nichols
%                - Diagramme de Nyquist  
% -------------------------------------------------------------------------------
%  TD 6 
% -------------------------------------------------------------------------------
%            - Réponse harmonique dans le domaine temporelle 
%            - Lecture et tracé d'échelle logarithmique
%            - Diagramme de Bode d'une FT 1/1+p
% ===============================================================================
%  Semaine 10 
% ===============================================================================
% -------------------------------------------------------------------------------
%  Cours 10
% -------------------------------------------------------------------------------
%  Chapitre : Analyse fréquentielle et représentation graphique 
%             de la réponse harmonique.
%            - Méthodologie générale pour tracer un diagramme de Bode
%            - Diagramme de Bode des modèles usuels :
%                - gain pur
%                - intégrateur pur
%                - dérivateur pur 
%                - retard pur
%                - 1er ordre
%                - 2 systèmes du premier ordre
% -------------------------------------------------------------------------------
%  TD 7 
% -------------------------------------------------------------------------------
%            - 
% ===============================================================================
%  Semaine 11 
% ===============================================================================
% -------------------------------------------------------------------------------
%  Cours 11
% -------------------------------------------------------------------------------
%  Chapitre : Analyse fréquentielle et représentation graphique 
%             de la réponse harmonique.
%            - Diagramme de Bode des modèles usuels (suites) :
%                - 2nd ordre / phénomène de résonance 
%                - système d'ordre quelconques à traiter en TD
% -------------------------------------------------------------------------------
%  TD 8
% -------------------------------------------------------------------------------
% ===============================================================================
%  Semaine 12 QCM3
% ===============================================================================
% -------------------------------------------------------------------------------
%  Cours 12
% -------------------------------------------------------------------------------
%  Chapitre : Analyse fréquentielle et représentation graphique 
%             de la réponse harmonique.
%            - Diagramme de Black-Nichols des modèles usuels :
%                - 1er ordre
%                - 2nd ordre
%            - Diagramme de Nyquist des modèles usuels :
%                - 1er ordre
%                - 2nd ordre
% -------------------------------------------------------------------------------
%  TP 3 
% -------------------------------------------------------------------------------
%            - Un TP pour tracer les réponses harmoniques Bode/Nyquist
%            - Scilab / Python  
% ===============================================================================
%  Semaine 13 
% ===============================================================================
% -------------------------------------------------------------------------------
%  Cours 13
% -------------------------------------------------------------------------------
%    Résumé du semestre
% -------------------------------------------------------------------------------
%  TD 9
% -------------------------------------------------------------------------------
%            Diagramme de Nyquist 
%            - /home/filipe/enseignement/auto/2018_2019_SPE/td/td_6
%            - Exercices de révisions pour l'examen
% ===============================================================================
% -------------------------------------------------------------------------------
%  DS 
% -------------------------------------------------------------------------------
%            - Le DS pourra être composé de plusieurs exercices :
%                - Décomposition d'un signal en signaux usuels + Transformée de Laplace.
%                - Détermination d'une fonction de transfert à partir
%                  de différentes équations différentielles.
%                - Réduction d'un schéma-bloc.
%                - Tracer des réponses temporelles du premier ordre.
%                - Détermination des caractéristiques d'une réponse 
%                  du second ordre.
\end{document}
