\documentclass[a4paper,12pt]{scrartcl}

\usepackage[utf8]{inputenc}
\usepackage[T1]{fontenc}
\usepackage{lmodern}
\usepackage[french]{babel}

\usepackage[a4paper,headheight=14pt]{geometry}

\title{Transformée inverse de Laplace}
\date{}
%--------------------------------------------------------------------------------CUERPO-----------------------------------------%
\begin{document}

\maketitle

\section{Contexte}
\label{sec:contexte}

La transformée de Laplace est un outil indispensable 
pour l'étude des SLCI (résolution d'équations différentielles, caractéristation et analyse 
des systèmes dynamiques\ldots). Ceci est dû au fait que les opérations courantes 
de l'analyse (dérivation, intégration, retard\ldots) sont largement simplifiés dans
le domaine de Laplace. 
de simplifier un grand nombre de calcul. 
a de nombreux avantages 
pour le

La transformée de Laplace d'une fonction temporelle



\end{document}


