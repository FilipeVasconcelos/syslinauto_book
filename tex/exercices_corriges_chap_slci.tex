%%%%%%%%%%%%%%%%%%%%%%%%%%%%%%%%%%%%%%%%%%%%%%%%%%%%%%%%%%%%%%%%%%%%%%%%%%%%%%%%
%%%%%%%%%%%%%%%%%%%%%%%%%%%%%%%%%%%%%%%%%%%%%%%%%%%%%%%%%%%%%%%%%%%%%%%%%%%%%%%%
\exercice{L'impulsion de Dirac approchée}
%%%%%%%%%%%%%%%%%%%%%%%%%%%%%%%%%%%%%%%%%%%%%%%%%%%%%%%%%%%%%%%%%%%%%%%%%%%%%%%%
%%%%%%%%%%%%%%%%%%%%%%%%%%%%%%%%%%%%%%%%%%%%%%%%%%%%%%%%%%%%%%%%%%%%%%%%%%%%%%%%
%%%%%%%%%%%%%%%%%%%%%%%%%%%%%%%%%%%%%%%%%%%%%%%%%%%%%%%%%%%%%%%%%%%%%%%%%%%%%%%%
\question{}
%%%%%%%%%%%%%%%%%%%%%%%%%%%%%%%%%%%%%%%%%%%%%%%%%%%%%%%%%%%%%%%%%%%%%%%%%%%%%%%%
%-------------------------------------------------------------------------------
\begin{figure}[!h]
    \centering
    \tikzsetnextfilename{fonction_porte_decomposition-chap-slci-ext}
    \input{tikz/fonction_porte_decomposition-chap-slci.tex}
\end{figure}
%-------------------------------------------------------------------------------
La fonction $\delta_a(t)$ est simplement donnée par 
\[
    \color{col1}p_a(t)
    =\color{col4}\dfrac{1}{a}u(t)\color{col3}-\dfrac{1}{a}u(t-a)
    =\normalcolor\dfrac{1}{a}\left(u(t)-u(t-a)\right)
\] 
où $u(t)$ est la fonction échelon unité.

%%%%%%%%%%%%%%%%%%%%%%%%%%%%%%%%%%%%%%%%%%%%%%%%%%%%%%%%%%%%%%%%%%%%%%%%%%%%%%%%
\question{}
%%%%%%%%%%%%%%%%%%%%%%%%%%%%%%%%%%%%%%%%%%%%%%%%%%%%%%%%%%%%%%%%%%%%%%%%%%%%%%%%
On rappel la transformée de Laplace d'une fonction retardée :
\[
    \laplace{f(t-\tau)} = e^{-\tau p}F(p)
\]
La transformée de Laplace de $\delta_a(t)$ est donc donnée par : 
\[
    \Delta_a(p) = \laplace{\delta_a(t)}
                = \dfrac{1}{a}\left(\laplace{u(t)}-\laplace{u(t-a)}\right)
                = \dfrac{1}{ap}(1-e^{-ap})
\]
%%%%%%%%%%%%%%%%%%%%%%%%%%%%%%%%%%%%%%%%%%%%%%%%%%%%%%%%%%%%%%%%%%%%%%%%%%%%%%%%
\question{}
%%%%%%%%%%%%%%%%%%%%%%%%%%%%%%%%%%%%%%%%%%%%%%%%%%%%%%%%%%%%%%%%%%%%%%%%%%%%%%%%
Le développement limité d'ordre 1 de $e^{x}=1+x+o\left(x\right)$ permet de
déterminer la limite pour $a\rightarrow0$ de la transformée de Laplace de 
$\delta_a(t)$. En effet, 
\[
    \lim_{a\to0} \Delta_a(p) = \dfrac{1}{ap}(1-1+ap)=1.
\]
Puisque la transformée de Laplace d'une impulsion de Dirac est égale à 1, on a
bien vérifié que $\lim_{a\to0} \delta_a(p)=\delta(t)$ où $\delta(t)$ est 
l'impulsion de Dirac.

%%%%%%%%%%%%%%%%%%%%%%%%%%%%%%%%%%%%%%%%%%%%%%%%%%%%%%%%%%%%%%%%%%%%%%%%%%%%%%%%
\question{Soit la fonction $g_a(t)$ définie par le graphe ci-contre. 
          Décomposer $g_a(t)$ à l'aide de fonctions échelons et/ou rampes. Donner sa 
          transformée de Laplace.}
%%%%%%%%%%%%%%%%%%%%%%%%%%%%%%%%%%%%%%%%%%%%%%%%%%%%%%%%%%%%%%%%%%%%%%%%%%%%%%%%
%-------------------------------------------------------------------------------
\begin{marginfigure}
    \centering
    \tikzsetnextfilename{fonction_creneau-chap-slci-ext}
    \begin{tikzpicture}[baseline=0]
   \begin{axis}[
   height=4cm,
   width=5cm,
      axis x line=center,
        axis y line=center,
        xmin=-1,
        xmax=5,
        ymin=-1.5,
        ymax=2.0,
        xlabel={$t$},
        ylabel={$g_a(t)$},
        xlabel style={below right},
        ylabel style={left},
        yticklabels={$\dfrac{1}{a}$},
        ytick={1},
        y tick label style={left},
        xticklabels={$\tau+a$,$\tau+2a$},
        xtick={2,4},
        x tick label style={below},
        ]
        \addplot [very thick,col1,domain=-1:0, samples=50]{0.01};
        \addplot [very thick,col1,domain=0:2, samples=50]{0.5*x};
        \addplot [very thick,col1,domain=2:4, samples=50]{-0.5*x+2};
        \addplot [very thick,col1,domain=4:5, samples=50]{0.01};
        \end{axis}
\end{tikzpicture}

\end{marginfigure}
%-------------------------------------------------------------------------------

\[
    g_a(t)=\dfrac{1}{a(\tau+a)}r(t)-\dfrac{2}{a(\tau+a)}r(t-(\tau+a))+\dfrac{1}{a(\tau+a)}r(t-(\tau+2a))
\]

\[
    g_a(t)=\dfrac{t}{a(\tau+a)}u(t)
\]

\[
    -\dfrac{t}{a(\tau+a)}-\dfrac{1}{\tau+a}-\dfrac{1}{a}=\dfrac{t}{a(\tau+a)}+f_2(t)
\]

\[
    -\dfrac{2t-(\tau+2a)}{a(\tau+a)}=f_2(t)
\]

\[
    g_a(t)=\dfrac{t}{a(\tau+a)}u(t)-\dfrac{2t-(\tau+2a)}{a(\tau+a)}u(t-(\tau+a))
\]

\[
    f_3(t)=\dfrac{t-(\tau+2a)}{a(\tau+a)}
\]

\[
    g_a(t)=\dfrac{t}{a(\tau+a)}u(t)-\dfrac{2t-(\tau+2a)}{a(\tau+a)}u(t-(\tau+a))+\dfrac{t-(\tau+2a)}{a(\tau+a)}u(t-(\tau+2a))
\]

\[
    g_a(t)=\dfrac{1}{a(\tau+a)}\left(t-(2t-(\tau+2a))u(t-(\tau+a))+(t-(\tau+2a))u(t-(\tau+2a))\right)
\]

\footnotesize
\[
    \laplace{g_a(t)}=\dfrac{1}{a(\tau+a)}\left(   \dfrac{1}{p^2}-\dfrac{2e^{-(\tau+a)p}}{p^2}+\dfrac{(\tau+2a)e^{-(\tau+a)p}}{p}
                                                + \dfrac{e^{-(\tau+2a)p}}{p^2} - \dfrac{(\tau+2a)e^{-(\tau+2a)p}}{p} \right)
\]
\[
     \laplace{g_a(t)}=\dfrac{1}{a(\tau+a)}\left( \dfrac{1}{p^2}\left(1-2e^{-(\tau+a)p}+e^{-(\tau+2a)p}\right) 
                                           + \dfrac{(\tau+2a)}{p}\left(e^{-(\tau+a)p}-e^{-(\tau+2a)p} \right) \right)
\]


\[
    e^{-(\tau+a)p}=e^{-\tau p}e^{-ap}=e^{-\tau p}\left(1-ap+\dfrac{(ap)^2}{2}\right)
\]

\[
    e^{-(\tau+2a)p}=e^{-\tau p}e^{-2ap}=e^{-\tau p}\left(1-2ap+2(ap)^2\right)
\]

\[
    1-2e^{-(\tau+a)p}+e^{-(\tau+2a)p}=1-e^{-\tau p}\left(1-ap+\dfrac{(ap)^2}{2}\right)+e^{-\tau p}\left(1-2ap+2(ap)^2\right)
\]

\[
    1-2e^{-(\tau+a)p}+e^{-(\tau+2a)p}=1-ape^{-\tau p}+\dfrac{3}{2}(ap)^2e^{-\tau p}
\]

\[
    e^{-(\tau+a)p}=e^{-\tau p}e^{-ap}=e^{-\tau p}\left(1-ap\right)
\]

\[
    e^{-(\tau+2a)p}=e^{-\tau p}e^{-ap}=e^{-\tau p}\left(1-2ap\right)
\]

\[
    e^{-(\tau+a)p}-e^{-(\tau+2a)p}=ape^{-\tau p}
\]

\[
    \laplace{g_a(t)}=\dfrac{1}{a(\tau+a)}\left(\right)
\]


\[
    g_a(t)=\dfrac{t}{a(\tau+a)}\left(u(t)-u(t-(\tau+a))\right) - \left(\dfrac{t-(\tau+2a)}{a(\tau+a)}\right)\left(u(t-(\tau+a))-u(t-(\tau+2a))\right)
\]

\[
    \laplace{g_a(t)}=\dfrac{1}{a(\tau+a)p^2}\left(1-e^{-(\tau+a)p}\right) - \dfrac{1}{a(\tau+a)}\left(\dfrac{1}{p^2}-\dfrac{(\tau+2a)}{p}\right)\left(e^{-(\tau+a)p}-e^{-(\tau+2a)p}\right)
\]
