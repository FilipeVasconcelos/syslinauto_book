%%%%%%%%%%%%%%%%%%%%%%%%%%%%%%%%%%%%%%%%%%%%%%%%%%%%%%%%%%%%%%%%%%%%%%%%%%%%%%%%
%%%%%%%%%%%%%%%%%%%%%%%%%%%%%%%%%%%%%%%%%%%%%%%%%%%%%%%%%%%%%%%%%%%%%%%%%%%%%%%%
\exercice{Régulation de la température d'une enceinte~\moyen}
%%%%%%%%%%%%%%%%%%%%%%%%%%%%%%%%%%%%%%%%%%%%%%%%%%%%%%%%%%%%%%%%%%%%%%%%%%%%%%%%
%%%%%%%%%%%%%%%%%%%%%%%%%%%%%%%%%%%%%%%%%%%%%%%%%%%%%%%%%%%%%%%%%%%%%%%%%%%%%%%%
On considère un système de chauffage qui doit faire passer la température 
d'une enceinte de $\theta_a$  (température ambiante) à $\theta_e$ (température 
de l'enceinte). 
La puissance totale $P_1$ fournie à la resistance de chauffage 
est proportionnelle de gain $A=\SI{200}{\watt\per\volt}$ à la tension de 
commande $u_c$. 
La puissance $P_2$ permettant de chauffer l'enceinte est telle que 
\[
    P_2=MC\devi{(\theta_e-\theta_a)}{}
\]
où $M$ est la masse du matériau de l'enceinte et $C$ sa capacité calorifique.
La puissance perdue $P_3$ par l'enceinte est telle que :
\[
    P_3=\dfrac{\theta_e-\theta_a}{R}
\]
où $R$ est la résistance thermique de l'enceinte.
Le bilan de puissance s'exprime donc par la relation $P_2=P_1-P_3$.

%%%%%%%%%%%%%%%%%%%%%%%%%%%%%%%%%%%%%%%%%%%%%%%%%%%%%%%%%%%%%%%%%%%%%%%%%%%%%%%%
\question{Exprimer l'équation différentielle donnant la différence 
de température $\Delta\theta=\theta_e-\theta_a$ en fonction de $u_c$}
%%%%%%%%%%%%%%%%%%%%%%%%%%%%%%%%%%%%%%%%%%%%%%%%%%%%%%%%%%%%%%%%%%%%%%%%%%%%%%%%

%%%%%%%%%%%%%%%%%%%%%%%%%%%%%%%%%%%%%%%%%%%%%%%%%%%%%%%%%%%%%%%%%%%%%%%%%%%%%%%%
\question{Déterminer la fonction de transfert en boucle ouverte $H(p)$ et 
montrer qu'elle peut se mettre sous la forme}
%%%%%%%%%%%%%%%%%%%%%%%%%%%%%%%%%%%%%%%%%%%%%%%%%%%%%%%%%%%%%%%%%%%%%%%%%%%%%%%%
\[
    H(p)=\dfrac{\Delta\theta(p)}{U_c(p)}=\dfrac{K}{1+\tau p}
\]

On réalise une mesure de la réponse indicielle, pour une tension de commande 
$u_c$ de \SI{5}{\volt}, on atteint 95\% de la valeur max en 3 minutes 
(\SI{180}{\second}). La valeur max de la température de l'enceinte est de 
\SI{80}{\celsius} pour une température extérieure de 20°C.
%%%%%%%%%%%%%%%%%%%%%%%%%%%%%%%%%%%%%%%%%%%%%%%%%%%%%%%%%%%%%%%%%%%%%%%%%%%%%%%%
\question{À partir de cette mesure, déterminer les valeurs numériques des 
paramètres de la fonction de transfert $H(p)$ $K$ et $\tau$.}
%%%%%%%%%%%%%%%%%%%%%%%%%%%%%%%%%%%%%%%%%%%%%%%%%%%%%%%%%%%%%%%%%%%%%%%%%%%%%%%%

Dans le but d'asservir la température on boucle le système par un ensemble 
thermocouple-amplificateur tel que représenté par le schéma-bloc suivant:

%-------------------------------------------------------------------------------
\begin{center}
    \tikzsetnextfilename{sb_bbr-chap_asserv-ext}
    \input{tikz/sb_bbr-chap_asserv.tex}
\end{center}
%-------------------------------------------------------------------------------
avec $B(p)=\SI{0.05}{\volt\per{\celsius}}$. 
%%%%%%%%%%%%%%%%%%%%%%%%%%%%%%%%%%%%%%%%%%%%%%%%%%%%%%%%%%%%%%%%%%%%%%%%%%%%%%%%
\question{Déterminer la forme canonique de la fonction de transfert en boucle 
fermée $G(p)$ de cet asservissement.}
%%%%%%%%%%%%%%%%%%%%%%%%%%%%%%%%%%%%%%%%%%%%%%%%%%%%%%%%%%%%%%%%%%%%%%%%%%%%%%%%

%%%%%%%%%%%%%%%%%%%%%%%%%%%%%%%%%%%%%%%%%%%%%%%%%%%%%%%%%%%%%%%%%%%%%%%%%%%%%%%%
\question{Déterminer la tension de commande $u_c$ nécessaire pour avoir une 
température de l'enceinte $\theta_e$ de 80° et déterminer le 
temps de réponse à 5\% (ainsi que la nouvelle constante de temps du système).}
%%%%%%%%%%%%%%%%%%%%%%%%%%%%%%%%%%%%%%%%%%%%%%%%%%%%%%%%%%%%%%%%%%%%%%%%%%%%%%%%

%%%%%%%%%%%%%%%%%%%%%%%%%%%%%%%%%%%%%%%%%%%%%%%%%%%%%%%%%%%%%%%%%%%%%%%%%%%%%%%%
\question{Déterminer l'écart statique (c'est à dire $u_c(\infty)-u_r(\infty)$) 
dans ces conditions.}
%%%%%%%%%%%%%%%%%%%%%%%%%%%%%%%%%%%%%%%%%%%%%%%%%%%%%%%%%%%%%%%%%%%%%%%%%%%%%%%%

