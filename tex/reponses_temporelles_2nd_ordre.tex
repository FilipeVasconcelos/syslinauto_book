\begin{landscape}
    \newcommand{\mysize}{\footnotesize}
    \captionsetup{width=1.4\linewidth}
    \small
    \centering
    \captionof{table}{Réponses temporelles d'un système du 2nd ordre 
                      pour les différents régimes.}
    \setlength{\ltmp}{0.15\linewidth}
    \setlength{\ldtmp}{0.25\linewidth}
    \setlength{\lctmp}{0.235\linewidth}
    \ra{0.001}
    \begin{tabular}{@{}P{\ltmp}P{\ldtmp}P{\ldtmp}P{\ldtmp}@{}}%
    \toprule
    Réponse   &
    Régime apérodique        ($\xi>1$)  & 
    Régime critique ($\xi=1$) & 
    Régime pseudo-périodique ($0<\xi<1$) \\
    \midrule
    Réponse impulsionnelle & 
    \tikzsetnextfilename{2nd_rep_1_1_ext}
    \raisebox{-.5\height}{\resizebox{\lctmp}{!}{\begin{tikzpicture}
    \begin{axis}
    [   legend style={draw=none},
        axis line style = thick,
        xmin=0,
        xmax=12,
        ymin=0,
        ymax=0.3,
        xlabel={$t$},
        ylabel={$s(t)$},
        label style={font=\Large},
        grid=both,
        grid style={line width=.4pt, draw=black},
        major grid style={line width=.4pt,draw=black},
    ]
    \addplot[signalb,domain=0:12] {(1/(3.73-0.26))*exp(-x/3.73)-exp(-x/0.26)};
    \end{axis}
\end{tikzpicture}
}} 
    {\mysize 
    \[
    s(t)=\dfrac{1}{\tau_1-\tau_2}
         \left(e^{-\frac{t}{\tau_1}}-e^{-\frac{t}{\tau_2}}\right)
    \]
    } &
    \tikzsetnextfilename{2nd_rep_1_2_ext}
    \raisebox{-.5\height}{\resizebox{\lctmp}{!}{    \begin{tikzpicture}
        \begin{axis}[
        legend style={draw=none},
        axis line style = thick,
        xmin=0,
        xmax=10,
        ymin=0,
        ymax=0.4,
        xlabel={$t$},
        ylabel={$s(t)$},
        label style={font=\Large},
        ]
            \addplot [thick,color=blue,domain=0:11.5, samples=101,unbounded coords=jump]{x*exp(-x)};
        \end{axis}
    \end{tikzpicture}
}} 
    {\mysize 
        \[ 
    s(t)=\dfrac{t}{\tau^2}e^{-\frac{t}{\tau}}
        \]} &  
    \tikzsetnextfilename{2nd_rep_1_3_ext}
    \raisebox{-.5\height}{\resizebox{\lctmp}{!}{\tikzsetnextfilename{2nd_rep_1_3_ext}
\begin{tikzpicture}
    \begin{axis}
    [   legend style={draw=none},
        axis line style = thick,
        xmin=0,
        xmax=12,
        ymin=-0.6,
        ymax=1.2,
        xlabel={$t$},
        ylabel={$s(t)$},
        label style={font=\Large},
        grid=both,
        grid style={line width=.4pt, draw=black},
        major grid style={line width=.4pt,draw=black},
    ]
    \def\a{0.3}            
    \def\b{0.91}           
    \def\w{0.953939201417} 
    \addplot[signalb,domain=0:12] {(\w/\b)*exp(-\a*x)*sin(deg(x)*\w)};
    \end{axis}
\end{tikzpicture}
}} 
    {\mysize 
        \[
    s(t)=\dfrac{\omega_d}{1-\xi^2}e^{-\xi\omega_0 t}\sin{\omega_d t} 
        \]} \\
    \midrule
    Réponse indicielle &  
    \tikzsetnextfilename{2nd_rep_2_1_ext}
    \raisebox{-.5\height}{\resizebox{\lctmp}{!}{\tikzsetnextfilename{2nd_rep_2_1_ext}
\begin{tikzpicture}
    \def\tu{2.0}
    \def\td{1.0}
    \begin{axis}
    [   legend style={draw=none},
        axis line style = thick,
        xmin=0,
        xmax=10,
        ymin=0,
        ymax=1.2,
        xlabel={$t$},
        ylabel={$s(t)$},
        label style={font=\Large},
    ]
    \addplot[very thick,color=blue,domain=0:11.5, samples=101]
    {1+(1/(3.73-0.26))*(0.26*exp(-x/0.26)-3.73*exp(-x/3.73))};
    \end{axis}
\end{tikzpicture}
}} 
    {\mysize 
        \[
    s(t)=1+\dfrac{1}{\tau_1-\tau_2}\left(\tau_2e^{-\frac{t}{\tau_2}}
                    -\tau_1e^{-\frac{t}{\tau_1}}\right)
        \]} &  
    \tikzsetnextfilename{2nd_rep_2_2_ext}
    \raisebox{-.5\height}{\resizebox{\lctmp}{!}{\begin{tikzpicture}
    \begin{axis}
    [   legend style={draw=none},
        axis line style = thick,
        xmin=0,
        xmax=10,
        ymin=0,
        ymax=1.2,
        xlabel={$t$},
        ylabel={$s(t)$},
        label style={font=\Large},
        grid=both,
        grid style={line width=.4pt, draw=black},
        major grid style={line width=.4pt,draw=black},
    ]
    \addplot[signalb,domain=0:10]  {1-exp(-x)-x*exp(-x)};
    \end{axis}
\end{tikzpicture}
}} 
    {\mysize 
        \[
    s(t)=1-e^{-\frac{t}{\tau}}-\dfrac{t}{\tau}e^{-\frac{t}{\tau}}
        \]} &  
    \tikzsetnextfilename{2nd_rep_2_3_ext}
    \raisebox{-.5\height}{\resizebox{\lctmp}{!}{\tikzsetnextfilename{2nd_rep_2_3_ext}
\begin{tikzpicture}
\begin{axis}
[   
    legend style={draw=none},
    axis line style = thick,
    xmin=0,
    xmax=12,
    ymin=0,
    ymax=1.5,
    xlabel={$t$},
    ylabel={$s(t)$},
    label style={font=\Large},
    grid=both,
    grid style={line width=.4pt, draw=black},
    major grid style={line width=.4pt,draw=black},
]
\def\a{0.3}            
\def\b{0.91}           
\def\w{0.954} 
\def\p{1.266}  
\addplot[signalb,domain=0:12] {1-((1./\w)*exp(-\a*x)*sin(deg(x)*\w+deg(\p)))};
\end{axis}
\end{tikzpicture}
}} 
    {\mysize 
        \[
    s(t) = 1 - \dfrac{e^{-\xi\omega_0 t}}{\sqrt{1-\xi^2}}\sin{(\omega_d t+\phi)}
        \]}\\
    \bottomrule
    \end{tabular}
    \begin{minipage}{0.8\linewidth}
    \noindent
    \scriptsize
    \textbf{Paramètres : 
    (pour tous) $K=1$, $E_0=1$ 
    (apériodique) $\xi=2$, $\omega_0=1$ (c.-à-d. $\tau_1=3.73$ 
                                              et $\tau_2=0.26$) 
    (critique) $\xi=1$, $\omega_0=1$ (c.-à-d. $\tau=1$) 
    (pseudo-périodique) $\xi=0.3$ et $\omega_0=1$}
    \end{minipage}
\end{landscape}
\captionsetup{width=\linewidth}

