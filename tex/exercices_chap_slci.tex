\makeatletter
\newcount\my@repeat@count
\newcommand{\myrepeat}[2]{%
    \begingroup
    \my@repeat@count=\z@
    \@whilenum\my@repeat@count<#1\do{#2\advance\my@repeat@count\@ne}%
    \endgroup
}
\makeatother
\newcommand{\mystar}{{\color{col1}\fontfamily{lmr}\selectfont$\star$}}
\newcommand{\facile}{\marginpar{\myrepeat{1}{\mystar}}}
\newcommand{\moyen}{\marginpar{\myrepeat{2}{\mystar}}}
\newcommand{\difficile}{\marginpar{\myrepeat{3}{\mystar}}}
\newcommand{\tresdifficile}{\marginpar{\myrepeat{4}{\mystar}}}
%%%%%%%%%%%%%%%%%%%%%%%%%%%%%%%%%%%%%%%%%%%%%%%%%%%%%%%%%%%%%%%%%%%%%%%%%%%%%%%%
%%%%%%%%%%%%%%%%%%%%%%%%%%%%%%%%%%%%%%%%%%%%%%%%%%%%%%%%%%%%%%%%%%%%%%%%%%%%%%%%
\moyen\exercice{L'impulsion de Dirac approchée}
%%%%%%%%%%%%%%%%%%%%%%%%%%%%%%%%%%%%%%%%%%%%%%%%%%%%%%%%%%%%%%%%%%%%%%%%%%%%%%%%
%%%%%%%%%%%%%%%%%%%%%%%%%%%%%%%%%%%%%%%%%%%%%%%%%%%%%%%%%%%%%%%%%%%%%%%%%%%%%%%%
L'impulsion de Dirac peut être approchée par la fonction porte 
$\delta_a(t)$ définie par le graphe ci-contre.

%%%%%%%%%%%%%%%%%%%%%%%%%%%%%%%%%%%%%%%%%%%%%%%%%%%%%%%%%%%%%%%%%%%%%%%%%%%%%%%%
\question{Décomposer $\delta_a(t)$ à l'aide de fonctions échelons.}
%%%%%%%%%%%%%%%%%%%%%%%%%%%%%%%%%%%%%%%%%%%%%%%%%%%%%%%%%%%%%%%%%%%%%%%%%%%%%%%%
%-------------------------------------------------------------------------------
\begin{marginfigure}
    \centering
    \tikzsetnextfilename{fonction_porte-chap-slci-ext}
    \begin{tikzpicture}[baseline=0]
   \begin{axis}[
        height=4cm,
        width=5cm,
        axis x line=center,
        axis y line=center,
        xmin=-1,
        xmax=5,
        ymin=-0.5,
        ymax=2.0,
        xlabel={$t$},
        ylabel={$\delta_a(t)$},
        xlabel style={below right},
        ylabel style={left},
        yticklabels={$\dfrac{1}{a}$},
        ytick={1},
        y tick label style={anchor=east},
        xticklabels={$a$},
        xtick={2},
        x tick label style={anchor=north},
        ]
        \addplot [very thick,col1,const plot] coordinates 
        {(-1,0.01) (0,0.01) (0,1) (2,1)  (2,0.01) (5,0.01) };
        \end{axis}
\end{tikzpicture}

\end{marginfigure}
%-------------------------------------------------------------------------------

%%%%%%%%%%%%%%%%%%%%%%%%%%%%%%%%%%%%%%%%%%%%%%%%%%%%%%%%%%%%%%%%%%%%%%%%%%%%%%%%
\question{Donner sa transformée de Laplace en fonction du paramètre $a$.}
%%%%%%%%%%%%%%%%%%%%%%%%%%%%%%%%%%%%%%%%%%%%%%%%%%%%%%%%%%%%%%%%%%%%%%%%%%%%%%%%

%%%%%%%%%%%%%%%%%%%%%%%%%%%%%%%%%%%%%%%%%%%%%%%%%%%%%%%%%%%%%%%%%%%%%%%%%%%%%%%%
\question{En considérant la limite $a\rightarrow0$, montrer que la transformée
          de Laplace de $\delta_a(t)$ tend bien vers 1.}
%%%%%%%%%%%%%%%%%%%%%%%%%%%%%%%%%%%%%%%%%%%%%%%%%%%%%%%%%%%%%%%%%%%%%%%%%%%%%%%%

%%%%%%%%%%%%%%%%%%%%%%%%%%%%%%%%%%%%%%%%%%%%%%%%%%%%%%%%%%%%%%%%%%%%%%%%%%%%%%%%
\question{Soit la fonction $g_a(t)$ définie par le graphe ci-contre. 
          Décomposer $g_a(t)$ à l'aide de fonctions échelons et/ou rampes. Donner sa 
          transformée de Laplace.}
%%%%%%%%%%%%%%%%%%%%%%%%%%%%%%%%%%%%%%%%%%%%%%%%%%%%%%%%%%%%%%%%%%%%%%%%%%%%%%%%
%-------------------------------------------------------------------------------
\begin{marginfigure}
    \centering
    \tikzsetnextfilename{fonction_creneau-chap-slci-ext}
    \input{tikz/fonction_creneau-chap-slci.tex}
\end{marginfigure}
%-------------------------------------------------------------------------------
