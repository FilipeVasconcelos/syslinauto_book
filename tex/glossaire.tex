%%%%%%%%%%%%%%%%%%%%%%%%%%%%%%%%%%%%%%%%%%%%%%%%%%%%%%%%%%%%%%%%%%%%%%%%%%%%%%%%
%%%%%%%%%%%%%%%%%%%%%%%%%%%%%%%%%%%%%%%%%%%%%%%%%%%%%%%%%%%%%%%%%%%%%%%%%%%%%%%%
%%%%%%%%%%%%%%%%%%%%%%%%%%%%%%%%%%%%%%%%%%%%%%%%%%%%%%%%%%%%%%%%%%%%%%%%%%%%%%%%
%%%%%%%%%%%%%%%%%%%%%%%%%%%%%%%%%%%%%%%%%%%%%%%%%%%%%%%%%%%%%%%%%%%%%%%%%%%%%%%%
%glossaire
\newglossaryentry{tempsdereponse}%
{% 
  name={Temps de réponse},% 
  description={Le temps de réponse est le temps écoulé entre le moment de 
  la sollicitation et le moment de sa réponse.
  }
}
\newglossaryentry{signal}%
{% 
  name={Signal},% 
  description={Un signal est la grandeur qui porte l'information. Elle peut être
  de différente nature selon le domaine d'application.
  }
}
\newglossaryentry{fonctiondetranfert}%
{% 
  name={Fonction de Transfert},% 
  description={La fonction de transfert défini la relation
  entre l'entrée et la sortie d'un système linéaire.
  }
}
\newglossaryentry{regulation}%
{% 
  name={Régulation},% 
  description={La régulation est un cas particulier d'asservissement 
  consistant à garder une consigne constante en présence de perturbation.}
}
\newglossaryentry{asservissement}%
{% 
  name={Asservissement},% 
  description={L'asservissment consiste à contrôler un système dynamique
  pour que sa réponse temporelle suive une consigne variable au cours du 
  temps.
  }
}
\newglossaryentry{schemabloc}%
{% 
  name={Schéma-bloc},% 
  description={Un schéma-bloc ou un schéma fonctionnel permet de 
  représenter graphiquement des relations mathématiques entre des grandeurs
  dans les domaines temporel ou de Laplace.
  }
}
\newglossaryentry{precision}%
{% 
  name={Précision},% 
  description={La précision est une performance recherchée par les automaticiens
  pour le système qu'ils souhaitent asservir. 
  Un système automatique est précis si la différence (l'écart) entre la 
  consigne et la sortie est nulle. Il est possible de définir une précision 
  statique (en régime permanent) ou une précision dynamique (l'écart instantané)
  }
}
\newglossaryentry{rapidite}%
{% 
  name={Rapidité},% 
  description={La rapidité est une performance recherchée par les automaticiens 
  pour le système qu'ils souhaitent asservir. 
  La rapidité peut être quantifier par le temps de réponse 
  ou le temps de montée. Un système est d'autant plus rapide que ces temps sont
  brefs.
  }
}
\newglossaryentry{stabilite}%
{% 
  name={Stabilité},% 
  description={Deux définitions sont possibles : (1) 
  Un système est dit stable lorsque écarté de son état d'équilibre, 
  il tend à y revenir. Un système est dit stable si à une entrée bornée 
  le système produit une sortie bornée. On parle alors de stabilité \gls{ebsb}.
  }
}
\newglossaryentry{depassement}%
{% 
  name={Dépassement},% 
  description={Le dépassement (ou plutôt le premier dépassement) 
  est lié à valeur maximale prise par une réponse temporelle en régime
  pseudo-périodique.  
  }
}
\newglossaryentry{correcteur}%
{% 
  name={Correcteur},% 
  description={
  }
}
\newglossaryentry{tempsdemontee}%
{% 
  name={Temps de montée},% 
  description={Le temps de montée est la durée de l'interval de temps d'une 
  réponse temporelle passe d'une réponse 
  }
}
\newglossaryentry{controle}%
{% 
  name={Contrôle},% 
  description={Le contrôle permet d'englober les notions d'asservissement et
  de régulation à l'idée de contrôle automatique des systèmes. C'est le terme
  qui est plus largement utilisé dans la litérature moderne.  
  }
}
\newglossaryentry{systemelineaire}%
{% 
  name={Système Linéaire},% 
  description={Un système est dit linéaire si il respecte le principe
  de proportionnalité et le principe de superposition.
  }
}
\newglossaryentry{performances}%
{% 
  name={Performances},% 
  description={Les performances attendues par un système asservis ou régulé 
  sont généralement liées à sa stabilité, sa précision et sa rapidité.
  }
}
\newglossaryentry{pole}%
{% 
  name={Pôle},% 
  description={Un pôle est formellement la racine du dénominateur d'une
  fraction rationnelle irreductible. Dans le contexte de l'étude d'un système
  linéaire, c'est la racine au dénominateur de
  la fonction de transfert du système
  }
}
\newglossaryentry{repharmonique}%
{% 
  name={Réponse harmonique},% 
  description={La réponse harmonique est la réponse d'un système à une 
  sollicitation harmonique, ou autrement du à une sollicitation sinusoïdale.
  }
}
\newglossaryentry{reptemporelle}%
{% 
  name={Réponse temporelle},% 
  description={La réponse temporelle correspond à la réponse d'un système 
  dans le domaine temporel à n'importe qu'elle sollicitation.
  }
}
\newglossaryentry{regtransitoire}%
{% 
  name={Régime transitoire},% 
  description={Le régime transitoire est la contribution de la réponse 
  temporelle qui disparait lorsque le régime permanent persiste.
  }
}
\newglossaryentry{regpermanent}%
{% 
  name={Régime permanent},% 
  description={Le régime permanent est la contribution de la réponse temporelle 
  qui persiste lorsque le régime transitoire disparé.
  }
}
\newglossaryentry{regaperiodique}%
{% 
  name={Régime apériodique},% 
  description={Le régime apériodique est le régime pour lequel la réponse 
  temporelle d'un système linéaire ne présente pas d'oscillations. On parle
  également de réponse fortement amortie.
  }
}
\newglossaryentry{regacritique}%
{% 
  name={Régime critique},% 
  description={Le régime critique est le régime intermédiaire entre le régime
      apériodique et le régime pseudo-périodique. Il correspond à un cas limite.
  }
}
\newglossaryentry{regpseudoperiodique}%
{% 
  name={Régime pseudo-périodique},% 
  description={Le régime pseudo-périodique est le régime pour lequel la réponse
  temporelle d'un système linéaire présente des oscillations faiblement 
  amorties.
  }
}
\newglossaryentry{diagbode}%
{% 
  name={Diagramme de Bode},% 
  description={C'est un diagramme pour lequel la réponse harmonique est 
  représentée par deux graphes semi-logarithmique donnant d'une part le gain 
  en décibel en fonction de la pulsation et d'autre part le déphasage en degré
  en fonction de la pulsation.
  }
}
\newglossaryentry{diagnyquist}%
{% 
  name={Diagramme de Nyquist},% 
  description={
  }
}
\newglossaryentry{diagblack}%
{% 
  name={Diagramme de Blach-Nichols},% 
  description={
  }
}
\newglossaryentry{impulsionnelle}%
{% 
  name={Impulsionnelle},% 
  description={voir Réponse impulsionnelle
  }
}
\newglossaryentry{indicielle}%
{% 
  name={Indicielle},% 
  description={voir Réponse indicielle
  }
}
\newglossaryentry{repimpuls}%
{% 
  name={Réponse impulsionnelle},% 
  description={La réponse impulsionnelle est la réponse temporelle
  à une impulsion de Dirac.
  }
}
\newglossaryentry{repind}%
{% 
  name={Réponse indicielle},% 
  description={La réponse indicielle est la réponse temporelle
  à un signal échelon.
  }
}
%%%%%%%%%%%%%%%%%%%%%%%%%%%%%%%%%%%%%%%%%%%%%%%%%%%%%%%%%%%%%%%%%%%%%%%%%%%%%%%%
%%%%%%%%%%%%%%%%%%%%%%%%%%%%%%%%%%%%%%%%%%%%%%%%%%%%%%%%%%%%%%%%%%%%%%%%%%%%%%%%
%%%%%%%%%%%%%%%%%%%%%%%%%%%%%%%%%%%%%%%%%%%%%%%%%%%%%%%%%%%%%%%%%%%%%%%%%%%%%%%%
%%%%%%%%%%%%%%%%%%%%%%%%%%%%%%%%%%%%%%%%%%%%%%%%%%%%%%%%%%%%%%%%%%%%%%%%%%%%%%%%
%%%%%%%%%%%%%%%%%%%%%%%%%%%%%%%%%%%%%%%%%%%%%%%%%%%%%%%%%%%%%%%%%%%%%%%%%%%%%%%%
%%%%%%%%%%%%%%%%%%%%%%%%%%%%%%%%%%%%%%%%%%%%%%%%%%%%%%%%%%%%%%%%%%%%%%%%%%%%%%%%
%%%%%%%%%%%%%%%%%%%%%%%%%%%%%%%%%%%%%%%%%%%%%%%%%%%%%%%%%%%%%%%%%%%%%%%%%%%%%%%%
%glossaire.tex
