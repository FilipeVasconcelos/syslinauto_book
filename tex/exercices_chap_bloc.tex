%%%%%%%%%%%%%%%%%%%%%%%%%%%%%%%%%%%%%%%%%%%%%%%%%%%%%%%%%%%%%%%%%%%%%%%%%%%%%%%%
%%%%%%%%%%%%%%%%%%%%%%%%%%%%%%%%%%%%%%%%%%%%%%%%%%%%%%%%%%%%%%%%%%%%%%%%%%%%%%%%
\exercice{Réduction de schéma-bloc~\moyen}
%%%%%%%%%%%%%%%%%%%%%%%%%%%%%%%%%%%%%%%%%%%%%%%%%%%%%%%%%%%%%%%%%%%%%%%%%%%%%%%%
%%%%%%%%%%%%%%%%%%%%%%%%%%%%%%%%%%%%%%%%%%%%%%%%%%%%%%%%%%%%%%%%%%%%%%%%%%%%%%%%

On cherche à déterminer la relation entre la sortie $S$ et les deux entrées 
$E$ et $P$ du schéma-bloc suivants :
%-------------------------------------------------------------------------------
\begin{center}
    \tikzsetnextfilename{ex1-chap_bloc-ext}
    \input{tikz/ex1-chap_bloc.tex}
\end{center}
%-------------------------------------------------------------------------------
Pour celà nous allons appliquer la propriété de linéarité des \textsc{SLCI}. 
C'est à dire que nous allons déterminer les fonctions de transfert en prenant 
en compte respectivement chaque entrée en considérant l'autre nulle.
La sortie globale du système sera alors la somme des sorties indépendantes. \\

%%%%%%%%%%%%%%%%%%%%%%%%%%%%%%%%%%%%%%%%%%%%%%%%%%%%%%%%%%%%%%%%%%%%%%%%%%%%%%%%
\question{Tracer le schéma-bloc dans le cas où l'entrée $P$ est nulle. 
On notera $S_E(p)$ la sortie de ce schéma-bloc.}
%%%%%%%%%%%%%%%%%%%%%%%%%%%%%%%%%%%%%%%%%%%%%%%%%%%%%%%%%%%%%%%%%%%%%%%%%%%%%%%%

%%%%%%%%%%%%%%%%%%%%%%%%%%%%%%%%%%%%%%%%%%%%%%%%%%%%%%%%%%%%%%%%%%%%%%%%%%%%%%%%
\question{Déterminer la fonction de transfert $H_E(p)$ 
du schéma-bloc précédent.}
%%%%%%%%%%%%%%%%%%%%%%%%%%%%%%%%%%%%%%%%%%%%%%%%%%%%%%%%%%%%%%%%%%%%%%%%%%%%%%%%

%%%%%%%%%%%%%%%%%%%%%%%%%%%%%%%%%%%%%%%%%%%%%%%%%%%%%%%%%%%%%%%%%%%%%%%%%%%%%%%%
\question{Tracer le schéma-bloc dans le cas où l'entrée $E$ est nulle. 
On notera $S_P(p)$ la sortie de ce schéma-bloc.}
%%%%%%%%%%%%%%%%%%%%%%%%%%%%%%%%%%%%%%%%%%%%%%%%%%%%%%%%%%%%%%%%%%%%%%%%%%%%%%%%

%%%%%%%%%%%%%%%%%%%%%%%%%%%%%%%%%%%%%%%%%%%%%%%%%%%%%%%%%%%%%%%%%%%%%%%%%%%%%%%%
\question{Déterminer la fonction de transfert $H_P(p)$ 
du schéma-bloc précédent.}
%%%%%%%%%%%%%%%%%%%%%%%%%%%%%%%%%%%%%%%%%%%%%%%%%%%%%%%%%%%%%%%%%%%%%%%%%%%%%%%%

%%%%%%%%%%%%%%%%%%%%%%%%%%%%%%%%%%%%%%%%%%%%%%%%%%%%%%%%%%%%%%%%%%%%%%%%%%%%%%%%
\question{Déterminer la sortie globale $S(p)=S_E(p)+S_P(p)$ en fonction 
des entrées $E$, $P$ et des blocs du système.}
%%%%%%%%%%%%%%%%%%%%%%%%%%%%%%%%%%%%%%%%%%%%%%%%%%%%%%%%%%%%%%%%%%%%%%%%%%%%%%%%

\clearpage
%%%%%%%%%%%%%%%%%%%%%%%%%%%%%%%%%%%%%%%%%%%%%%%%%%%%%%%%%%%%%%%%%%%%%%%%%%%%%%%%
%%%%%%%%%%%%%%%%%%%%%%%%%%%%%%%%%%%%%%%%%%%%%%%%%%%%%%%%%%%%%%%%%%%%%%%%%%%%%%%%
\exercice{Réduction de schémas fonctionnels~\difficile}
%%%%%%%%%%%%%%%%%%%%%%%%%%%%%%%%%%%%%%%%%%%%%%%%%%%%%%%%%%%%%%%%%%%%%%%%%%%%%%%%
%%%%%%%%%%%%%%%%%%%%%%%%%%%%%%%%%%%%%%%%%%%%%%%%%%%%%%%%%%%%%%%%%%%%%%%%%%%%%%%%
%%%%%%%%%%%%%%%%%%%%%%%%%%%%%%%%%%%%%%%%%%%%%%%%%%%%%%%%%%%%%%%%%%%%%%%%%%%%%%%%
\question{Donner la fonction de transfert globale des schémas-blocs suivants:}
%%%%%%%%%%%%%%%%%%%%%%%%%%%%%%%%%%%%%%%%%%%%%%%%%%%%%%%%%%%%%%%%%%%%%%%%%%%%%%%%
%%%%%%%%%%%%%%%%%%%%%%%%%%%%%%%%%%%%%%%%%%%%%%%%%%%%%%%%%%%%%%%%%%%%%%%%%%%%%%%%

\textbf{(a)}
%%%%%%%%%%%%%%%%%%%%%%%%%%%%%%%%%%%%%%%%%%%%%%%%%%%%%%%%%%%%%%%%%%%%%%%%%%%%%%%%
%-------------------------------------------------------------------------------
\begin{center}
    \tikzsetnextfilename{sb_exo2_1-chap-bloc-ext}
    \input{tikz/sb_exo2_1-chap_bloc.tex}
\end{center}
%-------------------------------------------------------------------------------
%%%%%%%%%%%%%%%%%%%%%%%%%%%%%%%%%%%%%%%%%%%%%%%%%%%%%%%%%%%%%%%%%%%%%%%%%%%%%%%%
\paragraph{(b)}
%%%%%%%%%%%%%%%%%%%%%%%%%%%%%%%%%%%%%%%%%%%%%%%%%%%%%%%%%%%%%%%%%%%%%%%%%%%%%%%%
%-------------------------------------------------------------------------------
\begin{center}
    \tikzsetnextfilename{sb_exo2_2-chap-bloc-ext}
    \begin{tikzpicture}
    \sbEntree{E}
    \sbComp{comp1}{E}
    \sbRelier[$E$]{E}{comp1}
    \sbBloc{h1}{$H_1$}{comp1}
    \sbRelier{comp1}{h1}
    \sbCompSum{comp2}{h1}{-}{}{+}{} 
    \sbRelier{h1}{comp2}
    \sbBloc{h2}{$H_2$}{comp2}
    \sbRelier{comp2}{h2}
    \sbBloc{h3}{$H_3$}{h2}
    \sbRelier{h2}{h3}
    \sbSortie[5]{S}{h3}
    \sbRelier{h3}{S}
    \sbNomLien[0.8]{S}{$S$}
    \sbDecaleNoeudy[-4]{h3}{u1}
    \sbBlocr{h4}{$H_4$}{u1}
    \sbRelierxy{h4}{comp2}
    \sbRelieryx{h3-S}{h4}
    \sbDecaleNoeudy[4]{h2}{u2}
    \sbBlocr[3]{h5}{$H_5$}{u2}
    \sbRelierxy{h5}{comp1}
    \sbRelieryx{h2-h3}{h5}
\end{tikzpicture}

\end{center}
%-------------------------------------------------------------------------------
%%%%%%%%%%%%%%%%%%%%%%%%%%%%%%%%%%%%%%%%%%%%%%%%%%%%%%%%%%%%%%%%%%%%%%%%%%%%%%%%
\paragraph{(c)}
%%%%%%%%%%%%%%%%%%%%%%%%%%%%%%%%%%%%%%%%%%%%%%%%%%%%%%%%%%%%%%%%%%%%%%%%%%%%%%%%
%-------------------------------------------------------------------------------
\begin{center}
    \tikzsetnextfilename{sb_exo2_3-chap-bloc-ext}
    \input{tikz/sb_exo2_3-chap_bloc.tex}
\end{center}
%-------------------------------------------------------------------------------
%%%%%%%%%%%%%%%%%%%%%%%%%%%%%%%%%%%%%%%%%%%%%%%%%%%%%%%%%%%%%%%%%%%%%%%%%%%%%%%%
\paragraph{(d)}
%%%%%%%%%%%%%%%%%%%%%%%%%%%%%%%%%%%%%%%%%%%%%%%%%%%%%%%%%%%%%%%%%%%%%%%%%%%%%%%%
%-------------------------------------------------------------------------------
\begin{center}
    \tikzsetnextfilename{sb_exo2_4-chap-bloc-ext}
    \begin{tikzpicture}
    \sbEntree{E}
    \sbComp{a}{E}
    \sbRelier[$E$]{E}{a}
    \sbBloc{b}{$H_1$}{a}
    \sbRelier{a}{b}
    \sbComp{c}{b}
    \sbRelier{b}{c}
    \sbBlocL{d}{$H_2$}{c}
    \sbComph{e}{d}
    \sbRelier{d}{e}
    \sbBlocL{f}{$H_3$}{e}
    \sbBlocL{g}{$H_4$}{f}
    \sbSortie[5]{S}{g}
    \sbRelier{g}{S}
    \sbDecaleNoeudy[-4]{g}{u}
    \sbBlocr{g4}{$G_4$}{u}
    \sbRelierxy{g4}{e}
    \sbRelieryx{g-S}{g4}
    \sbNomLien[0.8]{S}{$S$}
    \sbDecaleNoeudy{e}{v}
    \sbBlocr[2.5]{g3}{$G_3$}{v}
    \sbRelieryx{f-g}{g3}
    \sbRelierxy{g3}{c}
    \sbDecaleNoeudy[9]{e}{w}
    \sbBlocr[2.5]{g2}{$G_2$}{w}
    \sbRelieryx{f-g}{g2}
    \sbCompSum[-5]{h}{g2}{}{-}{}{+}
    \sbRelier{g2}{h}
    \sbRelierxy{h}{a}
    \sbDecaleNoeudy[13]{e}{x}
    \sbBlocr[2.5]{g1}{$G_1$}{x}
    \sbRelierxy{g1}{h}
    \sbRelieryx{g-S}{g1}
\end{tikzpicture}

\end{center}
%-------------------------------------------------------------------------------
\clearpage
%%%%%%%%%%%%%%%%%%%%%%%%%%%%%%%%%%%%%%%%%%%%%%%%%%%%%%%%%%%%%%%%%%%%%%%%%%%%%%%%
%%%%%%%%%%%%%%%%%%%%%%%%%%%%%%%%%%%%%%%%%%%%%%%%%%%%%%%%%%%%%%%%%%%%%%%%%%%%%%%%
\exercice{Graphe de fluence~\difficile}
%%%%%%%%%%%%%%%%%%%%%%%%%%%%%%%%%%%%%%%%%%%%%%%%%%%%%%%%%%%%%%%%%%%%%%%%%%%%%%%%
%%%%%%%%%%%%%%%%%%%%%%%%%%%%%%%%%%%%%%%%%%%%%%%%%%%%%%%%%%%%%%%%%%%%%%%%%%%%%%%%
Soit le circuit électrique suivant:
%-------------------------------------------------------------------------------
\begin{center}
    \tikzsetnextfilename{gf_exo3_1-chap-bloc-ext}
    %  (a) -- (ra) -- R1 -- (ca) -- (b) 
%   |                    |      (rb)
%   |                    C       R2
%   |                    |       |
%  (d) ---------------- (cb) -- (c)
\begin{tikzpicture}
    \coordinate (a) at (0,0);
    \coordinate (b) at (6,0);
    \coordinate (c) at (6,-2.2);
    \coordinate (d) at (0,-2.2);
    \coordinate (ra) at (2,0);
    \coordinate (rb) at (6,-1.2);
    \coordinate (ca) at (4,0);
    \coordinate (cb) at (4,-2.2);
    \node[thick,rectangle,
          draw,
          minimum width=1cm,
          minimum height=0.5cm] (R1) at (ra) {};
    \node[] (nR1) at (R1) {$R_1$};
    \node[thick,rectangle,
          draw,
          minimum width=1cm,
          minimum height=0.5cm,rotate=90]   (R2) at (rb) {};
    \node[] (nR2) at (R2) {$R_2$};
    \draw[thick] (d) circle(1.5pt);
    \draw[thick] (a) circle(1.5pt);
    \draw[thick] (a) -- (R1);
    \draw[thick] (R1) -- (b);
    \draw[thick] (b) -- (R2);
    \draw[thick] (c) -- (R2);
    \draw[thick] (d) -- (c);
    \draw[thick] (ca) -- ++ ($(0,-0.9)$) -- node[left,xshift=-1em] {$C$} 
                         ++ ($(-0.5,0)$) -- ++ ($(1,0)$) ;
    \draw[thick] (cb) -- ++ ($(0,0.9)$) -- ++ ($(-0.5,0)$) -- ++ ($(1,0)$) ;
    \draw[dblarw={black}{1.5pt}{1pt}] ($(d)+(0,0.1)$) -- node[left] {$v_1$} 
                            ($(a)+(0,-0.1)$);
    \draw[dblarw={black}{1.5pt}{1pt}] ($(c)+(0.5,0.1)$) -- node[right] {$v_2$} 
                            ($(b)+(0.5,-0.1)$);
    \draw[dblarw={black}{1.5pt}{1pt}] ($(a)+(0.1,0)$) --node[above] {$i_1$} 
                         ++ ($(0.5,0)$);
    \draw[dblarw={black}{1.5pt}{1pt}] ($(ca)+(0,-0.1)$) --node[left] {$i_2$} 
                         ++ ($(0,-0.5)$);
    \draw[dblarw={black}{1.5pt}{1pt}] ($(b)+(0,-0.1)$) --node[left] {$i_3$} 
                         ++ ($(0,-0.5)$);
\end{tikzpicture}

\end{center}
%-------------------------------------------------------------------------------
La tension $v_1(t)$ est imposée en entrée et on mesure la tension de sortie 
$v_2(t)$ aux bornes de la résistance $R_2$.
En appliquant les lois de Kirchoff, on obtient les relations suivantes 
entre les tensions et les courants:
%-------------------------------------------------------------------------------
\begin{align*}
    v_{R_1}(t)+v_2(t)-v_1(t)&=0\\
    i_1(t)-i_2(t)-i_3(t)&=0
\end{align*}
%-------------------------------------------------------------------------------
où  $v_{R_1}(t)$ est la tension aux bornes de la résistance $R_1$.

Aux bornes des dipoles, nous avons également :
%-------------------------------------------------------------------------------
\begin{align*}
    v_{R_1}(t)&=R_1i_1(t)\\
    v_{C}(t)&=\dfrac{1}{C}\int_0^t i_2(\tau) \dd{\tau}=v_2(t)\\
    v_2(t)&=R_2i_3(t)
\end{align*}
%-------------------------------------------------------------------------------
En remplaçant celles-ci dans les relations de Kirchoff, on obtient les 
équations régissant ce modèle électrique:
%-------------------------------------------------------------------------------
\begin{align*}
    i_1(t)&=\dfrac{1}{R_1}\big(v_1(t)-v_2(t)\big)\\
    i_2(t)&=i_1(t)-\dfrac{1}{R_2}v_2(t)\\
    v_2(t)&=\dfrac{1}{C}\int_0^t i_2(\tau) \dd{\tau}
\end{align*}
%-------------------------------------------------------------------------------
%%%%%%%%%%%%%%%%%%%%%%%%%%%%%%%%%%%%%%%%%%%%%%%%%%%%%%%%%%%%%%%%%%%%%%%%%%%%%%%%
\question{Donner les relations du modèle dans le domaine de Laplace}
%%%%%%%%%%%%%%%%%%%%%%%%%%%%%%%%%%%%%%%%%%%%%%%%%%%%%%%%%%%%%%%%%%%%%%%%%%%%%%%%
%%%%%%%%%%%%%%%%%%%%%%%%%%%%%%%%%%%%%%%%%%%%%%%%%%%%%%%%%%%%%%%%%%%%%%%%%%%%%%%%
\question{Tracer le graphe de fluence de ce modèle.}
%%%%%%%%%%%%%%%%%%%%%%%%%%%%%%%%%%%%%%%%%%%%%%%%%%%%%%%%%%%%%%%%%%%%%%%%%%%%%%%%
%%%%%%%%%%%%%%%%%%%%%%%%%%%%%%%%%%%%%%%%%%%%%%%%%%%%%%%%%%%%%%%%%%%%%%%%%%%%%%%%
\question{À partir de ce graphe et appliquant la règle de Mason, déterminer la
fonction de transfert global de ce système.}
\clearpage
%%%%%%%%%%%%%%%%%%%%%%%%%%%%%%%%%%%%%%%%%%%%%%%%%%%%%%%%%%%%%%%%%%%%%%%%%%%%%%%%
%%%%%%%%%%%%%%%%%%%%%%%%%%%%%%%%%%%%%%%%%%%%%%%%%%%%%%%%%%%%%%%%%%%%%%%%%%%%%%%%
%%%%%%%%%%%%%%%%%%%%%%%%%%%%%%%%%%%%%%%%%%%%%%%%%%%%%%%%%%%%%%%%%%%%%%%%%%%%%%%%
\exercice{Schéma-blocs dans le domaine temporel~\moyen}
%%%%%%%%%%%%%%%%%%%%%%%%%%%%%%%%%%%%%%%%%%%%%%%%%%%%%%%%%%%%%%%%%%%%%%%%%%%%%%%%
%%%%%%%%%%%%%%%%%%%%%%%%%%%%%%%%%%%%%%%%%%%%%%%%%%%%%%%%%%%%%%%%%%%%%%%%%%%%%%%%
%%%%%%%%%%%%%%%%%%%%%%%%%%%%%%%%%%%%%%%%%%%%%%%%%%%%%%%%%%%%%%%%%%%%%%%%%%%%%%%%
%%%%%%%%%%%%%%%%%%%%%%%%%%%%%%%%%%%%%%%%%%%%%%%%%%%%%%%%%%%%%%%%%%%%%%%%%%%%%%%%
\question{Tracer le schéma-bloc dans le domaine temporel de l'équation 
différentielle suivante:}
\[
    \dot{s}(t)=a\big(e(t)+s(t)\big)
\]
%%%%%%%%%%%%%%%%%%%%%%%%%%%%%%%%%%%%%%%%%%%%%%%%%%%%%%%%%%%%%%%%%%%%%%%%%%%%%%%%
%%%%%%%%%%%%%%%%%%%%%%%%%%%%%%%%%%%%%%%%%%%%%%%%%%%%%%%%%%%%%%%%%%%%%%%%%%%%%%%%
\question{Réduiser le schéma-bloc sous une forme faisant apparaitre l'opérateur
intégral}
%%%%%%%%%%%%%%%%%%%%%%%%%%%%%%%%%%%%%%%%%%%%%%%%%%%%%%%%%%%%%%%%%%%%%%%%%%%%%%%%
%%%%%%%%%%%%%%%%%%%%%%%%%%%%%%%%%%%%%%%%%%%%%%%%%%%%%%%%%%%%%%%%%%%%%%%%%%%%%%%%
\question{Donner le rapport de l'entrée sur la sortie}
%%%%%%%%%%%%%%%%%%%%%%%%%%%%%%%%%%%%%%%%%%%%%%%%%%%%%%%%%%%%%%%%%%%%%%%%%%%%%%%%
%%%%%%%%%%%%%%%%%%%%%%%%%%%%%%%%%%%%%%%%%%%%%%%%%%%%%%%%%%%%%%%%%%%%%%%%%%%%%%%%
%%%%%%%%%%%%%%%%%%%%%%%%%%%%%%%%%%%%%%%%%%%%%%%%%%%%%%%%%%%%%%%%%%%%%%%%%%%%%%%%
\exercice{Schéma-blocs dans le domaine temporel~\moyen}
%%%%%%%%%%%%%%%%%%%%%%%%%%%%%%%%%%%%%%%%%%%%%%%%%%%%%%%%%%%%%%%%%%%%%%%%%%%%%%%%
%%%%%%%%%%%%%%%%%%%%%%%%%%%%%%%%%%%%%%%%%%%%%%%%%%%%%%%%%%%%%%%%%%%%%%%%%%%%%%%%
%%%%%%%%%%%%%%%%%%%%%%%%%%%%%%%%%%%%%%%%%%%%%%%%%%%%%%%%%%%%%%%%%%%%%%%%%%%%%%%%
\question{Tracer le schéma-bloc dans le domaine temporel de l'équation 
différentielle suivante}
\[
    \dot{s}(t)=\dot{e}(t)+as(t)
\]
%%%%%%%%%%%%%%%%%%%%%%%%%%%%%%%%%%%%%%%%%%%%%%%%%%%%%%%%%%%%%%%%%%%%%%%%%%%%%%%%
%%%%%%%%%%%%%%%%%%%%%%%%%%%%%%%%%%%%%%%%%%%%%%%%%%%%%%%%%%%%%%%%%%%%%%%%%%%%%%%%
\question{Réduiser le schéma-bloc sous une forme faisant apparaitre 
l'opérateur intégral}
%%%%%%%%%%%%%%%%%%%%%%%%%%%%%%%%%%%%%%%%%%%%%%%%%%%%%%%%%%%%%%%%%%%%%%%%%%%%%%%%
%%%%%%%%%%%%%%%%%%%%%%%%%%%%%%%%%%%%%%%%%%%%%%%%%%%%%%%%%%%%%%%%%%%%%%%%%%%%%%%%
\question{Donner le rapport de l'entrée sur la sortie}
%%%%%%%%%%%%%%%%%%%%%%%%%%%%%%%%%%%%%%%%%%%%%%%%%%%%%%%%%%%%%%%%%%%%%%%%%%%%%%%%
