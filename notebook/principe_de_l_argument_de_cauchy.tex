\section{Principe de l'argument de
Cauchy}\label{principe-de-largument-de-cauchy}}

Ce notebook permet de tracer l'image d'un contour du plan complexe par
une fonction de transfert quelconque. Le but est de vérifier \textbf{le
principe de l'argument de Cauchy} en partant de cas simples. \textbf{Ce
principe est à la base du critère de Nyquist qui permet d'étudier la
stabilité des systèmes asservis.}

Le module qui s'occupe explicitement de calculer et tracer les contours
d'origine et image se trouve dans le fichier \texttt{cauchy\_main.py}.
Ici nous allons simplement importer les fonctions les plus importantes
et apprendre à les manipuler.

L'importation du module se fait par l'instruction suivante:

    \begin{tcolorbox}[breakable, size=fbox, boxrule=1pt, pad at break*=1mm,colback=cellbackground, colframe=cellborder]
\prompt{In}{incolor}{1}{\boxspacing}
\begin{Verbatim}[commandchars=\\\{\}]
\PY{k+kn}{from} \PY{n+nn}{ftransfert} \PY{k}{import} \PY{n}{Ftransfert}
\end{Verbatim}
\end{tcolorbox}

    La classe \texttt{Ftransfert} permet de définir une fonction de
transfert. Celle-ci sera simplement définie par son gain \(k\), ses
pôles \(p_i\) et ses zéros \(z_i\).

\[
F(p)=k\dfrac{(p-z_1)(p-z_2)(p-z_3)\ldots}{(p-p_1)(p-p_2)(p-p_3)\ldots}
\]

La définition d'une fonction de transfert se fait par l'instuction
suivante:

\begin{Shaded}
\begin{Highlighting}[]
\NormalTok{zeros}\OperatorTok{=}\NormalTok{[(}\DecValTok{1}\NormalTok{,}\DecValTok{0}\NormalTok{)]  }
\NormalTok{poles}\OperatorTok{=}\NormalTok{[(}\OperatorTok{-}\DecValTok{1}\NormalTok{,}\DecValTok{0}\NormalTok{),(}\OperatorTok{-}\DecValTok{2}\NormalTok{,}\DecValTok{0}\NormalTok{)]}
\NormalTok{gain}\OperatorTok{=}\FloatTok{0.25}
\NormalTok{F}\OperatorTok{=}\NormalTok{Ftransfert(zeros}\OperatorTok{=}\NormalTok{zeros,poles}\OperatorTok{=}\NormalTok{poles,gain}\OperatorTok{=}\NormalTok{gain)}
\end{Highlighting}
\end{Shaded}

où \texttt{zeros} et \texttt{poles} sont des listes de nombre complexe.
Un segment de points est une liste de nombres complexes modélisés par un
tuple de deux éléments \texttt{(réel,imaginaire)}.

    \begin{tcolorbox}[breakable, size=fbox, boxrule=1pt, pad at break*=1mm,colback=cellbackground, colframe=cellborder]
\prompt{In}{incolor}{2}{\boxspacing}
\begin{Verbatim}[commandchars=\\\{\}]
\PY{n}{zeros}\PY{o}{=}\PY{p}{[}\PY{p}{]}  
\PY{n}{poles}\PY{o}{=}\PY{p}{[}\PY{p}{(}\PY{o}{\PYZhy{}}\PY{l+m+mi}{1}\PY{p}{,}\PY{l+m+mi}{0}\PY{p}{)}\PY{p}{,}\PY{p}{(}\PY{o}{\PYZhy{}}\PY{l+m+mi}{2}\PY{p}{,}\PY{l+m+mi}{0}\PY{p}{)}\PY{p}{]}
\PY{n}{gain}\PY{o}{=}\PY{l+m+mi}{6}
\PY{n}{F}\PY{o}{=}\PY{n}{Ftransfert}\PY{p}{(}\PY{n}{zeros}\PY{o}{=}\PY{n}{zeros}\PY{p}{,}\PY{n}{poles}\PY{o}{=}\PY{n}{poles}\PY{p}{,}\PY{n}{gain}\PY{o}{=}\PY{n}{gain}\PY{p}{)}
\PY{n+nb}{print}\PY{p}{(}\PY{n+nb}{repr}\PY{p}{(}\PY{n}{F}\PY{p}{)}\PY{p}{)}
\PY{n+nb}{print}\PY{p}{(}\PY{n+nb}{str}\PY{p}{(}\PY{n}{F}\PY{p}{)}\PY{p}{)}
\end{Verbatim}
\end{tcolorbox}

    \begin{Verbatim}[commandchars=\\\{\}]
Ftranfert(zeros=[],poles=[(-1, 0), (-2, 0)],gain=6,name="F")

            6
F(p) = ----------
       (p+1)(p+2)

    \end{Verbatim}

    Pour tracer le diagramme de Nyquist on utilisera la méthode suivante :

    \begin{tcolorbox}[breakable, size=fbox, boxrule=1pt, pad at break*=1mm,colback=cellbackground, colframe=cellborder]
\prompt{In}{incolor}{3}{\boxspacing}
\begin{Verbatim}[commandchars=\\\{\}]
\PY{n}{F}\PY{o}{.}\PY{n}{nyquist}\PY{p}{(}\PY{n}{label}\PY{o}{=}\PY{l+s+sa}{r}\PY{l+s+s2}{\PYZdq{}}\PY{l+s+s2}{\PYZdl{}F(p)\PYZdl{}}\PY{l+s+s2}{\PYZdq{}}\PY{p}{,}\PY{n}{xlim}\PY{o}{=}\PY{p}{(}\PY{o}{\PYZhy{}}\PY{l+m+mi}{1}\PY{p}{,}\PY{l+m+mi}{4}\PY{p}{)}\PY{p}{,}\PY{n}{ylim}\PY{o}{=}\PY{p}{(}\PY{o}{\PYZhy{}}\PY{l+m+mi}{2}\PY{p}{,}\PY{l+m+mi}{2}\PY{p}{)}\PY{p}{,}\PY{n}{n}\PY{o}{=}\PY{l+m+mi}{4096}\PY{p}{,}\PY{n}{dw}\PY{o}{=}\PY{l+m+mf}{0.01}\PY{p}{)}
\PY{n+nb}{print}\PY{p}{(}\PY{n}{F}\PY{p}{)}
\end{Verbatim}
\end{tcolorbox}

    \begin{Verbatim}[commandchars=\\\{\}]
************************************************************
Nyquist plot : F(p)
Pulsation pas : 0.01
Interval des pulsations -20.48 20.48
Nombre de points 4096

            6
F(p) = ----------
       (p+1)(p+2)

************************************************************


            6
F(p) = ----------
       (p+1)(p+2)

    \end{Verbatim}

    \begin{center}
    \adjustimage{max size={0.9\linewidth}{0.9\paperheight}}{output_5_1.png}
    \end{center}
    { \hspace*{\fill} \\}
    
    \hypertarget{contour-dans-le-plan-complexe}{%
\section{Contour dans le plan
complexe}\label{contour-dans-le-plan-complexe}}

Le module est accompagné de quelques fonctions pour tracer des figures
de contour simple.

    \begin{tcolorbox}[breakable, size=fbox, boxrule=1pt, pad at break*=1mm,colback=cellbackground, colframe=cellborder]
\prompt{In}{incolor}{4}{\boxspacing}
\begin{Verbatim}[commandchars=\\\{\}]
\PY{k+kn}{from} \PY{n+nn}{contour} \PY{k}{import} \PY{n}{rectangle}\PY{p}{,}\PY{n}{circle}\PY{p}{,}\PY{n}{plot\PYZus{}contour}
\end{Verbatim}
\end{tcolorbox}

    Commençons par définir un contour de la forme d'un rectangle en
définissant deux coins de coordonnées \texttt{A=(-1.5,-1)} et
\texttt{B=(-0.5,1)} à l'aide de la fonction rectangle. Celle-ci retourne
une liste de quatre liste Python de 64 points chacuns (valeur par
défaut: 128).

    \begin{tcolorbox}[breakable, size=fbox, boxrule=1pt, pad at break*=1mm,colback=cellbackground, colframe=cellborder]
\prompt{In}{incolor}{5}{\boxspacing}
\begin{Verbatim}[commandchars=\\\{\}]
\PY{n}{C1}\PY{o}{=}\PY{n}{rectangle}\PY{p}{(}\PY{p}{(}\PY{l+m+mi}{1}\PY{p}{,}\PY{o}{\PYZhy{}}\PY{l+m+mi}{1}\PY{p}{)}\PY{p}{,}\PY{p}{(}\PY{l+m+mi}{2}\PY{p}{,}\PY{l+m+mi}{1}\PY{p}{)}\PY{p}{,}\PY{n}{npts}\PY{o}{=}\PY{l+m+mi}{64}\PY{p}{)}
\PY{n}{plot\PYZus{}contour}\PY{p}{(}\PY{n}{C1}\PY{p}{,}\PY{n}{xlim}\PY{o}{=}\PY{p}{(}\PY{o}{\PYZhy{}}\PY{l+m+mi}{2}\PY{p}{,}\PY{l+m+mf}{3.5}\PY{p}{)}\PY{p}{,}\PY{n}{ylim}\PY{o}{=}\PY{p}{(}\PY{o}{\PYZhy{}}\PY{l+m+mf}{1.5}\PY{p}{,}\PY{l+m+mf}{1.5}\PY{p}{)}\PY{p}{)}
\end{Verbatim}
\end{tcolorbox}

    \begin{center}
    \adjustimage{max size={0.9\linewidth}{0.9\paperheight}}{output_9_0.png}
    \end{center}
    { \hspace*{\fill} \\}
    
    On peut également définir le même contour mais parcouru dans le sens
contraire (ici le sens trigonométrique). Nous nommerons ce contour
\texttt{C1\_inv}

    \begin{tcolorbox}[breakable, size=fbox, boxrule=1pt, pad at break*=1mm,colback=cellbackground, colframe=cellborder]
\prompt{In}{incolor}{6}{\boxspacing}
\begin{Verbatim}[commandchars=\\\{\}]
\PY{n}{C1\PYZus{}inv}\PY{o}{=}\PY{n}{rectangle}\PY{p}{(}\PY{p}{(}\PY{l+m+mi}{1}\PY{p}{,}\PY{o}{\PYZhy{}}\PY{l+m+mi}{1}\PY{p}{)}\PY{p}{,}\PY{p}{(}\PY{l+m+mi}{2}\PY{p}{,}\PY{l+m+mi}{1}\PY{p}{)}\PY{p}{,}\PY{n}{npts}\PY{o}{=}\PY{l+m+mi}{64}\PY{p}{,}\PY{n}{inverse}\PY{o}{=}\PY{k+kc}{True}\PY{p}{)}
\PY{n}{plot\PYZus{}contour}\PY{p}{(}\PY{n}{C1\PYZus{}inv}\PY{p}{,}\PY{n}{xlim}\PY{o}{=}\PY{p}{(}\PY{o}{\PYZhy{}}\PY{l+m+mi}{2}\PY{p}{,}\PY{l+m+mf}{3.5}\PY{p}{)}\PY{p}{,}\PY{n}{ylim}\PY{o}{=}\PY{p}{(}\PY{o}{\PYZhy{}}\PY{l+m+mf}{1.5}\PY{p}{,}\PY{l+m+mf}{1.5}\PY{p}{)}\PY{p}{)}
\end{Verbatim}
\end{tcolorbox}

    \begin{center}
    \adjustimage{max size={0.9\linewidth}{0.9\paperheight}}{output_11_0.png}
    \end{center}
    { \hspace*{\fill} \\}
    
    \hypertarget{image-dun-contour-par-une-fonction-de-transfert}{%
\section{Image d'un contour par une fonction de
transfert}\label{image-dun-contour-par-une-fonction-de-transfert}}

\hypertarget{fonction-de-transfert-avec-un-seul-zuxe9ro}{%
\subsection{Fonction de transfert avec un seul
zéro}\label{fonction-de-transfert-avec-un-seul-zuxe9ro}}

Déterminons maintenant l'image de ces contours par une fonction de
transfert composée d'un seul zéro (\(z_1=-1\)).

\[
F_1(p)=(p+1)
\]

\hypertarget{contour-nentourant-pas-le-zuxe9ro-parcouru}{%
\subsubsection{Contour n'entourant pas le zéro
parcouru}\label{contour-nentourant-pas-le-zuxe9ro-parcouru}}

\hypertarget{dans-le-sens-horaire}{%
\paragraph{\ldots{} dans le sens horaire}\label{dans-le-sens-horaire}}

    \begin{tcolorbox}[breakable, size=fbox, boxrule=1pt, pad at break*=1mm,colback=cellbackground, colframe=cellborder]
\prompt{In}{incolor}{7}{\boxspacing}
\begin{Verbatim}[commandchars=\\\{\}]
\PY{n}{poles}\PY{o}{=}\PY{p}{[}\PY{p}{]}
\PY{n}{zeros}\PY{o}{=}\PY{p}{[}\PY{p}{(}\PY{o}{\PYZhy{}}\PY{l+m+mi}{1}\PY{p}{,}\PY{l+m+mi}{0}\PY{p}{)}\PY{p}{]}
\PY{n}{gain}\PY{o}{=}\PY{l+m+mi}{1}
\PY{n}{F\PYZus{}1}\PY{o}{=}\PY{n}{Ftransfert}\PY{p}{(}\PY{n}{zeros}\PY{o}{=}\PY{n}{zeros}\PY{p}{,}\PY{n}{poles}\PY{o}{=}\PY{n}{poles}\PY{p}{,}\PY{n}{gain}\PY{o}{=}\PY{n}{gain}\PY{p}{,}\PY{n}{name}\PY{o}{=}\PY{l+s+s2}{\PYZdq{}}\PY{l+s+s2}{F\PYZus{}1}\PY{l+s+s2}{\PYZdq{}}\PY{p}{)}
\PY{n}{F\PYZus{}1}\PY{o}{.}\PY{n}{cauchy}\PY{p}{(}\PY{n}{C1}\PY{p}{,}\PY{n}{xlim}\PY{o}{=}\PY{p}{(}\PY{o}{\PYZhy{}}\PY{l+m+mi}{2}\PY{p}{,}\PY{l+m+mf}{3.5}\PY{p}{)}\PY{p}{,}\PY{n}{ylim}\PY{o}{=}\PY{p}{(}\PY{o}{\PYZhy{}}\PY{l+m+mf}{1.5}\PY{p}{,}\PY{l+m+mf}{1.5}\PY{p}{)}\PY{p}{,}\PY{n}{contourLabel}\PY{o}{=}\PY{l+s+s2}{\PYZdq{}}\PY{l+s+s2}{C1}\PY{l+s+s2}{\PYZdq{}}\PY{p}{)}
\end{Verbatim}
\end{tcolorbox}

    \begin{Verbatim}[commandchars=\\\{\}]
************************************************************
Cauchy plot : contour C1

F\_1(p) = (p+1)


************************************************************

    \end{Verbatim}

    \begin{center}
    \adjustimage{max size={0.9\linewidth}{0.9\paperheight}}{output_13_1.png}
    \end{center}
    { \hspace*{\fill} \\}
    
    \hypertarget{dans-le-sens-trigonomuxe9trique}{%
\paragraph{\ldots{} Dans le sens
trigonométrique}\label{dans-le-sens-trigonomuxe9trique}}

    \begin{tcolorbox}[breakable, size=fbox, boxrule=1pt, pad at break*=1mm,colback=cellbackground, colframe=cellborder]
\prompt{In}{incolor}{8}{\boxspacing}
\begin{Verbatim}[commandchars=\\\{\}]
\PY{n}{F\PYZus{}1}\PY{o}{.}\PY{n}{cauchy}\PY{p}{(}\PY{n}{C1\PYZus{}inv}\PY{p}{,}\PY{n}{xlim}\PY{o}{=}\PY{p}{(}\PY{o}{\PYZhy{}}\PY{l+m+mi}{2}\PY{p}{,}\PY{l+m+mf}{3.5}\PY{p}{)}\PY{p}{,}\PY{n}{ylim}\PY{o}{=}\PY{p}{(}\PY{o}{\PYZhy{}}\PY{l+m+mf}{1.5}\PY{p}{,}\PY{l+m+mf}{1.5}\PY{p}{)}\PY{p}{,}\PY{n}{contourLabel}\PY{o}{=}\PY{l+s+s2}{\PYZdq{}}\PY{l+s+s2}{C1 (inverse)}\PY{l+s+s2}{\PYZdq{}}\PY{p}{)}
\end{Verbatim}
\end{tcolorbox}

    \begin{Verbatim}[commandchars=\\\{\}]
************************************************************
Cauchy plot : contour C1 (inverse)

F\_1(p) = (p+1)


************************************************************

    \end{Verbatim}

    \begin{center}
    \adjustimage{max size={0.9\linewidth}{0.9\paperheight}}{output_15_1.png}
    \end{center}
    { \hspace*{\fill} \\}
    
    (La carte des pôles et zéros de la fonction de transfert sera toujours
représentée sur le graphe de gauche. Sur le graphe de droite c'est
l'origine du plan qui sera toujours représentée par un `+'.)

Nous constatons que l'image du contour par une telle fonction de
transfert est simplement translatée d'une distance égale à la distance
entre le point du contour et la position du zéro. \textbf{Nous observons
également que le sens de parcours du contour de l'image n'est pas
modifié.}

\hypertarget{contour-contenant-le-zuxe9ro}{%
\subsubsection{Contour contenant le
zéro}\label{contour-contenant-le-zuxe9ro}}

On se donne un deuxième contour \texttt{C2} qui contient le zéro de la
fonction de transfert test.

    \begin{tcolorbox}[breakable, size=fbox, boxrule=1pt, pad at break*=1mm,colback=cellbackground, colframe=cellborder]
\prompt{In}{incolor}{9}{\boxspacing}
\begin{Verbatim}[commandchars=\\\{\}]
\PY{n}{C2}\PY{o}{=}\PY{n}{rectangle}\PY{p}{(}\PY{p}{(}\PY{o}{\PYZhy{}}\PY{l+m+mf}{1.5}\PY{p}{,}\PY{o}{\PYZhy{}}\PY{l+m+mi}{1}\PY{p}{)}\PY{p}{,}\PY{p}{(}\PY{o}{\PYZhy{}}\PY{l+m+mf}{0.5}\PY{p}{,}\PY{l+m+mi}{1}\PY{p}{)}\PY{p}{,}\PY{n}{npts}\PY{o}{=}\PY{l+m+mi}{64}\PY{p}{)}
\PY{n}{C2\PYZus{}inv}\PY{o}{=}\PY{n}{rectangle}\PY{p}{(}\PY{p}{(}\PY{o}{\PYZhy{}}\PY{l+m+mf}{1.5}\PY{p}{,}\PY{o}{\PYZhy{}}\PY{l+m+mi}{1}\PY{p}{)}\PY{p}{,}\PY{p}{(}\PY{o}{\PYZhy{}}\PY{l+m+mf}{0.5}\PY{p}{,}\PY{l+m+mi}{1}\PY{p}{)}\PY{p}{,}\PY{n}{npts}\PY{o}{=}\PY{l+m+mi}{64}\PY{p}{,}\PY{n}{inverse}\PY{o}{=}\PY{k+kc}{True}\PY{p}{)}
\PY{n}{F\PYZus{}1}\PY{o}{.}\PY{n}{cauchy}\PY{p}{(}\PY{n}{C2}\PY{p}{,}\PY{n}{xlim}\PY{o}{=}\PY{p}{(}\PY{o}{\PYZhy{}}\PY{l+m+mi}{2}\PY{p}{,}\PY{l+m+mi}{1}\PY{p}{)}\PY{p}{,}\PY{n}{ylim}\PY{o}{=}\PY{p}{(}\PY{o}{\PYZhy{}}\PY{l+m+mf}{1.5}\PY{p}{,}\PY{l+m+mf}{1.5}\PY{p}{)}\PY{p}{,}\PY{n}{contourLabel}\PY{o}{=}\PY{l+s+s2}{\PYZdq{}}\PY{l+s+s2}{C2}\PY{l+s+s2}{\PYZdq{}}\PY{p}{)}
\PY{n}{F\PYZus{}1}\PY{o}{.}\PY{n}{cauchy}\PY{p}{(}\PY{n}{C2\PYZus{}inv}\PY{p}{,}\PY{n}{xlim}\PY{o}{=}\PY{p}{(}\PY{o}{\PYZhy{}}\PY{l+m+mi}{2}\PY{p}{,}\PY{l+m+mi}{1}\PY{p}{)}\PY{p}{,}\PY{n}{ylim}\PY{o}{=}\PY{p}{(}\PY{o}{\PYZhy{}}\PY{l+m+mf}{1.5}\PY{p}{,}\PY{l+m+mf}{1.5}\PY{p}{)}\PY{p}{,}\PY{n}{contourLabel}\PY{o}{=}\PY{l+s+s2}{\PYZdq{}}\PY{l+s+s2}{C2 (inverse)}\PY{l+s+s2}{\PYZdq{}}\PY{p}{)}
\end{Verbatim}
\end{tcolorbox}

    \begin{Verbatim}[commandchars=\\\{\}]
************************************************************
Cauchy plot : contour C2

F\_1(p) = (p+1)


************************************************************

************************************************************
Cauchy plot : contour C2 (inverse)

F\_1(p) = (p+1)


************************************************************

    \end{Verbatim}

    \begin{center}
    \adjustimage{max size={0.9\linewidth}{0.9\paperheight}}{output_17_1.png}
    \end{center}
    { \hspace*{\fill} \\}
    
    \begin{center}
    \adjustimage{max size={0.9\linewidth}{0.9\paperheight}}{output_17_2.png}
    \end{center}
    { \hspace*{\fill} \\}
    
    \textbf{Ici le contour \(\mathcal{C}\) entoure le zéro de la fonction de
transfert. Nous observons que le contour image contient l'origine (et
est orienté dans le même sens que le contour origine).} \#\# Fonction de
transfert avec deux zéros

Traçons maintenant le résultat pour une fonction de transfert possédant
deux zéros (\(z_1=-1\) et \(z_2=-0.75\)):

\[
F_2(p)=\dfrac{1}{4}(p+1)(p+0.75).
\]

\hypertarget{contour-nentourant-pas-les-zuxe9ros}{%
\subsubsection{Contour n'entourant pas les
zéros}\label{contour-nentourant-pas-les-zuxe9ros}}

    \begin{tcolorbox}[breakable, size=fbox, boxrule=1pt, pad at break*=1mm,colback=cellbackground, colframe=cellborder]
\prompt{In}{incolor}{10}{\boxspacing}
\begin{Verbatim}[commandchars=\\\{\}]
\PY{n}{zeros}\PY{o}{=}\PY{p}{[}\PY{p}{(}\PY{o}{\PYZhy{}}\PY{l+m+mi}{1}\PY{p}{,}\PY{l+m+mi}{0}\PY{p}{)}\PY{p}{,}\PY{p}{(}\PY{o}{\PYZhy{}}\PY{l+m+mf}{0.75}\PY{p}{,}\PY{l+m+mi}{0}\PY{p}{)}\PY{p}{]}
\PY{n}{poles}\PY{o}{=}\PY{p}{[}\PY{p}{]}
\PY{n}{gain}\PY{o}{=}\PY{l+m+mf}{0.25}
\PY{n}{F\PYZus{}2}\PY{o}{=}\PY{n}{Ftransfert}\PY{p}{(}\PY{n}{zeros}\PY{o}{=}\PY{n}{zeros}\PY{p}{,}\PY{n}{poles}\PY{o}{=}\PY{n}{poles}\PY{p}{,}\PY{n}{gain}\PY{o}{=}\PY{n}{gain}\PY{p}{,}\PY{n}{name}\PY{o}{=}\PY{l+s+s2}{\PYZdq{}}\PY{l+s+s2}{F\PYZus{}2}\PY{l+s+s2}{\PYZdq{}}\PY{p}{)}
\PY{n}{F\PYZus{}2}\PY{o}{.}\PY{n}{cauchy}\PY{p}{(}\PY{n}{C1}\PY{p}{,}\PY{n}{xlim}\PY{o}{=}\PY{p}{(}\PY{o}{\PYZhy{}}\PY{l+m+mi}{2}\PY{p}{,}\PY{l+m+mf}{3.5}\PY{p}{)}\PY{p}{,}\PY{n}{ylim}\PY{o}{=}\PY{p}{(}\PY{o}{\PYZhy{}}\PY{l+m+mf}{1.5}\PY{p}{,}\PY{l+m+mf}{1.5}\PY{p}{)}\PY{p}{,}\PY{n}{contourLabel}\PY{o}{=}\PY{l+s+s2}{\PYZdq{}}\PY{l+s+s2}{C1}\PY{l+s+s2}{\PYZdq{}}\PY{p}{)}
\PY{n}{F\PYZus{}2}\PY{o}{.}\PY{n}{cauchy}\PY{p}{(}\PY{n}{C1\PYZus{}inv}\PY{p}{,}\PY{n}{xlim}\PY{o}{=}\PY{p}{(}\PY{o}{\PYZhy{}}\PY{l+m+mi}{2}\PY{p}{,}\PY{l+m+mf}{3.5}\PY{p}{)}\PY{p}{,}\PY{n}{ylim}\PY{o}{=}\PY{p}{(}\PY{o}{\PYZhy{}}\PY{l+m+mf}{1.5}\PY{p}{,}\PY{l+m+mf}{1.5}\PY{p}{)}\PY{p}{,}\PY{n}{contourLabel}\PY{o}{=}\PY{l+s+s2}{\PYZdq{}}\PY{l+s+s2}{C1 (inverse)}\PY{l+s+s2}{\PYZdq{}}\PY{p}{)}
\end{Verbatim}
\end{tcolorbox}

    \begin{Verbatim}[commandchars=\\\{\}]
************************************************************
Cauchy plot : contour C1

F\_2(p) = 0.25 (p+1)(p+0.75)


************************************************************

************************************************************
Cauchy plot : contour C1 (inverse)

F\_2(p) = 0.25 (p+1)(p+0.75)


************************************************************

    \end{Verbatim}

    \begin{center}
    \adjustimage{max size={0.9\linewidth}{0.9\paperheight}}{output_19_1.png}
    \end{center}
    { \hspace*{\fill} \\}
    
    \begin{center}
    \adjustimage{max size={0.9\linewidth}{0.9\paperheight}}{output_19_2.png}
    \end{center}
    { \hspace*{\fill} \\}
    
    \textbf{On observe comme précédemment que si le contour ne contient pas
les zéros de la fonction de transfert, l'image de celui-ci ne fait aucun
tour autour de l'origine.} \#\#\# Contour entourant les deux zéros Dans
cet exemple, nous allons utiliser un nouveau contour de la forme d'un
cercle.

    \begin{tcolorbox}[breakable, size=fbox, boxrule=1pt, pad at break*=1mm,colback=cellbackground, colframe=cellborder]
\prompt{In}{incolor}{11}{\boxspacing}
\begin{Verbatim}[commandchars=\\\{\}]
\PY{n}{C3}\PY{o}{=}\PY{n}{circle}\PY{p}{(}\PY{n}{center}\PY{o}{=}\PY{p}{(}\PY{o}{\PYZhy{}}\PY{l+m+mi}{1}\PY{p}{,}\PY{l+m+mi}{0}\PY{p}{)}\PY{p}{,}\PY{n}{radius}\PY{o}{=}\PY{l+m+mf}{0.75}\PY{p}{)}
\PY{n}{C3\PYZus{}inv}\PY{o}{=}\PY{n}{circle}\PY{p}{(}\PY{n}{center}\PY{o}{=}\PY{p}{(}\PY{o}{\PYZhy{}}\PY{l+m+mi}{1}\PY{p}{,}\PY{l+m+mi}{0}\PY{p}{)}\PY{p}{,}\PY{n}{radius}\PY{o}{=}\PY{l+m+mf}{0.75}\PY{p}{,}\PY{n}{inverse}\PY{o}{=}\PY{k+kc}{True}\PY{p}{)}
\PY{n}{F\PYZus{}2}\PY{o}{.}\PY{n}{cauchy}\PY{p}{(}\PY{n}{C3}\PY{p}{,}\PY{n}{xlim}\PY{o}{=}\PY{p}{[}\PY{p}{(}\PY{o}{\PYZhy{}}\PY{l+m+mf}{1.9}\PY{p}{,}\PY{l+m+mf}{1.0}\PY{p}{)}\PY{p}{,}\PY{p}{(}\PY{o}{\PYZhy{}}\PY{l+m+mf}{0.3}\PY{p}{,}\PY{l+m+mf}{0.3}\PY{p}{)}\PY{p}{]}\PY{p}{,}\PY{n}{ylim}\PY{o}{=}\PY{p}{[}\PY{p}{(}\PY{o}{\PYZhy{}}\PY{l+m+mi}{1}\PY{p}{,}\PY{l+m+mi}{1}\PY{p}{)}\PY{p}{,}\PY{p}{(}\PY{o}{\PYZhy{}}\PY{l+m+mf}{0.25}\PY{p}{,}\PY{l+m+mf}{0.25}\PY{p}{)}\PY{p}{]}\PY{p}{,}\PY{n}{contourLabel}\PY{o}{=}\PY{l+s+s2}{\PYZdq{}}\PY{l+s+s2}{C1}\PY{l+s+s2}{\PYZdq{}}\PY{p}{)}
\PY{n}{F\PYZus{}2}\PY{o}{.}\PY{n}{cauchy}\PY{p}{(}\PY{n}{C3\PYZus{}inv}\PY{p}{,}\PY{n}{xlim}\PY{o}{=}\PY{p}{[}\PY{p}{(}\PY{o}{\PYZhy{}}\PY{l+m+mf}{1.9}\PY{p}{,}\PY{l+m+mf}{1.0}\PY{p}{)}\PY{p}{,}\PY{p}{(}\PY{o}{\PYZhy{}}\PY{l+m+mf}{0.3}\PY{p}{,}\PY{l+m+mf}{0.3}\PY{p}{)}\PY{p}{]}\PY{p}{,}\PY{n}{ylim}\PY{o}{=}\PY{p}{[}\PY{p}{(}\PY{o}{\PYZhy{}}\PY{l+m+mi}{1}\PY{p}{,}\PY{l+m+mi}{1}\PY{p}{)}\PY{p}{,}\PY{p}{(}\PY{o}{\PYZhy{}}\PY{l+m+mf}{0.25}\PY{p}{,}\PY{l+m+mf}{0.25}\PY{p}{)}\PY{p}{]}\PY{p}{,}\PY{n}{contourLabel}\PY{o}{=}\PY{l+s+s2}{\PYZdq{}}\PY{l+s+s2}{C1 (inverse)}\PY{l+s+s2}{\PYZdq{}}\PY{p}{)}
\end{Verbatim}
\end{tcolorbox}

    \begin{Verbatim}[commandchars=\\\{\}]
************************************************************
Cauchy plot : contour C1

F\_2(p) = 0.25 (p+1)(p+0.75)


************************************************************

************************************************************
Cauchy plot : contour C1 (inverse)

F\_2(p) = 0.25 (p+1)(p+0.75)


************************************************************

    \end{Verbatim}

    \begin{center}
    \adjustimage{max size={0.9\linewidth}{0.9\paperheight}}{output_21_1.png}
    \end{center}
    { \hspace*{\fill} \\}
    
    \begin{center}
    \adjustimage{max size={0.9\linewidth}{0.9\paperheight}}{output_21_2.png}
    \end{center}
    { \hspace*{\fill} \\}
    
    Si le contour entoure deux zéros, l'image par la fonction de transfert
fait deux tours autour de l'origine. Plus généralement, \textbf{si le
contour entoure un nombre \(Z\) de zéros, l'image par la fonction de
transfert du contour fait \(Z\) tours autour de l'origine dans le même
sens.} \#\# Fonction de transfert possédant un pôle Nous allons
maintenant observer le comportement de ces tracés pour différents
contours dans le cas où la fonction de transfert présente un pôle,
notamment avec :

\[
F_3(p)=\dfrac{6}{p+1}
\] \#\#\# Contour n'entourant pas le pôle

    \begin{tcolorbox}[breakable, size=fbox, boxrule=1pt, pad at break*=1mm,colback=cellbackground, colframe=cellborder]
\prompt{In}{incolor}{12}{\boxspacing}
\begin{Verbatim}[commandchars=\\\{\}]
\PY{n}{zeros}\PY{o}{=}\PY{p}{[}\PY{p}{]}
\PY{n}{poles}\PY{o}{=}\PY{p}{[}\PY{p}{(}\PY{o}{\PYZhy{}}\PY{l+m+mi}{1}\PY{p}{,}\PY{l+m+mi}{0}\PY{p}{)}\PY{p}{]}
\PY{n}{gain}\PY{o}{=}\PY{l+m+mi}{6}
\PY{n}{F\PYZus{}3}\PY{o}{=}\PY{n}{Ftransfert}\PY{p}{(}\PY{n}{zeros}\PY{o}{=}\PY{n}{zeros}\PY{p}{,}\PY{n}{poles}\PY{o}{=}\PY{n}{poles}\PY{p}{,}\PY{n}{gain}\PY{o}{=}\PY{n}{gain}\PY{p}{,}\PY{n}{name}\PY{o}{=}\PY{l+s+s2}{\PYZdq{}}\PY{l+s+s2}{F\PYZus{}3}\PY{l+s+s2}{\PYZdq{}}\PY{p}{)}
\PY{n}{F\PYZus{}3}\PY{o}{.}\PY{n}{cauchy}\PY{p}{(}\PY{n}{C1}\PY{p}{,}\PY{n}{xlim}\PY{o}{=}\PY{p}{(}\PY{o}{\PYZhy{}}\PY{l+m+mi}{2}\PY{p}{,}\PY{l+m+mf}{3.5}\PY{p}{)}\PY{p}{,}\PY{n}{ylim}\PY{o}{=}\PY{p}{(}\PY{o}{\PYZhy{}}\PY{l+m+mf}{1.5}\PY{p}{,}\PY{l+m+mf}{1.5}\PY{p}{)}\PY{p}{,}\PY{n}{contourLabel}\PY{o}{=}\PY{l+s+s2}{\PYZdq{}}\PY{l+s+s2}{C1}\PY{l+s+s2}{\PYZdq{}}\PY{p}{)}
\PY{n}{F\PYZus{}3}\PY{o}{.}\PY{n}{cauchy}\PY{p}{(}\PY{n}{C1\PYZus{}inv}\PY{p}{,}\PY{n}{xlim}\PY{o}{=}\PY{p}{(}\PY{o}{\PYZhy{}}\PY{l+m+mi}{2}\PY{p}{,}\PY{l+m+mf}{3.5}\PY{p}{)}\PY{p}{,}\PY{n}{ylim}\PY{o}{=}\PY{p}{(}\PY{o}{\PYZhy{}}\PY{l+m+mf}{1.5}\PY{p}{,}\PY{l+m+mf}{1.5}\PY{p}{)}\PY{p}{,}\PY{n}{contourLabel}\PY{o}{=}\PY{l+s+s2}{\PYZdq{}}\PY{l+s+s2}{C1 (inverse)}\PY{l+s+s2}{\PYZdq{}}\PY{p}{)}
\end{Verbatim}
\end{tcolorbox}

    \begin{Verbatim}[commandchars=\\\{\}]
************************************************************
Cauchy plot : contour C1

           6
F\_3(p) = -----
         (p+1)

************************************************************

************************************************************
Cauchy plot : contour C1 (inverse)

           6
F\_3(p) = -----
         (p+1)

************************************************************

    \end{Verbatim}

    \begin{center}
    \adjustimage{max size={0.9\linewidth}{0.9\paperheight}}{output_23_1.png}
    \end{center}
    { \hspace*{\fill} \\}
    
    \begin{center}
    \adjustimage{max size={0.9\linewidth}{0.9\paperheight}}{output_23_2.png}
    \end{center}
    { \hspace*{\fill} \\}
    
    Nous remarquons tout d'abord que le contour image n'entoure pas
l'origine du plan complexe. De plus le contour image est parcouru dans
le même sens que le contour origine \#\#\# Contour entourant le pôle

    \begin{tcolorbox}[breakable, size=fbox, boxrule=1pt, pad at break*=1mm,colback=cellbackground, colframe=cellborder]
\prompt{In}{incolor}{13}{\boxspacing}
\begin{Verbatim}[commandchars=\\\{\}]
\PY{n}{F\PYZus{}3}\PY{o}{.}\PY{n}{cauchy}\PY{p}{(}\PY{n}{C2}\PY{p}{,}\PY{n}{xlim}\PY{o}{=}\PY{p}{[}\PY{p}{(}\PY{o}{\PYZhy{}}\PY{l+m+mi}{2}\PY{p}{,}\PY{l+m+mi}{2}\PY{p}{)}\PY{p}{,}\PY{p}{(}\PY{o}{\PYZhy{}}\PY{l+m+mi}{15}\PY{p}{,}\PY{l+m+mi}{15}\PY{p}{)}\PY{p}{]}\PY{p}{,}\PY{n}{ylim}\PY{o}{=}\PY{p}{[}\PY{p}{(}\PY{o}{\PYZhy{}}\PY{l+m+mf}{1.5}\PY{p}{,}\PY{l+m+mf}{1.5}\PY{p}{)}\PY{p}{,}\PY{p}{(}\PY{o}{\PYZhy{}}\PY{l+m+mi}{7}\PY{p}{,}\PY{l+m+mi}{7}\PY{p}{)}\PY{p}{]}\PY{p}{,}\PY{n}{contourLabel}\PY{o}{=}\PY{l+s+s2}{\PYZdq{}}\PY{l+s+s2}{C1}\PY{l+s+s2}{\PYZdq{}}\PY{p}{)}
\PY{n}{F\PYZus{}3}\PY{o}{.}\PY{n}{cauchy}\PY{p}{(}\PY{n}{C2\PYZus{}inv}\PY{p}{,}\PY{n}{xlim}\PY{o}{=}\PY{p}{[}\PY{p}{(}\PY{o}{\PYZhy{}}\PY{l+m+mi}{2}\PY{p}{,}\PY{l+m+mi}{2}\PY{p}{)}\PY{p}{,}\PY{p}{(}\PY{o}{\PYZhy{}}\PY{l+m+mi}{15}\PY{p}{,}\PY{l+m+mi}{15}\PY{p}{)}\PY{p}{]}\PY{p}{,}\PY{n}{ylim}\PY{o}{=}\PY{p}{[}\PY{p}{(}\PY{o}{\PYZhy{}}\PY{l+m+mf}{1.5}\PY{p}{,}\PY{l+m+mf}{1.5}\PY{p}{)}\PY{p}{,}\PY{p}{(}\PY{o}{\PYZhy{}}\PY{l+m+mi}{7}\PY{p}{,}\PY{l+m+mi}{7}\PY{p}{)}\PY{p}{]}\PY{p}{,}\PY{n}{contourLabel}\PY{o}{=}\PY{l+s+s2}{\PYZdq{}}\PY{l+s+s2}{C1 (inverse)}\PY{l+s+s2}{\PYZdq{}}\PY{p}{)}
\end{Verbatim}
\end{tcolorbox}

    \begin{Verbatim}[commandchars=\\\{\}]
************************************************************
Cauchy plot : contour C1

           6
F\_3(p) = -----
         (p+1)

************************************************************

************************************************************
Cauchy plot : contour C1 (inverse)

           6
F\_3(p) = -----
         (p+1)

************************************************************

    \end{Verbatim}

    \begin{center}
    \adjustimage{max size={0.9\linewidth}{0.9\paperheight}}{output_25_1.png}
    \end{center}
    { \hspace*{\fill} \\}
    
    \begin{center}
    \adjustimage{max size={0.9\linewidth}{0.9\paperheight}}{output_25_2.png}
    \end{center}
    { \hspace*{\fill} \\}
    
    \textbf{Nous remarquons que l'image du contour entoure l'origine dans le
sens opposé celui du contour d'origine.} Nous allons procédé de la même
manière avec une fonction de transfert possédant deux pôles.

\[
F_4(p)=\dfrac{6}{(p+1)(p+0.75)}
\] \#\#\# Contour n'entourant pas les pôles

    \begin{tcolorbox}[breakable, size=fbox, boxrule=1pt, pad at break*=1mm,colback=cellbackground, colframe=cellborder]
\prompt{In}{incolor}{14}{\boxspacing}
\begin{Verbatim}[commandchars=\\\{\}]
\PY{n}{zeros}\PY{o}{=}\PY{p}{[}\PY{p}{]}
\PY{n}{poles}\PY{o}{=}\PY{p}{[}\PY{p}{(}\PY{o}{\PYZhy{}}\PY{l+m+mi}{1}\PY{p}{,}\PY{l+m+mi}{0}\PY{p}{)}\PY{p}{,}\PY{p}{(}\PY{o}{\PYZhy{}}\PY{l+m+mf}{0.75}\PY{p}{,}\PY{l+m+mi}{0}\PY{p}{)}\PY{p}{]}
\PY{n}{gain}\PY{o}{=}\PY{l+m+mi}{6}
\PY{n}{F\PYZus{}4}\PY{o}{=}\PY{n}{Ftransfert}\PY{p}{(}\PY{n}{zeros}\PY{o}{=}\PY{n}{zeros}\PY{p}{,}\PY{n}{poles}\PY{o}{=}\PY{n}{poles}\PY{p}{,}\PY{n}{gain}\PY{o}{=}\PY{n}{gain}\PY{p}{,}\PY{n}{name}\PY{o}{=}\PY{l+s+s2}{\PYZdq{}}\PY{l+s+s2}{F\PYZus{}4}\PY{l+s+s2}{\PYZdq{}}\PY{p}{)}
\PY{n}{F\PYZus{}4}\PY{o}{.}\PY{n}{cauchy}\PY{p}{(}\PY{n}{C1}\PY{p}{,}\PY{n}{xlim}\PY{o}{=}\PY{p}{(}\PY{o}{\PYZhy{}}\PY{l+m+mi}{2}\PY{p}{,}\PY{l+m+mf}{3.5}\PY{p}{)}\PY{p}{,}\PY{n}{ylim}\PY{o}{=}\PY{p}{(}\PY{o}{\PYZhy{}}\PY{l+m+mf}{1.5}\PY{p}{,}\PY{l+m+mf}{1.5}\PY{p}{)}\PY{p}{,}\PY{n}{contourLabel}\PY{o}{=}\PY{l+s+s2}{\PYZdq{}}\PY{l+s+s2}{C1}\PY{l+s+s2}{\PYZdq{}}\PY{p}{)}
\PY{n}{F\PYZus{}4}\PY{o}{.}\PY{n}{cauchy}\PY{p}{(}\PY{n}{C1\PYZus{}inv}\PY{p}{,}\PY{n}{xlim}\PY{o}{=}\PY{p}{(}\PY{o}{\PYZhy{}}\PY{l+m+mi}{2}\PY{p}{,}\PY{l+m+mf}{3.5}\PY{p}{)}\PY{p}{,}\PY{n}{ylim}\PY{o}{=}\PY{p}{(}\PY{o}{\PYZhy{}}\PY{l+m+mf}{1.5}\PY{p}{,}\PY{l+m+mf}{1.5}\PY{p}{)}\PY{p}{,}\PY{n}{contourLabel}\PY{o}{=}\PY{l+s+s2}{\PYZdq{}}\PY{l+s+s2}{C1 (inverse)}\PY{l+s+s2}{\PYZdq{}}\PY{p}{)}
\end{Verbatim}
\end{tcolorbox}

    \begin{Verbatim}[commandchars=\\\{\}]
************************************************************
Cauchy plot : contour C1

               6
F\_4(p) = -------------
         (p+1)(p+0.75)

************************************************************

************************************************************
Cauchy plot : contour C1 (inverse)

               6
F\_4(p) = -------------
         (p+1)(p+0.75)

************************************************************

    \end{Verbatim}

    \begin{center}
    \adjustimage{max size={0.9\linewidth}{0.9\paperheight}}{output_27_1.png}
    \end{center}
    { \hspace*{\fill} \\}
    
    \begin{center}
    \adjustimage{max size={0.9\linewidth}{0.9\paperheight}}{output_27_2.png}
    \end{center}
    { \hspace*{\fill} \\}
    
    Nous observons comme précédemment que le contour de l'image n'entoure
pas l'origine. \#\#\# Contour entourant les pôles

    \begin{tcolorbox}[breakable, size=fbox, boxrule=1pt, pad at break*=1mm,colback=cellbackground, colframe=cellborder]
\prompt{In}{incolor}{15}{\boxspacing}
\begin{Verbatim}[commandchars=\\\{\}]
\PY{n}{F\PYZus{}4}\PY{o}{.}\PY{n}{cauchy}\PY{p}{(}\PY{n}{C3}\PY{p}{,}\PY{n}{xlim}\PY{o}{=}\PY{p}{[}\PY{p}{(}\PY{o}{\PYZhy{}}\PY{l+m+mi}{2}\PY{p}{,}\PY{l+m+mi}{2}\PY{p}{)}\PY{p}{,}\PY{p}{(}\PY{o}{\PYZhy{}}\PY{l+m+mi}{15}\PY{p}{,}\PY{l+m+mi}{20}\PY{p}{)}\PY{p}{]}\PY{p}{,}\PY{n}{ylim}\PY{o}{=}\PY{p}{[}\PY{p}{(}\PY{o}{\PYZhy{}}\PY{l+m+mf}{1.5}\PY{p}{,}\PY{l+m+mf}{1.5}\PY{p}{)}\PY{p}{,}\PY{p}{(}\PY{o}{\PYZhy{}}\PY{l+m+mi}{15}\PY{p}{,}\PY{l+m+mi}{15}\PY{p}{)}\PY{p}{]}\PY{p}{,}\PY{n}{contourLabel}\PY{o}{=}\PY{l+s+s2}{\PYZdq{}}\PY{l+s+s2}{C3}\PY{l+s+s2}{\PYZdq{}}\PY{p}{)}
\PY{n}{F\PYZus{}4}\PY{o}{.}\PY{n}{cauchy}\PY{p}{(}\PY{n}{C3\PYZus{}inv}\PY{p}{,}\PY{n}{xlim}\PY{o}{=}\PY{p}{[}\PY{p}{(}\PY{o}{\PYZhy{}}\PY{l+m+mi}{2}\PY{p}{,}\PY{l+m+mi}{2}\PY{p}{)}\PY{p}{,}\PY{p}{(}\PY{o}{\PYZhy{}}\PY{l+m+mi}{15}\PY{p}{,}\PY{l+m+mi}{20}\PY{p}{)}\PY{p}{]}\PY{p}{,}\PY{n}{ylim}\PY{o}{=}\PY{p}{[}\PY{p}{(}\PY{o}{\PYZhy{}}\PY{l+m+mf}{1.5}\PY{p}{,}\PY{l+m+mf}{1.5}\PY{p}{)}\PY{p}{,}\PY{p}{(}\PY{o}{\PYZhy{}}\PY{l+m+mi}{15}\PY{p}{,}\PY{l+m+mi}{15}\PY{p}{)}\PY{p}{]}\PY{p}{,}\PY{n}{contourLabel}\PY{o}{=}\PY{l+s+s2}{\PYZdq{}}\PY{l+s+s2}{C3 (inverse)}\PY{l+s+s2}{\PYZdq{}}\PY{p}{)}
\end{Verbatim}
\end{tcolorbox}

    \begin{Verbatim}[commandchars=\\\{\}]
************************************************************
Cauchy plot : contour C3

               6
F\_4(p) = -------------
         (p+1)(p+0.75)

************************************************************

************************************************************
Cauchy plot : contour C3 (inverse)

               6
F\_4(p) = -------------
         (p+1)(p+0.75)

************************************************************

    \end{Verbatim}

    \begin{center}
    \adjustimage{max size={0.9\linewidth}{0.9\paperheight}}{output_29_1.png}
    \end{center}
    { \hspace*{\fill} \\}
    
    \begin{center}
    \adjustimage{max size={0.9\linewidth}{0.9\paperheight}}{output_29_2.png}
    \end{center}
    { \hspace*{\fill} \\}
    
    \textbf{Si le contour entoure un nombre \(P\) de pôles, l'image par la
fonction de transfert de ce contour fait un nombre \(P\) de tours autour
de l'origine dans le sens opposé (à celui du contour).} \# Enoncé du
principe de l'argument de Cauchy On énonce alors le principe de Cauchy:

Si un contour \(\mathcal{C}\) entoure \(Z\) zéros et \(P\) pôles d'une
fonction analytique \(F(p)\) sans en traverser aucun, alors quand on le
parcourt dans le sens anti-trigonométrique, le contour image par
\(F(p)\), \(\Gamma=F(\mathcal{C})\) fait un nombre de tours \(N\) autour
de l'origine dans le sens trigonométrique égal à,

\[
    N=Z-P
\]

On vérifie le principe de Cauchy avec une fonction de transfert composé
de cinq zéros et deux pôles. Et un contour \(\mathcal{C}\) qui n'entoure
que \(Z=3\) zéros et \(P=1\) pôle. On a donc \(N=2\). Le signe négatif
indique que le nombre de tours est dans le sens opposé au sens
trigonométrique (lorque le contour d'origine tourne dans le sens
horaire.

    \begin{tcolorbox}[breakable, size=fbox, boxrule=1pt, pad at break*=1mm,colback=cellbackground, colframe=cellborder]
\prompt{In}{incolor}{16}{\boxspacing}
\begin{Verbatim}[commandchars=\\\{\}]
\PY{n}{zeros}\PY{o}{=}\PY{p}{[}\PY{p}{(}\PY{o}{\PYZhy{}}\PY{l+m+mf}{0.75}\PY{p}{,}\PY{l+m+mf}{0.5}\PY{p}{)}\PY{p}{,}\PY{p}{(}\PY{o}{\PYZhy{}}\PY{l+m+mf}{0.75}\PY{p}{,}\PY{o}{\PYZhy{}}\PY{l+m+mf}{0.5}\PY{p}{)}\PY{p}{,}\PY{p}{(}\PY{o}{\PYZhy{}}\PY{l+m+mf}{1.65}\PY{p}{,}\PY{l+m+mi}{0}\PY{p}{)}\PY{p}{,}\PY{p}{(}\PY{o}{\PYZhy{}}\PY{l+m+mi}{2}\PY{p}{,}\PY{l+m+mi}{1}\PY{p}{)}\PY{p}{,}\PY{p}{(}\PY{o}{\PYZhy{}}\PY{l+m+mi}{2}\PY{p}{,}\PY{o}{\PYZhy{}}\PY{l+m+mi}{1}\PY{p}{)}\PY{p}{]}
\PY{n}{poles}\PY{o}{=}\PY{p}{[}\PY{p}{(}\PY{o}{\PYZhy{}}\PY{l+m+mi}{1}\PY{p}{,}\PY{l+m+mi}{0}\PY{p}{)}\PY{p}{,}\PY{p}{(}\PY{o}{\PYZhy{}}\PY{l+m+mf}{2.25}\PY{p}{,}\PY{l+m+mi}{0}\PY{p}{)}\PY{p}{]}
\PY{n}{gain}\PY{o}{=}\PY{l+m+mf}{0.75}
\PY{n}{F\PYZus{}5}\PY{o}{=}\PY{n}{Ftransfert}\PY{p}{(}\PY{n}{zeros}\PY{o}{=}\PY{n}{zeros}\PY{p}{,}\PY{n}{poles}\PY{o}{=}\PY{n}{poles}\PY{p}{,}\PY{n}{gain}\PY{o}{=}\PY{n}{gain}\PY{p}{,}\PY{n}{name}\PY{o}{=}\PY{l+s+s2}{\PYZdq{}}\PY{l+s+s2}{F\PYZus{}5}\PY{l+s+s2}{\PYZdq{}}\PY{p}{)}
\PY{n}{F\PYZus{}5}\PY{o}{.}\PY{n}{cauchy}\PY{p}{(}\PY{n}{C3}\PY{p}{,}\PY{n}{xlim}\PY{o}{=}\PY{p}{(}\PY{o}{\PYZhy{}}\PY{l+m+mf}{2.5}\PY{p}{,}\PY{l+m+mf}{1.5}\PY{p}{)}\PY{p}{,}\PY{n}{ylim}\PY{o}{=}\PY{p}{(}\PY{o}{\PYZhy{}}\PY{l+m+mf}{1.5}\PY{p}{,}\PY{l+m+mf}{1.5}\PY{p}{)}\PY{p}{)}
\PY{n}{F\PYZus{}5}\PY{o}{.}\PY{n}{cauchy}\PY{p}{(}\PY{n}{C3\PYZus{}inv}\PY{p}{,}\PY{n}{xlim}\PY{o}{=}\PY{p}{(}\PY{o}{\PYZhy{}}\PY{l+m+mf}{2.5}\PY{p}{,}\PY{l+m+mf}{1.5}\PY{p}{)}\PY{p}{,}\PY{n}{ylim}\PY{o}{=}\PY{p}{(}\PY{o}{\PYZhy{}}\PY{l+m+mf}{1.5}\PY{p}{,}\PY{l+m+mf}{1.5}\PY{p}{)}\PY{p}{)}
\end{Verbatim}
\end{tcolorbox}

    \begin{Verbatim}[commandchars=\\\{\}]
************************************************************
Cauchy plot : contour

              (p+0.75-0.5j)(p+0.75+0.5j)(p+1.65)(p+2-j)(p+2+j)
F\_5(p) = 0.75 ------------------------------------------------
                               (p+1)(p+2.25)

************************************************************

************************************************************
Cauchy plot : contour

              (p+0.75-0.5j)(p+0.75+0.5j)(p+1.65)(p+2-j)(p+2+j)
F\_5(p) = 0.75 ------------------------------------------------
                               (p+1)(p+2.25)

************************************************************

    \end{Verbatim}

    \begin{center}
    \adjustimage{max size={0.9\linewidth}{0.9\paperheight}}{output_31_1.png}
    \end{center}
    { \hspace*{\fill} \\}
    
    \begin{center}
    \adjustimage{max size={0.9\linewidth}{0.9\paperheight}}{output_31_2.png}
    \end{center}
    { \hspace*{\fill} \\}
    

    % Add a bibliography block to the postdoc
    
    
    
