%%%%%%%%%%%%%%%%%%%%%%%%%%%%%%%%%%%%%%%%%%%%%%%%%%%%%%%%%%%%%%%%%%%%%%%%%%%%%%%%
%%%%%%%%%%%%%%%%%%%%%%%%%%%%%%%%%%%%%%%%%%%%%%%%%%%%%%%%%%%%%%%%%%%%%%%%%%%%%%%%
%%%%%%%%%%%%%%%%%%%%%%%%%%%%%%%%%%%%%%%%%%%%%%%%%%%%%%%%%%%%%%%%%%%%%%%%%%%%%%%%
%%%%%%%%%%%%%%%%%%%%%%%%%%%%%%%%%%%%%%%%%%%%%%%%%%%%%%%%%%%%%%%%%%%%%%%%%%%%%%%%
\chapter{Transformée de Laplace inverse~\label{annexe-invL}}
%%%%%%%%%%%%%%%%%%%%%%%%%%%%%%%%%%%%%%%%%%%%%%%%%%%%%%%%%%%%%%%%%%%%%%%%%%%%%%%%
%%%%%%%%%%%%%%%%%%%%%%%%%%%%%%%%%%%%%%%%%%%%%%%%%%%%%%%%%%%%%%%%%%%%%%%%%%%%%%%%
%%%%%%%%%%%%%%%%%%%%%%%%%%%%%%%%%%%%%%%%%%%%%%%%%%%%%%%%%%%%%%%%%%%%%%%%%%%%%%%%
%%%%%%%%%%%%%%%%%%%%%%%%%%%%%%%%%%%%%%%%%%%%%%%%%%%%%%%%%%%%%%%%%%%%%%%%%%%%%%%%

%%%%%%%%%%%%%%%%%%%%%%%%%%%%%%%%%%%%%%%%%%%%%%%%%%%%%%%%%%%%%%%%%%%%%%%%%%%%%%%%
%%%%%%%%%%%%%%%%%%%%%%%%%%%%%%%%%%%%%%%%%%%%%%%%%%%%%%%%%%%%%%%%%%%%%%%%%%%%%%%%
%%%%%%%%%%%%%%%%%%%%%%%%%%%%%%%%%%%%%%%%%%%%%%%%%%%%%%%%%%%%%%%%%%%%%%%%%%%%%%%%
\section{Contexte}
%%%%%%%%%%%%%%%%%%%%%%%%%%%%%%%%%%%%%%%%%%%%%%%%%%%%%%%%%%%%%%%%%%%%%%%%%%%%%%%%
%%%%%%%%%%%%%%%%%%%%%%%%%%%%%%%%%%%%%%%%%%%%%%%%%%%%%%%%%%%%%%%%%%%%%%%%%%%%%%%%
%%%%%%%%%%%%%%%%%%%%%%%%%%%%%%%%%%%%%%%%%%%%%%%%%%%%%%%%%%%%%%%%%%%%%%%%%%%%%%%%
Il existe une forme analytique de la transformée inverse basée sur
la formule de Mellin-Fourier\cite{Ostertag}:
\[
s(t)=\laplacei{S(p)}=\int_{c-j\infty}^{c+j\infty} e^{pt}S(p)\dd{p}
\]
Cette inversion se fait donc par le biais d'une intégrale dans le plan
complexe. Il existe différentes méthodes numériques pour obtenir des résultats
approchés de cette intégrale. Nous discuterons de deux algorithmes : 
l'algorithme de Gaver-Stehfest et l'algorithme fixe de Talbot comme présentés
par Abate et al.~\cite{abate2004,abate2006,frwiki:200676411}.
Nous présentons une implémention python de ces deux algorithmes.
%%%%%%%%%%%%%%%%%%%%%%%%%%%%%%%%%%%%%%%%%%%%%%%%%%%%%%%%%%%%%%%%%%%%%%%%%%%%%%%%
%%%%%%%%%%%%%%%%%%%%%%%%%%%%%%%%%%%%%%%%%%%%%%%%%%%%%%%%%%%%%%%%%%%%%%%%%%%%%%%%
%%%%%%%%%%%%%%%%%%%%%%%%%%%%%%%%%%%%%%%%%%%%%%%%%%%%%%%%%%%%%%%%%%%%%%%%%%%%%%%%
\section{Algorithme de Gaver-Stehfest}
%%%%%%%%%%%%%%%%%%%%%%%%%%%%%%%%%%%%%%%%%%%%%%%%%%%%%%%%%%%%%%%%%%%%%%%%%%%%%%%%
%%%%%%%%%%%%%%%%%%%%%%%%%%%%%%%%%%%%%%%%%%%%%%%%%%%%%%%%%%%%%%%%%%%%%%%%%%%%%%%%
%%%%%%%%%%%%%%%%%%%%%%%%%%%%%%%%%%%%%%%%%%%%%%%%%%%%%%%%%%%%%%%%%%%%%%%%%%%%%%%%
D'abord donnons nous une fonction \texttt{v\_stehfest} pour déterminer 
la relation (32) de~\cite{abate2006} c'est à dire :
\[
\zeta_k = (-1)^{M+k} \sum_{j=\lfloor(k+1)/2\rfloor}^{\min{(k,M)}} 
\dfrac{j^{M+1}}{M!}\begin{pmatrix} M\\j  \end{pmatrix}
                   \begin{pmatrix} 2j\\j \end{pmatrix}
                   \begin{pmatrix} j\\k-j\end{pmatrix}
\]
avec $\lfloor x \rfloor$ le plus grand entier inférieur ou égal à $x$. Il 
nous faudra définir les coefficients binomiaux.

\inputminted{python}{codes/python/coeff_bin-annexe_invL.py}
\inputminted{python}{codes/python/1er_ordre-annexe_invL.py}
\inputminted{python}{codes/python/2nd_ordre-annexe_invL.py}
\inputminted{python}{codes/python/coeff_bin-annexe_invL.py}
\inputminted{python}{codes/python/gaver_stehfest-annexe_invL.py}
\inputminted{python}{codes/python/talbot-annexe_invL.py}
\inputminted{python}{codes/python/v_stehfest-annexe_invL.py}


%%%%%%%%%%%%%%%%%%%%%%%%%%%%%%%%%%%%%%%%%%%%%%%%%%%%%%%%%%%%%%%%%%%%%%%%%%%%%%%%
%%%%%%%%%%%%%%%%%%%%%%%%%%%%%%%%%%%%%%%%%%%%%%%%%%%%%%%%%%%%%%%%%%%%%%%%%%%%%%%%
%%%%%%%%%%%%%%%%%%%%%%%%%%%%%%%%%%%%%%%%%%%%%%%%%%%%%%%%%%%%%%%%%%%%%%%%%%%%%%%%
\section{Algorithme fixe de Talbot}
%%%%%%%%%%%%%%%%%%%%%%%%%%%%%%%%%%%%%%%%%%%%%%%%%%%%%%%%%%%%%%%%%%%%%%%%%%%%%%%%
%%%%%%%%%%%%%%%%%%%%%%%%%%%%%%%%%%%%%%%%%%%%%%%%%%%%%%%%%%%%%%%%%%%%%%%%%%%%%%%%
%%%%%%%%%%%%%%%%%%%%%%%%%%%%%%%%%%%%%%%%%%%%%%%%%%%%%%%%%%%%%%%%%%%%%%%%%%%%%%%%
On souhaite se donner la relation (18) de~\cite{abate2004} 
qui se présente sous la forme :
\[
f(t) = \dfrac{r}{M} 
\left\{ 
    \dfrac{1}{2} F(r) e^{rt}+ \sum_{k=1}^{M-1} 
    \textrm{Re}\left[ e^{ts(\theta_k)} F(s(\theta_k)) (1+i\sigma(\theta_k))\right]
\right\}
\]
avec $\theta_k=\dfrac{k\pi}{M}$.

Les relations (19), (16) et (11) de [3] respectivements: 
\[
r=\dfrac{2M}{5t}
\]
\[
\sigma(\theta)=\theta+(\theta\cot\theta-1)\cot\theta
\] 
\[
s(\theta)=r\theta(\cot\theta+i)
\]
sont nécessaires pour implémenter la méthode.


%%%%%%%%%%%%%%%%%%%%%%%%%%%%%%%%%%%%%%%%%%%%%%%%%%%%%%%%%%%%%%%%%%%%%%%%%%%%%%%%
%%%%%%%%%%%%%%%%%%%%%%%%%%%%%%%%%%%%%%%%%%%%%%%%%%%%%%%%%%%%%%%%%%%%%%%%%%%%%%%%
%%%%%%%%%%%%%%%%%%%%%%%%%%%%%%%%%%%%%%%%%%%%%%%%%%%%%%%%%%%%%%%%%%%%%%%%%%%%%%%%
%%%%%%%%%%%%%%%%%%%%%%%%%%%%%%%%%%%%%%%%%%%%%%%%%%%%%%%%%%%%%%%%%%%%%%%%%%%%%%%%
%annexe_invL.tex
