%%%%%%%%%%%%%%%%%%%%%%%%%%%%%%%%%%%%%%%%%%%%%%%%%%%%%%%%%%%%%%%%%%%%%%%%%%%%%%%%
%%%%%%%%%%%%%%%%%%%%%%%%%%%%%%%%%%%%%%%%%%%%%%%%%%%%%%%%%%%%%%%%%%%%%%%%%%%%%%%%
\exercice{Diagrammes de Bode~\difficile}
%%%%%%%%%%%%%%%%%%%%%%%%%%%%%%%%%%%%%%%%%%%%%%%%%%%%%%%%%%%%%%%%%%%%%%%%%%%%%%%%
%%%%%%%%%%%%%%%%%%%%%%%%%%%%%%%%%%%%%%%%%%%%%%%%%%%%%%%%%%%%%%%%%%%%%%%%%%%%%%%%
%%%%%%%%%%%%%%%%%%%%%%%%%%%%%%%%%%%%%%%%%%%%%%%%%%%%%%%%%%%%%%%%%%%%%%%%%%%%%%%%
\question{\textbf{Tracer les diagrammes de Bode asymptotique et réel 
pour les fonctions de transfert suivantes :}}
%%%%%%%%%%%%%%%%%%%%%%%%%%%%%%%%%%%%%%%%%%%%%%%%%%%%%%%%%%%%%%%%%%%%%%%%%%%%%%%%
(a)
\[
    H_1(p)=\dfrac{100(10p+1)}{(0.1p+1)^2}
\]
%-------------------------------------------------------------------------------
\begin{center}
    \tikzsetnextfilename{bode_h1_gain-exercices_corriges-chap_repfreq-ext}
    \begin{tikzpicture}[trim axis left]
\begin{axis}
[
    ticklabel style = {font=\footnotesize},
    width=0.9\textwidth,
    height=0.22\textheight,
    grid=both,
    major grid style={black!40},
    %label style={font=\large},
    xmode=log,ymode=normal,ylabel={Gain(\si{\decibel})},
    xtick={1e-4,1e-3,1e-2,1e-1,1e0,1e1,1e2,1e3,1e4},
    ytick={20.0,30.0,40.0,50.0,60.0,70.0,80.0,90.0},
    xticklabels={$10^{-4}$,$10^{-3}$,$10^{-2}$,$10^{-1}$,
                 $10^{0}$,$10^{1}$,$10^{2}$,$10^{3}$,$10^{4}$},
    ytick={20.0,30.0,40.0,50.0,60.0,70.0,80.0,90.0},
    xmin=1e-4,xmax=1e4,
    ymin=20.0,ymax=90.0
]
\addplot[ultra thick, col1,domain=1e-4:1e4, samples=201] 
        {40+10*log10(1+100.0*x*x)-20*log10(1+0.010000000000000002*x*x)};
\addplot[line width=2pt,col4,dashed,domain=0.0001:0.1,samples=51] 
        {40.0+0.0*log10(x)};
\addplot[line width=2pt,col4,dashed,domain=0.1:10.0,samples=51] 
        {60.0+20.0*log10(x)};
\addplot[line width=2pt,col4,dashed,domain=10.0:10000,samples=51] 
        {100.0+-20.0*log10(x)};
%\draw[draw=none,fill=col1] (axis cs:0.01,40.043205051941136) circle (2pt);
%\draw[draw=none,fill=col1] (axis cs:0.1,43.00943141110256) circle (2pt);
%\draw[draw=none,fill=col1] (axis cs:10.0,73.979834359489) circle (2pt);
%\draw[draw=none,fill=col1] (axis cs:100.0,59.91357686728979) circle (2pt);
\end{axis}
\end{tikzpicture}


    \tikzsetnextfilename{bode_h1_phase-exercices_corriges-chap_repfreq-ext}
    \input{tikz/bode_h1_phase-exercices_corriges-chap_repfreq.tex}
\end{center}
%-------------------------------------------------------------------------------
(b)
\[ 
    H_2(p)=\dfrac{(p+1)}{p^2(1+0.3p+p^2)} 
\]
%-------------------------------------------------------------------------------
\begin{center}
    \tikzsetnextfilename{bode_h2_gain-exercices_corriges-chap_repfreq-ext}
    \begin{tikzpicture}[trim axis left]
\begin{axis}[
    ticklabel style = {font=\footnotesize},
    width=0.9\textwidth,
    height=0.22\textheight,
    ylabel={Gain (\si{\decibel})},
    xtick={1e-2,1e-1,1,1e1,1e2},
    xticklabels={$10^{-2}$,$10^{-1}$,$10^{0}$,$10^{1}$,$10^{2}$},
    ytick={-80,-60,-40,-20,0,20,40,60,80},
    yticklabels={-80,-60,-40,-20,0,20,40,60,80},
    xmode=log,ymode=normal,
    xmin=1e-2, xmax=1e2,
    ymin=-80, ymax=80,
    grid=both,
    major grid style={black!40}]

    \pgfmathsetmacro{\xi}{0.15}
    \pgfmathsetmacro{\txi}{2*\xi}
    \pgfmathsetmacro{\wz}{1}
    \pgfmathsetmacro{\K}{1}
\addplot[ultra thick,col1,domain=1e-2:1e2,samples=201] 
        {20*log10(1+x^2)+20*log10(\K*\wz^2)-
         10*log10((\wz^2-x^2)^2+(\txi*x)^2)-
          40*log10(x)};
\addplot[line width=2pt,col4,dashed,domain=1e-2:1e2, samples=51] {-40*log10(x)};
    %\draw[draw=none,fill=col1] (axis cs:0.5,2.32) circle (2pt);
    %\draw[draw=none,fill=col1] (axis cs:1,10.45)  circle (2pt);
    %\draw[draw=none,fill=col1] (axis cs:2,-9.7)   circle (2pt);
    %0.5 2.328441339178196 -11.309932474020213
    %1 10.457574905606752 -90.0
    %2 -9.712758487381052 -168.6900675259798
\end{axis}
\end{tikzpicture}


    \tikzsetnextfilename{bode_h2_phase-exercices_corriges-chap_repfreq-ext}
    \input{tikz/bode_h2_phase-exercices_corriges-chap_repfreq.tex}
\end{center}
%-------------------------------------------------------------------------------
(c)
\[
H_3(p)=\dfrac{100}{(1+0.1p)(1+0.01p)(1+0.001p)}
\]
%-------------------------------------------------------------------------------
\begin{center}
    \tikzsetnextfilename{bode_h3_gain-exercices_corriges-chap_repfreq-ext}
    \input{tikz/bode_h3_gain-exercices_corriges-chap_repfreq.tex}
    \tikzsetnextfilename{bode_h3_phase-exercices_corriges-chap_repfreq-ext}
    \begin{tikzpicture}[trim axis left]
\begin{axis}
    [
    ticklabel style = {font=\footnotesize},
    width=0.9\textwidth,
    height=0.25\textheight,
    grid=both,
    major grid style={black!40},
    %label style={font=\large},
    xmode=log,ymode=normal,xlabel={Pulsation (\si{\radian\per\second})},
    ylabel={Phase (\degreeSI)},
    xtick={1e-1,1e0,1e1,1e2,1e3,1e4,1e5},
    xticklabels={$10^{-1}$,$10^{0}$,$10^{1}$,
                 $10^{2}$,$10^{3}$,$10^{4}$,$10^{5}$},
    ytick={-270,-225,-180,-135,-90,-45,0},
    yticklabels={-270,-225,-180,-135,-90,-45,0},
    xmin=1e-1,xmax=1e5,
    ymin=-270.0,ymax=0.0
    ]
\addplot[ultra thick, col1,domain=1e-1:1e5, samples=201] 
        {-1*atan2(0.1*x,1)-atan2(0.01*x,1)-atan2(0.001*x,1)};
\addplot[line width=2pt,col4,dashed,domain=1e-2:1e1,samples=51] 
        {0.0};
\addplot[line width=2pt,col4,dashed,domain=1e1:1e2,samples=51] {-90.0};
\addplot[line width=2pt,col4,dashed,domain=1e2:1e3,samples=51] {-180.0};
\addplot[line width=2pt,col4,dashed,domain=1e3:1e6,samples=51] {-270.0};
\draw[line width=2pt,col4,dashed] (axis cs:1e1,0)  -- (axis cs:1e1,-90.0);
\draw[line width=2pt,col4,dashed] (axis cs:1e2,-90)  -- (axis cs:1e2,-180.0);
\draw[line width=2pt,col4,dashed] (axis cs:1e3,-180)  -- (axis cs:1e3,-270.0);
\end{axis}
\end{tikzpicture}

\end{center}
%-------------------------------------------------------------------------------

%%%%%%%%%%%%%%%%%%%%%%%%%%%%%%%%%%%%%%%%%%%%%%%%%%%%%%%%%%%%%%%%%%%%%%%%%%%%%%%%
\question{\textbf{Pour chacun des systèmes, déterminer la valeur de la 
pulsation $\omega_{0\si{\dB}}$, c'est à dire la pulsation pour laquelle le 
gain vaut 0~\si{\dB}.}}
%%%%%%%%%%%%%%%%%%%%%%%%%%%%%%%%%%%%%%%%%%%%%%%%%%%%%%%%%%%%%%%%%%%%%%%%%%%%%%%%
En lisant le diagramme de Bode 
$\omega_{0\si{\dB}}\sim\SI{300}{\radian\per\second}$

Par le calcul on cherche $\omega$ tel que 
\[
\left|(1+0.1\jw)(1+0.01\jw)(1+0.001\jw)\right|=100
\]
soit encore
\[
    (1+10^{-2}\omega)(1+10^{-4}\omega)(1+10^{-6}\omega)=10^4    
\]
La résolution de cette expression en $\omega$ donne
\[
    \omega_{0\si{\dB}}\sim301.307 
\]
