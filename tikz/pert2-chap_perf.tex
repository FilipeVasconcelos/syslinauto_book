%%%%%%%%%%%%%%%%%%%%%%%%%%%%%%%%%%%%%%%%%%%%%%%%%%%%%%%%%%%%%%%%%%%%%%%%%%%%%%%%
%   Nom des noeuds
%                        P1            P2 
%                        |             |
%                        |             | 
%   E ---- a ---- b ---- c ---- d ---- e ---- S
%          |                                  |
%          |                                  |
%           ------------ g ---- f ------------ 
%                        |
%                        |
%                        P3
% E  : entrée
% a  : comparateur/régulateur
% b  : correcteur 
% c  : sommateur1 (perturbation)
% d  : système 
% e  : sommateur2 (perturbation)
% f  : capteur
% g  : sommateur3 (perturbation/bruit)
% P1 : noeud décalé pour la perturbation
% P2 : noeud décalé pour la perturbation
% P3 : noeud décalé pour la perturbation
% S  : sortie
% r  : noeud décalé pour le retour
%%%%%%%%%%%%%%%%%%%%%%%%%%%%%%%%%%%%%%%%%%%%%%%%%%%%%%%%%%%%%%%%%%%%%%%%%%%%%%%%
\begin{tikzpicture}                  
    \sbEntree{E}
    \sbComp{comp1}{E}
    \sbRelier[$E(p)$]{E}{comp1}
    \sbBloc[3]{C}{$\dfrac{1}{p}$}{comp1}
    \sbRelier[$\epsilon(p)$]{comp1}{C}
    \sbBloc[3]{H}{$\dfrac{1}{p+1}$}{C}
    \sbRelier[]{C}{H}
    \sbSortie[3]{S}{H}
    \sbRelier[$S(p)$]{H}{S}
    \sbRenvoi[4]{H-S}{comp1}{}
\end{tikzpicture}                    
