\chapter{TODO}
\section{Notes:}

\begin{itemize}
\item notion de retard à présenter avant la présentation des signaux usuels
\item régime transitoire/permanent avant l'introduction de la TL
\item Définition de l'équation différentielle à coefficients constants avant l'introduction de la TL
\item Cas de CI non nulles (Réponse forcée/libre)
\item FTCD, FTBO à introduire dans le cas de l'asservissement
\item Ponctuation des équations en relation avec leur place dans le texte.
\item Regrouper les infos suivantes dans une annexe dédiée, pour chaque réponse temporelle du 1er et du 2nd ordre:
    \begin{itemize}
        \item valeur initiale (pente à l'origine)
        \item valeur finale (pente à l'infinie)
        \item forme analytique des réponses
    \end{itemize}
\item Place les définitions des temps de réponse, montée avant la modélisation, ou alors définition locale
    et traiter le problème dabs le chapitre des performances et caractéristiques d'un système asservi.
\item unité de $\tau$
\item Distinguer clairement les 3 régimes dans le cas de la réponse harmonique:
    \begin{itemize}
        \item apériodique $\Rightarrow$ 2 systèmes du 1er ordre
        \item critique $\Rightarrow$ rien de spécial
        \item pseudo-périodique $\Rightarrow$ phénomène de résoance
    \end{itemize}
\item \'Etablir le transitoire de la réponse harmonique du 1er ordre et/ou 2nd ordre.
\end{itemize}
